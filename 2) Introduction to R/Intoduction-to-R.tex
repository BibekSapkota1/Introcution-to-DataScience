% Options for packages loaded elsewhere
\PassOptionsToPackage{unicode}{hyperref}
\PassOptionsToPackage{hyphens}{url}
%
\documentclass[
]{article}
\usepackage{amsmath,amssymb}
\usepackage{iftex}
\ifPDFTeX
  \usepackage[T1]{fontenc}
  \usepackage[utf8]{inputenc}
  \usepackage{textcomp} % provide euro and other symbols
\else % if luatex or xetex
  \usepackage{unicode-math} % this also loads fontspec
  \defaultfontfeatures{Scale=MatchLowercase}
  \defaultfontfeatures[\rmfamily]{Ligatures=TeX,Scale=1}
\fi
\usepackage{lmodern}
\ifPDFTeX\else
  % xetex/luatex font selection
\fi
% Use upquote if available, for straight quotes in verbatim environments
\IfFileExists{upquote.sty}{\usepackage{upquote}}{}
\IfFileExists{microtype.sty}{% use microtype if available
  \usepackage[]{microtype}
  \UseMicrotypeSet[protrusion]{basicmath} % disable protrusion for tt fonts
}{}
\makeatletter
\@ifundefined{KOMAClassName}{% if non-KOMA class
  \IfFileExists{parskip.sty}{%
    \usepackage{parskip}
  }{% else
    \setlength{\parindent}{0pt}
    \setlength{\parskip}{6pt plus 2pt minus 1pt}}
}{% if KOMA class
  \KOMAoptions{parskip=half}}
\makeatother
\usepackage{xcolor}
\usepackage[margin=1in]{geometry}
\usepackage{color}
\usepackage{fancyvrb}
\newcommand{\VerbBar}{|}
\newcommand{\VERB}{\Verb[commandchars=\\\{\}]}
\DefineVerbatimEnvironment{Highlighting}{Verbatim}{commandchars=\\\{\}}
% Add ',fontsize=\small' for more characters per line
\usepackage{framed}
\definecolor{shadecolor}{RGB}{248,248,248}
\newenvironment{Shaded}{\begin{snugshade}}{\end{snugshade}}
\newcommand{\AlertTok}[1]{\textcolor[rgb]{0.94,0.16,0.16}{#1}}
\newcommand{\AnnotationTok}[1]{\textcolor[rgb]{0.56,0.35,0.01}{\textbf{\textit{#1}}}}
\newcommand{\AttributeTok}[1]{\textcolor[rgb]{0.13,0.29,0.53}{#1}}
\newcommand{\BaseNTok}[1]{\textcolor[rgb]{0.00,0.00,0.81}{#1}}
\newcommand{\BuiltInTok}[1]{#1}
\newcommand{\CharTok}[1]{\textcolor[rgb]{0.31,0.60,0.02}{#1}}
\newcommand{\CommentTok}[1]{\textcolor[rgb]{0.56,0.35,0.01}{\textit{#1}}}
\newcommand{\CommentVarTok}[1]{\textcolor[rgb]{0.56,0.35,0.01}{\textbf{\textit{#1}}}}
\newcommand{\ConstantTok}[1]{\textcolor[rgb]{0.56,0.35,0.01}{#1}}
\newcommand{\ControlFlowTok}[1]{\textcolor[rgb]{0.13,0.29,0.53}{\textbf{#1}}}
\newcommand{\DataTypeTok}[1]{\textcolor[rgb]{0.13,0.29,0.53}{#1}}
\newcommand{\DecValTok}[1]{\textcolor[rgb]{0.00,0.00,0.81}{#1}}
\newcommand{\DocumentationTok}[1]{\textcolor[rgb]{0.56,0.35,0.01}{\textbf{\textit{#1}}}}
\newcommand{\ErrorTok}[1]{\textcolor[rgb]{0.64,0.00,0.00}{\textbf{#1}}}
\newcommand{\ExtensionTok}[1]{#1}
\newcommand{\FloatTok}[1]{\textcolor[rgb]{0.00,0.00,0.81}{#1}}
\newcommand{\FunctionTok}[1]{\textcolor[rgb]{0.13,0.29,0.53}{\textbf{#1}}}
\newcommand{\ImportTok}[1]{#1}
\newcommand{\InformationTok}[1]{\textcolor[rgb]{0.56,0.35,0.01}{\textbf{\textit{#1}}}}
\newcommand{\KeywordTok}[1]{\textcolor[rgb]{0.13,0.29,0.53}{\textbf{#1}}}
\newcommand{\NormalTok}[1]{#1}
\newcommand{\OperatorTok}[1]{\textcolor[rgb]{0.81,0.36,0.00}{\textbf{#1}}}
\newcommand{\OtherTok}[1]{\textcolor[rgb]{0.56,0.35,0.01}{#1}}
\newcommand{\PreprocessorTok}[1]{\textcolor[rgb]{0.56,0.35,0.01}{\textit{#1}}}
\newcommand{\RegionMarkerTok}[1]{#1}
\newcommand{\SpecialCharTok}[1]{\textcolor[rgb]{0.81,0.36,0.00}{\textbf{#1}}}
\newcommand{\SpecialStringTok}[1]{\textcolor[rgb]{0.31,0.60,0.02}{#1}}
\newcommand{\StringTok}[1]{\textcolor[rgb]{0.31,0.60,0.02}{#1}}
\newcommand{\VariableTok}[1]{\textcolor[rgb]{0.00,0.00,0.00}{#1}}
\newcommand{\VerbatimStringTok}[1]{\textcolor[rgb]{0.31,0.60,0.02}{#1}}
\newcommand{\WarningTok}[1]{\textcolor[rgb]{0.56,0.35,0.01}{\textbf{\textit{#1}}}}
\usepackage{graphicx}
\makeatletter
\def\maxwidth{\ifdim\Gin@nat@width>\linewidth\linewidth\else\Gin@nat@width\fi}
\def\maxheight{\ifdim\Gin@nat@height>\textheight\textheight\else\Gin@nat@height\fi}
\makeatother
% Scale images if necessary, so that they will not overflow the page
% margins by default, and it is still possible to overwrite the defaults
% using explicit options in \includegraphics[width, height, ...]{}
\setkeys{Gin}{width=\maxwidth,height=\maxheight,keepaspectratio}
% Set default figure placement to htbp
\makeatletter
\def\fps@figure{htbp}
\makeatother
\setlength{\emergencystretch}{3em} % prevent overfull lines
\providecommand{\tightlist}{%
  \setlength{\itemsep}{0pt}\setlength{\parskip}{0pt}}
\setcounter{secnumdepth}{-\maxdimen} % remove section numbering
\ifLuaTeX
  \usepackage{selnolig}  % disable illegal ligatures
\fi
\IfFileExists{bookmark.sty}{\usepackage{bookmark}}{\usepackage{hyperref}}
\IfFileExists{xurl.sty}{\usepackage{xurl}}{} % add URL line breaks if available
\urlstyle{same}
\hypersetup{
  pdftitle={Introduction to R},
  pdfauthor={Bibek Sapkota},
  hidelinks,
  pdfcreator={LaTeX via pandoc}}

\title{Introduction to R}
\author{Bibek Sapkota}
\date{}

\begin{document}
\maketitle

\hypertarget{tibbles}{%
\section{Tibbles}\label{tibbles}}

Task 1:Loading the tidyverse package.

\begin{Shaded}
\begin{Highlighting}[]
\FunctionTok{library}\NormalTok{(tidyverse)}
\end{Highlighting}
\end{Shaded}

\begin{verbatim}
## -- Attaching core tidyverse packages ------------------------ tidyverse 2.0.0 --
## v dplyr     1.1.4     v readr     2.1.5
## v forcats   1.0.0     v stringr   1.5.1
## v ggplot2   3.5.0     v tibble    3.2.1
## v lubridate 1.9.3     v tidyr     1.3.1
## v purrr     1.0.2     
## -- Conflicts ------------------------------------------ tidyverse_conflicts() --
## x dplyr::filter() masks stats::filter()
## x dplyr::lag()    masks stats::lag()
## i Use the conflicted package (<http://conflicted.r-lib.org/>) to force all conflicts to become errors
\end{verbatim}

Task 2:Converting the iris dataset to a tibble.

\begin{Shaded}
\begin{Highlighting}[]
\FunctionTok{as\_tibble}\NormalTok{(iris)}
\end{Highlighting}
\end{Shaded}

\begin{verbatim}
## # A tibble: 150 x 5
##    Sepal.Length Sepal.Width Petal.Length Petal.Width Species
##           <dbl>       <dbl>        <dbl>       <dbl> <fct>  
##  1          5.1         3.5          1.4         0.2 setosa 
##  2          4.9         3            1.4         0.2 setosa 
##  3          4.7         3.2          1.3         0.2 setosa 
##  4          4.6         3.1          1.5         0.2 setosa 
##  5          5           3.6          1.4         0.2 setosa 
##  6          5.4         3.9          1.7         0.4 setosa 
##  7          4.6         3.4          1.4         0.3 setosa 
##  8          5           3.4          1.5         0.2 setosa 
##  9          4.4         2.9          1.4         0.2 setosa 
## 10          4.9         3.1          1.5         0.1 setosa 
## # i 140 more rows
\end{verbatim}

Task 3: Creating a tibble with columns ``x,'' ``y,'' and ``z,'' where
``x'' ranges from 1 to 5, ``y'' is 1 for all rows, and ``z'' is
calculated as the square of ``x'' plus ``y'' for each row.

\begin{Shaded}
\begin{Highlighting}[]
\FunctionTok{tibble}\NormalTok{(}
  \AttributeTok{x =} \DecValTok{1}\SpecialCharTok{:}\DecValTok{5}\NormalTok{, }
  \AttributeTok{y =} \DecValTok{1}\NormalTok{, }
  \AttributeTok{z =}\NormalTok{ x }\SpecialCharTok{\^{}} \DecValTok{2} \SpecialCharTok{+}\NormalTok{ y}
\NormalTok{)}
\end{Highlighting}
\end{Shaded}

\begin{verbatim}
## # A tibble: 5 x 3
##       x     y     z
##   <int> <dbl> <dbl>
## 1     1     1     2
## 2     2     1     5
## 3     3     1    10
## 4     4     1    17
## 5     5     1    26
\end{verbatim}

Task 4:Creating a tibble with columns named ``:)'' (representing
``smile''), '' '' (representing ``space''), and ``2000'' (representing
``number'').

\begin{Shaded}
\begin{Highlighting}[]
\NormalTok{tb }\OtherTok{\textless{}{-}} \FunctionTok{tibble}\NormalTok{(}
  \StringTok{\textasciigrave{}}\AttributeTok{:)}\StringTok{\textasciigrave{}} \OtherTok{=} \StringTok{"smile"}\NormalTok{, }
  \StringTok{\textasciigrave{}}\AttributeTok{ }\StringTok{\textasciigrave{}} \OtherTok{=} \StringTok{"space"}\NormalTok{,}
  \StringTok{\textasciigrave{}}\AttributeTok{2000}\StringTok{\textasciigrave{}} \OtherTok{=} \StringTok{"number"}
\NormalTok{)}
\NormalTok{tb}
\end{Highlighting}
\end{Shaded}

\begin{verbatim}
## # A tibble: 1 x 3
##   `:)`  ` `   `2000`
##   <chr> <chr> <chr> 
## 1 smile space number
\end{verbatim}

Task 5:Creating a tibble with columns ``x,'' ``y,'' and ``z,''
containing the values ``a,'' 2, 3.6 and ``b,'' 1, 8.5 respectively.

\begin{Shaded}
\begin{Highlighting}[]
\FunctionTok{tribble}\NormalTok{(}
  \SpecialCharTok{\textasciitilde{}}\NormalTok{x, }\SpecialCharTok{\textasciitilde{}}\NormalTok{y, }\SpecialCharTok{\textasciitilde{}}\NormalTok{z,}
  
  \StringTok{"a"}\NormalTok{, }\DecValTok{2}\NormalTok{, }\FloatTok{3.6}\NormalTok{,}
  \StringTok{"b"}\NormalTok{, }\DecValTok{1}\NormalTok{, }\FloatTok{8.5}
\NormalTok{)}
\end{Highlighting}
\end{Shaded}

\begin{verbatim}
## # A tibble: 2 x 3
##   x         y     z
##   <chr> <dbl> <dbl>
## 1 a         2   3.6
## 2 b         1   8.5
\end{verbatim}

\hypertarget{tibbles-vs.-data.frame}{%
\section{Tibbles vs.~data.frame}\label{tibbles-vs.-data.frame}}

Task-1:Creating a tibble with columns ``a,'' ``b,'' ``c,'' ``d,'' and
``e,'' containing 1000 randomly generated values for each column,
representing dates, numbers, and letters.

\begin{Shaded}
\begin{Highlighting}[]
\FunctionTok{tibble}\NormalTok{(}
  \AttributeTok{a =}\NormalTok{ lubridate}\SpecialCharTok{::}\FunctionTok{now}\NormalTok{() }\SpecialCharTok{+} \FunctionTok{runif}\NormalTok{(}\FloatTok{1e3}\NormalTok{) }\SpecialCharTok{*} \DecValTok{86400}\NormalTok{,}
  \AttributeTok{b =}\NormalTok{ lubridate}\SpecialCharTok{::}\FunctionTok{today}\NormalTok{() }\SpecialCharTok{+} \FunctionTok{runif}\NormalTok{(}\FloatTok{1e3}\NormalTok{) }\SpecialCharTok{*} \DecValTok{30}\NormalTok{,}
  \AttributeTok{c =} \DecValTok{1}\SpecialCharTok{:}\FloatTok{1e3}\NormalTok{,}
  \AttributeTok{d =} \FunctionTok{runif}\NormalTok{(}\FloatTok{1e3}\NormalTok{),}
  \AttributeTok{e =} \FunctionTok{sample}\NormalTok{(letters, }\FloatTok{1e3}\NormalTok{, }\AttributeTok{replace =} \ConstantTok{TRUE}\NormalTok{)}
\NormalTok{)}
\end{Highlighting}
\end{Shaded}

\begin{verbatim}
## # A tibble: 1,000 x 5
##    a                   b              c     d e    
##    <dttm>              <date>     <int> <dbl> <chr>
##  1 2024-06-13 10:22:11 2024-06-13     1 0.564 q    
##  2 2024-06-14 00:55:33 2024-07-04     2 0.912 o    
##  3 2024-06-13 21:19:50 2024-06-15     3 0.859 f    
##  4 2024-06-14 04:15:38 2024-07-02     4 0.990 o    
##  5 2024-06-13 21:03:18 2024-06-19     5 0.124 c    
##  6 2024-06-13 16:51:29 2024-07-02     6 0.836 w    
##  7 2024-06-13 20:08:12 2024-07-10     7 0.811 c    
##  8 2024-06-14 04:09:25 2024-06-29     8 0.250 e    
##  9 2024-06-13 21:56:36 2024-06-21     9 0.868 u    
## 10 2024-06-14 06:24:36 2024-06-14    10 0.499 q    
## # i 990 more rows
\end{verbatim}

Task 2: Tnstalling the package

\begin{Shaded}
\begin{Highlighting}[]
\NormalTok{package\_to\_install }\OtherTok{\textless{}{-}} \FunctionTok{c}\NormalTok{(}\StringTok{"nycflights13"}\NormalTok{)}

\ControlFlowTok{for}\NormalTok{ (package\_name }\ControlFlowTok{in}\NormalTok{ package\_to\_install) \{}
  \ControlFlowTok{if}\NormalTok{ (}\SpecialCharTok{!}\FunctionTok{requireNamespace}\NormalTok{(package\_name, }\AttributeTok{quietly =} \ConstantTok{TRUE}\NormalTok{)) \{}
    \FunctionTok{install.packages}\NormalTok{(package\_name)}
\NormalTok{  \}}
\NormalTok{\}}
\FunctionTok{library}\NormalTok{(nycflights13)}
\end{Highlighting}
\end{Shaded}

Task 3: Printing the first 10 rows of the nycflights13::flights dataset
with unlimited width.

\begin{Shaded}
\begin{Highlighting}[]
\NormalTok{nycflights13}\SpecialCharTok{::}\NormalTok{flights }\SpecialCharTok{\%\textgreater{}\%} 
  \FunctionTok{print}\NormalTok{(}\AttributeTok{n =} \DecValTok{10}\NormalTok{, }\AttributeTok{width =} \ConstantTok{Inf}\NormalTok{)}
\end{Highlighting}
\end{Shaded}

\begin{verbatim}
## # A tibble: 336,776 x 19
##     year month   day dep_time sched_dep_time dep_delay arr_time sched_arr_time
##    <int> <int> <int>    <int>          <int>     <dbl>    <int>          <int>
##  1  2013     1     1      517            515         2      830            819
##  2  2013     1     1      533            529         4      850            830
##  3  2013     1     1      542            540         2      923            850
##  4  2013     1     1      544            545        -1     1004           1022
##  5  2013     1     1      554            600        -6      812            837
##  6  2013     1     1      554            558        -4      740            728
##  7  2013     1     1      555            600        -5      913            854
##  8  2013     1     1      557            600        -3      709            723
##  9  2013     1     1      557            600        -3      838            846
## 10  2013     1     1      558            600        -2      753            745
##    arr_delay carrier flight tailnum origin dest  air_time distance  hour minute
##        <dbl> <chr>    <int> <chr>   <chr>  <chr>    <dbl>    <dbl> <dbl>  <dbl>
##  1        11 UA        1545 N14228  EWR    IAH        227     1400     5     15
##  2        20 UA        1714 N24211  LGA    IAH        227     1416     5     29
##  3        33 AA        1141 N619AA  JFK    MIA        160     1089     5     40
##  4       -18 B6         725 N804JB  JFK    BQN        183     1576     5     45
##  5       -25 DL         461 N668DN  LGA    ATL        116      762     6      0
##  6        12 UA        1696 N39463  EWR    ORD        150      719     5     58
##  7        19 B6         507 N516JB  EWR    FLL        158     1065     6      0
##  8       -14 EV        5708 N829AS  LGA    IAD         53      229     6      0
##  9        -8 B6          79 N593JB  JFK    MCO        140      944     6      0
## 10         8 AA         301 N3ALAA  LGA    ORD        138      733     6      0
##    time_hour          
##    <dttm>             
##  1 2013-01-01 05:00:00
##  2 2013-01-01 05:00:00
##  3 2013-01-01 05:00:00
##  4 2013-01-01 05:00:00
##  5 2013-01-01 06:00:00
##  6 2013-01-01 05:00:00
##  7 2013-01-01 06:00:00
##  8 2013-01-01 06:00:00
##  9 2013-01-01 06:00:00
## 10 2013-01-01 06:00:00
## # i 336,766 more rows
\end{verbatim}

Task 4: Viewing the nycflights13::flights dataset in a separate window
for interactive exploration.

\begin{Shaded}
\begin{Highlighting}[]
\NormalTok{nycflights13}\SpecialCharTok{::}\NormalTok{flights }\SpecialCharTok{\%\textgreater{}\%} 
  \FunctionTok{View}\NormalTok{()}
\end{Highlighting}
\end{Shaded}

\hypertarget{subsetting}{%
\subsection{Subsetting}\label{subsetting}}

Task 1: Creating a tibble named ``df'' with columns ``x'' and ``y,''
then accessing the ``x'' column using different methods:

\begin{Shaded}
\begin{Highlighting}[]
\NormalTok{df }\OtherTok{\textless{}{-}} \FunctionTok{tibble}\NormalTok{(}
  \AttributeTok{x =} \FunctionTok{runif}\NormalTok{(}\DecValTok{5}\NormalTok{),}\CommentTok{\#function that generates random numbers from a uniform distribution}
  \AttributeTok{y =} \FunctionTok{rnorm}\NormalTok{(}\DecValTok{5}\NormalTok{) }\CommentTok{\# function that generates random numbers from a normal (Gaussian) distribution}
\NormalTok{)}

\NormalTok{df}\SpecialCharTok{$}\NormalTok{x}
\end{Highlighting}
\end{Shaded}

\begin{verbatim}
## [1] 0.8860408 0.3144429 0.6872370 0.8185281 0.4759915
\end{verbatim}

\begin{Shaded}
\begin{Highlighting}[]
\NormalTok{df[[}\StringTok{"x"}\NormalTok{]]}
\end{Highlighting}
\end{Shaded}

\begin{verbatim}
## [1] 0.8860408 0.3144429 0.6872370 0.8185281 0.4759915
\end{verbatim}

\begin{Shaded}
\begin{Highlighting}[]
\NormalTok{df[[}\DecValTok{1}\NormalTok{]]}
\end{Highlighting}
\end{Shaded}

\begin{verbatim}
## [1] 0.8860408 0.3144429 0.6872370 0.8185281 0.4759915
\end{verbatim}

\begin{Shaded}
\begin{Highlighting}[]
\NormalTok{df }\SpecialCharTok{\%\textgreater{}\%}\NormalTok{ .}\SpecialCharTok{$}\NormalTok{x}
\end{Highlighting}
\end{Shaded}

\begin{verbatim}
## [1] 0.8860408 0.3144429 0.6872370 0.8185281 0.4759915
\end{verbatim}

\hypertarget{interacting-with-older-code}{%
\subsection{Interacting with older
code}\label{interacting-with-older-code}}

Task-1: Determining the class of the object ``tb'' after converting it
to a data frame.

\begin{Shaded}
\begin{Highlighting}[]
\FunctionTok{class}\NormalTok{(}\FunctionTok{as.data.frame}\NormalTok{(tb))}
\end{Highlighting}
\end{Shaded}

\begin{verbatim}
## [1] "data.frame"
\end{verbatim}

\hypertarget{exercises}{%
\subsection{Exercises}\label{exercises}}

Task-1: How can you tell if an object is a tibble? (Hint: try printing
mtcars, which is a regular data frame).

\begin{Shaded}
\begin{Highlighting}[]
\NormalTok{mtcars}
\end{Highlighting}
\end{Shaded}

\begin{verbatim}
##                      mpg cyl  disp  hp drat    wt  qsec vs am gear carb
## Mazda RX4           21.0   6 160.0 110 3.90 2.620 16.46  0  1    4    4
## Mazda RX4 Wag       21.0   6 160.0 110 3.90 2.875 17.02  0  1    4    4
## Datsun 710          22.8   4 108.0  93 3.85 2.320 18.61  1  1    4    1
## Hornet 4 Drive      21.4   6 258.0 110 3.08 3.215 19.44  1  0    3    1
## Hornet Sportabout   18.7   8 360.0 175 3.15 3.440 17.02  0  0    3    2
## Valiant             18.1   6 225.0 105 2.76 3.460 20.22  1  0    3    1
## Duster 360          14.3   8 360.0 245 3.21 3.570 15.84  0  0    3    4
## Merc 240D           24.4   4 146.7  62 3.69 3.190 20.00  1  0    4    2
## Merc 230            22.8   4 140.8  95 3.92 3.150 22.90  1  0    4    2
## Merc 280            19.2   6 167.6 123 3.92 3.440 18.30  1  0    4    4
## Merc 280C           17.8   6 167.6 123 3.92 3.440 18.90  1  0    4    4
## Merc 450SE          16.4   8 275.8 180 3.07 4.070 17.40  0  0    3    3
## Merc 450SL          17.3   8 275.8 180 3.07 3.730 17.60  0  0    3    3
## Merc 450SLC         15.2   8 275.8 180 3.07 3.780 18.00  0  0    3    3
## Cadillac Fleetwood  10.4   8 472.0 205 2.93 5.250 17.98  0  0    3    4
## Lincoln Continental 10.4   8 460.0 215 3.00 5.424 17.82  0  0    3    4
## Chrysler Imperial   14.7   8 440.0 230 3.23 5.345 17.42  0  0    3    4
## Fiat 128            32.4   4  78.7  66 4.08 2.200 19.47  1  1    4    1
## Honda Civic         30.4   4  75.7  52 4.93 1.615 18.52  1  1    4    2
## Toyota Corolla      33.9   4  71.1  65 4.22 1.835 19.90  1  1    4    1
## Toyota Corona       21.5   4 120.1  97 3.70 2.465 20.01  1  0    3    1
## Dodge Challenger    15.5   8 318.0 150 2.76 3.520 16.87  0  0    3    2
## AMC Javelin         15.2   8 304.0 150 3.15 3.435 17.30  0  0    3    2
## Camaro Z28          13.3   8 350.0 245 3.73 3.840 15.41  0  0    3    4
## Pontiac Firebird    19.2   8 400.0 175 3.08 3.845 17.05  0  0    3    2
## Fiat X1-9           27.3   4  79.0  66 4.08 1.935 18.90  1  1    4    1
## Porsche 914-2       26.0   4 120.3  91 4.43 2.140 16.70  0  1    5    2
## Lotus Europa        30.4   4  95.1 113 3.77 1.513 16.90  1  1    5    2
## Ford Pantera L      15.8   8 351.0 264 4.22 3.170 14.50  0  1    5    4
## Ferrari Dino        19.7   6 145.0 175 3.62 2.770 15.50  0  1    5    6
## Maserati Bora       15.0   8 301.0 335 3.54 3.570 14.60  0  1    5    8
## Volvo 142E          21.4   4 121.0 109 4.11 2.780 18.60  1  1    4    2
\end{verbatim}

Task-2

\begin{Shaded}
\begin{Highlighting}[]
\CommentTok{\# In a data.frame, extracting a non{-}existent column returns NULL,}
\CommentTok{\# whereas in a tibble, it raises an error, providing immediate feedback.}
\CommentTok{\# Other operations, such as extracting existing columns and subsets of columns,}
\CommentTok{\# behave similarly across both data frames and tibbles.}
\CommentTok{\# The default behavior of data.frames may lead to frustration}
\CommentTok{\# due to the lack of error feedback for non{-}existent columns,}
\CommentTok{\# potentially causing unnoticed mistakes and difficulty in debugging.}
\CommentTok{\# In contrast, tibbles offer more robust behavior, enhancing data integrity}
\CommentTok{\# and debugging efficiency.}

\NormalTok{df }\OtherTok{\textless{}{-}} \FunctionTok{data.frame}\NormalTok{(}\AttributeTok{abc =} \DecValTok{1}\NormalTok{, }\AttributeTok{xyz =} \StringTok{"a"}\NormalTok{)}

\CommentTok{\# Extracting non{-}existent column in a data.frame}
\NormalTok{df}\SpecialCharTok{$}\NormalTok{x  }\CommentTok{\# Returns NULL}
\end{Highlighting}
\end{Shaded}

\begin{verbatim}
## [1] "a"
\end{verbatim}

\begin{Shaded}
\begin{Highlighting}[]
\CommentTok{\# Extracting existing column in a data.frame}
\NormalTok{df[, }\StringTok{"xyz"}\NormalTok{]  }\CommentTok{\# Returns a data frame with one column containing the values of the "xyz" column}
\end{Highlighting}
\end{Shaded}

\begin{verbatim}
## [1] "a"
\end{verbatim}

\begin{Shaded}
\begin{Highlighting}[]
\CommentTok{\# Extracting multiple columns in a data.frame}
\NormalTok{df[, }\FunctionTok{c}\NormalTok{(}\StringTok{"abc"}\NormalTok{, }\StringTok{"xyz"}\NormalTok{)]  }\CommentTok{\# Returns a data frame containing only the specified columns}
\end{Highlighting}
\end{Shaded}

\begin{verbatim}
##   abc xyz
## 1   1   a
\end{verbatim}

Task-3:If you have the name of a variable stored in an object, e.g.~var
\textless- ``mpg'', how can you extract the reference variable from a
tibble?

\hypertarget{no-pacakages}{%
\section{No pacakages}\label{no-pacakages}}

\begin{Shaded}
\begin{Highlighting}[]
\CommentTok{\# heights \textless{}{-} read\_csv("data/heights.csv")}
\end{Highlighting}
\end{Shaded}

Task 1: listing several tables: table1, table2, table3, table4a, and
table4b.

\begin{Shaded}
\begin{Highlighting}[]
\NormalTok{table1}
\end{Highlighting}
\end{Shaded}

\begin{verbatim}
## # A tibble: 6 x 4
##   country      year  cases population
##   <chr>       <dbl>  <dbl>      <dbl>
## 1 Afghanistan  1999    745   19987071
## 2 Afghanistan  2000   2666   20595360
## 3 Brazil       1999  37737  172006362
## 4 Brazil       2000  80488  174504898
## 5 China        1999 212258 1272915272
## 6 China        2000 213766 1280428583
\end{verbatim}

\begin{Shaded}
\begin{Highlighting}[]
\NormalTok{table2}
\end{Highlighting}
\end{Shaded}

\begin{verbatim}
## # A tibble: 12 x 4
##    country      year type            count
##    <chr>       <dbl> <chr>           <dbl>
##  1 Afghanistan  1999 cases             745
##  2 Afghanistan  1999 population   19987071
##  3 Afghanistan  2000 cases            2666
##  4 Afghanistan  2000 population   20595360
##  5 Brazil       1999 cases           37737
##  6 Brazil       1999 population  172006362
##  7 Brazil       2000 cases           80488
##  8 Brazil       2000 population  174504898
##  9 China        1999 cases          212258
## 10 China        1999 population 1272915272
## 11 China        2000 cases          213766
## 12 China        2000 population 1280428583
\end{verbatim}

\begin{Shaded}
\begin{Highlighting}[]
\NormalTok{table3}
\end{Highlighting}
\end{Shaded}

\begin{verbatim}
## # A tibble: 6 x 3
##   country      year rate             
##   <chr>       <dbl> <chr>            
## 1 Afghanistan  1999 745/19987071     
## 2 Afghanistan  2000 2666/20595360    
## 3 Brazil       1999 37737/172006362  
## 4 Brazil       2000 80488/174504898  
## 5 China        1999 212258/1272915272
## 6 China        2000 213766/1280428583
\end{verbatim}

\begin{Shaded}
\begin{Highlighting}[]
\NormalTok{table4a}
\end{Highlighting}
\end{Shaded}

\begin{verbatim}
## # A tibble: 3 x 3
##   country     `1999` `2000`
##   <chr>        <dbl>  <dbl>
## 1 Afghanistan    745   2666
## 2 Brazil       37737  80488
## 3 China       212258 213766
\end{verbatim}

\begin{Shaded}
\begin{Highlighting}[]
\NormalTok{table4b}
\end{Highlighting}
\end{Shaded}

\begin{verbatim}
## # A tibble: 3 x 3
##   country         `1999`     `2000`
##   <chr>            <dbl>      <dbl>
## 1 Afghanistan   19987071   20595360
## 2 Brazil       172006362  174504898
## 3 China       1272915272 1280428583
\end{verbatim}

Task 2: Calculating the rate by dividing the number of cases by the
population and then multiplying by 10,000 for table1.

\begin{Shaded}
\begin{Highlighting}[]
\NormalTok{table1 }\SpecialCharTok{\%\textgreater{}\%} 
  \FunctionTok{mutate}\NormalTok{(}\AttributeTok{rate =}\NormalTok{ cases }\SpecialCharTok{/}\NormalTok{ population }\SpecialCharTok{*} \DecValTok{10000}\NormalTok{)}
\end{Highlighting}
\end{Shaded}

\begin{verbatim}
## # A tibble: 6 x 5
##   country      year  cases population  rate
##   <chr>       <dbl>  <dbl>      <dbl> <dbl>
## 1 Afghanistan  1999    745   19987071 0.373
## 2 Afghanistan  2000   2666   20595360 1.29 
## 3 Brazil       1999  37737  172006362 2.19 
## 4 Brazil       2000  80488  174504898 4.61 
## 5 China        1999 212258 1272915272 1.67 
## 6 China        2000 213766 1280428583 1.67
\end{verbatim}

Task 3: Counting the occurrences of each year in table1, using the
`cases' column as the weight.

\begin{Shaded}
\begin{Highlighting}[]
\NormalTok{table1 }\SpecialCharTok{\%\textgreater{}\%} 
  \FunctionTok{count}\NormalTok{(year, }\AttributeTok{wt =}\NormalTok{ cases)}
\end{Highlighting}
\end{Shaded}

\begin{verbatim}
## # A tibble: 2 x 2
##    year      n
##   <dbl>  <dbl>
## 1  1999 250740
## 2  2000 296920
\end{verbatim}

Task 4: Creating a ggplot using table1, plotting `year' against `cases'
with lines grouped by `country' and colored in grey50, along with points
colored by `country'.

\begin{Shaded}
\begin{Highlighting}[]
\FunctionTok{library}\NormalTok{(ggplot2)}
\FunctionTok{ggplot}\NormalTok{(table1, }\FunctionTok{aes}\NormalTok{(year, cases)) }\SpecialCharTok{+} 
  \FunctionTok{geom\_line}\NormalTok{(}\FunctionTok{aes}\NormalTok{(}\AttributeTok{group =}\NormalTok{ country), }\AttributeTok{colour =} \StringTok{"grey50"}\NormalTok{) }\SpecialCharTok{+} 
  \FunctionTok{geom\_point}\NormalTok{(}\FunctionTok{aes}\NormalTok{(}\AttributeTok{colour =}\NormalTok{ country))}
\end{Highlighting}
\end{Shaded}

\includegraphics{Intoduction-to-R_files/figure-latex/unnamed-chunk-18-1.pdf}
\# Pivoting \#\# Longer Task-1: referring to `table4a'

\begin{Shaded}
\begin{Highlighting}[]
\NormalTok{table4a}
\end{Highlighting}
\end{Shaded}

\begin{verbatim}
## # A tibble: 3 x 3
##   country     `1999` `2000`
##   <chr>        <dbl>  <dbl>
## 1 Afghanistan    745   2666
## 2 Brazil       37737  80488
## 3 China       212258 213766
\end{verbatim}

Task-2: Reshaping table4a using pivot\_longer for columns `1999' and
`2000' into `year' and `cases'.

\begin{Shaded}
\begin{Highlighting}[]
\NormalTok{table4a }\SpecialCharTok{\%\textgreater{}\%} 
  \FunctionTok{pivot\_longer}\NormalTok{(}\FunctionTok{c}\NormalTok{(}\StringTok{\textasciigrave{}}\AttributeTok{1999}\StringTok{\textasciigrave{}}\NormalTok{, }\StringTok{\textasciigrave{}}\AttributeTok{2000}\StringTok{\textasciigrave{}}\NormalTok{), }\AttributeTok{names\_to =} \StringTok{"year"}\NormalTok{, }\AttributeTok{values\_to =} \StringTok{"cases"}\NormalTok{)}
\end{Highlighting}
\end{Shaded}

\begin{verbatim}
## # A tibble: 6 x 3
##   country     year   cases
##   <chr>       <chr>  <dbl>
## 1 Afghanistan 1999     745
## 2 Afghanistan 2000    2666
## 3 Brazil      1999   37737
## 4 Brazil      2000   80488
## 5 China       1999  212258
## 6 China       2000  213766
\end{verbatim}

Task-3: Reshaping table4b with pivot\_longer for columns `1999' and
`2000' into `year' and `population'.

\begin{Shaded}
\begin{Highlighting}[]
\NormalTok{table4b }\SpecialCharTok{\%\textgreater{}\%} 
  \FunctionTok{pivot\_longer}\NormalTok{(}\FunctionTok{c}\NormalTok{(}\StringTok{\textasciigrave{}}\AttributeTok{1999}\StringTok{\textasciigrave{}}\NormalTok{, }\StringTok{\textasciigrave{}}\AttributeTok{2000}\StringTok{\textasciigrave{}}\NormalTok{), }\AttributeTok{names\_to =} \StringTok{"year"}\NormalTok{, }\AttributeTok{values\_to =} \StringTok{"population"}\NormalTok{)  }\CommentTok{\#function transforms wide data into long format by stacking multiple columns into two: one for variable names and one for their corresponding values}
\end{Highlighting}
\end{Shaded}

\begin{verbatim}
## # A tibble: 6 x 3
##   country     year  population
##   <chr>       <chr>      <dbl>
## 1 Afghanistan 1999    19987071
## 2 Afghanistan 2000    20595360
## 3 Brazil      1999   172006362
## 4 Brazil      2000   174504898
## 5 China       1999  1272915272
## 6 China       2000  1280428583
\end{verbatim}

Task-4: creating tidy datasets tidy4a and tidy4b by using pivot\_longer
on table4a and table4b to reshape them. Then, performing a left join on
tidy4a and tidy4b.

\begin{Shaded}
\begin{Highlighting}[]
\NormalTok{tidy4a }\OtherTok{\textless{}{-}}\NormalTok{ table4a }\SpecialCharTok{\%\textgreater{}\%} 
  \FunctionTok{pivot\_longer}\NormalTok{(}\FunctionTok{c}\NormalTok{(}\StringTok{\textasciigrave{}}\AttributeTok{1999}\StringTok{\textasciigrave{}}\NormalTok{, }\StringTok{\textasciigrave{}}\AttributeTok{2000}\StringTok{\textasciigrave{}}\NormalTok{), }\AttributeTok{names\_to =} \StringTok{"year"}\NormalTok{, }\AttributeTok{values\_to =} \StringTok{"cases"}\NormalTok{)}
\NormalTok{tidy4b }\OtherTok{\textless{}{-}}\NormalTok{ table4b }\SpecialCharTok{\%\textgreater{}\%} 
  \FunctionTok{pivot\_longer}\NormalTok{(}\FunctionTok{c}\NormalTok{(}\StringTok{\textasciigrave{}}\AttributeTok{1999}\StringTok{\textasciigrave{}}\NormalTok{, }\StringTok{\textasciigrave{}}\AttributeTok{2000}\StringTok{\textasciigrave{}}\NormalTok{), }\AttributeTok{names\_to =} \StringTok{"year"}\NormalTok{, }\AttributeTok{values\_to =} \StringTok{"population"}\NormalTok{)}
\FunctionTok{left\_join}\NormalTok{(tidy4a, tidy4b)}
\end{Highlighting}
\end{Shaded}

\begin{verbatim}
## Joining with `by = join_by(country, year)`
\end{verbatim}

\begin{verbatim}
## # A tibble: 6 x 4
##   country     year   cases population
##   <chr>       <chr>  <dbl>      <dbl>
## 1 Afghanistan 1999     745   19987071
## 2 Afghanistan 2000    2666   20595360
## 3 Brazil      1999   37737  172006362
## 4 Brazil      2000   80488  174504898
## 5 China       1999  212258 1272915272
## 6 China       2000  213766 1280428583
\end{verbatim}

\hypertarget{wider}{%
\subsection{Wider}\label{wider}}

Task-1:Displaying table 2

\begin{Shaded}
\begin{Highlighting}[]
\NormalTok{table2}
\end{Highlighting}
\end{Shaded}

\begin{verbatim}
## # A tibble: 12 x 4
##    country      year type            count
##    <chr>       <dbl> <chr>           <dbl>
##  1 Afghanistan  1999 cases             745
##  2 Afghanistan  1999 population   19987071
##  3 Afghanistan  2000 cases            2666
##  4 Afghanistan  2000 population   20595360
##  5 Brazil       1999 cases           37737
##  6 Brazil       1999 population  172006362
##  7 Brazil       2000 cases           80488
##  8 Brazil       2000 population  174504898
##  9 China        1999 cases          212258
## 10 China        1999 population 1272915272
## 11 China        2000 cases          213766
## 12 China        2000 population 1280428583
\end{verbatim}

Task-2: using the pivot\_wider function on table2 to transform it from
long to wide format, with `type' becoming the new column names and
`count' being the corresponding values.

\begin{Shaded}
\begin{Highlighting}[]
\NormalTok{table2 }\SpecialCharTok{\%\textgreater{}\%}
    \FunctionTok{pivot\_wider}\NormalTok{(}\AttributeTok{names\_from =}\NormalTok{ type, }\AttributeTok{values\_from =}\NormalTok{ count)}
\end{Highlighting}
\end{Shaded}

\begin{verbatim}
## # A tibble: 6 x 4
##   country      year  cases population
##   <chr>       <dbl>  <dbl>      <dbl>
## 1 Afghanistan  1999    745   19987071
## 2 Afghanistan  2000   2666   20595360
## 3 Brazil       1999  37737  172006362
## 4 Brazil       2000  80488  174504898
## 5 China        1999 212258 1272915272
## 6 China        2000 213766 1280428583
\end{verbatim}

\hypertarget{separating-and-uniting}{%
\section{Separating and uniting}\label{separating-and-uniting}}

\hypertarget{separate}{%
\subsection{Separate}\label{separate}}

Task-1:displaying table3

\begin{Shaded}
\begin{Highlighting}[]
\NormalTok{ table3}
\end{Highlighting}
\end{Shaded}

\begin{verbatim}
## # A tibble: 6 x 3
##   country      year rate             
##   <chr>       <dbl> <chr>            
## 1 Afghanistan  1999 745/19987071     
## 2 Afghanistan  2000 2666/20595360    
## 3 Brazil       1999 37737/172006362  
## 4 Brazil       2000 80488/174504898  
## 5 China        1999 212258/1272915272
## 6 China        2000 213766/1280428583
\end{verbatim}

Task-2: Using the separate function on table3 splits the `rate' column
into two separate columns named `cases' and `population'.

\begin{Shaded}
\begin{Highlighting}[]
\NormalTok{table3 }\SpecialCharTok{\%\textgreater{}\%} 
  \FunctionTok{separate}\NormalTok{(rate, }\AttributeTok{into =} \FunctionTok{c}\NormalTok{(}\StringTok{"cases"}\NormalTok{, }\StringTok{"population"}\NormalTok{))}
\end{Highlighting}
\end{Shaded}

\begin{verbatim}
## # A tibble: 6 x 4
##   country      year cases  population
##   <chr>       <dbl> <chr>  <chr>     
## 1 Afghanistan  1999 745    19987071  
## 2 Afghanistan  2000 2666   20595360  
## 3 Brazil       1999 37737  172006362 
## 4 Brazil       2000 80488  174504898 
## 5 China        1999 212258 1272915272
## 6 China        2000 213766 1280428583
\end{verbatim}

Task-3:Using the separate function on table3 splits the `rate' column
into two separate columns named `cases' and `population', using the `/'
character as the separator.

\begin{Shaded}
\begin{Highlighting}[]
\NormalTok{table3 }\SpecialCharTok{\%\textgreater{}\%} 
  \FunctionTok{separate}\NormalTok{(rate, }\AttributeTok{into =} \FunctionTok{c}\NormalTok{(}\StringTok{"cases"}\NormalTok{, }\StringTok{"population"}\NormalTok{), }\AttributeTok{sep =} \StringTok{"/"}\NormalTok{)}
\end{Highlighting}
\end{Shaded}

\begin{verbatim}
## # A tibble: 6 x 4
##   country      year cases  population
##   <chr>       <dbl> <chr>  <chr>     
## 1 Afghanistan  1999 745    19987071  
## 2 Afghanistan  2000 2666   20595360  
## 3 Brazil       1999 37737  172006362 
## 4 Brazil       2000 80488  174504898 
## 5 China        1999 212258 1272915272
## 6 China        2000 213766 1280428583
\end{verbatim}

Task-4:Using the separate function on table3 splits the `rate' column
into two separate columns named `cases' and `population', converting the
resulting columns to their appropriate data types.

\begin{Shaded}
\begin{Highlighting}[]
\NormalTok{table3 }\SpecialCharTok{\%\textgreater{}\%} 
  \FunctionTok{separate}\NormalTok{(rate, }\AttributeTok{into =} \FunctionTok{c}\NormalTok{(}\StringTok{"cases"}\NormalTok{, }\StringTok{"population"}\NormalTok{), }\AttributeTok{convert =} \ConstantTok{TRUE}\NormalTok{)}
\end{Highlighting}
\end{Shaded}

\begin{verbatim}
## # A tibble: 6 x 4
##   country      year  cases population
##   <chr>       <dbl>  <int>      <int>
## 1 Afghanistan  1999    745   19987071
## 2 Afghanistan  2000   2666   20595360
## 3 Brazil       1999  37737  172006362
## 4 Brazil       2000  80488  174504898
## 5 China        1999 212258 1272915272
## 6 China        2000 213766 1280428583
\end{verbatim}

Task-5: Applying the separate function to table3, the `year' column is
divided into two separate columns labeled `century' and `year', with the
separator defined as the second character.

\begin{Shaded}
\begin{Highlighting}[]
\NormalTok{table3 }\SpecialCharTok{\%\textgreater{}\%} 
  \FunctionTok{separate}\NormalTok{(year, }\AttributeTok{into =} \FunctionTok{c}\NormalTok{(}\StringTok{"century"}\NormalTok{, }\StringTok{"year"}\NormalTok{), }\AttributeTok{sep =} \DecValTok{2}\NormalTok{)}
\end{Highlighting}
\end{Shaded}

\begin{verbatim}
## # A tibble: 6 x 4
##   country     century year  rate             
##   <chr>       <chr>   <chr> <chr>            
## 1 Afghanistan 19      99    745/19987071     
## 2 Afghanistan 20      00    2666/20595360    
## 3 Brazil      19      99    37737/172006362  
## 4 Brazil      20      00    80488/174504898  
## 5 China       19      99    212258/1272915272
## 6 China       20      00    213766/1280428583
\end{verbatim}

\hypertarget{unite}{%
\subsection{Unite}\label{unite}}

Task-1: The unite function is applied to table5 to merge the `century'
and `year' columns into a single column named `new'.

\begin{Shaded}
\begin{Highlighting}[]
\NormalTok{table5 }\SpecialCharTok{\%\textgreater{}\%} 
  \FunctionTok{unite}\NormalTok{(new, century, year)}
\end{Highlighting}
\end{Shaded}

\begin{verbatim}
## # A tibble: 6 x 3
##   country     new   rate             
##   <chr>       <chr> <chr>            
## 1 Afghanistan 19_99 745/19987071     
## 2 Afghanistan 20_00 2666/20595360    
## 3 Brazil      19_99 37737/172006362  
## 4 Brazil      20_00 80488/174504898  
## 5 China       19_99 212258/1272915272
## 6 China       20_00 213766/1280428583
\end{verbatim}

Task-2: unite function is applied to table5 to merge the `century' and
`year' columns into a single column named `new', with no separator
between them.

\begin{Shaded}
\begin{Highlighting}[]
\NormalTok{table5 }\SpecialCharTok{\%\textgreater{}\%} 
  \FunctionTok{unite}\NormalTok{(new, century, year, }\AttributeTok{sep =} \StringTok{""}\NormalTok{)}
\end{Highlighting}
\end{Shaded}

\begin{verbatim}
## # A tibble: 6 x 3
##   country     new   rate             
##   <chr>       <chr> <chr>            
## 1 Afghanistan 1999  745/19987071     
## 2 Afghanistan 2000  2666/20595360    
## 3 Brazil      1999  37737/172006362  
## 4 Brazil      2000  80488/174504898  
## 5 China       1999  212258/1272915272
## 6 China       2000  213766/1280428583
\end{verbatim}

\hypertarget{missing-values}{%
\section{Missing values}\label{missing-values}}

Task-1: Create a tibble named ``stocks'' with columns ``year'', ``qtr''
(quarter), and ``return'', having data for 2015 and 2016, with quarterly
returns specified and some missing entries as NA.

\begin{Shaded}
\begin{Highlighting}[]
\NormalTok{stocks }\OtherTok{\textless{}{-}} \FunctionTok{tibble}\NormalTok{(}
  \AttributeTok{year   =} \FunctionTok{c}\NormalTok{(}\DecValTok{2015}\NormalTok{, }\DecValTok{2015}\NormalTok{, }\DecValTok{2015}\NormalTok{, }\DecValTok{2015}\NormalTok{, }\DecValTok{2016}\NormalTok{, }\DecValTok{2016}\NormalTok{, }\DecValTok{2016}\NormalTok{),}
  \AttributeTok{qtr    =} \FunctionTok{c}\NormalTok{(   }\DecValTok{1}\NormalTok{,    }\DecValTok{2}\NormalTok{,    }\DecValTok{3}\NormalTok{,    }\DecValTok{4}\NormalTok{,    }\DecValTok{2}\NormalTok{,    }\DecValTok{3}\NormalTok{,    }\DecValTok{4}\NormalTok{),}
  \AttributeTok{return =} \FunctionTok{c}\NormalTok{(}\FloatTok{1.88}\NormalTok{, }\FloatTok{0.59}\NormalTok{, }\FloatTok{0.35}\NormalTok{,   }\ConstantTok{NA}\NormalTok{, }\FloatTok{0.92}\NormalTok{, }\FloatTok{0.17}\NormalTok{, }\FloatTok{2.66}\NormalTok{)}
\NormalTok{)}
\end{Highlighting}
\end{Shaded}

Task-2:Pivoting the ``stocks'' tibble to widen the data, extracting
columns from the ``year'' variable and values from the ``return''
variable.

\begin{Shaded}
\begin{Highlighting}[]
\NormalTok{stocks }\SpecialCharTok{\%\textgreater{}\%} 
  \FunctionTok{pivot\_wider}\NormalTok{(}\AttributeTok{names\_from =}\NormalTok{ year, }\AttributeTok{values\_from =}\NormalTok{ return)}
\end{Highlighting}
\end{Shaded}

\begin{verbatim}
## # A tibble: 4 x 3
##     qtr `2015` `2016`
##   <dbl>  <dbl>  <dbl>
## 1     1   1.88  NA   
## 2     2   0.59   0.92
## 3     3   0.35   0.17
## 4     4  NA      2.66
\end{verbatim}

Task-3: pivot the data to a wide format with columns for each year's
returns, then reshape it back to a long format, keeping only the
non-missing values in the ``return'' column.

\begin{Shaded}
\begin{Highlighting}[]
\NormalTok{stocks }\SpecialCharTok{\%\textgreater{}\%} 
  \FunctionTok{pivot\_wider}\NormalTok{(}\AttributeTok{names\_from =}\NormalTok{ year, }\AttributeTok{values\_from =}\NormalTok{ return) }\SpecialCharTok{\%\textgreater{}\%} 
  \FunctionTok{pivot\_longer}\NormalTok{(}
    \AttributeTok{cols =} \FunctionTok{c}\NormalTok{(}\StringTok{\textasciigrave{}}\AttributeTok{2015}\StringTok{\textasciigrave{}}\NormalTok{, }\StringTok{\textasciigrave{}}\AttributeTok{2016}\StringTok{\textasciigrave{}}\NormalTok{), }
    \AttributeTok{names\_to =} \StringTok{"year"}\NormalTok{, }
    \AttributeTok{values\_to =} \StringTok{"return"}\NormalTok{, }
    \AttributeTok{values\_drop\_na =} \ConstantTok{TRUE}
\NormalTok{  )}
\end{Highlighting}
\end{Shaded}

\begin{verbatim}
## # A tibble: 6 x 3
##     qtr year  return
##   <dbl> <chr>  <dbl>
## 1     1 2015    1.88
## 2     2 2015    0.59
## 3     2 2016    0.92
## 4     3 2015    0.35
## 5     3 2016    0.17
## 6     4 2016    2.66
\end{verbatim}

Task-4:Filling missing combinations of ``year'' and ``qtr'' in the
``stocks'' dataset.

\begin{Shaded}
\begin{Highlighting}[]
\NormalTok{stocks }\SpecialCharTok{\%\textgreater{}\%} 
  \FunctionTok{complete}\NormalTok{(year, qtr)}
\end{Highlighting}
\end{Shaded}

\begin{verbatim}
## # A tibble: 8 x 3
##    year   qtr return
##   <dbl> <dbl>  <dbl>
## 1  2015     1   1.88
## 2  2015     2   0.59
## 3  2015     3   0.35
## 4  2015     4  NA   
## 5  2016     1  NA   
## 6  2016     2   0.92
## 7  2016     3   0.17
## 8  2016     4   2.66
\end{verbatim}

Task-5:Creating a tibble named ``treatment'' containing information
about individuals, their treatment groups, and their responses, with
some missing values for the ``person'' column.

\begin{Shaded}
\begin{Highlighting}[]
\NormalTok{treatment }\OtherTok{\textless{}{-}} \FunctionTok{tribble}\NormalTok{(}
  \SpecialCharTok{\textasciitilde{}}\NormalTok{ person,           }\SpecialCharTok{\textasciitilde{}}\NormalTok{ treatment, }\SpecialCharTok{\textasciitilde{}}\NormalTok{response,}
  \StringTok{"Derrick Whitmore"}\NormalTok{, }\DecValTok{1}\NormalTok{,           }\DecValTok{7}\NormalTok{,}
  \ConstantTok{NA}\NormalTok{,                 }\DecValTok{2}\NormalTok{,           }\DecValTok{10}\NormalTok{,}
  \ConstantTok{NA}\NormalTok{,                 }\DecValTok{3}\NormalTok{,           }\DecValTok{9}\NormalTok{,}
  \StringTok{"Katherine Burke"}\NormalTok{,  }\DecValTok{1}\NormalTok{,           }\DecValTok{4}
\NormalTok{)}
\end{Highlighting}
\end{Shaded}

Task-6: Filling the missing values in the ``person'' column of the
``treatment'' tibble.

\begin{Shaded}
\begin{Highlighting}[]
\NormalTok{treatment }\SpecialCharTok{\%\textgreater{}\%} 
  \FunctionTok{fill}\NormalTok{(person)}
\end{Highlighting}
\end{Shaded}

\begin{verbatim}
## # A tibble: 4 x 3
##   person           treatment response
##   <chr>                <dbl>    <dbl>
## 1 Derrick Whitmore         1        7
## 2 Derrick Whitmore         2       10
## 3 Derrick Whitmore         3        9
## 4 Katherine Burke          1        4
\end{verbatim}

\hypertarget{case-study}{%
\section{Case Study}\label{case-study}}

Task-1: Loading data set

\begin{Shaded}
\begin{Highlighting}[]
\NormalTok{who}
\end{Highlighting}
\end{Shaded}

\begin{verbatim}
## # A tibble: 7,240 x 60
##    country  iso2  iso3   year new_sp_m014 new_sp_m1524 new_sp_m2534 new_sp_m3544
##    <chr>    <chr> <chr> <dbl>       <dbl>        <dbl>        <dbl>        <dbl>
##  1 Afghani~ AF    AFG    1980          NA           NA           NA           NA
##  2 Afghani~ AF    AFG    1981          NA           NA           NA           NA
##  3 Afghani~ AF    AFG    1982          NA           NA           NA           NA
##  4 Afghani~ AF    AFG    1983          NA           NA           NA           NA
##  5 Afghani~ AF    AFG    1984          NA           NA           NA           NA
##  6 Afghani~ AF    AFG    1985          NA           NA           NA           NA
##  7 Afghani~ AF    AFG    1986          NA           NA           NA           NA
##  8 Afghani~ AF    AFG    1987          NA           NA           NA           NA
##  9 Afghani~ AF    AFG    1988          NA           NA           NA           NA
## 10 Afghani~ AF    AFG    1989          NA           NA           NA           NA
## # i 7,230 more rows
## # i 52 more variables: new_sp_m4554 <dbl>, new_sp_m5564 <dbl>,
## #   new_sp_m65 <dbl>, new_sp_f014 <dbl>, new_sp_f1524 <dbl>,
## #   new_sp_f2534 <dbl>, new_sp_f3544 <dbl>, new_sp_f4554 <dbl>,
## #   new_sp_f5564 <dbl>, new_sp_f65 <dbl>, new_sn_m014 <dbl>,
## #   new_sn_m1524 <dbl>, new_sn_m2534 <dbl>, new_sn_m3544 <dbl>,
## #   new_sn_m4554 <dbl>, new_sn_m5564 <dbl>, new_sn_m65 <dbl>, ...
\end{verbatim}

Task-2:Pivoting the ``who'' dataset from wide to long format, condensing
columns into ``cases'' and capturing the original column names in
``key''.

\begin{Shaded}
\begin{Highlighting}[]
\NormalTok{who1 }\OtherTok{\textless{}{-}}\NormalTok{ who }\SpecialCharTok{\%\textgreater{}\%} 
  \FunctionTok{pivot\_longer}\NormalTok{(}
    \AttributeTok{cols =}\NormalTok{ new\_sp\_m014}\SpecialCharTok{:}\NormalTok{newrel\_f65, }
    \AttributeTok{names\_to =} \StringTok{"key"}\NormalTok{, }
    \AttributeTok{values\_to =} \StringTok{"cases"}\NormalTok{, }
    \AttributeTok{values\_drop\_na =} \ConstantTok{TRUE}
\NormalTok{  )}
\NormalTok{who1}
\end{Highlighting}
\end{Shaded}

\begin{verbatim}
## # A tibble: 76,046 x 6
##    country     iso2  iso3   year key          cases
##    <chr>       <chr> <chr> <dbl> <chr>        <dbl>
##  1 Afghanistan AF    AFG    1997 new_sp_m014      0
##  2 Afghanistan AF    AFG    1997 new_sp_m1524    10
##  3 Afghanistan AF    AFG    1997 new_sp_m2534     6
##  4 Afghanistan AF    AFG    1997 new_sp_m3544     3
##  5 Afghanistan AF    AFG    1997 new_sp_m4554     5
##  6 Afghanistan AF    AFG    1997 new_sp_m5564     2
##  7 Afghanistan AF    AFG    1997 new_sp_m65       0
##  8 Afghanistan AF    AFG    1997 new_sp_f014      5
##  9 Afghanistan AF    AFG    1997 new_sp_f1524    38
## 10 Afghanistan AF    AFG    1997 new_sp_f2534    36
## # i 76,036 more rows
\end{verbatim}

Task-3:Counting the occurrences of each ``key'' in the ``who1'' dataset.

\begin{Shaded}
\begin{Highlighting}[]
\NormalTok{  who1 }\SpecialCharTok{\%\textgreater{}\%} 
    \FunctionTok{count}\NormalTok{(key)}
\end{Highlighting}
\end{Shaded}

\begin{verbatim}
## # A tibble: 56 x 2
##    key              n
##    <chr>        <int>
##  1 new_ep_f014   1032
##  2 new_ep_f1524  1021
##  3 new_ep_f2534  1021
##  4 new_ep_f3544  1021
##  5 new_ep_f4554  1017
##  6 new_ep_f5564  1017
##  7 new_ep_f65    1014
##  8 new_ep_m014   1038
##  9 new_ep_m1524  1026
## 10 new_ep_m2534  1020
## # i 46 more rows
\end{verbatim}

Task-4:Replacing ``newrel'' with ``new\_rel'' in the ``key'' column of
the ``who1'' dataset to create ``who2.''

\begin{Shaded}
\begin{Highlighting}[]
\NormalTok{who2 }\OtherTok{\textless{}{-}}\NormalTok{ who1 }\SpecialCharTok{\%\textgreater{}\%} 
  \FunctionTok{mutate}\NormalTok{(}\AttributeTok{key =}\NormalTok{ stringr}\SpecialCharTok{::}\FunctionTok{str\_replace}\NormalTok{(key, }\StringTok{"newrel"}\NormalTok{, }\StringTok{"new\_rel"}\NormalTok{))}
\NormalTok{who2}
\end{Highlighting}
\end{Shaded}

\begin{verbatim}
## # A tibble: 76,046 x 6
##    country     iso2  iso3   year key          cases
##    <chr>       <chr> <chr> <dbl> <chr>        <dbl>
##  1 Afghanistan AF    AFG    1997 new_sp_m014      0
##  2 Afghanistan AF    AFG    1997 new_sp_m1524    10
##  3 Afghanistan AF    AFG    1997 new_sp_m2534     6
##  4 Afghanistan AF    AFG    1997 new_sp_m3544     3
##  5 Afghanistan AF    AFG    1997 new_sp_m4554     5
##  6 Afghanistan AF    AFG    1997 new_sp_m5564     2
##  7 Afghanistan AF    AFG    1997 new_sp_m65       0
##  8 Afghanistan AF    AFG    1997 new_sp_f014      5
##  9 Afghanistan AF    AFG    1997 new_sp_f1524    38
## 10 Afghanistan AF    AFG    1997 new_sp_f2534    36
## # i 76,036 more rows
\end{verbatim}

Task-5:Separating the ``key'' column in the ``who2'' dataset into
``new,'' ``type,'' and ``sexage'' columns using ``\_'' as the separator
to create ``who3.''

\begin{Shaded}
\begin{Highlighting}[]
\NormalTok{who3 }\OtherTok{\textless{}{-}}\NormalTok{ who2 }\SpecialCharTok{\%\textgreater{}\%} 
  \FunctionTok{separate}\NormalTok{(key, }\FunctionTok{c}\NormalTok{(}\StringTok{"new"}\NormalTok{, }\StringTok{"type"}\NormalTok{, }\StringTok{"sexage"}\NormalTok{), }\AttributeTok{sep =} \StringTok{"\_"}\NormalTok{)}
\NormalTok{who3}
\end{Highlighting}
\end{Shaded}

\begin{verbatim}
## # A tibble: 76,046 x 8
##    country     iso2  iso3   year new   type  sexage cases
##    <chr>       <chr> <chr> <dbl> <chr> <chr> <chr>  <dbl>
##  1 Afghanistan AF    AFG    1997 new   sp    m014       0
##  2 Afghanistan AF    AFG    1997 new   sp    m1524     10
##  3 Afghanistan AF    AFG    1997 new   sp    m2534      6
##  4 Afghanistan AF    AFG    1997 new   sp    m3544      3
##  5 Afghanistan AF    AFG    1997 new   sp    m4554      5
##  6 Afghanistan AF    AFG    1997 new   sp    m5564      2
##  7 Afghanistan AF    AFG    1997 new   sp    m65        0
##  8 Afghanistan AF    AFG    1997 new   sp    f014       5
##  9 Afghanistan AF    AFG    1997 new   sp    f1524     38
## 10 Afghanistan AF    AFG    1997 new   sp    f2534     36
## # i 76,036 more rows
\end{verbatim}

Task-6:Counting the occurrences of each unique value in the ``new''
column of the ``who3'' dataset.

\begin{Shaded}
\begin{Highlighting}[]
\NormalTok{who3 }\SpecialCharTok{\%\textgreater{}\%} 
  \FunctionTok{count}\NormalTok{(new)}
\end{Highlighting}
\end{Shaded}

\begin{verbatim}
## # A tibble: 1 x 2
##   new       n
##   <chr> <int>
## 1 new   76046
\end{verbatim}

Task-7:Removing the ``new'', ``iso2'', and ``iso3'' columns from the
``who3'' dataset and assigning the result to ``who4''.

\begin{Shaded}
\begin{Highlighting}[]
\NormalTok{who4 }\OtherTok{\textless{}{-}}\NormalTok{ who3 }\SpecialCharTok{\%\textgreater{}\%} 
  \FunctionTok{select}\NormalTok{(}\SpecialCharTok{{-}}\NormalTok{new, }\SpecialCharTok{{-}}\NormalTok{iso2, }\SpecialCharTok{{-}}\NormalTok{iso3)}
\end{Highlighting}
\end{Shaded}

Task-8:Splitting the ``sexage'' column of the ``who4'' dataset into
``sex'' and ``age'' columns, separated by the first character, and
assigning the result to ``who5''.

\begin{Shaded}
\begin{Highlighting}[]
\NormalTok{who5 }\OtherTok{\textless{}{-}}\NormalTok{ who4 }\SpecialCharTok{\%\textgreater{}\%} 
  \FunctionTok{separate}\NormalTok{(sexage, }\FunctionTok{c}\NormalTok{(}\StringTok{"sex"}\NormalTok{, }\StringTok{"age"}\NormalTok{), }\AttributeTok{sep =} \DecValTok{1}\NormalTok{)}
\NormalTok{who5}
\end{Highlighting}
\end{Shaded}

\begin{verbatim}
## # A tibble: 76,046 x 6
##    country      year type  sex   age   cases
##    <chr>       <dbl> <chr> <chr> <chr> <dbl>
##  1 Afghanistan  1997 sp    m     014       0
##  2 Afghanistan  1997 sp    m     1524     10
##  3 Afghanistan  1997 sp    m     2534      6
##  4 Afghanistan  1997 sp    m     3544      3
##  5 Afghanistan  1997 sp    m     4554      5
##  6 Afghanistan  1997 sp    m     5564      2
##  7 Afghanistan  1997 sp    m     65        0
##  8 Afghanistan  1997 sp    f     014       5
##  9 Afghanistan  1997 sp    f     1524     38
## 10 Afghanistan  1997 sp    f     2534     36
## # i 76,036 more rows
\end{verbatim}

Task-9:Transforming the ``who'' dataset from wide to long format,
adjusting column names, extracting meaningful variables, dropping
unnecessary columns, and splitting the ``sexage'' column into ``sex''
and ``age''.

\begin{Shaded}
\begin{Highlighting}[]
\NormalTok{who }\SpecialCharTok{\%\textgreater{}\%}
  \FunctionTok{pivot\_longer}\NormalTok{(}
    \AttributeTok{cols =}\NormalTok{ new\_sp\_m014}\SpecialCharTok{:}\NormalTok{newrel\_f65, }
    \AttributeTok{names\_to =} \StringTok{"key"}\NormalTok{, }
    \AttributeTok{values\_to =} \StringTok{"cases"}\NormalTok{, }
    \AttributeTok{values\_drop\_na =} \ConstantTok{TRUE}
\NormalTok{  ) }\SpecialCharTok{\%\textgreater{}\%} 
  \FunctionTok{mutate}\NormalTok{(}
    \AttributeTok{key =}\NormalTok{ stringr}\SpecialCharTok{::}\FunctionTok{str\_replace}\NormalTok{(key, }\StringTok{"newrel"}\NormalTok{, }\StringTok{"new\_rel"}\NormalTok{)}
\NormalTok{  ) }\SpecialCharTok{\%\textgreater{}\%}
  \FunctionTok{separate}\NormalTok{(key, }\FunctionTok{c}\NormalTok{(}\StringTok{"new"}\NormalTok{, }\StringTok{"var"}\NormalTok{, }\StringTok{"sexage"}\NormalTok{)) }\SpecialCharTok{\%\textgreater{}\%} 
  \FunctionTok{select}\NormalTok{(}\SpecialCharTok{{-}}\NormalTok{new, }\SpecialCharTok{{-}}\NormalTok{iso2, }\SpecialCharTok{{-}}\NormalTok{iso3) }\SpecialCharTok{\%\textgreater{}\%} 
  \FunctionTok{separate}\NormalTok{(sexage, }\FunctionTok{c}\NormalTok{(}\StringTok{"sex"}\NormalTok{, }\StringTok{"age"}\NormalTok{), }\AttributeTok{sep =} \DecValTok{1}\NormalTok{)}
\end{Highlighting}
\end{Shaded}

\begin{verbatim}
## # A tibble: 76,046 x 6
##    country      year var   sex   age   cases
##    <chr>       <dbl> <chr> <chr> <chr> <dbl>
##  1 Afghanistan  1997 sp    m     014       0
##  2 Afghanistan  1997 sp    m     1524     10
##  3 Afghanistan  1997 sp    m     2534      6
##  4 Afghanistan  1997 sp    m     3544      3
##  5 Afghanistan  1997 sp    m     4554      5
##  6 Afghanistan  1997 sp    m     5564      2
##  7 Afghanistan  1997 sp    m     65        0
##  8 Afghanistan  1997 sp    f     014       5
##  9 Afghanistan  1997 sp    f     1524     38
## 10 Afghanistan  1997 sp    f     2534     36
## # i 76,036 more rows
\end{verbatim}

\hypertarget{ch-13-relational-data}{%
\subsection{CH-13: Relational data}\label{ch-13-relational-data}}

Task-1:Loding the libraries

\begin{Shaded}
\begin{Highlighting}[]
\FunctionTok{library}\NormalTok{(tidyverse)}
\FunctionTok{library}\NormalTok{(nycflights13)}
\end{Highlighting}
\end{Shaded}

\hypertarget{nycflights13}{%
\subsection{nycflights13}\label{nycflights13}}

Task-1: airlines data

\begin{Shaded}
\begin{Highlighting}[]
\NormalTok{airlines}
\end{Highlighting}
\end{Shaded}

\begin{verbatim}
## # A tibble: 16 x 2
##    carrier name                       
##    <chr>   <chr>                      
##  1 9E      Endeavor Air Inc.          
##  2 AA      American Airlines Inc.     
##  3 AS      Alaska Airlines Inc.       
##  4 B6      JetBlue Airways            
##  5 DL      Delta Air Lines Inc.       
##  6 EV      ExpressJet Airlines Inc.   
##  7 F9      Frontier Airlines Inc.     
##  8 FL      AirTran Airways Corporation
##  9 HA      Hawaiian Airlines Inc.     
## 10 MQ      Envoy Air                  
## 11 OO      SkyWest Airlines Inc.      
## 12 UA      United Air Lines Inc.      
## 13 US      US Airways Inc.            
## 14 VX      Virgin America             
## 15 WN      Southwest Airlines Co.     
## 16 YV      Mesa Airlines Inc.
\end{verbatim}

Task-2: airports data

\begin{Shaded}
\begin{Highlighting}[]
\NormalTok{airports}
\end{Highlighting}
\end{Shaded}

\begin{verbatim}
## # A tibble: 1,458 x 8
##    faa   name                             lat    lon   alt    tz dst   tzone    
##    <chr> <chr>                          <dbl>  <dbl> <dbl> <dbl> <chr> <chr>    
##  1 04G   Lansdowne Airport               41.1  -80.6  1044    -5 A     America/~
##  2 06A   Moton Field Municipal Airport   32.5  -85.7   264    -6 A     America/~
##  3 06C   Schaumburg Regional             42.0  -88.1   801    -6 A     America/~
##  4 06N   Randall Airport                 41.4  -74.4   523    -5 A     America/~
##  5 09J   Jekyll Island Airport           31.1  -81.4    11    -5 A     America/~
##  6 0A9   Elizabethton Municipal Airport  36.4  -82.2  1593    -5 A     America/~
##  7 0G6   Williams County Airport         41.5  -84.5   730    -5 A     America/~
##  8 0G7   Finger Lakes Regional Airport   42.9  -76.8   492    -5 A     America/~
##  9 0P2   Shoestring Aviation Airfield    39.8  -76.6  1000    -5 U     America/~
## 10 0S9   Jefferson County Intl           48.1 -123.    108    -8 A     America/~
## # i 1,448 more rows
\end{verbatim}

Task-3: planes data

\begin{Shaded}
\begin{Highlighting}[]
\NormalTok{planes }
\end{Highlighting}
\end{Shaded}

\begin{verbatim}
## # A tibble: 3,322 x 9
##    tailnum  year type              manufacturer model engines seats speed engine
##    <chr>   <int> <chr>             <chr>        <chr>   <int> <int> <int> <chr> 
##  1 N10156   2004 Fixed wing multi~ EMBRAER      EMB-~       2    55    NA Turbo~
##  2 N102UW   1998 Fixed wing multi~ AIRBUS INDU~ A320~       2   182    NA Turbo~
##  3 N103US   1999 Fixed wing multi~ AIRBUS INDU~ A320~       2   182    NA Turbo~
##  4 N104UW   1999 Fixed wing multi~ AIRBUS INDU~ A320~       2   182    NA Turbo~
##  5 N10575   2002 Fixed wing multi~ EMBRAER      EMB-~       2    55    NA Turbo~
##  6 N105UW   1999 Fixed wing multi~ AIRBUS INDU~ A320~       2   182    NA Turbo~
##  7 N107US   1999 Fixed wing multi~ AIRBUS INDU~ A320~       2   182    NA Turbo~
##  8 N108UW   1999 Fixed wing multi~ AIRBUS INDU~ A320~       2   182    NA Turbo~
##  9 N109UW   1999 Fixed wing multi~ AIRBUS INDU~ A320~       2   182    NA Turbo~
## 10 N110UW   1999 Fixed wing multi~ AIRBUS INDU~ A320~       2   182    NA Turbo~
## # i 3,312 more rows
\end{verbatim}

Task-4: weather data

\begin{Shaded}
\begin{Highlighting}[]
\NormalTok{weather }
\end{Highlighting}
\end{Shaded}

\begin{verbatim}
## # A tibble: 26,115 x 15
##    origin  year month   day  hour  temp  dewp humid wind_dir wind_speed
##    <chr>  <int> <int> <int> <int> <dbl> <dbl> <dbl>    <dbl>      <dbl>
##  1 EWR     2013     1     1     1  39.0  26.1  59.4      270      10.4 
##  2 EWR     2013     1     1     2  39.0  27.0  61.6      250       8.06
##  3 EWR     2013     1     1     3  39.0  28.0  64.4      240      11.5 
##  4 EWR     2013     1     1     4  39.9  28.0  62.2      250      12.7 
##  5 EWR     2013     1     1     5  39.0  28.0  64.4      260      12.7 
##  6 EWR     2013     1     1     6  37.9  28.0  67.2      240      11.5 
##  7 EWR     2013     1     1     7  39.0  28.0  64.4      240      15.0 
##  8 EWR     2013     1     1     8  39.9  28.0  62.2      250      10.4 
##  9 EWR     2013     1     1     9  39.9  28.0  62.2      260      15.0 
## 10 EWR     2013     1     1    10  41    28.0  59.6      260      13.8 
## # i 26,105 more rows
## # i 5 more variables: wind_gust <dbl>, precip <dbl>, pressure <dbl>,
## #   visib <dbl>, time_hour <dttm>
\end{verbatim}

\hypertarget{keys}{%
\section{Keys}\label{keys}}

Task-1Counting the occurrences of each tail number in the ``planes''
table and filtering for those with more than one occurrence.

\begin{Shaded}
\begin{Highlighting}[]
\NormalTok{planes }\SpecialCharTok{\%\textgreater{}\%} 
  \FunctionTok{count}\NormalTok{(tailnum) }\SpecialCharTok{\%\textgreater{}\%} 
  \FunctionTok{filter}\NormalTok{(n }\SpecialCharTok{\textgreater{}} \DecValTok{1}\NormalTok{)}
\end{Highlighting}
\end{Shaded}

\begin{verbatim}
## # A tibble: 0 x 2
## # i 2 variables: tailnum <chr>, n <int>
\end{verbatim}

Task-2:Counting the occurrences of each combination of year, month, day,
hour, and origin in the ``weather'' table and filtering for those with
more than one occurrence.

\begin{Shaded}
\begin{Highlighting}[]
\NormalTok{weather }\SpecialCharTok{\%\textgreater{}\%} 
  \FunctionTok{count}\NormalTok{(year, month, day, hour, origin) }\SpecialCharTok{\%\textgreater{}\%} 
  \FunctionTok{filter}\NormalTok{(n }\SpecialCharTok{\textgreater{}} \DecValTok{1}\NormalTok{)}
\end{Highlighting}
\end{Shaded}

\begin{verbatim}
## # A tibble: 3 x 6
##    year month   day  hour origin     n
##   <int> <int> <int> <int> <chr>  <int>
## 1  2013    11     3     1 EWR        2
## 2  2013    11     3     1 JFK        2
## 3  2013    11     3     1 LGA        2
\end{verbatim}

Task-3:Counting the occurrences of each combination of year, month, day,
and flight in the ``flights'' table and filtering for those with more
than one occurrence.

\begin{Shaded}
\begin{Highlighting}[]
\NormalTok{flights }\SpecialCharTok{\%\textgreater{}\%} 
  \FunctionTok{count}\NormalTok{(year, month, day, flight) }\SpecialCharTok{\%\textgreater{}\%} 
  \FunctionTok{filter}\NormalTok{(n }\SpecialCharTok{\textgreater{}} \DecValTok{1}\NormalTok{)}
\end{Highlighting}
\end{Shaded}

\begin{verbatim}
## # A tibble: 29,768 x 5
##     year month   day flight     n
##    <int> <int> <int>  <int> <int>
##  1  2013     1     1      1     2
##  2  2013     1     1      3     2
##  3  2013     1     1      4     2
##  4  2013     1     1     11     3
##  5  2013     1     1     15     2
##  6  2013     1     1     21     2
##  7  2013     1     1     27     4
##  8  2013     1     1     31     2
##  9  2013     1     1     32     2
## 10  2013     1     1     35     2
## # i 29,758 more rows
\end{verbatim}

Task-4:Counting the occurrences of each combination of year, month, day,
and tail number in the ``flights'' table and filtering for those with
more than one occurrence.

\begin{Shaded}
\begin{Highlighting}[]
\NormalTok{flights }\SpecialCharTok{\%\textgreater{}\%} 
  \FunctionTok{count}\NormalTok{(year, month, day, tailnum) }\SpecialCharTok{\%\textgreater{}\%} 
  \FunctionTok{filter}\NormalTok{(n }\SpecialCharTok{\textgreater{}} \DecValTok{1}\NormalTok{)}
\end{Highlighting}
\end{Shaded}

\begin{verbatim}
## # A tibble: 64,928 x 5
##     year month   day tailnum     n
##    <int> <int> <int> <chr>   <int>
##  1  2013     1     1 N0EGMQ      2
##  2  2013     1     1 N11189      2
##  3  2013     1     1 N11536      2
##  4  2013     1     1 N11544      3
##  5  2013     1     1 N11551      2
##  6  2013     1     1 N12540      2
##  7  2013     1     1 N12567      2
##  8  2013     1     1 N13123      2
##  9  2013     1     1 N13538      3
## 10  2013     1     1 N13566      3
## # i 64,918 more rows
\end{verbatim}

\hypertarget{mutating-joins}{%
\section{Mutating joins}\label{mutating-joins}}

Task-1: Creating a subset of the ``flights'' table named ``flights2''
containing columns from ``year'' to ``day'', ``hour'', ``origin'',
``dest'', ``tailnum'', and ``carrier''.

\begin{Shaded}
\begin{Highlighting}[]
\NormalTok{flights2 }\OtherTok{\textless{}{-}}\NormalTok{ flights }\SpecialCharTok{\%\textgreater{}\%} 
  \FunctionTok{select}\NormalTok{(year}\SpecialCharTok{:}\NormalTok{day, hour, origin, dest, tailnum, carrier)}
\NormalTok{flights2}
\end{Highlighting}
\end{Shaded}

\begin{verbatim}
## # A tibble: 336,776 x 8
##     year month   day  hour origin dest  tailnum carrier
##    <int> <int> <int> <dbl> <chr>  <chr> <chr>   <chr>  
##  1  2013     1     1     5 EWR    IAH   N14228  UA     
##  2  2013     1     1     5 LGA    IAH   N24211  UA     
##  3  2013     1     1     5 JFK    MIA   N619AA  AA     
##  4  2013     1     1     5 JFK    BQN   N804JB  B6     
##  5  2013     1     1     6 LGA    ATL   N668DN  DL     
##  6  2013     1     1     5 EWR    ORD   N39463  UA     
##  7  2013     1     1     6 EWR    FLL   N516JB  B6     
##  8  2013     1     1     6 LGA    IAD   N829AS  EV     
##  9  2013     1     1     6 JFK    MCO   N593JB  B6     
## 10  2013     1     1     6 LGA    ORD   N3ALAA  AA     
## # i 336,766 more rows
\end{verbatim}

Task-2:Removing the ``origin'' and ``dest'' columns from ``flights2''
table and then performing a left join with the ``airlines'' table, using
the ``carrier'' column as the key for matching.

\begin{Shaded}
\begin{Highlighting}[]
\NormalTok{flights2 }\SpecialCharTok{\%\textgreater{}\%}
  \FunctionTok{select}\NormalTok{(}\SpecialCharTok{{-}}\NormalTok{origin, }\SpecialCharTok{{-}}\NormalTok{dest) }\SpecialCharTok{\%\textgreater{}\%} 
  \FunctionTok{left\_join}\NormalTok{(airlines, }\AttributeTok{by =} \StringTok{"carrier"}\NormalTok{)}
\end{Highlighting}
\end{Shaded}

\begin{verbatim}
## # A tibble: 336,776 x 7
##     year month   day  hour tailnum carrier name                    
##    <int> <int> <int> <dbl> <chr>   <chr>   <chr>                   
##  1  2013     1     1     5 N14228  UA      United Air Lines Inc.   
##  2  2013     1     1     5 N24211  UA      United Air Lines Inc.   
##  3  2013     1     1     5 N619AA  AA      American Airlines Inc.  
##  4  2013     1     1     5 N804JB  B6      JetBlue Airways         
##  5  2013     1     1     6 N668DN  DL      Delta Air Lines Inc.    
##  6  2013     1     1     5 N39463  UA      United Air Lines Inc.   
##  7  2013     1     1     6 N516JB  B6      JetBlue Airways         
##  8  2013     1     1     6 N829AS  EV      ExpressJet Airlines Inc.
##  9  2013     1     1     6 N593JB  B6      JetBlue Airways         
## 10  2013     1     1     6 N3ALAA  AA      American Airlines Inc.  
## # i 336,766 more rows
\end{verbatim}

Task-3:Shortening the command by removing ``selecting'' and directly
``mutating'' the ``name'' column with the corresponding airline names
from the ``airlines'' table based on the ``carrier'' column.

\begin{Shaded}
\begin{Highlighting}[]
\NormalTok{flights2 }\SpecialCharTok{\%\textgreater{}\%}
  \FunctionTok{select}\NormalTok{(}\SpecialCharTok{{-}}\NormalTok{origin, }\SpecialCharTok{{-}}\NormalTok{dest) }\SpecialCharTok{\%\textgreater{}\%} 
  \FunctionTok{mutate}\NormalTok{(}\AttributeTok{name =}\NormalTok{ airlines}\SpecialCharTok{$}\NormalTok{name[}\FunctionTok{match}\NormalTok{(carrier, airlines}\SpecialCharTok{$}\NormalTok{carrier)])}
\end{Highlighting}
\end{Shaded}

\begin{verbatim}
## # A tibble: 336,776 x 7
##     year month   day  hour tailnum carrier name                    
##    <int> <int> <int> <dbl> <chr>   <chr>   <chr>                   
##  1  2013     1     1     5 N14228  UA      United Air Lines Inc.   
##  2  2013     1     1     5 N24211  UA      United Air Lines Inc.   
##  3  2013     1     1     5 N619AA  AA      American Airlines Inc.  
##  4  2013     1     1     5 N804JB  B6      JetBlue Airways         
##  5  2013     1     1     6 N668DN  DL      Delta Air Lines Inc.    
##  6  2013     1     1     5 N39463  UA      United Air Lines Inc.   
##  7  2013     1     1     6 N516JB  B6      JetBlue Airways         
##  8  2013     1     1     6 N829AS  EV      ExpressJet Airlines Inc.
##  9  2013     1     1     6 N593JB  B6      JetBlue Airways         
## 10  2013     1     1     6 N3ALAA  AA      American Airlines Inc.  
## # i 336,766 more rows
\end{verbatim}

\hypertarget{understanding-joins}{%
\section{Understanding joins}\label{understanding-joins}}

Task-1:Creating two tibbles, ``x'' and ``y'', each with a ``key'' column
and an associated ``val\_x'' or ``val\_y'' column, respectively.

\begin{Shaded}
\begin{Highlighting}[]
\NormalTok{x }\OtherTok{\textless{}{-}} \FunctionTok{tribble}\NormalTok{(}
  \SpecialCharTok{\textasciitilde{}}\NormalTok{key, }\SpecialCharTok{\textasciitilde{}}\NormalTok{val\_x,}
     \DecValTok{1}\NormalTok{, }\StringTok{"x1"}\NormalTok{,}
     \DecValTok{2}\NormalTok{, }\StringTok{"x2"}\NormalTok{,}
     \DecValTok{3}\NormalTok{, }\StringTok{"x3"}
\NormalTok{)}
\NormalTok{y }\OtherTok{\textless{}{-}} \FunctionTok{tribble}\NormalTok{(}
  \SpecialCharTok{\textasciitilde{}}\NormalTok{key, }\SpecialCharTok{\textasciitilde{}}\NormalTok{val\_y,}
     \DecValTok{1}\NormalTok{, }\StringTok{"y1"}\NormalTok{,}
     \DecValTok{2}\NormalTok{, }\StringTok{"y2"}\NormalTok{,}
     \DecValTok{4}\NormalTok{, }\StringTok{"y3"}
\NormalTok{)}

\NormalTok{x}
\end{Highlighting}
\end{Shaded}

\begin{verbatim}
## # A tibble: 3 x 2
##     key val_x
##   <dbl> <chr>
## 1     1 x1   
## 2     2 x2   
## 3     3 x3
\end{verbatim}

\begin{Shaded}
\begin{Highlighting}[]
\NormalTok{y}
\end{Highlighting}
\end{Shaded}

\begin{verbatim}
## # A tibble: 3 x 2
##     key val_y
##   <dbl> <chr>
## 1     1 y1   
## 2     2 y2   
## 3     4 y3
\end{verbatim}

\hypertarget{inner-join}{%
\subsection{Inner join}\label{inner-join}}

Task-1:Joining tibbles \texttt{x} and \texttt{y} using an inner join
operation based on the ``key'' column.

\begin{Shaded}
\begin{Highlighting}[]
\NormalTok{x }\SpecialCharTok{\%\textgreater{}\%} 
  \FunctionTok{inner\_join}\NormalTok{(y, }\AttributeTok{by =} \StringTok{"key"}\NormalTok{)}
\end{Highlighting}
\end{Shaded}

\begin{verbatim}
## # A tibble: 2 x 3
##     key val_x val_y
##   <dbl> <chr> <chr>
## 1     1 x1    y1   
## 2     2 x2    y2
\end{verbatim}

\hypertarget{duplicate-keys}{%
\subsection{Duplicate keys}\label{duplicate-keys}}

Task-1: Joining tibble x with tibble y using the common column ``key''.

\begin{Shaded}
\begin{Highlighting}[]
\NormalTok{x }\OtherTok{\textless{}{-}} \FunctionTok{tribble}\NormalTok{(}
  \SpecialCharTok{\textasciitilde{}}\NormalTok{key, }\SpecialCharTok{\textasciitilde{}}\NormalTok{val\_x,}
     \DecValTok{1}\NormalTok{, }\StringTok{"x1"}\NormalTok{,}
     \DecValTok{2}\NormalTok{, }\StringTok{"x2"}\NormalTok{,}
     \DecValTok{2}\NormalTok{, }\StringTok{"x3"}\NormalTok{,}
     \DecValTok{1}\NormalTok{, }\StringTok{"x4"}
\NormalTok{)}
\NormalTok{y }\OtherTok{\textless{}{-}} \FunctionTok{tribble}\NormalTok{(}
  \SpecialCharTok{\textasciitilde{}}\NormalTok{key, }\SpecialCharTok{\textasciitilde{}}\NormalTok{val\_y,}
     \DecValTok{1}\NormalTok{, }\StringTok{"y1"}\NormalTok{,}
     \DecValTok{2}\NormalTok{, }\StringTok{"y2"}
\NormalTok{)}
\end{Highlighting}
\end{Shaded}

Task-2:Performing a left join between tibble \texttt{x} and tibble
\texttt{y} based on the common column ``key''.

\begin{Shaded}
\begin{Highlighting}[]
\FunctionTok{left\_join}\NormalTok{(x, y, }\AttributeTok{by =} \StringTok{"key"}\NormalTok{)}
\end{Highlighting}
\end{Shaded}

\begin{verbatim}
## # A tibble: 4 x 3
##     key val_x val_y
##   <dbl> <chr> <chr>
## 1     1 x1    y1   
## 2     2 x2    y2   
## 3     2 x3    y2   
## 4     1 x4    y1
\end{verbatim}

Task-3:Creating two tibbles, \texttt{x} and \texttt{y}, with columns
``key'', ``val\_x'', and ``val\_y'', populated with corresponding
values.

\begin{Shaded}
\begin{Highlighting}[]
\NormalTok{x }\OtherTok{\textless{}{-}} \FunctionTok{tribble}\NormalTok{(}
  \SpecialCharTok{\textasciitilde{}}\NormalTok{key, }\SpecialCharTok{\textasciitilde{}}\NormalTok{val\_x,}
     \DecValTok{1}\NormalTok{, }\StringTok{"x1"}\NormalTok{,}
     \DecValTok{2}\NormalTok{, }\StringTok{"x2"}\NormalTok{,}
     \DecValTok{2}\NormalTok{, }\StringTok{"x3"}\NormalTok{,}
     \DecValTok{3}\NormalTok{, }\StringTok{"x4"}
\NormalTok{)}
\NormalTok{y }\OtherTok{\textless{}{-}} \FunctionTok{tribble}\NormalTok{(}
  \SpecialCharTok{\textasciitilde{}}\NormalTok{key, }\SpecialCharTok{\textasciitilde{}}\NormalTok{val\_y,}
     \DecValTok{1}\NormalTok{, }\StringTok{"y1"}\NormalTok{,}
     \DecValTok{2}\NormalTok{, }\StringTok{"y2"}\NormalTok{,}
     \DecValTok{2}\NormalTok{, }\StringTok{"y3"}\NormalTok{,}
     \DecValTok{3}\NormalTok{, }\StringTok{"y4"}
\NormalTok{)}
\end{Highlighting}
\end{Shaded}

Task-4:Performing a left join on tibbles \texttt{x} and \texttt{y} using
the ``key'' column as the join key.

\begin{Shaded}
\begin{Highlighting}[]
\FunctionTok{left\_join}\NormalTok{(x, y, }\AttributeTok{by =} \StringTok{"key"}\NormalTok{)}
\end{Highlighting}
\end{Shaded}

\begin{verbatim}
## Warning in left_join(x, y, by = "key"): Detected an unexpected many-to-many relationship between `x` and `y`.
## i Row 2 of `x` matches multiple rows in `y`.
## i Row 2 of `y` matches multiple rows in `x`.
## i If a many-to-many relationship is expected, set `relationship =
##   "many-to-many"` to silence this warning.
\end{verbatim}

\begin{verbatim}
## # A tibble: 6 x 3
##     key val_x val_y
##   <dbl> <chr> <chr>
## 1     1 x1    y1   
## 2     2 x2    y2   
## 3     2 x2    y3   
## 4     2 x3    y2   
## 5     2 x3    y3   
## 6     3 x4    y4
\end{verbatim}

\hypertarget{defining-the-key-columns}{%
\section{Defining the key columns}\label{defining-the-key-columns}}

Task-1:Performing a left join between the \texttt{flights2} tibble and
the \texttt{weather} tibble.

\begin{Shaded}
\begin{Highlighting}[]
\NormalTok{flights2 }\SpecialCharTok{\%\textgreater{}\%} 
  \FunctionTok{left\_join}\NormalTok{(weather)}
\end{Highlighting}
\end{Shaded}

\begin{verbatim}
## Joining with `by = join_by(year, month, day, hour, origin)`
\end{verbatim}

\begin{verbatim}
## # A tibble: 336,776 x 18
##     year month   day  hour origin dest  tailnum carrier  temp  dewp humid
##    <int> <int> <int> <dbl> <chr>  <chr> <chr>   <chr>   <dbl> <dbl> <dbl>
##  1  2013     1     1     5 EWR    IAH   N14228  UA       39.0  28.0  64.4
##  2  2013     1     1     5 LGA    IAH   N24211  UA       39.9  25.0  54.8
##  3  2013     1     1     5 JFK    MIA   N619AA  AA       39.0  27.0  61.6
##  4  2013     1     1     5 JFK    BQN   N804JB  B6       39.0  27.0  61.6
##  5  2013     1     1     6 LGA    ATL   N668DN  DL       39.9  25.0  54.8
##  6  2013     1     1     5 EWR    ORD   N39463  UA       39.0  28.0  64.4
##  7  2013     1     1     6 EWR    FLL   N516JB  B6       37.9  28.0  67.2
##  8  2013     1     1     6 LGA    IAD   N829AS  EV       39.9  25.0  54.8
##  9  2013     1     1     6 JFK    MCO   N593JB  B6       37.9  27.0  64.3
## 10  2013     1     1     6 LGA    ORD   N3ALAA  AA       39.9  25.0  54.8
## # i 336,766 more rows
## # i 7 more variables: wind_dir <dbl>, wind_speed <dbl>, wind_gust <dbl>,
## #   precip <dbl>, pressure <dbl>, visib <dbl>, time_hour <dttm>
\end{verbatim}

Task-2:Performing a left join between the \texttt{flights2} tibble and
the \texttt{planes} tibble using the ``tailnum'' column as the key.

\begin{Shaded}
\begin{Highlighting}[]
\NormalTok{flights2 }\SpecialCharTok{\%\textgreater{}\%} 
  \FunctionTok{left\_join}\NormalTok{(planes, }\AttributeTok{by =} \StringTok{"tailnum"}\NormalTok{)}
\end{Highlighting}
\end{Shaded}

\begin{verbatim}
## # A tibble: 336,776 x 16
##    year.x month   day  hour origin dest  tailnum carrier year.y type            
##     <int> <int> <int> <dbl> <chr>  <chr> <chr>   <chr>    <int> <chr>           
##  1   2013     1     1     5 EWR    IAH   N14228  UA        1999 Fixed wing mult~
##  2   2013     1     1     5 LGA    IAH   N24211  UA        1998 Fixed wing mult~
##  3   2013     1     1     5 JFK    MIA   N619AA  AA        1990 Fixed wing mult~
##  4   2013     1     1     5 JFK    BQN   N804JB  B6        2012 Fixed wing mult~
##  5   2013     1     1     6 LGA    ATL   N668DN  DL        1991 Fixed wing mult~
##  6   2013     1     1     5 EWR    ORD   N39463  UA        2012 Fixed wing mult~
##  7   2013     1     1     6 EWR    FLL   N516JB  B6        2000 Fixed wing mult~
##  8   2013     1     1     6 LGA    IAD   N829AS  EV        1998 Fixed wing mult~
##  9   2013     1     1     6 JFK    MCO   N593JB  B6        2004 Fixed wing mult~
## 10   2013     1     1     6 LGA    ORD   N3ALAA  AA          NA <NA>            
## # i 336,766 more rows
## # i 6 more variables: manufacturer <chr>, model <chr>, engines <int>,
## #   seats <int>, speed <int>, engine <chr>
\end{verbatim}

Task-3:Performing a left join between the \texttt{flights2} tibble and
the \texttt{airports} tibble, matching the ``dest'' column from
\texttt{flights2} with the ``faa'' column from \texttt{airports}.

\begin{Shaded}
\begin{Highlighting}[]
\NormalTok{flights2 }\SpecialCharTok{\%\textgreater{}\%} 
  \FunctionTok{left\_join}\NormalTok{(airports, }\FunctionTok{c}\NormalTok{(}\StringTok{"dest"} \OtherTok{=} \StringTok{"faa"}\NormalTok{))}
\end{Highlighting}
\end{Shaded}

\begin{verbatim}
## # A tibble: 336,776 x 15
##     year month   day  hour origin dest  tailnum carrier name     lat   lon   alt
##    <int> <int> <int> <dbl> <chr>  <chr> <chr>   <chr>   <chr>  <dbl> <dbl> <dbl>
##  1  2013     1     1     5 EWR    IAH   N14228  UA      Georg~  30.0 -95.3    97
##  2  2013     1     1     5 LGA    IAH   N24211  UA      Georg~  30.0 -95.3    97
##  3  2013     1     1     5 JFK    MIA   N619AA  AA      Miami~  25.8 -80.3     8
##  4  2013     1     1     5 JFK    BQN   N804JB  B6      <NA>    NA    NA      NA
##  5  2013     1     1     6 LGA    ATL   N668DN  DL      Harts~  33.6 -84.4  1026
##  6  2013     1     1     5 EWR    ORD   N39463  UA      Chica~  42.0 -87.9   668
##  7  2013     1     1     6 EWR    FLL   N516JB  B6      Fort ~  26.1 -80.2     9
##  8  2013     1     1     6 LGA    IAD   N829AS  EV      Washi~  38.9 -77.5   313
##  9  2013     1     1     6 JFK    MCO   N593JB  B6      Orlan~  28.4 -81.3    96
## 10  2013     1     1     6 LGA    ORD   N3ALAA  AA      Chica~  42.0 -87.9   668
## # i 336,766 more rows
## # i 3 more variables: tz <dbl>, dst <chr>, tzone <chr>
\end{verbatim}

Task-4:Performing a left join between the \texttt{flights2} tibble and
the \texttt{airports} tibble, matching the ``origin'' column from
\texttt{flights2} with the ``faa'' column from \texttt{airports}.

\begin{Shaded}
\begin{Highlighting}[]
\NormalTok{flights2 }\SpecialCharTok{\%\textgreater{}\%} 
  \FunctionTok{left\_join}\NormalTok{(airports, }\FunctionTok{c}\NormalTok{(}\StringTok{"origin"} \OtherTok{=} \StringTok{"faa"}\NormalTok{))}
\end{Highlighting}
\end{Shaded}

\begin{verbatim}
## # A tibble: 336,776 x 15
##     year month   day  hour origin dest  tailnum carrier name     lat   lon   alt
##    <int> <int> <int> <dbl> <chr>  <chr> <chr>   <chr>   <chr>  <dbl> <dbl> <dbl>
##  1  2013     1     1     5 EWR    IAH   N14228  UA      Newar~  40.7 -74.2    18
##  2  2013     1     1     5 LGA    IAH   N24211  UA      La Gu~  40.8 -73.9    22
##  3  2013     1     1     5 JFK    MIA   N619AA  AA      John ~  40.6 -73.8    13
##  4  2013     1     1     5 JFK    BQN   N804JB  B6      John ~  40.6 -73.8    13
##  5  2013     1     1     6 LGA    ATL   N668DN  DL      La Gu~  40.8 -73.9    22
##  6  2013     1     1     5 EWR    ORD   N39463  UA      Newar~  40.7 -74.2    18
##  7  2013     1     1     6 EWR    FLL   N516JB  B6      Newar~  40.7 -74.2    18
##  8  2013     1     1     6 LGA    IAD   N829AS  EV      La Gu~  40.8 -73.9    22
##  9  2013     1     1     6 JFK    MCO   N593JB  B6      John ~  40.6 -73.8    13
## 10  2013     1     1     6 LGA    ORD   N3ALAA  AA      La Gu~  40.8 -73.9    22
## # i 336,766 more rows
## # i 3 more variables: tz <dbl>, dst <chr>, tzone <chr>
\end{verbatim}

\hypertarget{filtering-joins}{%
\section{Filtering joins}\label{filtering-joins}}

Task-1: Calculating the top 10 destinations by counting the occurrences
in the ``dest'' column of the \texttt{flights} tibble, sorted in
descending order, and then displaying the result.

\begin{Shaded}
\begin{Highlighting}[]
\NormalTok{top\_dest }\OtherTok{\textless{}{-}}\NormalTok{ flights }\SpecialCharTok{\%\textgreater{}\%}
  \FunctionTok{count}\NormalTok{(dest, }\AttributeTok{sort =} \ConstantTok{TRUE}\NormalTok{) }\SpecialCharTok{\%\textgreater{}\%}
  \FunctionTok{head}\NormalTok{(}\DecValTok{10}\NormalTok{)}
\NormalTok{top\_dest}
\end{Highlighting}
\end{Shaded}

\begin{verbatim}
## # A tibble: 10 x 2
##    dest      n
##    <chr> <int>
##  1 ORD   17283
##  2 ATL   17215
##  3 LAX   16174
##  4 BOS   15508
##  5 MCO   14082
##  6 CLT   14064
##  7 SFO   13331
##  8 FLL   12055
##  9 MIA   11728
## 10 DCA    9705
\end{verbatim}

Task-2: Filtering the \texttt{flights} tibble to include only rows where
the destination (\texttt{dest}) matches any of the top 10 destinations
identified in the previous step.

\begin{Shaded}
\begin{Highlighting}[]
\NormalTok{flights }\SpecialCharTok{\%\textgreater{}\%} 
  \FunctionTok{filter}\NormalTok{(dest }\SpecialCharTok{\%in\%}\NormalTok{ top\_dest}\SpecialCharTok{$}\NormalTok{dest)}
\end{Highlighting}
\end{Shaded}

\begin{verbatim}
## # A tibble: 141,145 x 19
##     year month   day dep_time sched_dep_time dep_delay arr_time sched_arr_time
##    <int> <int> <int>    <int>          <int>     <dbl>    <int>          <int>
##  1  2013     1     1      542            540         2      923            850
##  2  2013     1     1      554            600        -6      812            837
##  3  2013     1     1      554            558        -4      740            728
##  4  2013     1     1      555            600        -5      913            854
##  5  2013     1     1      557            600        -3      838            846
##  6  2013     1     1      558            600        -2      753            745
##  7  2013     1     1      558            600        -2      924            917
##  8  2013     1     1      558            600        -2      923            937
##  9  2013     1     1      559            559         0      702            706
## 10  2013     1     1      600            600         0      851            858
## # i 141,135 more rows
## # i 11 more variables: arr_delay <dbl>, carrier <chr>, flight <int>,
## #   tailnum <chr>, origin <chr>, dest <chr>, air_time <dbl>, distance <dbl>,
## #   hour <dbl>, minute <dbl>, time_hour <dttm>
\end{verbatim}

\begin{Shaded}
\begin{Highlighting}[]
\CommentTok{\#\%in\% operator in R is used to check if elements in one vector are present in another vector}
\end{Highlighting}
\end{Shaded}

Task-3: Selecting rows from the \texttt{flights} dataset where the
destination airport matches one of the top 10 destinations previously
identified.

\begin{Shaded}
\begin{Highlighting}[]
\NormalTok{flights }\SpecialCharTok{\%\textgreater{}\%} 
  \FunctionTok{semi\_join}\NormalTok{(top\_dest)}
\end{Highlighting}
\end{Shaded}

\begin{verbatim}
## Joining with `by = join_by(dest)`
\end{verbatim}

\begin{verbatim}
## # A tibble: 141,145 x 19
##     year month   day dep_time sched_dep_time dep_delay arr_time sched_arr_time
##    <int> <int> <int>    <int>          <int>     <dbl>    <int>          <int>
##  1  2013     1     1      542            540         2      923            850
##  2  2013     1     1      554            600        -6      812            837
##  3  2013     1     1      554            558        -4      740            728
##  4  2013     1     1      555            600        -5      913            854
##  5  2013     1     1      557            600        -3      838            846
##  6  2013     1     1      558            600        -2      753            745
##  7  2013     1     1      558            600        -2      924            917
##  8  2013     1     1      558            600        -2      923            937
##  9  2013     1     1      559            559         0      702            706
## 10  2013     1     1      600            600         0      851            858
## # i 141,135 more rows
## # i 11 more variables: arr_delay <dbl>, carrier <chr>, flight <int>,
## #   tailnum <chr>, origin <chr>, dest <chr>, air_time <dbl>, distance <dbl>,
## #   hour <dbl>, minute <dbl>, time_hour <dttm>
\end{verbatim}

Task-4: Filtering out flights with tail numbers present in the planes
dataset and counting the occurrences of each unique tail number, sorting
the result.

\begin{Shaded}
\begin{Highlighting}[]
\NormalTok{flights }\SpecialCharTok{\%\textgreater{}\%}
  \FunctionTok{anti\_join}\NormalTok{(planes, }\AttributeTok{by =} \StringTok{"tailnum"}\NormalTok{) }\SpecialCharTok{\%\textgreater{}\%}
  \FunctionTok{count}\NormalTok{(tailnum, }\AttributeTok{sort =} \ConstantTok{TRUE}\NormalTok{)}
\end{Highlighting}
\end{Shaded}

\begin{verbatim}
## # A tibble: 722 x 2
##    tailnum     n
##    <chr>   <int>
##  1 <NA>     2512
##  2 N725MQ    575
##  3 N722MQ    513
##  4 N723MQ    507
##  5 N713MQ    483
##  6 N735MQ    396
##  7 N0EGMQ    371
##  8 N534MQ    364
##  9 N542MQ    363
## 10 N531MQ    349
## # i 712 more rows
\end{verbatim}

\hypertarget{set-operations}{%
\section{Set operations}\label{set-operations}}

Task-1:creating two tibbles, df1 and df2, each with columns x and y,
containing sample data.

\begin{Shaded}
\begin{Highlighting}[]
\NormalTok{df1 }\OtherTok{\textless{}{-}} \FunctionTok{tribble}\NormalTok{(}
  \SpecialCharTok{\textasciitilde{}}\NormalTok{x, }\SpecialCharTok{\textasciitilde{}}\NormalTok{y,}
   \DecValTok{1}\NormalTok{,  }\DecValTok{1}\NormalTok{,}
   \DecValTok{2}\NormalTok{,  }\DecValTok{1}
\NormalTok{)}
\NormalTok{df2 }\OtherTok{\textless{}{-}} \FunctionTok{tribble}\NormalTok{(}
  \SpecialCharTok{\textasciitilde{}}\NormalTok{x, }\SpecialCharTok{\textasciitilde{}}\NormalTok{y,}
   \DecValTok{1}\NormalTok{,  }\DecValTok{1}\NormalTok{,}
   \DecValTok{1}\NormalTok{,  }\DecValTok{2}
\NormalTok{)}
\end{Highlighting}
\end{Shaded}

Task-2:performing set operations on the tibbles df1 and df2, including
intersection, union, and set differences.

\begin{Shaded}
\begin{Highlighting}[]
\FunctionTok{intersect}\NormalTok{(df1, df2)}
\end{Highlighting}
\end{Shaded}

\begin{verbatim}
## # A tibble: 1 x 2
##       x     y
##   <dbl> <dbl>
## 1     1     1
\end{verbatim}

\begin{Shaded}
\begin{Highlighting}[]
\FunctionTok{union}\NormalTok{(df1, df2)}
\end{Highlighting}
\end{Shaded}

\begin{verbatim}
## # A tibble: 3 x 2
##       x     y
##   <dbl> <dbl>
## 1     1     1
## 2     2     1
## 3     1     2
\end{verbatim}

\begin{Shaded}
\begin{Highlighting}[]
\FunctionTok{setdiff}\NormalTok{(df1, df2)}
\end{Highlighting}
\end{Shaded}

\begin{verbatim}
## # A tibble: 1 x 2
##       x     y
##   <dbl> <dbl>
## 1     2     1
\end{verbatim}

\begin{Shaded}
\begin{Highlighting}[]
\FunctionTok{setdiff}\NormalTok{(df2, df1)}
\end{Highlighting}
\end{Shaded}

\begin{verbatim}
## # A tibble: 1 x 2
##       x     y
##   <dbl> <dbl>
## 1     1     2
\end{verbatim}

\hypertarget{ch-14-strings}{%
\section{CH-14: Strings}\label{ch-14-strings}}

Basic \url{Info:string1} \textless- ``This is a string'' string2
\textless- `If I want to include a ``quote'' inside a string, I use
single quotes'

Task-1:To include a literal single or double quote in a string you can
use ~to ``escape'' it

\begin{Shaded}
\begin{Highlighting}[]
\NormalTok{double\_quote }\OtherTok{\textless{}{-}} \StringTok{"}\SpecialCharTok{\textbackslash{}"}\StringTok{"} \CommentTok{\# or \textquotesingle{}"\textquotesingle{}}
\NormalTok{single\_quote }\OtherTok{\textless{}{-}} \StringTok{\textquotesingle{}}\SpecialCharTok{\textbackslash{}\textquotesingle{}}\StringTok{\textquotesingle{}} \CommentTok{\# or "\textquotesingle{}"}
\end{Highlighting}
\end{Shaded}

Task-2: Understanding the character

\begin{Shaded}
\begin{Highlighting}[]
\NormalTok{x }\OtherTok{\textless{}{-}} \FunctionTok{c}\NormalTok{(}\StringTok{"}\SpecialCharTok{\textbackslash{}"}\StringTok{"}\NormalTok{, }\StringTok{"}\SpecialCharTok{\textbackslash{}\textbackslash{}}\StringTok{"}\NormalTok{) }\CommentTok{\#backslash is escape character}
\NormalTok{x}
\end{Highlighting}
\end{Shaded}

\begin{verbatim}
## [1] "\"" "\\"
\end{verbatim}

\begin{Shaded}
\begin{Highlighting}[]
\FunctionTok{writeLines}\NormalTok{(x)}
\end{Highlighting}
\end{Shaded}

\begin{verbatim}
## "
## \
\end{verbatim}

\hypertarget{string-length}{%
\section{String length}\label{string-length}}

Task-1:

\begin{Shaded}
\begin{Highlighting}[]
\FunctionTok{str\_length}\NormalTok{(}\FunctionTok{c}\NormalTok{(}\StringTok{"a"}\NormalTok{, }\StringTok{"R for data science"}\NormalTok{, }\ConstantTok{NA}\NormalTok{))}
\end{Highlighting}
\end{Shaded}

\begin{verbatim}
## [1]  1 18 NA
\end{verbatim}

\hypertarget{combining-strings}{%
\section{Combining strings}\label{combining-strings}}

Task-1:Combining the strings

\begin{Shaded}
\begin{Highlighting}[]
\FunctionTok{str\_c}\NormalTok{(}\StringTok{"x"}\NormalTok{, }\StringTok{"y"}\NormalTok{)}
\end{Highlighting}
\end{Shaded}

\begin{verbatim}
## [1] "xy"
\end{verbatim}

\begin{Shaded}
\begin{Highlighting}[]
\FunctionTok{str\_c}\NormalTok{(}\StringTok{"x"}\NormalTok{, }\StringTok{"y"}\NormalTok{, }\StringTok{"z"}\NormalTok{)}
\end{Highlighting}
\end{Shaded}

\begin{verbatim}
## [1] "xyz"
\end{verbatim}

Task-2:Using the sep argument to control how they're separated.

\begin{Shaded}
\begin{Highlighting}[]
\FunctionTok{str\_c}\NormalTok{(}\StringTok{"x"}\NormalTok{, }\StringTok{"y"}\NormalTok{, }\AttributeTok{sep =} \StringTok{", "}\NormalTok{)}
\end{Highlighting}
\end{Shaded}

\begin{verbatim}
## [1] "x, y"
\end{verbatim}

Task-3:Performing concatenation with ``\textbar{}'' and ``-'' at both
ends of each element of vector x, and replacing NA values with empty
strings before concatenation.

\begin{Shaded}
\begin{Highlighting}[]
\NormalTok{x }\OtherTok{\textless{}{-}} \FunctionTok{c}\NormalTok{(}\StringTok{"abc"}\NormalTok{, }\ConstantTok{NA}\NormalTok{)}
\FunctionTok{str\_c}\NormalTok{(}\StringTok{"|{-}"}\NormalTok{, x, }\StringTok{"{-}|"}\NormalTok{)}
\end{Highlighting}
\end{Shaded}

\begin{verbatim}
## [1] "|-abc-|" NA
\end{verbatim}

\begin{Shaded}
\begin{Highlighting}[]
\FunctionTok{str\_c}\NormalTok{(}\StringTok{"|{-}"}\NormalTok{, }\FunctionTok{str\_replace\_na}\NormalTok{(x), }\StringTok{"{-}|"}\NormalTok{)}
\end{Highlighting}
\end{Shaded}

\begin{verbatim}
## [1] "|-abc-|" "|-NA-|"
\end{verbatim}

Task-4: concatenating each element of the vector c(``a'', ``b'', ``c'')
with a prefix ``prefix-'' and a suffix ``-suffix''.

\begin{Shaded}
\begin{Highlighting}[]
\FunctionTok{str\_c}\NormalTok{(}\StringTok{"prefix{-}"}\NormalTok{, }\FunctionTok{c}\NormalTok{(}\StringTok{"a"}\NormalTok{, }\StringTok{"b"}\NormalTok{, }\StringTok{"c"}\NormalTok{), }\StringTok{"{-}suffix"}\NormalTok{)}
\end{Highlighting}
\end{Shaded}

\begin{verbatim}
## [1] "prefix-a-suffix" "prefix-b-suffix" "prefix-c-suffix"
\end{verbatim}

Task-5: combining strings

\begin{Shaded}
\begin{Highlighting}[]
\NormalTok{name }\OtherTok{\textless{}{-}} \StringTok{"Hadley"}
\NormalTok{time\_of\_day }\OtherTok{\textless{}{-}} \StringTok{"morning"}
\NormalTok{birthday }\OtherTok{\textless{}{-}} \ConstantTok{FALSE}

\FunctionTok{str\_c}\NormalTok{(}
  \StringTok{"Good "}\NormalTok{, time\_of\_day, }\StringTok{" "}\NormalTok{, name,}
  \ControlFlowTok{if}\NormalTok{ (birthday) }\StringTok{" and HAPPY BIRTHDAY"}\NormalTok{,}
  \StringTok{"."}
\NormalTok{)}
\end{Highlighting}
\end{Shaded}

\begin{verbatim}
## [1] "Good morning Hadley."
\end{verbatim}

\hypertarget{subsetting-strings}{%
\section{Subsetting strings}\label{subsetting-strings}}

Task-1:Extracting the first three characters from each element in the
vector \texttt{x} using \texttt{str\_sub}.

\begin{Shaded}
\begin{Highlighting}[]
\NormalTok{x }\OtherTok{\textless{}{-}} \FunctionTok{c}\NormalTok{(}\StringTok{"Apple"}\NormalTok{, }\StringTok{"Banana"}\NormalTok{, }\StringTok{"Pear"}\NormalTok{)}
\FunctionTok{str\_sub}\NormalTok{(x, }\DecValTok{1}\NormalTok{, }\DecValTok{3}\NormalTok{)}
\end{Highlighting}
\end{Shaded}

\begin{verbatim}
## [1] "App" "Ban" "Pea"
\end{verbatim}

Task-2:negative numbers count backwards from end

\begin{Shaded}
\begin{Highlighting}[]
\FunctionTok{str\_sub}\NormalTok{(x, }\SpecialCharTok{{-}}\DecValTok{3}\NormalTok{, }\SpecialCharTok{{-}}\DecValTok{1}\NormalTok{)}
\end{Highlighting}
\end{Shaded}

\begin{verbatim}
## [1] "ple" "ana" "ear"
\end{verbatim}

Task-3:using the assignment form of str\_sub() to modify strings

\begin{Shaded}
\begin{Highlighting}[]
\FunctionTok{str\_sub}\NormalTok{(x, }\DecValTok{1}\NormalTok{, }\DecValTok{1}\NormalTok{) }\OtherTok{\textless{}{-}} \FunctionTok{str\_to\_lower}\NormalTok{(}\FunctionTok{str\_sub}\NormalTok{(x, }\DecValTok{1}\NormalTok{, }\DecValTok{1}\NormalTok{))}
\NormalTok{x}
\end{Highlighting}
\end{Shaded}

\begin{verbatim}
## [1] "apple"  "banana" "pear"
\end{verbatim}

\hypertarget{locales}{%
\section{Locales}\label{locales}}

Task-1:Changing the case

\begin{Shaded}
\begin{Highlighting}[]
\FunctionTok{str\_to\_upper}\NormalTok{(}\FunctionTok{c}\NormalTok{(}\StringTok{"i"}\NormalTok{, }\StringTok{"ı"}\NormalTok{))}
\end{Highlighting}
\end{Shaded}

\begin{verbatim}
## [1] "I" "I"
\end{verbatim}

\begin{Shaded}
\begin{Highlighting}[]
\FunctionTok{str\_to\_upper}\NormalTok{(}\FunctionTok{c}\NormalTok{(}\StringTok{"i"}\NormalTok{, }\StringTok{"ı"}\NormalTok{), }\AttributeTok{locale =} \StringTok{"tr"}\NormalTok{)}
\end{Highlighting}
\end{Shaded}

\begin{verbatim}
## [1] "İ" "I"
\end{verbatim}

Task-2:Sorting the character vector x alphabetically using the English
(en) locale and the Hawaiian (haw) locale.

\begin{Shaded}
\begin{Highlighting}[]
\NormalTok{x }\OtherTok{\textless{}{-}} \FunctionTok{c}\NormalTok{(}\StringTok{"apple"}\NormalTok{, }\StringTok{"eggplant"}\NormalTok{, }\StringTok{"banana"}\NormalTok{)}
\FunctionTok{str\_sort}\NormalTok{(x, }\AttributeTok{locale =} \StringTok{"en"}\NormalTok{) }
\end{Highlighting}
\end{Shaded}

\begin{verbatim}
## [1] "apple"    "banana"   "eggplant"
\end{verbatim}

\begin{Shaded}
\begin{Highlighting}[]
\FunctionTok{str\_sort}\NormalTok{(x, }\AttributeTok{locale =} \StringTok{"haw"}\NormalTok{) }
\end{Highlighting}
\end{Shaded}

\begin{verbatim}
## [1] "apple"    "eggplant" "banana"
\end{verbatim}

\hypertarget{matching-patterns-with-regular-expressions}{%
\section{Matching patterns with regular
expressions}\label{matching-patterns-with-regular-expressions}}

\hypertarget{basic-matches}{%
\subsection{Basic matches}\label{basic-matches}}

Task-1:Searching for the pattern ``an'' within each element of
\texttt{x} and displaying the matches.

\begin{Shaded}
\begin{Highlighting}[]
\NormalTok{x }\OtherTok{\textless{}{-}} \FunctionTok{c}\NormalTok{(}\StringTok{"apple"}\NormalTok{, }\StringTok{"banana"}\NormalTok{, }\StringTok{"pear"}\NormalTok{)}
\FunctionTok{str\_view}\NormalTok{(x, }\StringTok{"an"}\NormalTok{)}
\end{Highlighting}
\end{Shaded}

\begin{verbatim}
## [2] | b<an><an>a
\end{verbatim}

Task-2:Displaying elements in \texttt{x} where any character is followed
by ``a'' and then any character.

\begin{Shaded}
\begin{Highlighting}[]
\FunctionTok{str\_view}\NormalTok{(x, }\StringTok{".a."}\NormalTok{)}
\end{Highlighting}
\end{Shaded}

\begin{verbatim}
## [2] | <ban>ana
## [3] | p<ear>
\end{verbatim}

Task-3

\begin{Shaded}
\begin{Highlighting}[]
\CommentTok{\# To create the regular expression, we need \textbackslash{}\textbackslash{}}
\NormalTok{dot }\OtherTok{\textless{}{-}} \StringTok{"}\SpecialCharTok{\textbackslash{}\textbackslash{}}\StringTok{."}

\CommentTok{\# But the expression itself only contains one:}
\FunctionTok{writeLines}\NormalTok{(dot)}
\end{Highlighting}
\end{Shaded}

\begin{verbatim}
## \.
\end{verbatim}

\begin{Shaded}
\begin{Highlighting}[]
\CommentTok{\# And this tells R to look for an explicit .}
\FunctionTok{str\_view}\NormalTok{(}\FunctionTok{c}\NormalTok{(}\StringTok{"abc"}\NormalTok{, }\StringTok{"a.c"}\NormalTok{, }\StringTok{"bef"}\NormalTok{), }\StringTok{"a}\SpecialCharTok{\textbackslash{}\textbackslash{}}\StringTok{.c"}\NormalTok{)}
\end{Highlighting}
\end{Shaded}

\begin{verbatim}
## [2] | <a.c>
\end{verbatim}

Task-4: Displaying elements in \texttt{x} where the sequence
``\textbackslash{}'' occurs.

\begin{Shaded}
\begin{Highlighting}[]
\NormalTok{x }\OtherTok{\textless{}{-}} \StringTok{"a}\SpecialCharTok{\textbackslash{}\textbackslash{}}\StringTok{b"}
\FunctionTok{writeLines}\NormalTok{(x)}
\end{Highlighting}
\end{Shaded}

\begin{verbatim}
## a\b
\end{verbatim}

\begin{Shaded}
\begin{Highlighting}[]
\FunctionTok{str\_view}\NormalTok{(x, }\StringTok{"}\SpecialCharTok{\textbackslash{}\textbackslash{}\textbackslash{}\textbackslash{}}\StringTok{"}\NormalTok{)}
\end{Highlighting}
\end{Shaded}

\begin{verbatim}
## [1] | a<\>b
\end{verbatim}

\hypertarget{anchors}{%
\subsection{Anchors}\label{anchors}}

Task-1: Displaying elements in \texttt{x} that start with ``a'' and end
with ``a'' respectively.

\begin{Shaded}
\begin{Highlighting}[]
\NormalTok{x }\OtherTok{\textless{}{-}} \FunctionTok{c}\NormalTok{(}\StringTok{"apple"}\NormalTok{, }\StringTok{"banana"}\NormalTok{, }\StringTok{"pear"}\NormalTok{)}
\FunctionTok{str\_view}\NormalTok{(x, }\StringTok{"\^{}a"}\NormalTok{)}
\end{Highlighting}
\end{Shaded}

\begin{verbatim}
## [1] | <a>pple
\end{verbatim}

\begin{Shaded}
\begin{Highlighting}[]
\FunctionTok{str\_view}\NormalTok{(x, }\StringTok{"a$"}\NormalTok{)}
\end{Highlighting}
\end{Shaded}

\begin{verbatim}
## [2] | banan<a>
\end{verbatim}

Task-2: Highlighting ``apple'' occurrences in \texttt{x} and instances
where it's the only content.

\begin{Shaded}
\begin{Highlighting}[]
\NormalTok{x }\OtherTok{\textless{}{-}} \FunctionTok{c}\NormalTok{(}\StringTok{"apple pie"}\NormalTok{, }\StringTok{"apple"}\NormalTok{, }\StringTok{"apple cake"}\NormalTok{)}
\FunctionTok{str\_view}\NormalTok{(x, }\StringTok{"apple"}\NormalTok{)}
\end{Highlighting}
\end{Shaded}

\begin{verbatim}
## [1] | <apple> pie
## [2] | <apple>
## [3] | <apple> cake
\end{verbatim}

\begin{Shaded}
\begin{Highlighting}[]
\FunctionTok{str\_view}\NormalTok{(x, }\StringTok{"\^{}apple$"}\NormalTok{)}
\end{Highlighting}
\end{Shaded}

\begin{verbatim}
## [2] | <apple>
\end{verbatim}

\hypertarget{character-classes-and-alternatives}{%
\subsection{Character classes and
alternatives}\label{character-classes-and-alternatives}}

Task-1: Visualizing patterns matching ``a.c'', ``a*c'', and ``a c'' in
the provided character vector.

\begin{Shaded}
\begin{Highlighting}[]
\FunctionTok{str\_view}\NormalTok{(}\FunctionTok{c}\NormalTok{(}\StringTok{"abc"}\NormalTok{, }\StringTok{"a.c"}\NormalTok{, }\StringTok{"a*c"}\NormalTok{, }\StringTok{"a c"}\NormalTok{), }\StringTok{"a[.]c"}\NormalTok{)}
\end{Highlighting}
\end{Shaded}

\begin{verbatim}
## [2] | <a.c>
\end{verbatim}

\begin{Shaded}
\begin{Highlighting}[]
\FunctionTok{str\_view}\NormalTok{(}\FunctionTok{c}\NormalTok{(}\StringTok{"abc"}\NormalTok{, }\StringTok{"a.c"}\NormalTok{, }\StringTok{"a*c"}\NormalTok{, }\StringTok{"a c"}\NormalTok{), }\StringTok{".[*]c"}\NormalTok{)}
\end{Highlighting}
\end{Shaded}

\begin{verbatim}
## [3] | <a*c>
\end{verbatim}

\begin{Shaded}
\begin{Highlighting}[]
\FunctionTok{str\_view}\NormalTok{(}\FunctionTok{c}\NormalTok{(}\StringTok{"abc"}\NormalTok{, }\StringTok{"a.c"}\NormalTok{, }\StringTok{"a*c"}\NormalTok{, }\StringTok{"a c"}\NormalTok{), }\StringTok{"a[ ]"}\NormalTok{)}
\end{Highlighting}
\end{Shaded}

\begin{verbatim}
## [4] | <a >c
\end{verbatim}

Task-2: Visualizing patterns matching ``grey'' or ``gray'' in the
provided character vector.

\begin{Shaded}
\begin{Highlighting}[]
\FunctionTok{str\_view}\NormalTok{(}\FunctionTok{c}\NormalTok{(}\StringTok{"grey"}\NormalTok{, }\StringTok{"gray"}\NormalTok{), }\StringTok{"gr(e|a)y"}\NormalTok{)}
\end{Highlighting}
\end{Shaded}

\begin{verbatim}
## [1] | <grey>
## [2] | <gray>
\end{verbatim}

\hypertarget{repetition}{%
\subsection{Repetition}\label{repetition}}

Task-1:Identifying patterns ``CC'' or ``C'' in the string ``1888 is the
longest year in Roman numerals

\begin{Shaded}
\begin{Highlighting}[]
\NormalTok{x }\OtherTok{\textless{}{-}} \StringTok{"1888 is the longest year in Roman numerals: MDCCCLXXXVIII"}
\FunctionTok{str\_view}\NormalTok{(x, }\StringTok{"CC?"}\NormalTok{)}
\end{Highlighting}
\end{Shaded}

\begin{verbatim}
## [1] | 1888 is the longest year in Roman numerals: MD<CC><C>LXXXVIII
\end{verbatim}

Task-2: Viewing the pattern ``CC''

\begin{Shaded}
\begin{Highlighting}[]
\FunctionTok{str\_view}\NormalTok{(x, }\StringTok{"CC+"}\NormalTok{)}
\end{Highlighting}
\end{Shaded}

\begin{verbatim}
## [1] | 1888 is the longest year in Roman numerals: MD<CCC>LXXXVIII
\end{verbatim}

Task-3: Viewing the pattern ``C{[}LX{]}+''

\begin{Shaded}
\begin{Highlighting}[]
\FunctionTok{str\_view}\NormalTok{(x, }\StringTok{\textquotesingle{}C[LX]+\textquotesingle{}}\NormalTok{)}
\end{Highlighting}
\end{Shaded}

\begin{verbatim}
## [1] | 1888 is the longest year in Roman numerals: MDCC<CLXXX>VIII
\end{verbatim}

Task-4:Viewing the pattern ``C\{2\},C\{2,\},c\{2,3\}''

\begin{Shaded}
\begin{Highlighting}[]
\FunctionTok{str\_view}\NormalTok{(x, }\StringTok{"C\{2\}"}\NormalTok{)}
\end{Highlighting}
\end{Shaded}

\begin{verbatim}
## [1] | 1888 is the longest year in Roman numerals: MD<CC>CLXXXVIII
\end{verbatim}

\begin{Shaded}
\begin{Highlighting}[]
\FunctionTok{str\_view}\NormalTok{(x, }\StringTok{"C\{2,\}"}\NormalTok{)}
\end{Highlighting}
\end{Shaded}

\begin{verbatim}
## [1] | 1888 is the longest year in Roman numerals: MD<CCC>LXXXVIII
\end{verbatim}

\begin{Shaded}
\begin{Highlighting}[]
\FunctionTok{str\_view}\NormalTok{(x, }\StringTok{"C\{2,3\}"}\NormalTok{)}
\end{Highlighting}
\end{Shaded}

\begin{verbatim}
## [1] | 1888 is the longest year in Roman numerals: MD<CCC>LXXXVIII
\end{verbatim}

\hypertarget{grouping-and-backreferences}{%
\subsection{Grouping and
backreferences}\label{grouping-and-backreferences}}

Task-1:Grouping

\begin{Shaded}
\begin{Highlighting}[]
\FunctionTok{str\_view}\NormalTok{(fruit, }\StringTok{"(..)}\SpecialCharTok{\textbackslash{}\textbackslash{}}\StringTok{1"}\NormalTok{, }\AttributeTok{match =} \ConstantTok{TRUE}\NormalTok{)}
\end{Highlighting}
\end{Shaded}

\begin{verbatim}
##  [4] | b<anan>a
## [20] | <coco>nut
## [22] | <cucu>mber
## [41] | <juju>be
## [56] | <papa>ya
## [73] | s<alal> berry
\end{verbatim}

\hypertarget{detect-matches}{%
\subsection{Detect matches}\label{detect-matches}}

Task-1: Checking for the presence of the letter ``e'' in each word

\begin{Shaded}
\begin{Highlighting}[]
\NormalTok{x }\OtherTok{\textless{}{-}} \FunctionTok{c}\NormalTok{(}\StringTok{"apple"}\NormalTok{, }\StringTok{"banana"}\NormalTok{, }\StringTok{"pear"}\NormalTok{)}
\FunctionTok{str\_detect}\NormalTok{(x, }\StringTok{"e"}\NormalTok{)}
\end{Highlighting}
\end{Shaded}

\begin{verbatim}
## [1]  TRUE FALSE  TRUE
\end{verbatim}

Task-2:Checking how many common words start with t

\begin{Shaded}
\begin{Highlighting}[]
\FunctionTok{sum}\NormalTok{(}\FunctionTok{str\_detect}\NormalTok{(words, }\StringTok{"\^{}t"}\NormalTok{))}
\end{Highlighting}
\end{Shaded}

\begin{verbatim}
## [1] 65
\end{verbatim}

Task-3: Checking proportion of common words end with a vowel

\begin{Shaded}
\begin{Highlighting}[]
\FunctionTok{mean}\NormalTok{(}\FunctionTok{str\_detect}\NormalTok{(words, }\StringTok{"[aeiou]$"}\NormalTok{))}
\end{Highlighting}
\end{Shaded}

\begin{verbatim}
## [1] 0.2765306
\end{verbatim}

Task-4:Finding all words containing at least one vowel, and negate

\begin{Shaded}
\begin{Highlighting}[]
\NormalTok{no\_vowels\_1 }\OtherTok{\textless{}{-}} \SpecialCharTok{!}\FunctionTok{str\_detect}\NormalTok{(words, }\StringTok{"[aeiou]"}\NormalTok{)}
\end{Highlighting}
\end{Shaded}

Task-5:Finding all words consisting only of consonants (non-vowels)

\begin{Shaded}
\begin{Highlighting}[]
\NormalTok{no\_vowels\_2 }\OtherTok{\textless{}{-}} \FunctionTok{str\_detect}\NormalTok{(words, }\StringTok{"\^{}[\^{}aeiou]+$"}\NormalTok{)}
\FunctionTok{identical}\NormalTok{(no\_vowels\_1, no\_vowels\_2)}
\end{Highlighting}
\end{Shaded}

\begin{verbatim}
## [1] TRUE
\end{verbatim}

Task-6: Filtering words that end with the letter ``x'' from a list of
words.

\begin{Shaded}
\begin{Highlighting}[]
\NormalTok{words[}\FunctionTok{str\_detect}\NormalTok{(words, }\StringTok{"x$"}\NormalTok{)]}
\end{Highlighting}
\end{Shaded}

\begin{verbatim}
## [1] "box" "sex" "six" "tax"
\end{verbatim}

\begin{Shaded}
\begin{Highlighting}[]
\FunctionTok{str\_subset}\NormalTok{(words, }\StringTok{"x$"}\NormalTok{)}
\end{Highlighting}
\end{Shaded}

\begin{verbatim}
## [1] "box" "sex" "six" "tax"
\end{verbatim}

Task-7: Filtering a tibble for words that end with ``x''.

\begin{Shaded}
\begin{Highlighting}[]
\NormalTok{df }\OtherTok{\textless{}{-}} \FunctionTok{tibble}\NormalTok{(}
  \AttributeTok{word =}\NormalTok{ words, }
  \AttributeTok{i =} \FunctionTok{seq\_along}\NormalTok{(word)}
\NormalTok{)}
\NormalTok{df }\SpecialCharTok{\%\textgreater{}\%} 
  \FunctionTok{filter}\NormalTok{(}\FunctionTok{str\_detect}\NormalTok{(word, }\StringTok{"x$"}\NormalTok{))}
\end{Highlighting}
\end{Shaded}

\begin{verbatim}
## # A tibble: 4 x 2
##   word      i
##   <chr> <int>
## 1 box     108
## 2 sex     747
## 3 six     772
## 4 tax     841
\end{verbatim}

Task-8:Counting the occurrences of ``a'' in each element of a character
vector.

\begin{Shaded}
\begin{Highlighting}[]
\NormalTok{x }\OtherTok{\textless{}{-}} \FunctionTok{c}\NormalTok{(}\StringTok{"apple"}\NormalTok{, }\StringTok{"banana"}\NormalTok{, }\StringTok{"pear"}\NormalTok{)}
\FunctionTok{str\_count}\NormalTok{(x, }\StringTok{"a"}\NormalTok{)}
\end{Highlighting}
\end{Shaded}

\begin{verbatim}
## [1] 1 3 1
\end{verbatim}

Task-9: Seeing average of how many vowels per word

\begin{Shaded}
\begin{Highlighting}[]
\FunctionTok{mean}\NormalTok{(}\FunctionTok{str\_count}\NormalTok{(words, }\StringTok{"[aeiou]"}\NormalTok{))}
\end{Highlighting}
\end{Shaded}

\begin{verbatim}
## [1] 1.991837
\end{verbatim}

Task-10: Adding columns to a tibble to count vowels and consonants in
each word.

\begin{Shaded}
\begin{Highlighting}[]
\NormalTok{df }\SpecialCharTok{\%\textgreater{}\%} 
  \FunctionTok{mutate}\NormalTok{(}
    \AttributeTok{vowels =} \FunctionTok{str\_count}\NormalTok{(word, }\StringTok{"[aeiou]"}\NormalTok{),}
    \AttributeTok{consonants =} \FunctionTok{str\_count}\NormalTok{(word, }\StringTok{"[\^{}aeiou]"}\NormalTok{)}
\NormalTok{  )}
\end{Highlighting}
\end{Shaded}

\begin{verbatim}
## # A tibble: 980 x 4
##    word         i vowels consonants
##    <chr>    <int>  <int>      <int>
##  1 a            1      1          0
##  2 able         2      2          2
##  3 about        3      3          2
##  4 absolute     4      4          4
##  5 accept       5      2          4
##  6 account      6      3          4
##  7 achieve      7      4          3
##  8 across       8      2          4
##  9 act          9      1          2
## 10 active      10      3          3
## # i 970 more rows
\end{verbatim}

Task-11:Counting ``aba'' occurrences in ``abababa'' and showing all
``aba'' instances.

\begin{Shaded}
\begin{Highlighting}[]
\FunctionTok{str\_count}\NormalTok{(}\StringTok{"abababa"}\NormalTok{, }\StringTok{"aba"}\NormalTok{)}
\end{Highlighting}
\end{Shaded}

\begin{verbatim}
## [1] 2
\end{verbatim}

\begin{Shaded}
\begin{Highlighting}[]
\FunctionTok{str\_view\_all}\NormalTok{(}\StringTok{"abababa"}\NormalTok{, }\StringTok{"aba"}\NormalTok{)}
\end{Highlighting}
\end{Shaded}

\begin{verbatim}
## Warning: `str_view_all()` was deprecated in stringr 1.5.0.
## i Please use `str_view()` instead.
## This warning is displayed once every 8 hours.
## Call `lifecycle::last_lifecycle_warnings()` to see where this warning was
## generated.
\end{verbatim}

\begin{verbatim}
## [1] | <aba>b<aba>
\end{verbatim}

\hypertarget{extract-matches}{%
\subsection{Extract matches}\label{extract-matches}}

Task-1: Displaying the length of sentences and showing the first few
sentences.

\begin{Shaded}
\begin{Highlighting}[]
\FunctionTok{length}\NormalTok{(sentences)}
\end{Highlighting}
\end{Shaded}

\begin{verbatim}
## [1] 720
\end{verbatim}

\begin{Shaded}
\begin{Highlighting}[]
\FunctionTok{head}\NormalTok{(sentences)}
\end{Highlighting}
\end{Shaded}

\begin{verbatim}
## [1] "The birch canoe slid on the smooth planks." 
## [2] "Glue the sheet to the dark blue background."
## [3] "It's easy to tell the depth of a well."     
## [4] "These days a chicken leg is a rare dish."   
## [5] "Rice is often served in round bowls."       
## [6] "The juice of lemons makes fine punch."
\end{verbatim}

Task-2: Creating a string pattern to match colors by concatenating them
with a pipe delimiter.

\begin{Shaded}
\begin{Highlighting}[]
\NormalTok{colours }\OtherTok{\textless{}{-}} \FunctionTok{c}\NormalTok{(}\StringTok{"red"}\NormalTok{, }\StringTok{"orange"}\NormalTok{, }\StringTok{"yellow"}\NormalTok{, }\StringTok{"green"}\NormalTok{, }\StringTok{"blue"}\NormalTok{, }\StringTok{"purple"}\NormalTok{)}
\NormalTok{colour\_match }\OtherTok{\textless{}{-}} \FunctionTok{str\_c}\NormalTok{(colours, }\AttributeTok{collapse =} \StringTok{"|"}\NormalTok{)}
\NormalTok{colour\_match}
\end{Highlighting}
\end{Shaded}

\begin{verbatim}
## [1] "red|orange|yellow|green|blue|purple"
\end{verbatim}

Task-3: Filter sentences for colors and extract matches, showing the
first few.

\begin{Shaded}
\begin{Highlighting}[]
\NormalTok{has\_colour }\OtherTok{\textless{}{-}} \FunctionTok{str\_subset}\NormalTok{(sentences, colour\_match)}
\NormalTok{matches }\OtherTok{\textless{}{-}} \FunctionTok{str\_extract}\NormalTok{(has\_colour, colour\_match)}
\FunctionTok{head}\NormalTok{(matches)}
\end{Highlighting}
\end{Shaded}

\begin{verbatim}
## [1] "blue" "blue" "red"  "red"  "red"  "blue"
\end{verbatim}

Task-4:Showing all sentences containing multiple colors and highlight
the matches.

\begin{Shaded}
\begin{Highlighting}[]
\NormalTok{more }\OtherTok{\textless{}{-}}\NormalTok{ sentences[}\FunctionTok{str\_count}\NormalTok{(sentences, colour\_match) }\SpecialCharTok{\textgreater{}} \DecValTok{1}\NormalTok{]}
\FunctionTok{str\_view\_all}\NormalTok{(more, colour\_match)}
\end{Highlighting}
\end{Shaded}

\begin{verbatim}
## [1] | It is hard to erase <blue> or <red> ink.
## [2] | The <green> light in the brown box flicke<red>.
## [3] | The sky in the west is tinged with <orange> <red>.
\end{verbatim}

Task-5:Extracting all color matches from the subset of sentences
containing multiple colors.

\begin{Shaded}
\begin{Highlighting}[]
\FunctionTok{str\_extract}\NormalTok{(more, colour\_match)}
\end{Highlighting}
\end{Shaded}

\begin{verbatim}
## [1] "blue"   "green"  "orange"
\end{verbatim}

Task-6:Extracting all occurrences of colors from the subset of sentences
containing multiple colors.

\begin{Shaded}
\begin{Highlighting}[]
\FunctionTok{str\_extract\_all}\NormalTok{(more, colour\_match)}
\end{Highlighting}
\end{Shaded}

\begin{verbatim}
## [[1]]
## [1] "blue" "red" 
## 
## [[2]]
## [1] "green" "red"  
## 
## [[3]]
## [1] "orange" "red"
\end{verbatim}

Task-7: Extracting colors from sentences with multiple colors and
simplify, also extract lowercase letters from each element in x and
simplify.

\begin{Shaded}
\begin{Highlighting}[]
\FunctionTok{str\_extract\_all}\NormalTok{(more, colour\_match, }\AttributeTok{simplify =} \ConstantTok{TRUE}\NormalTok{)}
\end{Highlighting}
\end{Shaded}

\begin{verbatim}
##      [,1]     [,2] 
## [1,] "blue"   "red"
## [2,] "green"  "red"
## [3,] "orange" "red"
\end{verbatim}

\begin{Shaded}
\begin{Highlighting}[]
\NormalTok{x }\OtherTok{\textless{}{-}} \FunctionTok{c}\NormalTok{(}\StringTok{"a"}\NormalTok{, }\StringTok{"a b"}\NormalTok{, }\StringTok{"a b c"}\NormalTok{)}
\FunctionTok{str\_extract\_all}\NormalTok{(x, }\StringTok{"[a{-}z]"}\NormalTok{, }\AttributeTok{simplify =} \ConstantTok{TRUE}\NormalTok{)}
\end{Highlighting}
\end{Shaded}

\begin{verbatim}
##      [,1] [,2] [,3]
## [1,] "a"  ""   ""  
## [2,] "a"  "b"  ""  
## [3,] "a"  "b"  "c"
\end{verbatim}

\hypertarget{grouped-matches}{%
\subsection{Grouped matches}\label{grouped-matches}}

Task-1: Extracting sentences containing nouns defined by a pattern, then
extracts the nouns from those sentences.

\begin{Shaded}
\begin{Highlighting}[]
\NormalTok{noun }\OtherTok{\textless{}{-}} \StringTok{"(a|the) ([\^{} ]+)"}

\NormalTok{has\_noun }\OtherTok{\textless{}{-}}\NormalTok{ sentences }\SpecialCharTok{\%\textgreater{}\%}
  \FunctionTok{str\_subset}\NormalTok{(noun) }\SpecialCharTok{\%\textgreater{}\%}
  \FunctionTok{head}\NormalTok{(}\DecValTok{10}\NormalTok{)}
\NormalTok{has\_noun }\SpecialCharTok{\%\textgreater{}\%} 
  \FunctionTok{str\_extract}\NormalTok{(noun)}
\end{Highlighting}
\end{Shaded}

\begin{verbatim}
##  [1] "the smooth" "the sheet"  "the depth"  "a chicken"  "the parked"
##  [6] "the sun"    "the huge"   "the ball"   "the woman"  "a helps"
\end{verbatim}

Task-2:

\begin{Shaded}
\begin{Highlighting}[]
\NormalTok{has\_noun }\SpecialCharTok{\%\textgreater{}\%} 
  \FunctionTok{str\_match}\NormalTok{(noun)}
\end{Highlighting}
\end{Shaded}

\begin{verbatim}
##       [,1]         [,2]  [,3]     
##  [1,] "the smooth" "the" "smooth" 
##  [2,] "the sheet"  "the" "sheet"  
##  [3,] "the depth"  "the" "depth"  
##  [4,] "a chicken"  "a"   "chicken"
##  [5,] "the parked" "the" "parked" 
##  [6,] "the sun"    "the" "sun"    
##  [7,] "the huge"   "the" "huge"   
##  [8,] "the ball"   "the" "ball"   
##  [9,] "the woman"  "the" "woman"  
## [10,] "a helps"    "a"   "helps"
\end{verbatim}

Task-3:Creating a tibble with columns `article' and `noun' extracted
from sentences based on a pattern.

\begin{Shaded}
\begin{Highlighting}[]
\FunctionTok{tibble}\NormalTok{(}\AttributeTok{sentence =}\NormalTok{ sentences) }\SpecialCharTok{\%\textgreater{}\%} 
\NormalTok{  tidyr}\SpecialCharTok{::}\FunctionTok{extract}\NormalTok{(}
\NormalTok{    sentence, }\FunctionTok{c}\NormalTok{(}\StringTok{"article"}\NormalTok{, }\StringTok{"noun"}\NormalTok{), }\StringTok{"(a|the) ([\^{} ]+)"}\NormalTok{, }
    \AttributeTok{remove =} \ConstantTok{FALSE}
\NormalTok{  )}
\end{Highlighting}
\end{Shaded}

\begin{verbatim}
## # A tibble: 720 x 3
##    sentence                                    article noun   
##    <chr>                                       <chr>   <chr>  
##  1 The birch canoe slid on the smooth planks.  the     smooth 
##  2 Glue the sheet to the dark blue background. the     sheet  
##  3 It's easy to tell the depth of a well.      the     depth  
##  4 These days a chicken leg is a rare dish.    a       chicken
##  5 Rice is often served in round bowls.        <NA>    <NA>   
##  6 The juice of lemons makes fine punch.       <NA>    <NA>   
##  7 The box was thrown beside the parked truck. the     parked 
##  8 The hogs were fed chopped corn and garbage. <NA>    <NA>   
##  9 Four hours of steady work faced us.         <NA>    <NA>   
## 10 A large size in stockings is hard to sell.  <NA>    <NA>   
## # i 710 more rows
\end{verbatim}

\hypertarget{replacing-matches}{%
\subsection{Replacing matches}\label{replacing-matches}}

Task-1: Replacing the first vowel in each word of x with a hyphen.
Replacing all vowels in each word of x with a hyphen.

\begin{Shaded}
\begin{Highlighting}[]
\NormalTok{x }\OtherTok{\textless{}{-}} \FunctionTok{c}\NormalTok{(}\StringTok{"apple"}\NormalTok{, }\StringTok{"pear"}\NormalTok{, }\StringTok{"banana"}\NormalTok{)}
\FunctionTok{str\_replace}\NormalTok{(x, }\StringTok{"[aeiou]"}\NormalTok{, }\StringTok{"{-}"}\NormalTok{)}
\end{Highlighting}
\end{Shaded}

\begin{verbatim}
## [1] "-pple"  "p-ar"   "b-nana"
\end{verbatim}

\begin{Shaded}
\begin{Highlighting}[]
\FunctionTok{str\_replace\_all}\NormalTok{(x, }\StringTok{"[aeiou]"}\NormalTok{, }\StringTok{"{-}"}\NormalTok{)}
\end{Highlighting}
\end{Shaded}

\begin{verbatim}
## [1] "-ppl-"  "p--r"   "b-n-n-"
\end{verbatim}

Task-2: Replacing numeric values in x with their corresponding word
representations.

\begin{Shaded}
\begin{Highlighting}[]
\NormalTok{x }\OtherTok{\textless{}{-}} \FunctionTok{c}\NormalTok{(}\StringTok{"1 house"}\NormalTok{, }\StringTok{"2 cars"}\NormalTok{, }\StringTok{"3 people"}\NormalTok{)}
\FunctionTok{str\_replace\_all}\NormalTok{(x, }\FunctionTok{c}\NormalTok{(}\StringTok{"1"} \OtherTok{=} \StringTok{"one"}\NormalTok{, }\StringTok{"2"} \OtherTok{=} \StringTok{"two"}\NormalTok{, }\StringTok{"3"} \OtherTok{=} \StringTok{"three"}\NormalTok{))}
\end{Highlighting}
\end{Shaded}

\begin{verbatim}
## [1] "one house"    "two cars"     "three people"
\end{verbatim}

Task-3:Reordering words in sentences by swapping the second and third
word positions.

\begin{Shaded}
\begin{Highlighting}[]
\NormalTok{sentences }\SpecialCharTok{\%\textgreater{}\%} 
  \FunctionTok{str\_replace}\NormalTok{(}\StringTok{"([\^{} ]+) ([\^{} ]+) ([\^{} ]+)"}\NormalTok{, }\StringTok{"}\SpecialCharTok{\textbackslash{}\textbackslash{}}\StringTok{1 }\SpecialCharTok{\textbackslash{}\textbackslash{}}\StringTok{3 }\SpecialCharTok{\textbackslash{}\textbackslash{}}\StringTok{2"}\NormalTok{) }\SpecialCharTok{\%\textgreater{}\%} 
  \FunctionTok{head}\NormalTok{(}\DecValTok{5}\NormalTok{)}
\end{Highlighting}
\end{Shaded}

\begin{verbatim}
## [1] "The canoe birch slid on the smooth planks." 
## [2] "Glue sheet the to the dark blue background."
## [3] "It's to easy tell the depth of a well."     
## [4] "These a days chicken leg is a rare dish."   
## [5] "Rice often is served in round bowls."
\end{verbatim}

\hypertarget{splitting}{%
\section{Splitting}\label{splitting}}

Task-1: Splitting the first five sentences into words.

\begin{Shaded}
\begin{Highlighting}[]
\NormalTok{sentences }\SpecialCharTok{\%\textgreater{}\%}
  \FunctionTok{head}\NormalTok{(}\DecValTok{5}\NormalTok{) }\SpecialCharTok{\%\textgreater{}\%} 
  \FunctionTok{str\_split}\NormalTok{(}\StringTok{" "}\NormalTok{)}
\end{Highlighting}
\end{Shaded}

\begin{verbatim}
## [[1]]
## [1] "The"     "birch"   "canoe"   "slid"    "on"      "the"     "smooth" 
## [8] "planks."
## 
## [[2]]
## [1] "Glue"        "the"         "sheet"       "to"          "the"        
## [6] "dark"        "blue"        "background."
## 
## [[3]]
## [1] "It's"  "easy"  "to"    "tell"  "the"   "depth" "of"    "a"     "well."
## 
## [[4]]
## [1] "These"   "days"    "a"       "chicken" "leg"     "is"      "a"      
## [8] "rare"    "dish."  
## 
## [[5]]
## [1] "Rice"   "is"     "often"  "served" "in"     "round"  "bowls."
\end{verbatim}

Task-2:Splitting the string `a\textbar b\textbar c\textbar d' by
`\textbar{}' into a vector of elements.

\begin{Shaded}
\begin{Highlighting}[]
\StringTok{"a|b|c|d"} \SpecialCharTok{\%\textgreater{}\%} 
  \FunctionTok{str\_split}\NormalTok{(}\StringTok{"}\SpecialCharTok{\textbackslash{}\textbackslash{}}\StringTok{|"}\NormalTok{) }\SpecialCharTok{\%\textgreater{}\%} 
\NormalTok{  .[[}\DecValTok{1}\NormalTok{]]}
\end{Highlighting}
\end{Shaded}

\begin{verbatim}
## [1] "a" "b" "c" "d"
\end{verbatim}

Task-3:Splitting the first 5 sentences by space into a matrix of words.

\begin{Shaded}
\begin{Highlighting}[]
\NormalTok{sentences }\SpecialCharTok{\%\textgreater{}\%}
  \FunctionTok{head}\NormalTok{(}\DecValTok{5}\NormalTok{) }\SpecialCharTok{\%\textgreater{}\%} 
  \FunctionTok{str\_split}\NormalTok{(}\StringTok{" "}\NormalTok{, }\AttributeTok{simplify =} \ConstantTok{TRUE}\NormalTok{)}
\end{Highlighting}
\end{Shaded}

\begin{verbatim}
##      [,1]    [,2]    [,3]    [,4]      [,5]  [,6]    [,7]     [,8]         
## [1,] "The"   "birch" "canoe" "slid"    "on"  "the"   "smooth" "planks."    
## [2,] "Glue"  "the"   "sheet" "to"      "the" "dark"  "blue"   "background."
## [3,] "It's"  "easy"  "to"    "tell"    "the" "depth" "of"     "a"          
## [4,] "These" "days"  "a"     "chicken" "leg" "is"    "a"      "rare"       
## [5,] "Rice"  "is"    "often" "served"  "in"  "round" "bowls." ""           
##      [,9]   
## [1,] ""     
## [2,] ""     
## [3,] "well."
## [4,] "dish."
## [5,] ""
\end{verbatim}

Task-4:Splitting each field string into two parts at the first
occurrence of `:'.

\begin{Shaded}
\begin{Highlighting}[]
\NormalTok{fields }\OtherTok{\textless{}{-}} \FunctionTok{c}\NormalTok{(}\StringTok{"Name: Hadley"}\NormalTok{, }\StringTok{"Country: NZ"}\NormalTok{, }\StringTok{"Age: 35"}\NormalTok{)}
\NormalTok{fields }\SpecialCharTok{\%\textgreater{}\%} \FunctionTok{str\_split}\NormalTok{(}\StringTok{": "}\NormalTok{, }\AttributeTok{n =} \DecValTok{2}\NormalTok{, }\AttributeTok{simplify =} \ConstantTok{TRUE}\NormalTok{)}
\end{Highlighting}
\end{Shaded}

\begin{verbatim}
##      [,1]      [,2]    
## [1,] "Name"    "Hadley"
## [2,] "Country" "NZ"    
## [3,] "Age"     "35"
\end{verbatim}

Task-5: Display word boundaries, split by spaces, and split by word
boundaries, respectively.

\begin{Shaded}
\begin{Highlighting}[]
\NormalTok{x }\OtherTok{\textless{}{-}} \StringTok{"This is a sentence.  This is another sentence."}
\FunctionTok{str\_view\_all}\NormalTok{(x, }\FunctionTok{boundary}\NormalTok{(}\StringTok{"word"}\NormalTok{))}
\end{Highlighting}
\end{Shaded}

\begin{verbatim}
## [1] | <This> <is> <a> <sentence>.  <This> <is> <another> <sentence>.
\end{verbatim}

\begin{Shaded}
\begin{Highlighting}[]
\FunctionTok{str\_split}\NormalTok{(x, }\StringTok{" "}\NormalTok{)[[}\DecValTok{1}\NormalTok{]]}
\end{Highlighting}
\end{Shaded}

\begin{verbatim}
## [1] "This"      "is"        "a"         "sentence." ""          "This"     
## [7] "is"        "another"   "sentence."
\end{verbatim}

\begin{Shaded}
\begin{Highlighting}[]
\FunctionTok{str\_split}\NormalTok{(x, }\FunctionTok{boundary}\NormalTok{(}\StringTok{"word"}\NormalTok{))[[}\DecValTok{1}\NormalTok{]]}
\end{Highlighting}
\end{Shaded}

\begin{verbatim}
## [1] "This"     "is"       "a"        "sentence" "This"     "is"       "another" 
## [8] "sentence"
\end{verbatim}

\hypertarget{other-types-of-pattern}{%
\section{Other types of pattern}\label{other-types-of-pattern}}

Task-1:

\begin{Shaded}
\begin{Highlighting}[]
\CommentTok{\# The regular call:}
\FunctionTok{str\_view}\NormalTok{(fruit, }\StringTok{"nana"}\NormalTok{)}
\end{Highlighting}
\end{Shaded}

\begin{verbatim}
## [4] | ba<nana>
\end{verbatim}

\begin{Shaded}
\begin{Highlighting}[]
\CommentTok{\# Is shorthand for}
\FunctionTok{str\_view}\NormalTok{(fruit, }\FunctionTok{regex}\NormalTok{(}\StringTok{"nana"}\NormalTok{))}
\end{Highlighting}
\end{Shaded}

\begin{verbatim}
## [4] | ba<nana>
\end{verbatim}

Task-2:Visualizing occurrences of ``banana'' in different case
variations.

\begin{Shaded}
\begin{Highlighting}[]
\NormalTok{bananas }\OtherTok{\textless{}{-}} \FunctionTok{c}\NormalTok{(}\StringTok{"banana"}\NormalTok{, }\StringTok{"Banana"}\NormalTok{, }\StringTok{"BANANA"}\NormalTok{)}
\FunctionTok{str\_view}\NormalTok{(bananas, }\StringTok{"banana"}\NormalTok{)}
\end{Highlighting}
\end{Shaded}

\begin{verbatim}
## [1] | <banana>
\end{verbatim}

\begin{Shaded}
\begin{Highlighting}[]
\FunctionTok{str\_view}\NormalTok{(bananas, }\FunctionTok{regex}\NormalTok{(}\StringTok{"banana"}\NormalTok{, }\AttributeTok{ignore\_case =} \ConstantTok{TRUE}\NormalTok{))}
\end{Highlighting}
\end{Shaded}

\begin{verbatim}
## [1] | <banana>
## [2] | <Banana>
## [3] | <BANANA>
\end{verbatim}

Task-3: Extracting all lines starting with ``Line'' from the text.

\begin{Shaded}
\begin{Highlighting}[]
\NormalTok{x }\OtherTok{\textless{}{-}} \StringTok{"Line 1}\SpecialCharTok{\textbackslash{}n}\StringTok{Line 2}\SpecialCharTok{\textbackslash{}n}\StringTok{Line 3"}
\FunctionTok{str\_extract\_all}\NormalTok{(x, }\StringTok{"\^{}Line"}\NormalTok{)[[}\DecValTok{1}\NormalTok{]]}
\end{Highlighting}
\end{Shaded}

\begin{verbatim}
## [1] "Line"
\end{verbatim}

Task-4: Extracting all occurrences of lines starting with ``Line'' from
the text, considering each line separately.

\begin{Shaded}
\begin{Highlighting}[]
\FunctionTok{str\_extract\_all}\NormalTok{(x, }\FunctionTok{regex}\NormalTok{(}\StringTok{"\^{}Line"}\NormalTok{, }\AttributeTok{multiline =} \ConstantTok{TRUE}\NormalTok{))[[}\DecValTok{1}\NormalTok{]]}
\end{Highlighting}
\end{Shaded}

\begin{verbatim}
## [1] "Line" "Line" "Line"
\end{verbatim}

Task-5:Creating a regular expression pattern for phone numbers, allowing
for variations in formatting, and attempting to match it against the
provided phone number.

\begin{Shaded}
\begin{Highlighting}[]
\NormalTok{phone }\OtherTok{\textless{}{-}} \FunctionTok{regex}\NormalTok{(}\StringTok{"}
\StringTok{  }\SpecialCharTok{\textbackslash{}\textbackslash{}}\StringTok{(?     \# optional opening parens}
\StringTok{  (}\SpecialCharTok{\textbackslash{}\textbackslash{}}\StringTok{d\{3\}) \# area code}
\StringTok{  [) {-}]?   \# optional closing parens, space, or dash}
\StringTok{  (}\SpecialCharTok{\textbackslash{}\textbackslash{}}\StringTok{d\{3\}) \# another three numbers}
\StringTok{  [ {-}]?    \# optional space or dash}
\StringTok{  (}\SpecialCharTok{\textbackslash{}\textbackslash{}}\StringTok{d\{3\}) \# three more numbers}
\StringTok{  "}\NormalTok{, }\AttributeTok{comments =} \ConstantTok{TRUE}\NormalTok{)}

\FunctionTok{str\_match}\NormalTok{(}\StringTok{"514{-}791{-}8141"}\NormalTok{, phone)}
\end{Highlighting}
\end{Shaded}

\begin{verbatim}
##      [,1]          [,2]  [,3]  [,4] 
## [1,] "514-791-814" "514" "791" "814"
\end{verbatim}

Task-6:Installling the package and Benchmarking string detection in
``sentences'' using fixed and regex patterns 20 times each, comparing
performance with microbenchmark.

\begin{Shaded}
\begin{Highlighting}[]
\NormalTok{package\_to\_install }\OtherTok{\textless{}{-}} \FunctionTok{c}\NormalTok{(}\StringTok{"microbenchmark"}\NormalTok{)}

\ControlFlowTok{for}\NormalTok{ (package\_name }\ControlFlowTok{in}\NormalTok{ package\_to\_install) \{}
  \ControlFlowTok{if}\NormalTok{ (}\SpecialCharTok{!}\FunctionTok{requireNamespace}\NormalTok{(package\_name, }\AttributeTok{quietly =} \ConstantTok{TRUE}\NormalTok{)) \{}
    \FunctionTok{install.packages}\NormalTok{(package\_name)}
\NormalTok{  \}}
\NormalTok{\}}
\FunctionTok{library}\NormalTok{(microbenchmark)}

\NormalTok{microbenchmark}\SpecialCharTok{::}\FunctionTok{microbenchmark}\NormalTok{(}
  \AttributeTok{fixed =} \FunctionTok{str\_detect}\NormalTok{(sentences, }\FunctionTok{fixed}\NormalTok{(}\StringTok{"the"}\NormalTok{)),}
  \AttributeTok{regex =} \FunctionTok{str\_detect}\NormalTok{(sentences, }\StringTok{"the"}\NormalTok{),}
  \AttributeTok{times =} \DecValTok{20}
\NormalTok{  )}
\end{Highlighting}
\end{Shaded}

\begin{verbatim}
## Unit: microseconds
##   expr     min       lq     mean   median       uq       max neval
##  fixed 206.601 212.1515 1935.566 219.8510  230.351 34420.101    20
##  regex 967.701 972.1010 1024.976 975.2515 1010.851  1338.501    20
\end{verbatim}

Task-7:Starting with a1 being ``\u00e1'' and a2 being ``a\u0301'', both
representing the character ``á'', they are compared for equality.

\begin{Shaded}
\begin{Highlighting}[]
\NormalTok{a1 }\OtherTok{\textless{}{-}} \StringTok{"\textbackslash{}u00e1"}
\NormalTok{a2 }\OtherTok{\textless{}{-}} \StringTok{"a\textbackslash{}u0301"}
\FunctionTok{c}\NormalTok{(a1, a2)}
\end{Highlighting}
\end{Shaded}

\begin{verbatim}
## [1] "á" "á"
\end{verbatim}

\begin{Shaded}
\begin{Highlighting}[]
\NormalTok{a1 }\SpecialCharTok{==}\NormalTok{ a2}
\end{Highlighting}
\end{Shaded}

\begin{verbatim}
## [1] FALSE
\end{verbatim}

Task-8: Checking if \texttt{a1} contains the fixed string \texttt{a2}
returns \texttt{FALSE}, whereas using collation rules returns
\texttt{TRUE}.

\begin{Shaded}
\begin{Highlighting}[]
\FunctionTok{str\_detect}\NormalTok{(a1, }\FunctionTok{fixed}\NormalTok{(a2))}
\end{Highlighting}
\end{Shaded}

\begin{verbatim}
## [1] FALSE
\end{verbatim}

\begin{Shaded}
\begin{Highlighting}[]
\FunctionTok{str\_detect}\NormalTok{(a1, }\FunctionTok{coll}\NormalTok{(a2))}
\end{Highlighting}
\end{Shaded}

\begin{verbatim}
## [1] TRUE
\end{verbatim}

Task-9:Creating a vector \texttt{i} with different forms of the letter
``i'', then using \texttt{str\_subset} to filter them based on
collation.

\begin{Shaded}
\begin{Highlighting}[]
\NormalTok{i }\OtherTok{\textless{}{-}} \FunctionTok{c}\NormalTok{(}\StringTok{"I"}\NormalTok{, }\StringTok{"İ"}\NormalTok{, }\StringTok{"i"}\NormalTok{, }\StringTok{"ı"}\NormalTok{)}
\NormalTok{i}
\end{Highlighting}
\end{Shaded}

\begin{verbatim}
## [1] "I" "İ" "i" "ı"
\end{verbatim}

\begin{Shaded}
\begin{Highlighting}[]
\FunctionTok{str\_subset}\NormalTok{(i, }\FunctionTok{coll}\NormalTok{(}\StringTok{"i"}\NormalTok{, }\AttributeTok{ignore\_case =} \ConstantTok{TRUE}\NormalTok{))}
\end{Highlighting}
\end{Shaded}

\begin{verbatim}
## [1] "I" "i"
\end{verbatim}

\begin{Shaded}
\begin{Highlighting}[]
\FunctionTok{str\_subset}\NormalTok{(i, }\FunctionTok{coll}\NormalTok{(}\StringTok{"i"}\NormalTok{, }\AttributeTok{ignore\_case =} \ConstantTok{TRUE}\NormalTok{, }\AttributeTok{locale =} \StringTok{"tr"}\NormalTok{))}
\end{Highlighting}
\end{Shaded}

\begin{verbatim}
## [1] "İ" "i"
\end{verbatim}

Task-10: Fetching locale information.

\begin{Shaded}
\begin{Highlighting}[]
\NormalTok{stringi}\SpecialCharTok{::}\FunctionTok{stri\_locale\_info}\NormalTok{()}
\end{Highlighting}
\end{Shaded}

\begin{verbatim}
## $Language
## [1] "en"
## 
## $Country
## [1] "US"
## 
## $Variant
## [1] ""
## 
## $Name
## [1] "en_US"
\end{verbatim}

Task-11:Visualizing word boundaries and extracts all words from the
string.

\begin{Shaded}
\begin{Highlighting}[]
\NormalTok{x }\OtherTok{\textless{}{-}} \StringTok{"This is a sentence."}
\FunctionTok{str\_view\_all}\NormalTok{(x, }\FunctionTok{boundary}\NormalTok{(}\StringTok{"word"}\NormalTok{))}
\end{Highlighting}
\end{Shaded}

\begin{verbatim}
## [1] | <This> <is> <a> <sentence>.
\end{verbatim}

\begin{Shaded}
\begin{Highlighting}[]
\FunctionTok{str\_extract\_all}\NormalTok{(x, }\FunctionTok{boundary}\NormalTok{(}\StringTok{"word"}\NormalTok{))}
\end{Highlighting}
\end{Shaded}

\begin{verbatim}
## [[1]]
## [1] "This"     "is"       "a"        "sentence"
\end{verbatim}

\hypertarget{ch-15-factors}{%
\section{CH-15: Factors}\label{ch-15-factors}}

\hypertarget{creatig-factors}{%
\subsection{Creatig factors}\label{creatig-factors}}

Task-1:Adding character vector in variable x1

\begin{Shaded}
\begin{Highlighting}[]
\NormalTok{x1 }\OtherTok{\textless{}{-}} \FunctionTok{c}\NormalTok{(}\StringTok{"Dec"}\NormalTok{, }\StringTok{"Apr"}\NormalTok{, }\StringTok{"Jan"}\NormalTok{, }\StringTok{"Mar"}\NormalTok{)}
\end{Highlighting}
\end{Shaded}

Task-2:Adding character vector in variable x2

\begin{Shaded}
\begin{Highlighting}[]
\NormalTok{x2 }\OtherTok{\textless{}{-}} \FunctionTok{c}\NormalTok{(}\StringTok{"Dec"}\NormalTok{, }\StringTok{"Apr"}\NormalTok{, }\StringTok{"Jam"}\NormalTok{, }\StringTok{"Mar"}\NormalTok{)}
\end{Highlighting}
\end{Shaded}

Task-3:Sorting X1

\begin{Shaded}
\begin{Highlighting}[]
\FunctionTok{sort}\NormalTok{(x1)}
\end{Highlighting}
\end{Shaded}

\begin{verbatim}
## [1] "Apr" "Dec" "Jan" "Mar"
\end{verbatim}

Task-4:Adding Character vector in month\_levels

\begin{Shaded}
\begin{Highlighting}[]
\NormalTok{month\_levels }\OtherTok{\textless{}{-}} \FunctionTok{c}\NormalTok{(}
  \StringTok{"Jan"}\NormalTok{, }\StringTok{"Feb"}\NormalTok{, }\StringTok{"Mar"}\NormalTok{, }\StringTok{"Apr"}\NormalTok{, }\StringTok{"May"}\NormalTok{, }\StringTok{"Jun"}\NormalTok{, }
  \StringTok{"Jul"}\NormalTok{, }\StringTok{"Aug"}\NormalTok{, }\StringTok{"Sep"}\NormalTok{, }\StringTok{"Oct"}\NormalTok{, }\StringTok{"Nov"}\NormalTok{, }\StringTok{"Dec"}
\NormalTok{)}
\end{Highlighting}
\end{Shaded}

Task-5:Assigning the factor levels to the variable x1, using the
predefined month\_levels.

\begin{Shaded}
\begin{Highlighting}[]
\NormalTok{y1 }\OtherTok{\textless{}{-}} \FunctionTok{factor}\NormalTok{(x1, }\AttributeTok{levels =}\NormalTok{ month\_levels)}
\NormalTok{y}
\end{Highlighting}
\end{Shaded}

\begin{verbatim}
## # A tibble: 4 x 2
##     key val_y
##   <dbl> <chr>
## 1     1 y1   
## 2     2 y2   
## 3     2 y3   
## 4     3 y4
\end{verbatim}

Task-6:Sorting the factor levels in y1.

\begin{Shaded}
\begin{Highlighting}[]
\FunctionTok{sort}\NormalTok{(y1)}
\end{Highlighting}
\end{Shaded}

\begin{verbatim}
## [1] Jan Mar Apr Dec
## Levels: Jan Feb Mar Apr May Jun Jul Aug Sep Oct Nov Dec
\end{verbatim}

Task-7:creating a factor y2 from x2 with custom levels specified by
month\_levels.

\begin{Shaded}
\begin{Highlighting}[]
\NormalTok{y2 }\OtherTok{\textless{}{-}} \FunctionTok{factor}\NormalTok{(x2, }\AttributeTok{levels =}\NormalTok{ month\_levels)}
\NormalTok{y2}
\end{Highlighting}
\end{Shaded}

\begin{verbatim}
## [1] Dec  Apr  <NA> Mar 
## Levels: Jan Feb Mar Apr May Jun Jul Aug Sep Oct Nov Dec
\end{verbatim}

Task-8:parsing the values in x2 as factors

\begin{Shaded}
\begin{Highlighting}[]
\NormalTok{y2 }\OtherTok{\textless{}{-}} \FunctionTok{parse\_factor}\NormalTok{(x2, }\AttributeTok{levels =}\NormalTok{ month\_levels)}
\end{Highlighting}
\end{Shaded}

\begin{verbatim}
## Warning: 1 parsing failure.
## row col           expected actual
##   3  -- value in level set    Jam
\end{verbatim}

Task-9: omitting the levels.

\begin{Shaded}
\begin{Highlighting}[]
\FunctionTok{factor}\NormalTok{(x1)}
\end{Highlighting}
\end{Shaded}

\begin{verbatim}
## [1] Dec Apr Jan Mar
## Levels: Apr Dec Jan Mar
\end{verbatim}

Task-10:Creating a factor f1 from the values in x1, using the unique
values of x1 as levels.

\begin{Shaded}
\begin{Highlighting}[]
\NormalTok{f1 }\OtherTok{\textless{}{-}} \FunctionTok{factor}\NormalTok{(x1, }\AttributeTok{levels =} \FunctionTok{unique}\NormalTok{(x1))}
\NormalTok{f1}
\end{Highlighting}
\end{Shaded}

\begin{verbatim}
## [1] Dec Apr Jan Mar
## Levels: Dec Apr Jan Mar
\end{verbatim}

Task-11: creating a factor f2 from the values in x1, ordering them
according to their appearance in x1.

\begin{Shaded}
\begin{Highlighting}[]
\NormalTok{f2 }\OtherTok{\textless{}{-}}\NormalTok{ x1 }\SpecialCharTok{\%\textgreater{}\%} \FunctionTok{factor}\NormalTok{() }\SpecialCharTok{\%\textgreater{}\%} \FunctionTok{fct\_inorder}\NormalTok{()}
\NormalTok{f2}
\end{Highlighting}
\end{Shaded}

\begin{verbatim}
## [1] Dec Apr Jan Mar
## Levels: Dec Apr Jan Mar
\end{verbatim}

Task-12:Omitting levels2

\begin{Shaded}
\begin{Highlighting}[]
\FunctionTok{levels}\NormalTok{(f2)}
\end{Highlighting}
\end{Shaded}

\begin{verbatim}
## [1] "Dec" "Apr" "Jan" "Mar"
\end{verbatim}

\hypertarget{general-social-survey}{%
\section{General Social Survey}\label{general-social-survey}}

Task-1:Loading datasets

\begin{Shaded}
\begin{Highlighting}[]
\NormalTok{gss\_cat}
\end{Highlighting}
\end{Shaded}

\begin{verbatim}
## # A tibble: 21,483 x 9
##     year marital         age race  rincome        partyid    relig denom tvhours
##    <int> <fct>         <int> <fct> <fct>          <fct>      <fct> <fct>   <int>
##  1  2000 Never married    26 White $8000 to 9999  Ind,near ~ Prot~ Sout~      12
##  2  2000 Divorced         48 White $8000 to 9999  Not str r~ Prot~ Bapt~      NA
##  3  2000 Widowed          67 White Not applicable Independe~ Prot~ No d~       2
##  4  2000 Never married    39 White Not applicable Ind,near ~ Orth~ Not ~       4
##  5  2000 Divorced         25 White Not applicable Not str d~ None  Not ~       1
##  6  2000 Married          25 White $20000 - 24999 Strong de~ Prot~ Sout~      NA
##  7  2000 Never married    36 White $25000 or more Not str r~ Chri~ Not ~       3
##  8  2000 Divorced         44 White $7000 to 7999  Ind,near ~ Prot~ Luth~      NA
##  9  2000 Married          44 White $25000 or more Not str d~ Prot~ Other       0
## 10  2000 Married          47 White $25000 or more Strong re~ Prot~ Sout~       3
## # i 21,473 more rows
\end{verbatim}

Task-2:Seeing levels through count()

\begin{Shaded}
\begin{Highlighting}[]
\NormalTok{gss\_cat }\SpecialCharTok{\%\textgreater{}\%}
  \FunctionTok{count}\NormalTok{(race)}
\end{Highlighting}
\end{Shaded}

\begin{verbatim}
## # A tibble: 3 x 2
##   race      n
##   <fct> <int>
## 1 Other  1959
## 2 Black  3129
## 3 White 16395
\end{verbatim}

Task-3:Also seeing through bar()

\begin{Shaded}
\begin{Highlighting}[]
\FunctionTok{ggplot}\NormalTok{(gss\_cat, }\FunctionTok{aes}\NormalTok{(race)) }\SpecialCharTok{+}
  \FunctionTok{geom\_bar}\NormalTok{()}
\end{Highlighting}
\end{Shaded}

\includegraphics{Intoduction-to-R_files/figure-latex/unnamed-chunk-155-1.pdf}

Task-4:Generating a bar plot using ggplot()

\begin{Shaded}
\begin{Highlighting}[]
\FunctionTok{ggplot}\NormalTok{(gss\_cat,}\FunctionTok{aes}\NormalTok{(race))}\SpecialCharTok{+}\FunctionTok{geom\_bar}\NormalTok{()}\SpecialCharTok{+}\FunctionTok{scale\_x\_discrete}\NormalTok{(}\AttributeTok{drop=}\ConstantTok{FALSE}\NormalTok{)}
\end{Highlighting}
\end{Shaded}

\includegraphics{Intoduction-to-R_files/figure-latex/unnamed-chunk-156-1.pdf}
\# Modifying factor order Task-1:calculating summary statistics and then
creating scatter plot

\begin{Shaded}
\begin{Highlighting}[]
\NormalTok{relig\_summary }\OtherTok{\textless{}{-}}\NormalTok{ gss\_cat }\SpecialCharTok{\%\textgreater{}\%}
  \FunctionTok{group\_by}\NormalTok{(relig) }\SpecialCharTok{\%\textgreater{}\%}
  \FunctionTok{summarise}\NormalTok{(}
    \AttributeTok{age =} \FunctionTok{mean}\NormalTok{(age, }\AttributeTok{na.rm =} \ConstantTok{TRUE}\NormalTok{),}
    \AttributeTok{tvhours =} \FunctionTok{mean}\NormalTok{(tvhours, }\AttributeTok{na.rm =} \ConstantTok{TRUE}\NormalTok{),}
    \AttributeTok{n =} \FunctionTok{n}\NormalTok{()}
\NormalTok{  )}

\FunctionTok{ggplot}\NormalTok{(relig\_summary, }\FunctionTok{aes}\NormalTok{(tvhours, relig)) }\SpecialCharTok{+} \FunctionTok{geom\_point}\NormalTok{()}
\end{Highlighting}
\end{Shaded}

\includegraphics{Intoduction-to-R_files/figure-latex/unnamed-chunk-157-1.pdf}

Task-2:Generating a scatter plot using \texttt{ggplot}, where the x-axis
represents the mean TV hours (\texttt{tvhours}), and the y-axis
represents the \texttt{relig} variable reordered by mean TV hours.

\begin{Shaded}
\begin{Highlighting}[]
\FunctionTok{ggplot}\NormalTok{(relig\_summary, }\FunctionTok{aes}\NormalTok{(tvhours, }\FunctionTok{fct\_reorder}\NormalTok{(relig, tvhours))) }\SpecialCharTok{+}
  \FunctionTok{geom\_point}\NormalTok{()}
\end{Highlighting}
\end{Shaded}

\includegraphics{Intoduction-to-R_files/figure-latex/unnamed-chunk-158-1.pdf}
Task-3:Creating a scatter plot using ggplot.

\begin{Shaded}
\begin{Highlighting}[]
\NormalTok{relig\_summary }\SpecialCharTok{\%\textgreater{}\%}
  \FunctionTok{mutate}\NormalTok{(}\AttributeTok{relig =} \FunctionTok{fct\_reorder}\NormalTok{(relig, tvhours)) }\SpecialCharTok{\%\textgreater{}\%}
  \FunctionTok{ggplot}\NormalTok{(}\FunctionTok{aes}\NormalTok{(tvhours, relig)) }\SpecialCharTok{+}
    \FunctionTok{geom\_point}\NormalTok{()}
\end{Highlighting}
\end{Shaded}

\includegraphics{Intoduction-to-R_files/figure-latex/unnamed-chunk-159-1.pdf}
Task-4:Generating a scatter plot using ggplot

\begin{Shaded}
\begin{Highlighting}[]
\NormalTok{rincome\_summary }\OtherTok{\textless{}{-}}\NormalTok{ gss\_cat }\SpecialCharTok{\%\textgreater{}\%}
  \FunctionTok{group\_by}\NormalTok{(rincome) }\SpecialCharTok{\%\textgreater{}\%}
  \FunctionTok{summarise}\NormalTok{(}
    \AttributeTok{age =} \FunctionTok{mean}\NormalTok{(age, }\AttributeTok{na.rm =} \ConstantTok{TRUE}\NormalTok{),}
    \AttributeTok{tvhours =} \FunctionTok{mean}\NormalTok{(tvhours, }\AttributeTok{na.rm =} \ConstantTok{TRUE}\NormalTok{),}
    \AttributeTok{n =} \FunctionTok{n}\NormalTok{()}
\NormalTok{  )}

\FunctionTok{ggplot}\NormalTok{(rincome\_summary, }\FunctionTok{aes}\NormalTok{(age, }\FunctionTok{fct\_reorder}\NormalTok{(rincome, age))) }\SpecialCharTok{+} \FunctionTok{geom\_point}\NormalTok{()}
\end{Highlighting}
\end{Shaded}

\includegraphics{Intoduction-to-R_files/figure-latex/unnamed-chunk-160-1.pdf}
Task-5: creates a scatter plot of the average age by income level, with
``Not applicable'' as the reference level for income

\begin{Shaded}
\begin{Highlighting}[]
\FunctionTok{ggplot}\NormalTok{(rincome\_summary, }\FunctionTok{aes}\NormalTok{(age, }\FunctionTok{fct\_relevel}\NormalTok{(rincome, }\StringTok{"Not applicable"}\NormalTok{))) }\SpecialCharTok{+}
  \FunctionTok{geom\_point}\NormalTok{()}
\end{Highlighting}
\end{Shaded}

\includegraphics{Intoduction-to-R_files/figure-latex/unnamed-chunk-161-1.pdf}
Task-6:calculating the proportion of each marital status group across
different age groups and creates a line plot showing the distribution of
marital status proportions by age.

\begin{Shaded}
\begin{Highlighting}[]
\NormalTok{by\_age }\OtherTok{\textless{}{-}}\NormalTok{ gss\_cat }\SpecialCharTok{\%\textgreater{}\%}
  \FunctionTok{filter}\NormalTok{(}\SpecialCharTok{!}\FunctionTok{is.na}\NormalTok{(age)) }\SpecialCharTok{\%\textgreater{}\%}
  \FunctionTok{count}\NormalTok{(age, marital) }\SpecialCharTok{\%\textgreater{}\%}
  \FunctionTok{group\_by}\NormalTok{(age) }\SpecialCharTok{\%\textgreater{}\%}
  \FunctionTok{mutate}\NormalTok{(}\AttributeTok{prop =}\NormalTok{ n }\SpecialCharTok{/} \FunctionTok{sum}\NormalTok{(n))}

\FunctionTok{ggplot}\NormalTok{(by\_age, }\FunctionTok{aes}\NormalTok{(age, prop, }\AttributeTok{colour =}\NormalTok{ marital)) }\SpecialCharTok{+}
  \FunctionTok{geom\_line}\NormalTok{(}\AttributeTok{na.rm =} \ConstantTok{TRUE}\NormalTok{)}
\end{Highlighting}
\end{Shaded}

\includegraphics{Intoduction-to-R_files/figure-latex/unnamed-chunk-162-1.pdf}

\begin{Shaded}
\begin{Highlighting}[]
\FunctionTok{ggplot}\NormalTok{(by\_age, }\FunctionTok{aes}\NormalTok{(age, prop, }\AttributeTok{colour =} \FunctionTok{fct\_reorder2}\NormalTok{(marital, age, prop))) }\SpecialCharTok{+}
  \FunctionTok{geom\_line}\NormalTok{() }\SpecialCharTok{+}
  \FunctionTok{labs}\NormalTok{(}\AttributeTok{colour =} \StringTok{"marital"}\NormalTok{)}
\end{Highlighting}
\end{Shaded}

\includegraphics{Intoduction-to-R_files/figure-latex/unnamed-chunk-162-2.pdf}
Task-7: Adjusting the order of the ``marital'' variable based on
frequency and then reverses the order before generating a bar plot
illustrating the distribution of marital status.

\begin{Shaded}
\begin{Highlighting}[]
\NormalTok{gss\_cat }\SpecialCharTok{\%\textgreater{}\%}
  \FunctionTok{mutate}\NormalTok{(}\AttributeTok{marital =}\NormalTok{ marital }\SpecialCharTok{\%\textgreater{}\%} \FunctionTok{fct\_infreq}\NormalTok{() }\SpecialCharTok{\%\textgreater{}\%} \FunctionTok{fct\_rev}\NormalTok{()) }\SpecialCharTok{\%\textgreater{}\%}
  \FunctionTok{ggplot}\NormalTok{(}\FunctionTok{aes}\NormalTok{(marital)) }\SpecialCharTok{+}
    \FunctionTok{geom\_bar}\NormalTok{()}
\end{Highlighting}
\end{Shaded}

\includegraphics{Intoduction-to-R_files/figure-latex/unnamed-chunk-163-1.pdf}
\# Modifying factor levels

Task-1: counting the frequency of each unique value in the ``partyid''
variable of the ``gss\_cat'' dataset.

\begin{Shaded}
\begin{Highlighting}[]
\NormalTok{gss\_cat}\SpecialCharTok{\%\textgreater{}\%}\FunctionTok{count}\NormalTok{(partyid)}
\end{Highlighting}
\end{Shaded}

\begin{verbatim}
## # A tibble: 10 x 2
##    partyid                n
##    <fct>              <int>
##  1 No answer            154
##  2 Don't know             1
##  3 Other party          393
##  4 Strong republican   2314
##  5 Not str republican  3032
##  6 Ind,near rep        1791
##  7 Independent         4119
##  8 Ind,near dem        2499
##  9 Not str democrat    3690
## 10 Strong democrat     3490
\end{verbatim}

Task-2:Recording the levels of the ``partyid'' variable in the
``gss\_cat'' dataset and then counts the frequency of each unique
recorded value.

\begin{Shaded}
\begin{Highlighting}[]
\NormalTok{gss\_cat }\SpecialCharTok{\%\textgreater{}\%}
  \FunctionTok{mutate}\NormalTok{( }\AttributeTok{partyid=}\FunctionTok{fct\_recode}\NormalTok{(partyid,}
    \StringTok{"Republican, strong"}    \OtherTok{=} \StringTok{"Strong republican"}\NormalTok{,}
    \StringTok{"Republican, weak"}      \OtherTok{=} \StringTok{"Not str republican"}\NormalTok{,}
    \StringTok{"Independent, near rep"} \OtherTok{=} \StringTok{"Ind,near rep"}\NormalTok{,}
    \StringTok{"Independent, near dem"} \OtherTok{=} \StringTok{"Ind,near dem"}\NormalTok{,}
    \StringTok{"Democrat, weak"}        \OtherTok{=} \StringTok{"Not str democrat"}\NormalTok{,}
    \StringTok{"Democrat, strong"}      \OtherTok{=} \StringTok{"Strong democrat"}
\NormalTok{    ))}\SpecialCharTok{\%\textgreater{}\%}
  \FunctionTok{count}\NormalTok{(partyid)}
\end{Highlighting}
\end{Shaded}

\begin{verbatim}
## # A tibble: 10 x 2
##    partyid                   n
##    <fct>                 <int>
##  1 No answer               154
##  2 Don't know                1
##  3 Other party             393
##  4 Republican, strong     2314
##  5 Republican, weak       3032
##  6 Independent, near rep  1791
##  7 Independent            4119
##  8 Independent, near dem  2499
##  9 Democrat, weak         3690
## 10 Democrat, strong       3490
\end{verbatim}

Task-3:Recategorizing and counting party affiliations in the
``gss\_cat'' dataset.

\begin{Shaded}
\begin{Highlighting}[]
\NormalTok{gss\_cat }\SpecialCharTok{\%\textgreater{}\%}
  \FunctionTok{mutate}\NormalTok{(}\AttributeTok{partyid =} \FunctionTok{fct\_recode}\NormalTok{(partyid,}
    \StringTok{"Republican, strong"}    \OtherTok{=} \StringTok{"Strong republican"}\NormalTok{,}
    \StringTok{"Republican, weak"}      \OtherTok{=} \StringTok{"Not str republican"}\NormalTok{,}
    \StringTok{"Independent, near rep"} \OtherTok{=} \StringTok{"Ind,near rep"}\NormalTok{,}
    \StringTok{"Independent, near dem"} \OtherTok{=} \StringTok{"Ind,near dem"}\NormalTok{,}
    \StringTok{"Democrat, weak"}        \OtherTok{=} \StringTok{"Not str democrat"}\NormalTok{,}
    \StringTok{"Democrat, strong"}      \OtherTok{=} \StringTok{"Strong democrat"}\NormalTok{,}
    \StringTok{"Other"}                 \OtherTok{=} \StringTok{"No answer"}\NormalTok{,}
    \StringTok{"Other"}                 \OtherTok{=} \StringTok{"Don\textquotesingle{}t know"}\NormalTok{,}
    \StringTok{"Other"}                 \OtherTok{=} \StringTok{"Other party"}
\NormalTok{  )) }\SpecialCharTok{\%\textgreater{}\%}
  \FunctionTok{count}\NormalTok{(partyid)}
\end{Highlighting}
\end{Shaded}

\begin{verbatim}
## # A tibble: 8 x 2
##   partyid                   n
##   <fct>                 <int>
## 1 Other                   548
## 2 Republican, strong     2314
## 3 Republican, weak       3032
## 4 Independent, near rep  1791
## 5 Independent            4119
## 6 Independent, near dem  2499
## 7 Democrat, weak         3690
## 8 Democrat, strong       3490
\end{verbatim}

Task-4: Collapsing categories within the ``partyid'' variable in the
``gss\_cat'' dataset into broader groups and then counting the frequency
of each collapsed category.

\begin{Shaded}
\begin{Highlighting}[]
\NormalTok{gss\_cat}\SpecialCharTok{\%\textgreater{}\%}
  \FunctionTok{mutate}\NormalTok{(}\AttributeTok{partyid=}\FunctionTok{fct\_collapse}\NormalTok{(partyid,}
                              \AttributeTok{other=}\FunctionTok{c}\NormalTok{(}\StringTok{"No answer"}\NormalTok{, }\StringTok{"Don\textquotesingle{}t know"}\NormalTok{, }\StringTok{"Other party"}\NormalTok{),}
                              \AttributeTok{rep=}\FunctionTok{c}\NormalTok{(}\StringTok{"Strong republican"}\NormalTok{, }\StringTok{"Not str republican"}\NormalTok{),}
                              \AttributeTok{ind=}\FunctionTok{c}\NormalTok{(}\StringTok{"Ind,near rep"}\NormalTok{, }\StringTok{"Independent"}\NormalTok{, }\StringTok{"Ind,near dem"}\NormalTok{),}
                              \AttributeTok{dem=}\FunctionTok{c}\NormalTok{(}\StringTok{"Not str democrat"}\NormalTok{, }\StringTok{"Strong democrat"}\NormalTok{))) }\SpecialCharTok{\%\textgreater{}\%}
  \FunctionTok{count}\NormalTok{(partyid)}
\end{Highlighting}
\end{Shaded}

\begin{verbatim}
## # A tibble: 4 x 2
##   partyid     n
##   <fct>   <int>
## 1 other     548
## 2 rep      5346
## 3 ind      8409
## 4 dem      7180
\end{verbatim}

Task-5:Counting and aggregating religious affiliations in the
``gss\_cat'' dataset after lumping together less frequent categories.

\begin{Shaded}
\begin{Highlighting}[]
\NormalTok{gss\_cat }\SpecialCharTok{\%\textgreater{}\%}
  \FunctionTok{mutate}\NormalTok{(}\AttributeTok{relig =} \FunctionTok{fct\_lump}\NormalTok{(relig)) }\SpecialCharTok{\%\textgreater{}\%}
  \FunctionTok{count}\NormalTok{(relig)}
\end{Highlighting}
\end{Shaded}

\begin{verbatim}
## # A tibble: 2 x 2
##   relig          n
##   <fct>      <int>
## 1 Protestant 10846
## 2 Other      10637
\end{verbatim}

Task-6:``Summarizing religious affiliations after lumping infrequent
categories and sort.''

\begin{Shaded}
\begin{Highlighting}[]
\NormalTok{gss\_cat }\SpecialCharTok{\%\textgreater{}\%}
  \FunctionTok{mutate}\NormalTok{(}\AttributeTok{relig =} \FunctionTok{fct\_lump}\NormalTok{(relig, }\AttributeTok{n =} \DecValTok{10}\NormalTok{)) }\SpecialCharTok{\%\textgreater{}\%}
  \FunctionTok{count}\NormalTok{(relig, }\AttributeTok{sort =} \ConstantTok{TRUE}\NormalTok{) }\SpecialCharTok{\%\textgreater{}\%}
  \FunctionTok{print}\NormalTok{(}\AttributeTok{n =} \ConstantTok{Inf}\NormalTok{)}
\end{Highlighting}
\end{Shaded}

\begin{verbatim}
## # A tibble: 10 x 2
##    relig                       n
##    <fct>                   <int>
##  1 Protestant              10846
##  2 Catholic                 5124
##  3 None                     3523
##  4 Christian                 689
##  5 Other                     458
##  6 Jewish                    388
##  7 Buddhism                  147
##  8 Inter-nondenominational   109
##  9 Moslem/islam              104
## 10 Orthodox-christian         95
\end{verbatim}

\hypertarget{ch-data-and-times}{%
\section{CH-Data and Times}\label{ch-data-and-times}}

Task-1:Loading library

\begin{Shaded}
\begin{Highlighting}[]
\FunctionTok{library}\NormalTok{(tidyverse)}

\FunctionTok{library}\NormalTok{(lubridate)}
\FunctionTok{library}\NormalTok{(nycflights13)}
\end{Highlighting}
\end{Shaded}

\hypertarget{creating-datestimes}{%
\subsection{Creating dates/times}\label{creating-datestimes}}

Task-1: Printing current date or date-time

\begin{Shaded}
\begin{Highlighting}[]
\FunctionTok{today}\NormalTok{()}
\end{Highlighting}
\end{Shaded}

\begin{verbatim}
## [1] "2024-06-13"
\end{verbatim}

\begin{Shaded}
\begin{Highlighting}[]
\FunctionTok{now}\NormalTok{()}
\end{Highlighting}
\end{Shaded}

\begin{verbatim}
## [1] "2024-06-13 08:41:51 +0545"
\end{verbatim}

\hypertarget{form-strings}{%
\subsection{Form strings}\label{form-strings}}

Task-2:converting date strings to date objects in different formats.

\begin{Shaded}
\begin{Highlighting}[]
\FunctionTok{ymd}\NormalTok{(}\StringTok{"2017{-}01{-}31"}\NormalTok{)}
\end{Highlighting}
\end{Shaded}

\begin{verbatim}
## [1] "2017-01-31"
\end{verbatim}

\begin{Shaded}
\begin{Highlighting}[]
\FunctionTok{mdy}\NormalTok{(}\StringTok{"January 31st, 2017"}\NormalTok{)}
\end{Highlighting}
\end{Shaded}

\begin{verbatim}
## [1] "2017-01-31"
\end{verbatim}

\begin{Shaded}
\begin{Highlighting}[]
\FunctionTok{dmy}\NormalTok{(}\StringTok{"31{-}Jan{-}2017"}\NormalTok{)}
\end{Highlighting}
\end{Shaded}

\begin{verbatim}
## [1] "2017-01-31"
\end{verbatim}

\begin{Shaded}
\begin{Highlighting}[]
\FunctionTok{ymd}\NormalTok{(}\DecValTok{20170131}\NormalTok{)}
\end{Highlighting}
\end{Shaded}

\begin{verbatim}
## [1] "2017-01-31"
\end{verbatim}

\begin{Shaded}
\begin{Highlighting}[]
\FunctionTok{ymd\_hms}\NormalTok{(}\StringTok{"2017{-}01{-}31 20:11:59"}\NormalTok{)}
\end{Highlighting}
\end{Shaded}

\begin{verbatim}
## [1] "2017-01-31 20:11:59 UTC"
\end{verbatim}

\begin{Shaded}
\begin{Highlighting}[]
\FunctionTok{mdy\_hm}\NormalTok{(}\StringTok{"01/31/2017 08:01"}\NormalTok{)}
\end{Highlighting}
\end{Shaded}

\begin{verbatim}
## [1] "2017-01-31 08:01:00 UTC"
\end{verbatim}

\begin{Shaded}
\begin{Highlighting}[]
\NormalTok{flights }\SpecialCharTok{\%\textgreater{}\%} 
  \FunctionTok{select}\NormalTok{(year, month, day, hour, minute)}
\end{Highlighting}
\end{Shaded}

\begin{verbatim}
## # A tibble: 336,776 x 5
##     year month   day  hour minute
##    <int> <int> <int> <dbl>  <dbl>
##  1  2013     1     1     5     15
##  2  2013     1     1     5     29
##  3  2013     1     1     5     40
##  4  2013     1     1     5     45
##  5  2013     1     1     6      0
##  6  2013     1     1     5     58
##  7  2013     1     1     6      0
##  8  2013     1     1     6      0
##  9  2013     1     1     6      0
## 10  2013     1     1     6      0
## # i 336,766 more rows
\end{verbatim}

\begin{Shaded}
\begin{Highlighting}[]
\NormalTok{flights }\SpecialCharTok{\%\textgreater{}\%} 
  \FunctionTok{select}\NormalTok{(year, month, day, hour, minute) }\SpecialCharTok{\%\textgreater{}\%} 
  \FunctionTok{mutate}\NormalTok{(}\AttributeTok{departure =} \FunctionTok{make\_datetime}\NormalTok{(year, month, day, hour, minute))}
\end{Highlighting}
\end{Shaded}

\begin{verbatim}
## # A tibble: 336,776 x 6
##     year month   day  hour minute departure          
##    <int> <int> <int> <dbl>  <dbl> <dttm>             
##  1  2013     1     1     5     15 2013-01-01 05:15:00
##  2  2013     1     1     5     29 2013-01-01 05:29:00
##  3  2013     1     1     5     40 2013-01-01 05:40:00
##  4  2013     1     1     5     45 2013-01-01 05:45:00
##  5  2013     1     1     6      0 2013-01-01 06:00:00
##  6  2013     1     1     5     58 2013-01-01 05:58:00
##  7  2013     1     1     6      0 2013-01-01 06:00:00
##  8  2013     1     1     6      0 2013-01-01 06:00:00
##  9  2013     1     1     6      0 2013-01-01 06:00:00
## 10  2013     1     1     6      0 2013-01-01 06:00:00
## # i 336,766 more rows
\end{verbatim}

Task: Creating date-time objects from hour-minute time data in the
`flights' dataset and filtering out rows with missing departure or
arrival times

\begin{Shaded}
\begin{Highlighting}[]
\NormalTok{make\_datetime\_100 }\OtherTok{\textless{}{-}} \ControlFlowTok{function}\NormalTok{(year, month, day, time) \{}
  \FunctionTok{make\_datetime}\NormalTok{(year, month, day, time }\SpecialCharTok{\%/\%} \DecValTok{100}\NormalTok{, time }\SpecialCharTok{\%\%} \DecValTok{100}\NormalTok{)}
\NormalTok{\}}

\NormalTok{flights\_dt }\OtherTok{\textless{}{-}}\NormalTok{ flights }\SpecialCharTok{\%\textgreater{}\%} 
  \FunctionTok{filter}\NormalTok{(}\SpecialCharTok{!}\FunctionTok{is.na}\NormalTok{(dep\_time), }\SpecialCharTok{!}\FunctionTok{is.na}\NormalTok{(arr\_time)) }\SpecialCharTok{\%\textgreater{}\%} 
  \FunctionTok{mutate}\NormalTok{(}
    \AttributeTok{dep\_time =} \FunctionTok{make\_datetime\_100}\NormalTok{(year, month, day, dep\_time),}
    \AttributeTok{arr\_time =} \FunctionTok{make\_datetime\_100}\NormalTok{(year, month, day, arr\_time),}
    \AttributeTok{sched\_dep\_time =} \FunctionTok{make\_datetime\_100}\NormalTok{(year, month, day, sched\_dep\_time),}
    \AttributeTok{sched\_arr\_time =} \FunctionTok{make\_datetime\_100}\NormalTok{(year, month, day, sched\_arr\_time)}
\NormalTok{  ) }\SpecialCharTok{\%\textgreater{}\%} 
  \FunctionTok{select}\NormalTok{(origin, dest, }\FunctionTok{ends\_with}\NormalTok{(}\StringTok{"delay"}\NormalTok{), }\FunctionTok{ends\_with}\NormalTok{(}\StringTok{"time"}\NormalTok{))}

\NormalTok{flights\_dt}
\end{Highlighting}
\end{Shaded}

\begin{verbatim}
## # A tibble: 328,063 x 9
##    origin dest  dep_delay arr_delay dep_time            sched_dep_time     
##    <chr>  <chr>     <dbl>     <dbl> <dttm>              <dttm>             
##  1 EWR    IAH           2        11 2013-01-01 05:17:00 2013-01-01 05:15:00
##  2 LGA    IAH           4        20 2013-01-01 05:33:00 2013-01-01 05:29:00
##  3 JFK    MIA           2        33 2013-01-01 05:42:00 2013-01-01 05:40:00
##  4 JFK    BQN          -1       -18 2013-01-01 05:44:00 2013-01-01 05:45:00
##  5 LGA    ATL          -6       -25 2013-01-01 05:54:00 2013-01-01 06:00:00
##  6 EWR    ORD          -4        12 2013-01-01 05:54:00 2013-01-01 05:58:00
##  7 EWR    FLL          -5        19 2013-01-01 05:55:00 2013-01-01 06:00:00
##  8 LGA    IAD          -3       -14 2013-01-01 05:57:00 2013-01-01 06:00:00
##  9 JFK    MCO          -3        -8 2013-01-01 05:57:00 2013-01-01 06:00:00
## 10 LGA    ORD          -2         8 2013-01-01 05:58:00 2013-01-01 06:00:00
## # i 328,053 more rows
## # i 3 more variables: arr_time <dttm>, sched_arr_time <dttm>, air_time <dbl>
\end{verbatim}

Task: Plotting the frequency of flights over time using departure
date-time

\begin{Shaded}
\begin{Highlighting}[]
\NormalTok{flights\_dt }\SpecialCharTok{\%\textgreater{}\%} 
  \FunctionTok{ggplot}\NormalTok{(}\FunctionTok{aes}\NormalTok{(dep\_time)) }\SpecialCharTok{+} 
  \FunctionTok{geom\_freqpoly}\NormalTok{(}\AttributeTok{binwidth =} \DecValTok{86400}\NormalTok{) }
\end{Highlighting}
\end{Shaded}

\includegraphics{Intoduction-to-R_files/figure-latex/unnamed-chunk-177-1.pdf}
Task: Plotting the frequency of flights over time for a specific period
using departure date-time

\begin{Shaded}
\begin{Highlighting}[]
\NormalTok{flights\_dt }\SpecialCharTok{\%\textgreater{}\%} 
  \FunctionTok{filter}\NormalTok{(dep\_time }\SpecialCharTok{\textless{}} \FunctionTok{ymd}\NormalTok{(}\DecValTok{20130102}\NormalTok{)) }\SpecialCharTok{\%\textgreater{}\%} 
  \FunctionTok{ggplot}\NormalTok{(}\FunctionTok{aes}\NormalTok{(dep\_time)) }\SpecialCharTok{+} 
  \FunctionTok{geom\_freqpoly}\NormalTok{(}\AttributeTok{binwidth =} \DecValTok{600}\NormalTok{) }\CommentTok{\# 600 s = 10 minutes}
\end{Highlighting}
\end{Shaded}

\includegraphics{Intoduction-to-R_files/figure-latex/unnamed-chunk-178-1.pdf}
Task: to convert today's date to date-time object

\begin{Shaded}
\begin{Highlighting}[]
\FunctionTok{as\_datetime}\NormalTok{(}\FunctionTok{today}\NormalTok{())}
\end{Highlighting}
\end{Shaded}

\begin{verbatim}
## [1] "2024-06-13 UTC"
\end{verbatim}

\begin{Shaded}
\begin{Highlighting}[]
\FunctionTok{as\_date}\NormalTok{(}\FunctionTok{now}\NormalTok{())}
\end{Highlighting}
\end{Shaded}

\begin{verbatim}
## [1] "2024-06-13"
\end{verbatim}

\begin{Shaded}
\begin{Highlighting}[]
\FunctionTok{as\_date}\NormalTok{(}\DecValTok{365} \SpecialCharTok{*} \DecValTok{10} \SpecialCharTok{+} \DecValTok{2}\NormalTok{)}
\end{Highlighting}
\end{Shaded}

\begin{verbatim}
## [1] "1980-01-01"
\end{verbatim}

Date-time components Task: Extracting various components of a date-time
object

\begin{Shaded}
\begin{Highlighting}[]
\NormalTok{datetime }\OtherTok{\textless{}{-}} \FunctionTok{ymd\_hms}\NormalTok{(}\StringTok{"2016{-}07{-}08 12:34:56"}\NormalTok{)}
\FunctionTok{year}\NormalTok{(datetime)}
\end{Highlighting}
\end{Shaded}

\begin{verbatim}
## [1] 2016
\end{verbatim}

\begin{Shaded}
\begin{Highlighting}[]
\FunctionTok{month}\NormalTok{(datetime)}
\end{Highlighting}
\end{Shaded}

\begin{verbatim}
## [1] 7
\end{verbatim}

\begin{Shaded}
\begin{Highlighting}[]
\FunctionTok{mday}\NormalTok{(datetime)}
\end{Highlighting}
\end{Shaded}

\begin{verbatim}
## [1] 8
\end{verbatim}

\begin{Shaded}
\begin{Highlighting}[]
\FunctionTok{yday}\NormalTok{(datetime)}
\end{Highlighting}
\end{Shaded}

\begin{verbatim}
## [1] 190
\end{verbatim}

\begin{Shaded}
\begin{Highlighting}[]
\FunctionTok{wday}\NormalTok{(datetime)}
\end{Highlighting}
\end{Shaded}

\begin{verbatim}
## [1] 6
\end{verbatim}

\begin{Shaded}
\begin{Highlighting}[]
\FunctionTok{month}\NormalTok{(datetime, }\AttributeTok{label =} \ConstantTok{TRUE}\NormalTok{)}
\end{Highlighting}
\end{Shaded}

\begin{verbatim}
## [1] Jul
## 12 Levels: Jan < Feb < Mar < Apr < May < Jun < Jul < Aug < Sep < ... < Dec
\end{verbatim}

\begin{Shaded}
\begin{Highlighting}[]
\FunctionTok{wday}\NormalTok{(datetime, }\AttributeTok{label =} \ConstantTok{TRUE}\NormalTok{, }\AttributeTok{abbr =} \ConstantTok{FALSE}\NormalTok{)}
\end{Highlighting}
\end{Shaded}

\begin{verbatim}
## [1] Friday
## 7 Levels: Sunday < Monday < Tuesday < Wednesday < Thursday < ... < Saturday
\end{verbatim}

Task: Plotting the frequency of flights by day of the week

\begin{Shaded}
\begin{Highlighting}[]
\NormalTok{flights\_dt }\SpecialCharTok{\%\textgreater{}\%} 
  \FunctionTok{mutate}\NormalTok{(}\AttributeTok{wday =} \FunctionTok{wday}\NormalTok{(dep\_time, }\AttributeTok{label =} \ConstantTok{TRUE}\NormalTok{)) }\SpecialCharTok{\%\textgreater{}\%} 
  \FunctionTok{ggplot}\NormalTok{(}\FunctionTok{aes}\NormalTok{(}\AttributeTok{x =}\NormalTok{ wday)) }\SpecialCharTok{+}
    \FunctionTok{geom\_bar}\NormalTok{()}
\end{Highlighting}
\end{Shaded}

\includegraphics{Intoduction-to-R_files/figure-latex/unnamed-chunk-181-1.pdf}
Task: Plotting average delay by minute of departure time

\begin{Shaded}
\begin{Highlighting}[]
\NormalTok{flights\_dt }\SpecialCharTok{\%\textgreater{}\%} 
  \FunctionTok{mutate}\NormalTok{(}\AttributeTok{minute =} \FunctionTok{minute}\NormalTok{(dep\_time)) }\SpecialCharTok{\%\textgreater{}\%} 
  \FunctionTok{group\_by}\NormalTok{(minute) }\SpecialCharTok{\%\textgreater{}\%} 
  \FunctionTok{summarise}\NormalTok{(}
    \AttributeTok{avg\_delay =} \FunctionTok{mean}\NormalTok{(arr\_delay, }\AttributeTok{na.rm =} \ConstantTok{TRUE}\NormalTok{),}
    \AttributeTok{n =} \FunctionTok{n}\NormalTok{()) }\SpecialCharTok{\%\textgreater{}\%} 
  \FunctionTok{ggplot}\NormalTok{(}\FunctionTok{aes}\NormalTok{(minute, avg\_delay)) }\SpecialCharTok{+}
    \FunctionTok{geom\_line}\NormalTok{()}
\end{Highlighting}
\end{Shaded}

\includegraphics{Intoduction-to-R_files/figure-latex/unnamed-chunk-182-1.pdf}
Task: Plotting average delay by minute of scheduled departure time

\begin{Shaded}
\begin{Highlighting}[]
\NormalTok{sched\_dep }\OtherTok{\textless{}{-}}\NormalTok{ flights\_dt }\SpecialCharTok{\%\textgreater{}\%} 
  \FunctionTok{mutate}\NormalTok{(}\AttributeTok{minute =} \FunctionTok{minute}\NormalTok{(sched\_dep\_time)) }\SpecialCharTok{\%\textgreater{}\%} 
  \FunctionTok{group\_by}\NormalTok{(minute) }\SpecialCharTok{\%\textgreater{}\%} 
  \FunctionTok{summarise}\NormalTok{(}
    \AttributeTok{avg\_delay =} \FunctionTok{mean}\NormalTok{(arr\_delay, }\AttributeTok{na.rm =} \ConstantTok{TRUE}\NormalTok{),}
    \AttributeTok{n =} \FunctionTok{n}\NormalTok{())}

\FunctionTok{ggplot}\NormalTok{(sched\_dep, }\FunctionTok{aes}\NormalTok{(minute, avg\_delay)) }\SpecialCharTok{+}
  \FunctionTok{geom\_line}\NormalTok{()}
\end{Highlighting}
\end{Shaded}

\includegraphics{Intoduction-to-R_files/figure-latex/unnamed-chunk-183-1.pdf}
Task: Plotting the number of flights by minute of scheduled departure
time

\begin{Shaded}
\begin{Highlighting}[]
\FunctionTok{ggplot}\NormalTok{(sched\_dep, }\FunctionTok{aes}\NormalTok{(minute, n)) }\SpecialCharTok{+}
  \FunctionTok{geom\_line}\NormalTok{()}
\end{Highlighting}
\end{Shaded}

\includegraphics{Intoduction-to-R_files/figure-latex/unnamed-chunk-184-1.pdf}
Rounding Task:Plotting the number of flights by week, rounding to the
nearest week

\begin{Shaded}
\begin{Highlighting}[]
\NormalTok{flights\_dt }\SpecialCharTok{\%\textgreater{}\%} 
  \FunctionTok{count}\NormalTok{(}\AttributeTok{week =} \FunctionTok{floor\_date}\NormalTok{(dep\_time, }\StringTok{"week"}\NormalTok{)) }\SpecialCharTok{\%\textgreater{}\%} 
  \FunctionTok{ggplot}\NormalTok{(}\FunctionTok{aes}\NormalTok{(week, n)) }\SpecialCharTok{+}
    \FunctionTok{geom\_line}\NormalTok{()}
\end{Highlighting}
\end{Shaded}

\includegraphics{Intoduction-to-R_files/figure-latex/unnamed-chunk-185-1.pdf}
setting compounds Task: Setting up a date-time object

\begin{Shaded}
\begin{Highlighting}[]
\NormalTok{(datetime }\OtherTok{\textless{}{-}} \FunctionTok{ymd\_hms}\NormalTok{(}\StringTok{"2016{-}07{-}08 12:34:56"}\NormalTok{))}
\end{Highlighting}
\end{Shaded}

\begin{verbatim}
## [1] "2016-07-08 12:34:56 UTC"
\end{verbatim}

\begin{Shaded}
\begin{Highlighting}[]
\FunctionTok{year}\NormalTok{(datetime) }\OtherTok{\textless{}{-}} \DecValTok{2020}
\NormalTok{datetime}
\end{Highlighting}
\end{Shaded}

\begin{verbatim}
## [1] "2020-07-08 12:34:56 UTC"
\end{verbatim}

\begin{Shaded}
\begin{Highlighting}[]
\FunctionTok{month}\NormalTok{(datetime) }\OtherTok{\textless{}{-}} \DecValTok{01}
\NormalTok{datetime}
\end{Highlighting}
\end{Shaded}

\begin{verbatim}
## [1] "2020-01-08 12:34:56 UTC"
\end{verbatim}

\begin{Shaded}
\begin{Highlighting}[]
\FunctionTok{hour}\NormalTok{(datetime) }\OtherTok{\textless{}{-}} \FunctionTok{hour}\NormalTok{(datetime) }\SpecialCharTok{+} \DecValTok{1}
\NormalTok{datetime}
\end{Highlighting}
\end{Shaded}

\begin{verbatim}
## [1] "2020-01-08 13:34:56 UTC"
\end{verbatim}

\begin{Shaded}
\begin{Highlighting}[]
\FunctionTok{update}\NormalTok{(datetime, }\AttributeTok{year =} \DecValTok{2020}\NormalTok{, }\AttributeTok{month =} \DecValTok{2}\NormalTok{, }\AttributeTok{mday =} \DecValTok{2}\NormalTok{, }\AttributeTok{hour =} \DecValTok{2}\NormalTok{)}
\end{Highlighting}
\end{Shaded}

\begin{verbatim}
## [1] "2020-02-02 02:34:56 UTC"
\end{verbatim}

\begin{Shaded}
\begin{Highlighting}[]
\FunctionTok{ymd}\NormalTok{(}\StringTok{"2015{-}02{-}01"}\NormalTok{) }\SpecialCharTok{\%\textgreater{}\%} 
  \FunctionTok{update}\NormalTok{(}\AttributeTok{mday =} \DecValTok{30}\NormalTok{)}
\end{Highlighting}
\end{Shaded}

\begin{verbatim}
## [1] "2015-03-02"
\end{verbatim}

\begin{Shaded}
\begin{Highlighting}[]
\FunctionTok{ymd}\NormalTok{(}\StringTok{"2015{-}02{-}01"}\NormalTok{) }\SpecialCharTok{\%\textgreater{}\%} 
  \FunctionTok{update}\NormalTok{(}\AttributeTok{hour =} \DecValTok{400}\NormalTok{)}
\end{Highlighting}
\end{Shaded}

\begin{verbatim}
## [1] "2015-02-17 16:00:00 UTC"
\end{verbatim}

Task: Creating a new variable `dep\_hour' by updating the `dep\_time' to
the first day of the year

\begin{Shaded}
\begin{Highlighting}[]
\NormalTok{flights\_dt }\SpecialCharTok{\%\textgreater{}\%} 
  \FunctionTok{mutate}\NormalTok{(}\AttributeTok{dep\_hour =} \FunctionTok{update}\NormalTok{(dep\_time, }\AttributeTok{yday =} \DecValTok{1}\NormalTok{)) }\SpecialCharTok{\%\textgreater{}\%} 
  \FunctionTok{ggplot}\NormalTok{(}\FunctionTok{aes}\NormalTok{(dep\_hour)) }\SpecialCharTok{+}
    \FunctionTok{geom\_freqpoly}\NormalTok{(}\AttributeTok{binwidth =} \DecValTok{300}\NormalTok{)}
\end{Highlighting}
\end{Shaded}

\includegraphics{Intoduction-to-R_files/figure-latex/unnamed-chunk-188-1.pdf}
Time Spans Compute the age of a person based on their birthdate and
today's date

\begin{Shaded}
\begin{Highlighting}[]
\NormalTok{h\_age }\OtherTok{\textless{}{-}} \FunctionTok{today}\NormalTok{() }\SpecialCharTok{{-}} \FunctionTok{ymd}\NormalTok{(}\DecValTok{19791014}\NormalTok{)}
\NormalTok{h\_age}
\end{Highlighting}
\end{Shaded}

\begin{verbatim}
## Time difference of 16314 days
\end{verbatim}

\begin{Shaded}
\begin{Highlighting}[]
\FunctionTok{as.duration}\NormalTok{(h\_age)}
\end{Highlighting}
\end{Shaded}

\begin{verbatim}
## [1] "1409529600s (~44.67 years)"
\end{verbatim}

\begin{Shaded}
\begin{Highlighting}[]
\FunctionTok{dseconds}\NormalTok{(}\DecValTok{15}\NormalTok{)}
\end{Highlighting}
\end{Shaded}

\begin{verbatim}
## [1] "15s"
\end{verbatim}

\begin{Shaded}
\begin{Highlighting}[]
\FunctionTok{dminutes}\NormalTok{(}\DecValTok{10}\NormalTok{)}
\end{Highlighting}
\end{Shaded}

\begin{verbatim}
## [1] "600s (~10 minutes)"
\end{verbatim}

\begin{Shaded}
\begin{Highlighting}[]
\FunctionTok{dhours}\NormalTok{(}\FunctionTok{c}\NormalTok{(}\DecValTok{12}\NormalTok{, }\DecValTok{24}\NormalTok{))}
\end{Highlighting}
\end{Shaded}

\begin{verbatim}
## [1] "43200s (~12 hours)" "86400s (~1 days)"
\end{verbatim}

\begin{Shaded}
\begin{Highlighting}[]
\FunctionTok{ddays}\NormalTok{(}\DecValTok{0}\SpecialCharTok{:}\DecValTok{5}\NormalTok{)}
\end{Highlighting}
\end{Shaded}

\begin{verbatim}
## [1] "0s"                "86400s (~1 days)"  "172800s (~2 days)"
## [4] "259200s (~3 days)" "345600s (~4 days)" "432000s (~5 days)"
\end{verbatim}

\begin{Shaded}
\begin{Highlighting}[]
\FunctionTok{dweeks}\NormalTok{(}\DecValTok{3}\NormalTok{)}
\end{Highlighting}
\end{Shaded}

\begin{verbatim}
## [1] "1814400s (~3 weeks)"
\end{verbatim}

\begin{Shaded}
\begin{Highlighting}[]
\FunctionTok{dyears}\NormalTok{(}\DecValTok{1}\NormalTok{)}
\end{Highlighting}
\end{Shaded}

\begin{verbatim}
## [1] "31557600s (~1 years)"
\end{verbatim}

\begin{Shaded}
\begin{Highlighting}[]
\DecValTok{2} \SpecialCharTok{*} \FunctionTok{dyears}\NormalTok{(}\DecValTok{1}\NormalTok{)}
\end{Highlighting}
\end{Shaded}

\begin{verbatim}
## [1] "63115200s (~2 years)"
\end{verbatim}

\begin{Shaded}
\begin{Highlighting}[]
\FunctionTok{dyears}\NormalTok{(}\DecValTok{1}\NormalTok{) }\SpecialCharTok{+} \FunctionTok{dweeks}\NormalTok{(}\DecValTok{12}\NormalTok{) }\SpecialCharTok{+} \FunctionTok{dhours}\NormalTok{(}\DecValTok{15}\NormalTok{)}
\end{Highlighting}
\end{Shaded}

\begin{verbatim}
## [1] "38869200s (~1.23 years)"
\end{verbatim}

\begin{Shaded}
\begin{Highlighting}[]
\NormalTok{tomorrow }\OtherTok{\textless{}{-}} \FunctionTok{today}\NormalTok{() }\SpecialCharTok{+} \FunctionTok{ddays}\NormalTok{(}\DecValTok{1}\NormalTok{)}
\NormalTok{last\_year }\OtherTok{\textless{}{-}} \FunctionTok{today}\NormalTok{() }\SpecialCharTok{{-}} \FunctionTok{dyears}\NormalTok{(}\DecValTok{1}\NormalTok{)}
\NormalTok{one\_pm }\OtherTok{\textless{}{-}} \FunctionTok{ymd\_hms}\NormalTok{(}\StringTok{"2016{-}03{-}12 13:00:00"}\NormalTok{, }\AttributeTok{tz =} \StringTok{"America/New\_York"}\NormalTok{)}
\NormalTok{one\_pm}
\end{Highlighting}
\end{Shaded}

\begin{verbatim}
## [1] "2016-03-12 13:00:00 EST"
\end{verbatim}

\begin{Shaded}
\begin{Highlighting}[]
\NormalTok{one\_pm }\SpecialCharTok{+} \FunctionTok{ddays}\NormalTok{(}\DecValTok{1}\NormalTok{)}
\end{Highlighting}
\end{Shaded}

\begin{verbatim}
## [1] "2016-03-13 14:00:00 EDT"
\end{verbatim}

Periods Create period objects representing different time spans and
Perform arithmetic operations with period objects

\begin{Shaded}
\begin{Highlighting}[]
\NormalTok{one\_pm}
\end{Highlighting}
\end{Shaded}

\begin{verbatim}
## [1] "2016-03-12 13:00:00 EST"
\end{verbatim}

\begin{Shaded}
\begin{Highlighting}[]
\NormalTok{one\_om }\OtherTok{=} \FunctionTok{days}\NormalTok{(}\DecValTok{1}\NormalTok{)}
\end{Highlighting}
\end{Shaded}

\begin{Shaded}
\begin{Highlighting}[]
\FunctionTok{seconds}\NormalTok{(}\DecValTok{15}\NormalTok{)}
\end{Highlighting}
\end{Shaded}

\begin{verbatim}
## [1] "15S"
\end{verbatim}

\begin{Shaded}
\begin{Highlighting}[]
\FunctionTok{minutes}\NormalTok{(}\DecValTok{10}\NormalTok{)}
\end{Highlighting}
\end{Shaded}

\begin{verbatim}
## [1] "10M 0S"
\end{verbatim}

\begin{Shaded}
\begin{Highlighting}[]
\FunctionTok{hours}\NormalTok{(}\FunctionTok{c}\NormalTok{(}\DecValTok{12}\NormalTok{, }\DecValTok{24}\NormalTok{))}
\end{Highlighting}
\end{Shaded}

\begin{verbatim}
## [1] "12H 0M 0S" "24H 0M 0S"
\end{verbatim}

\begin{Shaded}
\begin{Highlighting}[]
\FunctionTok{days}\NormalTok{(}\DecValTok{7}\NormalTok{)}
\end{Highlighting}
\end{Shaded}

\begin{verbatim}
## [1] "7d 0H 0M 0S"
\end{verbatim}

\begin{Shaded}
\begin{Highlighting}[]
\FunctionTok{months}\NormalTok{(}\DecValTok{1}\SpecialCharTok{:}\DecValTok{6}\NormalTok{)}
\end{Highlighting}
\end{Shaded}

\begin{verbatim}
## [1] "1m 0d 0H 0M 0S" "2m 0d 0H 0M 0S" "3m 0d 0H 0M 0S" "4m 0d 0H 0M 0S"
## [5] "5m 0d 0H 0M 0S" "6m 0d 0H 0M 0S"
\end{verbatim}

\begin{Shaded}
\begin{Highlighting}[]
\FunctionTok{weeks}\NormalTok{(}\DecValTok{3}\NormalTok{)}
\end{Highlighting}
\end{Shaded}

\begin{verbatim}
## [1] "21d 0H 0M 0S"
\end{verbatim}

\begin{Shaded}
\begin{Highlighting}[]
\FunctionTok{years}\NormalTok{(}\DecValTok{1}\NormalTok{)}
\end{Highlighting}
\end{Shaded}

\begin{verbatim}
## [1] "1y 0m 0d 0H 0M 0S"
\end{verbatim}

\begin{Shaded}
\begin{Highlighting}[]
\DecValTok{10} \SpecialCharTok{*}\NormalTok{ (}\FunctionTok{months}\NormalTok{(}\DecValTok{6}\NormalTok{) }\SpecialCharTok{+} \FunctionTok{days}\NormalTok{(}\DecValTok{1}\NormalTok{))}
\end{Highlighting}
\end{Shaded}

\begin{verbatim}
## [1] "60m 10d 0H 0M 0S"
\end{verbatim}

\begin{Shaded}
\begin{Highlighting}[]
\FunctionTok{days}\NormalTok{(}\DecValTok{50}\NormalTok{) }\SpecialCharTok{+} \FunctionTok{hours}\NormalTok{(}\DecValTok{25}\NormalTok{) }\SpecialCharTok{+} \FunctionTok{minutes}\NormalTok{(}\DecValTok{2}\NormalTok{)}
\end{Highlighting}
\end{Shaded}

\begin{verbatim}
## [1] "50d 25H 2M 0S"
\end{verbatim}

\begin{Shaded}
\begin{Highlighting}[]
\FunctionTok{ymd}\NormalTok{(}\StringTok{"2016{-}01{-}01"}\NormalTok{) }\SpecialCharTok{+} \FunctionTok{dyears}\NormalTok{(}\DecValTok{1}\NormalTok{)}
\end{Highlighting}
\end{Shaded}

\begin{verbatim}
## [1] "2016-12-31 06:00:00 UTC"
\end{verbatim}

\begin{Shaded}
\begin{Highlighting}[]
\FunctionTok{ymd}\NormalTok{(}\StringTok{"2016{-}01{-}01"}\NormalTok{) }\SpecialCharTok{+} \FunctionTok{years}\NormalTok{(}\DecValTok{1}\NormalTok{)}
\end{Highlighting}
\end{Shaded}

\begin{verbatim}
## [1] "2017-01-01"
\end{verbatim}

\begin{Shaded}
\begin{Highlighting}[]
\NormalTok{one\_pm }\SpecialCharTok{+} \FunctionTok{ddays}\NormalTok{(}\DecValTok{1}\NormalTok{)}
\end{Highlighting}
\end{Shaded}

\begin{verbatim}
## [1] "2016-03-13 14:00:00 EDT"
\end{verbatim}

\begin{Shaded}
\begin{Highlighting}[]
\NormalTok{one\_pm }\SpecialCharTok{+} \FunctionTok{days}\NormalTok{(}\DecValTok{1}\NormalTok{)}
\end{Highlighting}
\end{Shaded}

\begin{verbatim}
## [1] "2016-03-13 13:00:00 EDT"
\end{verbatim}

Filter flights where arrival time is before departure time

\begin{Shaded}
\begin{Highlighting}[]
\NormalTok{flights\_dt }\SpecialCharTok{\%\textgreater{}\%} 
  \FunctionTok{filter}\NormalTok{(arr\_time }\SpecialCharTok{\textless{}}\NormalTok{ dep\_time) }
\end{Highlighting}
\end{Shaded}

\begin{verbatim}
## # A tibble: 10,633 x 9
##    origin dest  dep_delay arr_delay dep_time            sched_dep_time     
##    <chr>  <chr>     <dbl>     <dbl> <dttm>              <dttm>             
##  1 EWR    BQN           9        -4 2013-01-01 19:29:00 2013-01-01 19:20:00
##  2 JFK    DFW          59        NA 2013-01-01 19:39:00 2013-01-01 18:40:00
##  3 EWR    TPA          -2         9 2013-01-01 20:58:00 2013-01-01 21:00:00
##  4 EWR    SJU          -6       -12 2013-01-01 21:02:00 2013-01-01 21:08:00
##  5 EWR    SFO          11       -14 2013-01-01 21:08:00 2013-01-01 20:57:00
##  6 LGA    FLL         -10        -2 2013-01-01 21:20:00 2013-01-01 21:30:00
##  7 EWR    MCO          41        43 2013-01-01 21:21:00 2013-01-01 20:40:00
##  8 JFK    LAX          -7       -24 2013-01-01 21:28:00 2013-01-01 21:35:00
##  9 EWR    FLL          49        28 2013-01-01 21:34:00 2013-01-01 20:45:00
## 10 EWR    FLL          -9       -14 2013-01-01 21:36:00 2013-01-01 21:45:00
## # i 10,623 more rows
## # i 3 more variables: arr_time <dttm>, sched_arr_time <dttm>, air_time <dbl>
\end{verbatim}

Update flights data to correct overnight flights

\begin{Shaded}
\begin{Highlighting}[]
\NormalTok{flights\_dt }\OtherTok{\textless{}{-}}\NormalTok{ flights\_dt }\SpecialCharTok{\%\textgreater{}\%} 
  \FunctionTok{mutate}\NormalTok{(}
    \AttributeTok{overnight =}\NormalTok{ arr\_time }\SpecialCharTok{\textless{}}\NormalTok{ dep\_time,}
    \AttributeTok{arr\_time =}\NormalTok{ arr\_time }\SpecialCharTok{+} \FunctionTok{days}\NormalTok{(overnight }\SpecialCharTok{*} \DecValTok{1}\NormalTok{),}
    \AttributeTok{sched\_arr\_time =}\NormalTok{ sched\_arr\_time }\SpecialCharTok{+} \FunctionTok{days}\NormalTok{(overnight }\SpecialCharTok{*} \DecValTok{1}\NormalTok{)}
\NormalTok{  )}
\end{Highlighting}
\end{Shaded}

Filter flights where overnight condition is true and arrival time is
before departure time

\begin{Shaded}
\begin{Highlighting}[]
\NormalTok{flights\_dt }\SpecialCharTok{\%\textgreater{}\%} 
  \FunctionTok{filter}\NormalTok{(overnight, arr\_time }\SpecialCharTok{\textless{}}\NormalTok{ dep\_time) }
\end{Highlighting}
\end{Shaded}

\begin{verbatim}
## # A tibble: 0 x 10
## # i 10 variables: origin <chr>, dest <chr>, dep_delay <dbl>, arr_delay <dbl>,
## #   dep_time <dttm>, sched_dep_time <dttm>, arr_time <dttm>,
## #   sched_arr_time <dttm>, air_time <dbl>, overnight <lgl>
\end{verbatim}

Intervals Calculate the ratio of one year in days

\begin{Shaded}
\begin{Highlighting}[]
\FunctionTok{years}\NormalTok{(}\DecValTok{1}\NormalTok{) }\SpecialCharTok{/} \FunctionTok{days}\NormalTok{(}\DecValTok{1}\NormalTok{)}
\end{Highlighting}
\end{Shaded}

\begin{verbatim}
## [1] 365.25
\end{verbatim}

\begin{Shaded}
\begin{Highlighting}[]
\NormalTok{next\_year }\OtherTok{\textless{}{-}} \FunctionTok{today}\NormalTok{() }\SpecialCharTok{+} \FunctionTok{years}\NormalTok{(}\DecValTok{1}\NormalTok{)}
\NormalTok{(}\FunctionTok{today}\NormalTok{() }\SpecialCharTok{\%{-}{-}\%}\NormalTok{ next\_year) }\SpecialCharTok{/} \FunctionTok{ddays}\NormalTok{(}\DecValTok{1}\NormalTok{)}
\end{Highlighting}
\end{Shaded}

\begin{verbatim}
## [1] 365
\end{verbatim}

\begin{Shaded}
\begin{Highlighting}[]
\NormalTok{(}\FunctionTok{today}\NormalTok{() }\SpecialCharTok{\%{-}{-}\%}\NormalTok{ next\_year) }\SpecialCharTok{\%/\%} \FunctionTok{days}\NormalTok{(}\DecValTok{1}\NormalTok{)}
\end{Highlighting}
\end{Shaded}

\begin{verbatim}
## [1] 365
\end{verbatim}

Display time zone information

\begin{Shaded}
\begin{Highlighting}[]
\FunctionTok{Sys.timezone}\NormalTok{()}
\end{Highlighting}
\end{Shaded}

\begin{verbatim}
## [1] "Asia/Katmandu"
\end{verbatim}

\begin{Shaded}
\begin{Highlighting}[]
\FunctionTok{length}\NormalTok{(}\FunctionTok{OlsonNames}\NormalTok{())}
\end{Highlighting}
\end{Shaded}

\begin{verbatim}
## [1] 596
\end{verbatim}

\begin{Shaded}
\begin{Highlighting}[]
\FunctionTok{head}\NormalTok{(}\FunctionTok{OlsonNames}\NormalTok{())}
\end{Highlighting}
\end{Shaded}

\begin{verbatim}
## [1] "Africa/Abidjan"     "Africa/Accra"       "Africa/Addis_Ababa"
## [4] "Africa/Algiers"     "Africa/Asmara"      "Africa/Asmera"
\end{verbatim}

\begin{Shaded}
\begin{Highlighting}[]
\NormalTok{(x1 }\OtherTok{\textless{}{-}} \FunctionTok{ymd\_hms}\NormalTok{(}\StringTok{"2015{-}06{-}01 12:00:00"}\NormalTok{, }\AttributeTok{tz =} \StringTok{"America/New\_York"}\NormalTok{))}
\end{Highlighting}
\end{Shaded}

\begin{verbatim}
## [1] "2015-06-01 12:00:00 EDT"
\end{verbatim}

\begin{Shaded}
\begin{Highlighting}[]
\NormalTok{(x2 }\OtherTok{\textless{}{-}} \FunctionTok{ymd\_hms}\NormalTok{(}\StringTok{"2015{-}06{-}01 18:00:00"}\NormalTok{, }\AttributeTok{tz =} \StringTok{"Europe/Copenhagen"}\NormalTok{))}
\end{Highlighting}
\end{Shaded}

\begin{verbatim}
## [1] "2015-06-01 18:00:00 CEST"
\end{verbatim}

\begin{Shaded}
\begin{Highlighting}[]
\NormalTok{(x3 }\OtherTok{\textless{}{-}} \FunctionTok{ymd\_hms}\NormalTok{(}\StringTok{"2015{-}06{-}02 04:00:00"}\NormalTok{, }\AttributeTok{tz =} \StringTok{"Pacific/Auckland"}\NormalTok{))}
\end{Highlighting}
\end{Shaded}

\begin{verbatim}
## [1] "2015-06-02 04:00:00 NZST"
\end{verbatim}

\begin{Shaded}
\begin{Highlighting}[]
\NormalTok{x1 }\SpecialCharTok{{-}}\NormalTok{ x2}
\end{Highlighting}
\end{Shaded}

\begin{verbatim}
## Time difference of 0 secs
\end{verbatim}

\begin{Shaded}
\begin{Highlighting}[]
\NormalTok{x1 }\SpecialCharTok{{-}}\NormalTok{ x3}
\end{Highlighting}
\end{Shaded}

\begin{verbatim}
## Time difference of 0 secs
\end{verbatim}

\hypertarget{pipes}{%
\section{Pipes}\label{pipes}}

Task: To import the required library

\begin{Shaded}
\begin{Highlighting}[]
\NormalTok{packages\_to\_install }\OtherTok{\textless{}{-}} \FunctionTok{c}\NormalTok{(}\StringTok{"tidyverse"}\NormalTok{, }\StringTok{"pryr"}\NormalTok{)}
\ControlFlowTok{for}\NormalTok{ (package\_name }\ControlFlowTok{in}\NormalTok{ packages\_to\_install) \{}
  \ControlFlowTok{if}\NormalTok{ (}\SpecialCharTok{!}\FunctionTok{requireNamespace}\NormalTok{(package\_name, }\AttributeTok{quietly =} \ConstantTok{TRUE}\NormalTok{)) \{}
    \FunctionTok{install.packages}\NormalTok{(package\_name)}
\NormalTok{  \}}
  \FunctionTok{library}\NormalTok{(package\_name, }\AttributeTok{character.only =} \ConstantTok{TRUE}\NormalTok{)}
\NormalTok{\}}
\end{Highlighting}
\end{Shaded}

\begin{verbatim}
## 
## Attaching package: 'pryr'
\end{verbatim}

\begin{verbatim}
## The following object is masked from 'package:dplyr':
## 
##     where
\end{verbatim}

\begin{verbatim}
## The following objects are masked from 'package:purrr':
## 
##     compose, partial
\end{verbatim}

\begin{Shaded}
\begin{Highlighting}[]
\FunctionTok{library}\NormalTok{(magrittr)}
\end{Highlighting}
\end{Shaded}

\begin{verbatim}
## 
## Attaching package: 'magrittr'
\end{verbatim}

\begin{verbatim}
## The following object is masked from 'package:purrr':
## 
##     set_names
\end{verbatim}

\begin{verbatim}
## The following object is masked from 'package:tidyr':
## 
##     extract
\end{verbatim}

Create diamond data and calculate the object sizes

\begin{Shaded}
\begin{Highlighting}[]
\NormalTok{diamonds }\OtherTok{\textless{}{-}}\NormalTok{ ggplot2}\SpecialCharTok{::}\NormalTok{diamonds}
\NormalTok{diamonds2 }\OtherTok{\textless{}{-}}\NormalTok{ diamonds }\SpecialCharTok{\%\textgreater{}\%} 
\NormalTok{  dplyr}\SpecialCharTok{::}\FunctionTok{mutate}\NormalTok{(}\AttributeTok{price\_per\_carat =}\NormalTok{ price }\SpecialCharTok{/}\NormalTok{ carat)}

\NormalTok{pryr}\SpecialCharTok{::}\FunctionTok{object\_size}\NormalTok{(diamonds)}
\end{Highlighting}
\end{Shaded}

\begin{verbatim}
## 3.46 MB
\end{verbatim}

\begin{Shaded}
\begin{Highlighting}[]
\NormalTok{pryr}\SpecialCharTok{::}\FunctionTok{object\_size}\NormalTok{(diamonds2)}
\end{Highlighting}
\end{Shaded}

\begin{verbatim}
## 3.89 MB
\end{verbatim}

\begin{Shaded}
\begin{Highlighting}[]
\NormalTok{pryr}\SpecialCharTok{::}\FunctionTok{object\_size}\NormalTok{(diamonds, diamonds2)}
\end{Highlighting}
\end{Shaded}

\begin{verbatim}
## 3.89 MB
\end{verbatim}

Functions Normalize the columns of a data frame

\begin{Shaded}
\begin{Highlighting}[]
\NormalTok{df }\OtherTok{\textless{}{-}}\NormalTok{ tibble}\SpecialCharTok{::}\FunctionTok{tibble}\NormalTok{(}
  \AttributeTok{a =} \FunctionTok{rnorm}\NormalTok{(}\DecValTok{10}\NormalTok{),}
  \AttributeTok{b =} \FunctionTok{rnorm}\NormalTok{(}\DecValTok{10}\NormalTok{),}
  \AttributeTok{c =} \FunctionTok{rnorm}\NormalTok{(}\DecValTok{10}\NormalTok{),}
  \AttributeTok{d =} \FunctionTok{rnorm}\NormalTok{(}\DecValTok{10}\NormalTok{)}
\NormalTok{)}

\NormalTok{df}\SpecialCharTok{$}\NormalTok{a }\OtherTok{\textless{}{-}}\NormalTok{ (df}\SpecialCharTok{$}\NormalTok{a }\SpecialCharTok{{-}} \FunctionTok{min}\NormalTok{(df}\SpecialCharTok{$}\NormalTok{a, }\AttributeTok{na.rm =} \ConstantTok{TRUE}\NormalTok{)) }\SpecialCharTok{/} 
\NormalTok{  (}\FunctionTok{max}\NormalTok{(df}\SpecialCharTok{$}\NormalTok{a, }\AttributeTok{na.rm =} \ConstantTok{TRUE}\NormalTok{) }\SpecialCharTok{{-}} \FunctionTok{min}\NormalTok{(df}\SpecialCharTok{$}\NormalTok{a, }\AttributeTok{na.rm =} \ConstantTok{TRUE}\NormalTok{))}
\NormalTok{df}\SpecialCharTok{$}\NormalTok{b }\OtherTok{\textless{}{-}}\NormalTok{ (df}\SpecialCharTok{$}\NormalTok{b }\SpecialCharTok{{-}} \FunctionTok{min}\NormalTok{(df}\SpecialCharTok{$}\NormalTok{b, }\AttributeTok{na.rm =} \ConstantTok{TRUE}\NormalTok{)) }\SpecialCharTok{/} 
\NormalTok{  (}\FunctionTok{max}\NormalTok{(df}\SpecialCharTok{$}\NormalTok{b, }\AttributeTok{na.rm =} \ConstantTok{TRUE}\NormalTok{) }\SpecialCharTok{{-}} \FunctionTok{min}\NormalTok{(df}\SpecialCharTok{$}\NormalTok{a, }\AttributeTok{na.rm =} \ConstantTok{TRUE}\NormalTok{))}
\NormalTok{df}\SpecialCharTok{$}\NormalTok{c }\OtherTok{\textless{}{-}}\NormalTok{ (df}\SpecialCharTok{$}\NormalTok{c }\SpecialCharTok{{-}} \FunctionTok{min}\NormalTok{(df}\SpecialCharTok{$}\NormalTok{c, }\AttributeTok{na.rm =} \ConstantTok{TRUE}\NormalTok{)) }\SpecialCharTok{/} 
\NormalTok{  (}\FunctionTok{max}\NormalTok{(df}\SpecialCharTok{$}\NormalTok{c, }\AttributeTok{na.rm =} \ConstantTok{TRUE}\NormalTok{) }\SpecialCharTok{{-}} \FunctionTok{min}\NormalTok{(df}\SpecialCharTok{$}\NormalTok{c, }\AttributeTok{na.rm =} \ConstantTok{TRUE}\NormalTok{))}
\NormalTok{df}\SpecialCharTok{$}\NormalTok{d }\OtherTok{\textless{}{-}}\NormalTok{ (df}\SpecialCharTok{$}\NormalTok{d }\SpecialCharTok{{-}} \FunctionTok{min}\NormalTok{(df}\SpecialCharTok{$}\NormalTok{d, }\AttributeTok{na.rm =} \ConstantTok{TRUE}\NormalTok{)) }\SpecialCharTok{/} 
\NormalTok{  (}\FunctionTok{max}\NormalTok{(df}\SpecialCharTok{$}\NormalTok{d, }\AttributeTok{na.rm =} \ConstantTok{TRUE}\NormalTok{) }\SpecialCharTok{{-}} \FunctionTok{min}\NormalTok{(df}\SpecialCharTok{$}\NormalTok{d, }\AttributeTok{na.rm =} \ConstantTok{TRUE}\NormalTok{))}
\end{Highlighting}
\end{Shaded}

Normalize a single column of a data frame

\begin{Shaded}
\begin{Highlighting}[]
\NormalTok{(df}\SpecialCharTok{$}\NormalTok{a }\SpecialCharTok{{-}} \FunctionTok{min}\NormalTok{(df}\SpecialCharTok{$}\NormalTok{a, }\AttributeTok{na.rm =} \ConstantTok{TRUE}\NormalTok{)) }\SpecialCharTok{/}
\NormalTok{  (}\FunctionTok{max}\NormalTok{(df}\SpecialCharTok{$}\NormalTok{a, }\AttributeTok{na.rm =} \ConstantTok{TRUE}\NormalTok{) }\SpecialCharTok{{-}} \FunctionTok{min}\NormalTok{(df}\SpecialCharTok{$}\NormalTok{a, }\AttributeTok{na.rm =} \ConstantTok{TRUE}\NormalTok{))}
\end{Highlighting}
\end{Shaded}

\begin{verbatim}
##  [1] 0.4210166 0.4488734 0.3229973 0.5835642 0.3246167 0.2816776 1.0000000
##  [8] 0.0000000 0.4056213 0.5478119
\end{verbatim}

\begin{Shaded}
\begin{Highlighting}[]
\NormalTok{x }\OtherTok{\textless{}{-}}\NormalTok{ df}\SpecialCharTok{$}\NormalTok{a}
\NormalTok{(x }\SpecialCharTok{{-}} \FunctionTok{min}\NormalTok{(x, }\AttributeTok{na.rm =} \ConstantTok{TRUE}\NormalTok{)) }\SpecialCharTok{/}\NormalTok{ (}\FunctionTok{max}\NormalTok{(x, }\AttributeTok{na.rm =} \ConstantTok{TRUE}\NormalTok{) }\SpecialCharTok{{-}} \FunctionTok{min}\NormalTok{(x, }\AttributeTok{na.rm =} \ConstantTok{TRUE}\NormalTok{))}
\end{Highlighting}
\end{Shaded}

\begin{verbatim}
##  [1] 0.4210166 0.4488734 0.3229973 0.5835642 0.3246167 0.2816776 1.0000000
##  [8] 0.0000000 0.4056213 0.5478119
\end{verbatim}

\begin{Shaded}
\begin{Highlighting}[]
\NormalTok{rng }\OtherTok{\textless{}{-}} \FunctionTok{range}\NormalTok{(x, }\AttributeTok{na.rm =} \ConstantTok{TRUE}\NormalTok{)}
\NormalTok{(x }\SpecialCharTok{{-}}\NormalTok{ rng[}\DecValTok{1}\NormalTok{]) }\SpecialCharTok{/}\NormalTok{ (rng[}\DecValTok{2}\NormalTok{] }\SpecialCharTok{{-}}\NormalTok{ rng[}\DecValTok{1}\NormalTok{])}
\end{Highlighting}
\end{Shaded}

\begin{verbatim}
##  [1] 0.4210166 0.4488734 0.3229973 0.5835642 0.3246167 0.2816776 1.0000000
##  [8] 0.0000000 0.4056213 0.5478119
\end{verbatim}

\begin{Shaded}
\begin{Highlighting}[]
\NormalTok{rescale01 }\OtherTok{\textless{}{-}} \ControlFlowTok{function}\NormalTok{(x) \{}
\NormalTok{  rng }\OtherTok{\textless{}{-}} \FunctionTok{range}\NormalTok{(x, }\AttributeTok{na.rm =} \ConstantTok{TRUE}\NormalTok{)}
\NormalTok{  (x }\SpecialCharTok{{-}}\NormalTok{ rng[}\DecValTok{1}\NormalTok{]) }\SpecialCharTok{/}\NormalTok{ (rng[}\DecValTok{2}\NormalTok{] }\SpecialCharTok{{-}}\NormalTok{ rng[}\DecValTok{1}\NormalTok{])}
\NormalTok{\}}
\FunctionTok{rescale01}\NormalTok{(}\FunctionTok{c}\NormalTok{(}\DecValTok{0}\NormalTok{, }\DecValTok{5}\NormalTok{, }\DecValTok{10}\NormalTok{))}
\end{Highlighting}
\end{Shaded}

\begin{verbatim}
## [1] 0.0 0.5 1.0
\end{verbatim}

Rescale a vector to the range {[}0, 1{]}

\begin{Shaded}
\begin{Highlighting}[]
\FunctionTok{rescale01}\NormalTok{(}\FunctionTok{c}\NormalTok{(}\SpecialCharTok{{-}}\DecValTok{10}\NormalTok{, }\DecValTok{0}\NormalTok{, }\DecValTok{10}\NormalTok{))}
\end{Highlighting}
\end{Shaded}

\begin{verbatim}
## [1] 0.0 0.5 1.0
\end{verbatim}

\begin{Shaded}
\begin{Highlighting}[]
\FunctionTok{rescale01}\NormalTok{(}\FunctionTok{c}\NormalTok{(}\DecValTok{1}\NormalTok{, }\DecValTok{2}\NormalTok{, }\DecValTok{3}\NormalTok{, }\ConstantTok{NA}\NormalTok{, }\DecValTok{5}\NormalTok{))}
\end{Highlighting}
\end{Shaded}

\begin{verbatim}
## [1] 0.00 0.25 0.50   NA 1.00
\end{verbatim}

Rescale each column of a DataFrame to the range {[}0, 1{]}

\begin{Shaded}
\begin{Highlighting}[]
\NormalTok{df}\SpecialCharTok{$}\NormalTok{a }\OtherTok{\textless{}{-}} \FunctionTok{rescale01}\NormalTok{(df}\SpecialCharTok{$}\NormalTok{a)}
\NormalTok{df}\SpecialCharTok{$}\NormalTok{b }\OtherTok{\textless{}{-}} \FunctionTok{rescale01}\NormalTok{(df}\SpecialCharTok{$}\NormalTok{b)}
\NormalTok{df}\SpecialCharTok{$}\NormalTok{c }\OtherTok{\textless{}{-}} \FunctionTok{rescale01}\NormalTok{(df}\SpecialCharTok{$}\NormalTok{c)}
\NormalTok{df}\SpecialCharTok{$}\NormalTok{d }\OtherTok{\textless{}{-}} \FunctionTok{rescale01}\NormalTok{(df}\SpecialCharTok{$}\NormalTok{d)}
\end{Highlighting}
\end{Shaded}

\begin{Shaded}
\begin{Highlighting}[]
\NormalTok{x }\OtherTok{\textless{}{-}} \FunctionTok{c}\NormalTok{(}\DecValTok{1}\SpecialCharTok{:}\DecValTok{10}\NormalTok{, }\ConstantTok{Inf}\NormalTok{)}
\FunctionTok{rescale01}\NormalTok{(x)}
\end{Highlighting}
\end{Shaded}

\begin{verbatim}
##  [1]   0   0   0   0   0   0   0   0   0   0 NaN
\end{verbatim}

Define the rescale01 function and apply it

\begin{Shaded}
\begin{Highlighting}[]
\NormalTok{rescale01 }\OtherTok{\textless{}{-}} \ControlFlowTok{function}\NormalTok{(x) \{}
\NormalTok{  rng }\OtherTok{\textless{}{-}} \FunctionTok{range}\NormalTok{(x, }\AttributeTok{na.rm =} \ConstantTok{TRUE}\NormalTok{, }\AttributeTok{finite =} \ConstantTok{TRUE}\NormalTok{)}
\NormalTok{  (x }\SpecialCharTok{{-}}\NormalTok{ rng[}\DecValTok{1}\NormalTok{]) }\SpecialCharTok{/}\NormalTok{ (rng[}\DecValTok{2}\NormalTok{] }\SpecialCharTok{{-}}\NormalTok{ rng[}\DecValTok{1}\NormalTok{])}
\NormalTok{\}}
\FunctionTok{rescale01}\NormalTok{(x)}
\end{Highlighting}
\end{Shaded}

\begin{verbatim}
##  [1] 0.0000000 0.1111111 0.2222222 0.3333333 0.4444444 0.5555556 0.6666667
##  [8] 0.7777778 0.8888889 1.0000000       Inf
\end{verbatim}

Load required libraries and packages

\begin{Shaded}
\begin{Highlighting}[]
\FunctionTok{library}\NormalTok{(tidyverse)}
\FunctionTok{library}\NormalTok{(purrr)}
\FunctionTok{library}\NormalTok{(magrittr)}

\CommentTok{\# install.packages("pryr")}
\FunctionTok{library}\NormalTok{(pryr)}
\end{Highlighting}
\end{Shaded}

\hypertarget{piping-alternatives}{%
\subsection{18.2 Piping alternatives}\label{piping-alternatives}}

This is a popular Children's poem that is accompanied by hand
actions.We'll start by defining an object to represent little bunny Foo
Foo:

\begin{Shaded}
\begin{Highlighting}[]
\CommentTok{\# foo\_foo \textless{}{-} little\_bunny()}
\end{Highlighting}
\end{Shaded}

\hypertarget{intermediate-steps}{%
\subsubsection{18.2.1 Intermediate steps}\label{intermediate-steps}}

The simplest approach is to save each step as a new object:

\begin{Shaded}
\begin{Highlighting}[]
\CommentTok{\# foo\_foo\_1 \textless{}{-} hop(foo\_foo,through=forest)}
\CommentTok{\# foo\_foo\_2 \textless{}{-} scoop(foo\_foo\_1, up = field\_mice)}
\CommentTok{\# foo\_foo\_3 \textless{}{-} bop(foo\_foo\_2, on = head)}
\end{Highlighting}
\end{Shaded}

Create diamonds dataset and calculate price per carat

\begin{Shaded}
\begin{Highlighting}[]
\NormalTok{diamonds }\OtherTok{\textless{}{-}}\NormalTok{ ggplot2}\SpecialCharTok{::}\NormalTok{diamonds}
\NormalTok{diamonds2 }\OtherTok{\textless{}{-}}\NormalTok{ diamonds }\SpecialCharTok{\%\textgreater{}\%} 
\NormalTok{  dplyr}\SpecialCharTok{::}\FunctionTok{mutate}\NormalTok{(}\AttributeTok{price\_per\_carat=}\NormalTok{price}\SpecialCharTok{/}\NormalTok{carat)}

\NormalTok{pryr}\SpecialCharTok{::}\FunctionTok{object\_size}\NormalTok{(diamonds)}
\end{Highlighting}
\end{Shaded}

\begin{verbatim}
## 3.46 MB
\end{verbatim}

\begin{Shaded}
\begin{Highlighting}[]
\NormalTok{pryr}\SpecialCharTok{::}\FunctionTok{object\_size}\NormalTok{(diamonds2)}
\end{Highlighting}
\end{Shaded}

\begin{verbatim}
## 3.89 MB
\end{verbatim}

\begin{Shaded}
\begin{Highlighting}[]
\NormalTok{pryr}\SpecialCharTok{::}\FunctionTok{object\_size}\NormalTok{(diamonds,diamonds2)}
\end{Highlighting}
\end{Shaded}

\begin{verbatim}
## 3.89 MB
\end{verbatim}

Introduce NA value into diamonds\$carat and check object sizes

\begin{Shaded}
\begin{Highlighting}[]
\NormalTok{diamonds}\SpecialCharTok{$}\NormalTok{carat[}\DecValTok{1}\NormalTok{] }\OtherTok{\textless{}{-}} \ConstantTok{NA}
\NormalTok{pryr}\SpecialCharTok{::}\FunctionTok{object\_size}\NormalTok{(diamonds)}
\end{Highlighting}
\end{Shaded}

\begin{verbatim}
## 3.46 MB
\end{verbatim}

\begin{Shaded}
\begin{Highlighting}[]
\NormalTok{pryr}\SpecialCharTok{::}\FunctionTok{object\_size}\NormalTok{(diamonds2)}
\end{Highlighting}
\end{Shaded}

\begin{verbatim}
## 3.89 MB
\end{verbatim}

\begin{Shaded}
\begin{Highlighting}[]
\NormalTok{pryr}\SpecialCharTok{::}\FunctionTok{object\_size}\NormalTok{(diamonds,diamonds2)}
\end{Highlighting}
\end{Shaded}

\begin{verbatim}
## 4.32 MB
\end{verbatim}

\hypertarget{overwrite-the-original}{%
\subsubsection{18.2.2 Overwrite the
original}\label{overwrite-the-original}}

Instead of creating intermediate objects at each step, we could
overwrite the original object:

\begin{Shaded}
\begin{Highlighting}[]
\CommentTok{\# foo\_foo \textless{}{-} hop(foo\_foo, through = forest)}
\CommentTok{\# foo\_foo \textless{}{-} scoop(foo\_foo, up = field\_mice)}
\CommentTok{\# foo\_foo \textless{}{-} bop(foo\_foo, on = head)}
\end{Highlighting}
\end{Shaded}

\hypertarget{function-composition}{%
\subsubsection{18.2.3 Function composition}\label{function-composition}}

Another approach is to abandon assignment and just string the function
calls together:

\begin{Shaded}
\begin{Highlighting}[]
\CommentTok{\# bop(}
\CommentTok{\#   scoop(}
\CommentTok{\#     hop(foo\_foo, through = forest),}
\CommentTok{\#     up = field\_mice}
\CommentTok{\#   ), }
\CommentTok{\#   on = head}
\CommentTok{\# )}
\end{Highlighting}
\end{Shaded}

Here the disadvantage is that you have to read from inside-out, from
right-to-left, and that the arguments end up spread far apart
(evocatively called the dagwood sandwhich problem). In short, this code
is hard for a human to consume.

\hypertarget{use-the-pipe}{%
\subsubsection{18.2.4 Use the pipe}\label{use-the-pipe}}

Finally, we can use the pipe:

\begin{Shaded}
\begin{Highlighting}[]
\CommentTok{\# foo\_foo \%\textgreater{}\%}
\CommentTok{\#   hop(through = forest) \%\textgreater{}\%}
\CommentTok{\#   scoop(up = field\_mice) \%\textgreater{}\%}
\CommentTok{\#   bop(on = head)}
\end{Highlighting}
\end{Shaded}

\begin{Shaded}
\begin{Highlighting}[]
\CommentTok{\# my\_pipe \textless{}{-} function(.) \{}
\CommentTok{\#   . \textless{}{-} hop(., through = forest)}
\CommentTok{\#   . \textless{}{-} scoop(., up = field\_mice)}
\CommentTok{\#   bop(., on = head)}
\CommentTok{\# \}}
\CommentTok{\# my\_pipe(foo\_foo)}
\end{Highlighting}
\end{Shaded}

TASK: Functions that use the current environment. For example,
\texttt{assign()} will create a new variable with the given name in the
current environment:

\begin{Shaded}
\begin{Highlighting}[]
\FunctionTok{assign}\NormalTok{(}\StringTok{"x"}\NormalTok{,}\DecValTok{10}\NormalTok{)}
\NormalTok{x}
\end{Highlighting}
\end{Shaded}

\begin{verbatim}
## [1] 10
\end{verbatim}

\begin{Shaded}
\begin{Highlighting}[]
\StringTok{"x"} \SpecialCharTok{\%\textgreater{}\%} \FunctionTok{assign}\NormalTok{(}\DecValTok{100}\NormalTok{)}
\NormalTok{x}
\end{Highlighting}
\end{Shaded}

\begin{verbatim}
## [1] 10
\end{verbatim}

Assign value to ``x'' in the specified environment and check its value
and Generate random numbers, create a matrix, plot it, and inspect its
structure

\begin{Shaded}
\begin{Highlighting}[]
\NormalTok{env }\OtherTok{\textless{}{-}} \FunctionTok{environment}\NormalTok{()}
\StringTok{"x"} \SpecialCharTok{\%\textgreater{}\%} \FunctionTok{assign}\NormalTok{(}\DecValTok{100}\NormalTok{,}\AttributeTok{envir=}\NormalTok{env)}
\NormalTok{x}
\end{Highlighting}
\end{Shaded}

\begin{verbatim}
## [1] 100
\end{verbatim}

\begin{Shaded}
\begin{Highlighting}[]
\FunctionTok{rnorm}\NormalTok{(}\DecValTok{100}\NormalTok{) }\SpecialCharTok{\%\textgreater{}\%} 
  \FunctionTok{matrix}\NormalTok{(}\AttributeTok{ncol=}\DecValTok{2}\NormalTok{) }\SpecialCharTok{\%\textgreater{}\%} 
  \FunctionTok{plot}\NormalTok{() }\SpecialCharTok{\%\textgreater{}\%} 
  \FunctionTok{str}\NormalTok{()}
\end{Highlighting}
\end{Shaded}

\includegraphics{Intoduction-to-R_files/figure-latex/18.4-1-1.pdf}

\begin{verbatim}
##  NULL
\end{verbatim}

\begin{Shaded}
\begin{Highlighting}[]
\FunctionTok{rnorm}\NormalTok{(}\DecValTok{100}\NormalTok{) }\SpecialCharTok{\%\textgreater{}\%} 
  \FunctionTok{matrix}\NormalTok{(}\AttributeTok{ncol=}\DecValTok{2}\NormalTok{) }\SpecialCharTok{\%\textgreater{}\%} 
  \FunctionTok{plot}\NormalTok{() }\SpecialCharTok{\%\textgreater{}\%} 
  \FunctionTok{str}\NormalTok{()}
\end{Highlighting}
\end{Shaded}

\includegraphics{Intoduction-to-R_files/figure-latex/18.4-1-2.pdf}

\begin{verbatim}
##  NULL
\end{verbatim}

\begin{Shaded}
\begin{Highlighting}[]
\NormalTok{ndist }\OtherTok{\textless{}{-}} \FunctionTok{rnorm}\NormalTok{(}\DecValTok{100000}\NormalTok{)}
\FunctionTok{hist}\NormalTok{(ndist)}
\end{Highlighting}
\end{Shaded}

\includegraphics{Intoduction-to-R_files/figure-latex/18.4-1-3.pdf}

Calculate the correlation between two variables in mtcars dataset

\begin{Shaded}
\begin{Highlighting}[]
\NormalTok{mtcars }\SpecialCharTok{\%$\%}
  \FunctionTok{cor}\NormalTok{(disp, mpg)}
\end{Highlighting}
\end{Shaded}

\begin{verbatim}
## [1] -0.8475514
\end{verbatim}

\begin{itemize}
\tightlist
\item
  For assignment magrittr provides the
  \texttt{\%\textless{}\textgreater{}\%} operator which allows you to
  replace code like:
\end{itemize}

\begin{Shaded}
\begin{Highlighting}[]
\NormalTok{mtcars }\OtherTok{\textless{}{-}}\NormalTok{ mtcars }\SpecialCharTok{\%\textgreater{}\%} 
  \FunctionTok{transform}\NormalTok{(}\AttributeTok{cyl=}\NormalTok{cyl}\SpecialCharTok{*}\DecValTok{2}\NormalTok{)}
\end{Highlighting}
\end{Shaded}

\begin{Shaded}
\begin{Highlighting}[]
\NormalTok{mtcars }\SpecialCharTok{\%\textless{}\textgreater{}\%} \FunctionTok{transform}\NormalTok{(}\AttributeTok{cyl=}\NormalTok{cyl}\SpecialCharTok{*}\DecValTok{2}\NormalTok{)}
\end{Highlighting}
\end{Shaded}

\hypertarget{chapter-19-functions}{%
\section{Chapter 19 Functions}\label{chapter-19-functions}}

\hypertarget{introduction}{%
\subsection{19.1 Introduction}\label{introduction}}

\hypertarget{when-should-you-write-a-function}{%
\subsection{19.2 When should you write a
function?}\label{when-should-you-write-a-function}}

\begin{Shaded}
\begin{Highlighting}[]
\NormalTok{df }\OtherTok{\textless{}{-}}\NormalTok{ tibble}\SpecialCharTok{::}\FunctionTok{tibble}\NormalTok{(}
  \AttributeTok{a =} \FunctionTok{rnorm}\NormalTok{(}\DecValTok{10}\NormalTok{),}
  \AttributeTok{b =} \FunctionTok{rnorm}\NormalTok{(}\DecValTok{10}\NormalTok{),}
  \AttributeTok{c =} \FunctionTok{rnorm}\NormalTok{(}\DecValTok{10}\NormalTok{),}
  \AttributeTok{d =} \FunctionTok{rnorm}\NormalTok{(}\DecValTok{10}\NormalTok{)}
\NormalTok{)}
\NormalTok{df}
\end{Highlighting}
\end{Shaded}

\begin{verbatim}
## # A tibble: 10 x 4
##         a        b       c      d
##     <dbl>    <dbl>   <dbl>  <dbl>
##  1  0.464 -0.134    0.794  -1.19 
##  2 -0.683 -0.348   -1.41   -0.350
##  3  0.945 -0.923    3.46    0.217
##  4  0.682  0.687    0.0654 -0.903
##  5 -0.936  0.413    0.0554  1.01 
##  6  1.36   1.73    -0.842  -1.70 
##  7  0.818 -0.405    1.26   -0.296
##  8 -0.369  1.29    -0.880  -0.260
##  9  0.894 -1.02    -0.303   0.492
## 10 -2.23  -0.00704 -1.20   -1.14
\end{verbatim}

\begin{Shaded}
\begin{Highlighting}[]
\NormalTok{df}\SpecialCharTok{$}\NormalTok{a }\OtherTok{\textless{}{-}}\NormalTok{ (df}\SpecialCharTok{$}\NormalTok{a }\SpecialCharTok{{-}} \FunctionTok{min}\NormalTok{(df}\SpecialCharTok{$}\NormalTok{a, }\AttributeTok{na.rm =} \ConstantTok{TRUE}\NormalTok{)) }\SpecialCharTok{/} 
\NormalTok{  (}\FunctionTok{max}\NormalTok{(df}\SpecialCharTok{$}\NormalTok{a, }\AttributeTok{na.rm =} \ConstantTok{TRUE}\NormalTok{) }\SpecialCharTok{{-}} \FunctionTok{min}\NormalTok{(df}\SpecialCharTok{$}\NormalTok{a, }\AttributeTok{na.rm =} \ConstantTok{TRUE}\NormalTok{))}
\NormalTok{df}\SpecialCharTok{$}\NormalTok{b }\OtherTok{\textless{}{-}}\NormalTok{ (df}\SpecialCharTok{$}\NormalTok{b }\SpecialCharTok{{-}} \FunctionTok{min}\NormalTok{(df}\SpecialCharTok{$}\NormalTok{b, }\AttributeTok{na.rm =} \ConstantTok{TRUE}\NormalTok{)) }\SpecialCharTok{/} 
\NormalTok{  (}\FunctionTok{max}\NormalTok{(df}\SpecialCharTok{$}\NormalTok{b, }\AttributeTok{na.rm =} \ConstantTok{TRUE}\NormalTok{) }\SpecialCharTok{{-}} \FunctionTok{min}\NormalTok{(df}\SpecialCharTok{$}\NormalTok{a, }\AttributeTok{na.rm =} \ConstantTok{TRUE}\NormalTok{))}
\NormalTok{df}\SpecialCharTok{$}\NormalTok{c }\OtherTok{\textless{}{-}}\NormalTok{ (df}\SpecialCharTok{$}\NormalTok{c }\SpecialCharTok{{-}} \FunctionTok{min}\NormalTok{(df}\SpecialCharTok{$}\NormalTok{c, }\AttributeTok{na.rm =} \ConstantTok{TRUE}\NormalTok{)) }\SpecialCharTok{/} 
\NormalTok{  (}\FunctionTok{max}\NormalTok{(df}\SpecialCharTok{$}\NormalTok{c, }\AttributeTok{na.rm =} \ConstantTok{TRUE}\NormalTok{) }\SpecialCharTok{{-}} \FunctionTok{min}\NormalTok{(df}\SpecialCharTok{$}\NormalTok{c, }\AttributeTok{na.rm =} \ConstantTok{TRUE}\NormalTok{))}
\NormalTok{df}\SpecialCharTok{$}\NormalTok{d }\OtherTok{\textless{}{-}}\NormalTok{ (df}\SpecialCharTok{$}\NormalTok{d }\SpecialCharTok{{-}} \FunctionTok{min}\NormalTok{(df}\SpecialCharTok{$}\NormalTok{d, }\AttributeTok{na.rm =} \ConstantTok{TRUE}\NormalTok{)) }\SpecialCharTok{/} 
\NormalTok{  (}\FunctionTok{max}\NormalTok{(df}\SpecialCharTok{$}\NormalTok{d, }\AttributeTok{na.rm =} \ConstantTok{TRUE}\NormalTok{) }\SpecialCharTok{{-}} \FunctionTok{min}\NormalTok{(df}\SpecialCharTok{$}\NormalTok{d, }\AttributeTok{na.rm =} \ConstantTok{TRUE}\NormalTok{))}
\end{Highlighting}
\end{Shaded}

Rescale a single variable in a data frame

\begin{Shaded}
\begin{Highlighting}[]
\NormalTok{(df}\SpecialCharTok{$}\NormalTok{a }\SpecialCharTok{{-}} \FunctionTok{min}\NormalTok{(df}\SpecialCharTok{$}\NormalTok{a, }\AttributeTok{na.rm =} \ConstantTok{TRUE}\NormalTok{)) }\SpecialCharTok{/}
\NormalTok{  (}\FunctionTok{max}\NormalTok{(df}\SpecialCharTok{$}\NormalTok{a, }\AttributeTok{na.rm =} \ConstantTok{TRUE}\NormalTok{) }\SpecialCharTok{{-}} \FunctionTok{min}\NormalTok{(df}\SpecialCharTok{$}\NormalTok{a, }\AttributeTok{na.rm =} \ConstantTok{TRUE}\NormalTok{))}
\end{Highlighting}
\end{Shaded}

\begin{verbatim}
##  [1] 0.7495722 0.4304521 0.8832264 0.8100031 0.3601140 1.0000000 0.8479950
##  [8] 0.5177630 0.8690661 0.0000000
\end{verbatim}

Rescale a single variable without creating a new object

\begin{Shaded}
\begin{Highlighting}[]
\NormalTok{x }\OtherTok{\textless{}{-}}\NormalTok{ df}\SpecialCharTok{$}\NormalTok{a}
\NormalTok{(x }\SpecialCharTok{{-}} \FunctionTok{min}\NormalTok{(x, }\AttributeTok{na.rm =}\NormalTok{ T)) }\SpecialCharTok{/}\NormalTok{ (}\FunctionTok{max}\NormalTok{(x, }\AttributeTok{na.rm =}\NormalTok{ T)}\SpecialCharTok{{-}}\FunctionTok{min}\NormalTok{(x, }\AttributeTok{na.rm =}\NormalTok{ T))}
\end{Highlighting}
\end{Shaded}

\begin{verbatim}
##  [1] 0.7495722 0.4304521 0.8832264 0.8100031 0.3601140 1.0000000 0.8479950
##  [8] 0.5177630 0.8690661 0.0000000
\end{verbatim}

Task: There is some duplication in this code. We're computing the range
of the data three times, so it makes sense to do it in one step:

\begin{Shaded}
\begin{Highlighting}[]
\NormalTok{rng }\OtherTok{\textless{}{-}} \FunctionTok{range}\NormalTok{(x, }\AttributeTok{na.rm =}\NormalTok{ T)}
\NormalTok{(x}\SpecialCharTok{{-}}\NormalTok{rng[}\DecValTok{1}\NormalTok{])}\SpecialCharTok{/}\NormalTok{(rng[}\DecValTok{2}\NormalTok{]}\SpecialCharTok{{-}}\NormalTok{rng[}\DecValTok{1}\NormalTok{])}
\end{Highlighting}
\end{Shaded}

\begin{verbatim}
##  [1] 0.7495722 0.4304521 0.8832264 0.8100031 0.3601140 1.0000000 0.8479950
##  [8] 0.5177630 0.8690661 0.0000000
\end{verbatim}

Pulling out intermediate calculations into named variables is a good
practice because it makes it more clear what the code is doing. Now that
I've simplified the code, and checked that it still works, I can turn it
into a function:

\begin{Shaded}
\begin{Highlighting}[]
\NormalTok{rescale01 }\OtherTok{\textless{}{-}} \ControlFlowTok{function}\NormalTok{(x)\{}
\NormalTok{  rng }\OtherTok{\textless{}{-}} \FunctionTok{range}\NormalTok{(x, }\AttributeTok{na.rm =}\NormalTok{ T)}
\NormalTok{  (x}\SpecialCharTok{{-}}\NormalTok{rng[}\DecValTok{1}\NormalTok{])}\SpecialCharTok{/}\NormalTok{(rng[}\DecValTok{2}\NormalTok{]}\SpecialCharTok{{-}}\NormalTok{rng[}\DecValTok{1}\NormalTok{])}
\NormalTok{\}}
\FunctionTok{rescale01}\NormalTok{(}\FunctionTok{c}\NormalTok{(}\DecValTok{0}\NormalTok{,}\DecValTok{5}\NormalTok{,}\DecValTok{10}\NormalTok{))}
\end{Highlighting}
\end{Shaded}

\begin{verbatim}
## [1] 0.0 0.5 1.0
\end{verbatim}

Test the rescale01 function with various inputs

\begin{Shaded}
\begin{Highlighting}[]
\FunctionTok{rescale01}\NormalTok{(}\FunctionTok{c}\NormalTok{(}\SpecialCharTok{{-}}\DecValTok{10}\NormalTok{,}\DecValTok{0}\NormalTok{,}\DecValTok{10}\NormalTok{))}
\end{Highlighting}
\end{Shaded}

\begin{verbatim}
## [1] 0.0 0.5 1.0
\end{verbatim}

\begin{Shaded}
\begin{Highlighting}[]
\FunctionTok{rescale01}\NormalTok{(}\FunctionTok{c}\NormalTok{(}\DecValTok{1}\NormalTok{,}\DecValTok{2}\NormalTok{,}\DecValTok{3}\NormalTok{,}\ConstantTok{NA}\NormalTok{,}\DecValTok{5}\NormalTok{))}
\end{Highlighting}
\end{Shaded}

\begin{verbatim}
## [1] 0.00 0.25 0.50   NA 1.00
\end{verbatim}

We can simplify the original example now that we have a function:

\begin{Shaded}
\begin{Highlighting}[]
\NormalTok{df}\SpecialCharTok{$}\NormalTok{a }\OtherTok{\textless{}{-}} \FunctionTok{rescale01}\NormalTok{(df}\SpecialCharTok{$}\NormalTok{a)}
\NormalTok{df}\SpecialCharTok{$}\NormalTok{b }\OtherTok{\textless{}{-}} \FunctionTok{rescale01}\NormalTok{(df}\SpecialCharTok{$}\NormalTok{b)}
\NormalTok{df}\SpecialCharTok{$}\NormalTok{c }\OtherTok{\textless{}{-}} \FunctionTok{rescale01}\NormalTok{(df}\SpecialCharTok{$}\NormalTok{c)}
\NormalTok{df}\SpecialCharTok{$}\NormalTok{d }\OtherTok{\textless{}{-}} \FunctionTok{rescale01}\NormalTok{(df}\SpecialCharTok{$}\NormalTok{d)}
\end{Highlighting}
\end{Shaded}

Rescale a vector with infinite values

\begin{Shaded}
\begin{Highlighting}[]
\NormalTok{x }\OtherTok{\textless{}{-}} \FunctionTok{c}\NormalTok{(}\DecValTok{1}\SpecialCharTok{:}\DecValTok{10}\NormalTok{,}\ConstantTok{Inf}\NormalTok{)}
\FunctionTok{rescale01}\NormalTok{(x)}
\end{Highlighting}
\end{Shaded}

\begin{verbatim}
##  [1]   0   0   0   0   0   0   0   0   0   0 NaN
\end{verbatim}

Because we've extracted the code into a function, we only need to make
the fix in one place:

\begin{Shaded}
\begin{Highlighting}[]
\NormalTok{rescale01 }\OtherTok{\textless{}{-}} \ControlFlowTok{function}\NormalTok{(x)\{}
\NormalTok{  rng }\OtherTok{\textless{}{-}} \FunctionTok{range}\NormalTok{(x,}\AttributeTok{na.rm=}\NormalTok{T,}\AttributeTok{finite=}\NormalTok{T)}
\NormalTok{  (x}\SpecialCharTok{{-}}\NormalTok{rng[}\DecValTok{1}\NormalTok{])}\SpecialCharTok{/}\NormalTok{(rng[}\DecValTok{2}\NormalTok{]}\SpecialCharTok{{-}}\NormalTok{rng[}\DecValTok{1}\NormalTok{])}
\NormalTok{\}}
\FunctionTok{rescale01}\NormalTok{(x)}
\end{Highlighting}
\end{Shaded}

\begin{verbatim}
##  [1] 0.0000000 0.1111111 0.2222222 0.3333333 0.4444444 0.5555556 0.6666667
##  [8] 0.7777778 0.8888889 1.0000000       Inf
\end{verbatim}

\hypertarget{conditional-execution}{%
\subsection{19.4 Conditional execution}\label{conditional-execution}}

An \texttt{if} statement allows you to conditionally execute code. It
looks like this:

\begin{Shaded}
\begin{Highlighting}[]
\CommentTok{\# if (condition) \{}
  \CommentTok{\# code executed when condition is TRUE}
\CommentTok{\# \} else \{}
  \CommentTok{\# code executed when condition is FALSE}
\CommentTok{\# \}}
\end{Highlighting}
\end{Shaded}

Define a function to check if an object has names

\begin{Shaded}
\begin{Highlighting}[]
\NormalTok{has\_name }\OtherTok{\textless{}{-}} \ControlFlowTok{function}\NormalTok{(x)\{}
\NormalTok{  nms }\OtherTok{\textless{}{-}} \FunctionTok{names}\NormalTok{(x)}
  \ControlFlowTok{if}\NormalTok{(}\FunctionTok{is.null}\NormalTok{(nms))\{}
    \FunctionTok{rep}\NormalTok{(}\ConstantTok{FALSE}\NormalTok{,}\FunctionTok{length}\NormalTok{(x))}
\NormalTok{  \}}\ControlFlowTok{else}\NormalTok{ \{}
    \SpecialCharTok{!}\FunctionTok{is.na}\NormalTok{(nms) }\SpecialCharTok{\&}\NormalTok{ nms }\SpecialCharTok{!=}\StringTok{""}
\NormalTok{  \}}
\NormalTok{\}}
\end{Highlighting}
\end{Shaded}

\hypertarget{conditions}{%
\subsubsection{19.4.1 Conditions}\label{conditions}}

how if condition works with warnings

\begin{Shaded}
\begin{Highlighting}[]
\CommentTok{\# if (c(TRUE,FALSE))\{\}}
\CommentTok{\#\textgreater{} Warning in if (c(TRUE, FALSE)) \{: the condition has length \textgreater{} 1 and only the}
\CommentTok{\#\textgreater{} first element will be used}
\CommentTok{\#\textgreater{} NULL}

\CommentTok{\# if (NA) \{\}}
\end{Highlighting}
\end{Shaded}

Check if two objects are identical

\begin{Shaded}
\begin{Highlighting}[]
\FunctionTok{identical}\NormalTok{(0L,}\DecValTok{0}\NormalTok{)}
\end{Highlighting}
\end{Shaded}

\begin{verbatim}
## [1] FALSE
\end{verbatim}

\begin{Shaded}
\begin{Highlighting}[]
\NormalTok{x }\OtherTok{\textless{}{-}} \FunctionTok{sqrt}\NormalTok{(}\DecValTok{2}\NormalTok{)}\SpecialCharTok{\^{}}\DecValTok{2}
\NormalTok{x}\SpecialCharTok{==}\DecValTok{2}
\end{Highlighting}
\end{Shaded}

\begin{verbatim}
## [1] FALSE
\end{verbatim}

\begin{Shaded}
\begin{Highlighting}[]
\NormalTok{x}\DecValTok{{-}2}
\end{Highlighting}
\end{Shaded}

\begin{verbatim}
## [1] 4.440892e-16
\end{verbatim}

\hypertarget{multiple-conditions}{%
\subsubsection{19.4.2 Multiple conditions}\label{multiple-conditions}}

You can chain multiple if statement together:

\begin{Shaded}
\begin{Highlighting}[]
\CommentTok{\# if (this) \{}
\CommentTok{\#   \# do that}
\CommentTok{\# \} else if (that) \{}
\CommentTok{\#   \# do something else}
\CommentTok{\# \} else \{}
\CommentTok{\#   \# }
\CommentTok{\# \}}
\end{Highlighting}
\end{Shaded}

\begin{Shaded}
\begin{Highlighting}[]
\CommentTok{\#\textgreater{} function(x, y, op) \{}
\CommentTok{\#\textgreater{}   switch(op,}
\CommentTok{\#\textgreater{}     plus = x + y,}
\CommentTok{\#\textgreater{}     minus = x {-} y,}
\CommentTok{\#\textgreater{}     times = x * y,}
\CommentTok{\#\textgreater{}     divide = x / y,}
\CommentTok{\#\textgreater{}     stop("Unknown op!")}
\CommentTok{\#\textgreater{}   )}
\CommentTok{\#\textgreater{} \}}
\end{Highlighting}
\end{Shaded}

\hypertarget{code-style}{%
\subsubsection{19.4.3 Code style}\label{code-style}}

Good practice for writing if statements

\begin{Shaded}
\begin{Highlighting}[]
\CommentTok{\# Good}
\CommentTok{\# if (y \textless{} 0 \&\& debug) \{}
\CommentTok{\#   message("Y is negative")}
\CommentTok{\# \}}
\CommentTok{\# }
\CommentTok{\# if (y == 0) \{}
\CommentTok{\#   log(x)}
\CommentTok{\# \} else \{}
\CommentTok{\#   y \^{} x}
\CommentTok{\# \}}
\CommentTok{\# }
\CommentTok{\# \# Bad}
\CommentTok{\# if (y \textless{} 0 \&\& debug)}
\CommentTok{\# message("Y is negative")}
\CommentTok{\# }
\CommentTok{\# if (y == 0) \{}
\CommentTok{\#   log(x)}
\CommentTok{\# \} }
\CommentTok{\# else \{}
\CommentTok{\#   y \^{} x}
\CommentTok{\# \}}
\end{Highlighting}
\end{Shaded}

It's ok to drop the curly braces if you have a very short if statement
that can fit on one line:

\begin{Shaded}
\begin{Highlighting}[]
\NormalTok{y }\OtherTok{\textless{}{-}} \DecValTok{10}
\NormalTok{x }\OtherTok{\textless{}{-}} \ControlFlowTok{if}\NormalTok{ (y }\SpecialCharTok{\textless{}} \DecValTok{20}\NormalTok{) }\StringTok{"Too low"} \ControlFlowTok{else} \StringTok{"Too high"}
\end{Highlighting}
\end{Shaded}

I recommend this only for very brief \texttt{if} statements. Otherwise,
the full form is easier to read:

\begin{Shaded}
\begin{Highlighting}[]
\ControlFlowTok{if}\NormalTok{ (y }\SpecialCharTok{\textless{}} \DecValTok{20}\NormalTok{) \{}
\NormalTok{  x }\OtherTok{\textless{}{-}} \StringTok{"Too low"} 
\NormalTok{\} }\ControlFlowTok{else}\NormalTok{ \{}
\NormalTok{  x }\OtherTok{\textless{}{-}} \StringTok{"Too high"}
\NormalTok{\}}
\end{Highlighting}
\end{Shaded}

\hypertarget{function-arguments}{%
\subsection{19.5 Function arguments}\label{function-arguments}}

\begin{Shaded}
\begin{Highlighting}[]
\CommentTok{\# Compute confidence interval around mean using normal approximation}
\NormalTok{mean\_ci }\OtherTok{\textless{}{-}} \ControlFlowTok{function}\NormalTok{(x, }\AttributeTok{conf =} \FloatTok{0.95}\NormalTok{) \{}
\NormalTok{  se }\OtherTok{\textless{}{-}} \FunctionTok{sd}\NormalTok{(x) }\SpecialCharTok{/} \FunctionTok{sqrt}\NormalTok{(}\FunctionTok{length}\NormalTok{(x))}
\NormalTok{  alpha }\OtherTok{\textless{}{-}} \DecValTok{1} \SpecialCharTok{{-}}\NormalTok{ conf}
  \FunctionTok{mean}\NormalTok{(x) }\SpecialCharTok{+}\NormalTok{ se }\SpecialCharTok{*} \FunctionTok{qnorm}\NormalTok{(}\FunctionTok{c}\NormalTok{(alpha }\SpecialCharTok{/} \DecValTok{2}\NormalTok{, }\DecValTok{1} \SpecialCharTok{{-}}\NormalTok{ alpha }\SpecialCharTok{/} \DecValTok{2}\NormalTok{))}
\NormalTok{\}}

\NormalTok{x }\OtherTok{\textless{}{-}} \FunctionTok{runif}\NormalTok{(}\DecValTok{100}\NormalTok{)}
\FunctionTok{mean\_ci}\NormalTok{(x)}
\end{Highlighting}
\end{Shaded}

\begin{verbatim}
## [1] 0.3822472 0.5010310
\end{verbatim}

\begin{Shaded}
\begin{Highlighting}[]
\FunctionTok{mean\_ci}\NormalTok{(x, }\AttributeTok{conf =} \FloatTok{0.99}\NormalTok{)}
\end{Highlighting}
\end{Shaded}

\begin{verbatim}
## [1] 0.3635849 0.5196933
\end{verbatim}

\hypertarget{choosing-names}{%
\subsubsection{19.5.1 Choosing names}\label{choosing-names}}

\hypertarget{cheking-values}{%
\subsubsection{19.5.2 Cheking values}\label{cheking-values}}

\begin{Shaded}
\begin{Highlighting}[]
\NormalTok{wt\_mean }\OtherTok{\textless{}{-}} \ControlFlowTok{function}\NormalTok{(x, w) \{}
  \FunctionTok{sum}\NormalTok{(x }\SpecialCharTok{*}\NormalTok{ w) }\SpecialCharTok{/} \FunctionTok{sum}\NormalTok{(w)}
\NormalTok{\}}
\NormalTok{wt\_var }\OtherTok{\textless{}{-}} \ControlFlowTok{function}\NormalTok{(x, w) \{}
\NormalTok{  mu }\OtherTok{\textless{}{-}} \FunctionTok{wt\_mean}\NormalTok{(x, w)}
  \FunctionTok{sum}\NormalTok{(w }\SpecialCharTok{*}\NormalTok{ (x }\SpecialCharTok{{-}}\NormalTok{ mu) }\SpecialCharTok{\^{}} \DecValTok{2}\NormalTok{) }\SpecialCharTok{/} \FunctionTok{sum}\NormalTok{(w)}
\NormalTok{\}}
\NormalTok{wt\_sd }\OtherTok{\textless{}{-}} \ControlFlowTok{function}\NormalTok{(x, w) \{}
  \FunctionTok{sqrt}\NormalTok{(}\FunctionTok{wt\_var}\NormalTok{(x, w))}
\NormalTok{\}}
\end{Highlighting}
\end{Shaded}

What happens if x and w are not the same length?

\begin{Shaded}
\begin{Highlighting}[]
\FunctionTok{wt\_mean}\NormalTok{(}\DecValTok{1}\SpecialCharTok{:}\DecValTok{6}\NormalTok{, }\DecValTok{1}\SpecialCharTok{:}\DecValTok{3}\NormalTok{)}
\end{Highlighting}
\end{Shaded}

\begin{verbatim}
## [1] 7.666667
\end{verbatim}

In this case, because of R's vector recycling rules, we don't get an
error.

It's good practice to check important preconditions, and throw an error
(with \texttt{stop()}), if they are not true:

\begin{Shaded}
\begin{Highlighting}[]
\NormalTok{wt\_mean }\OtherTok{\textless{}{-}} \ControlFlowTok{function}\NormalTok{(x, w) \{}
  \ControlFlowTok{if}\NormalTok{ (}\FunctionTok{length}\NormalTok{(x) }\SpecialCharTok{!=} \FunctionTok{length}\NormalTok{(w)) \{}
    \FunctionTok{stop}\NormalTok{(}\StringTok{"\textasciigrave{}x\textasciigrave{} and \textasciigrave{}w\textasciigrave{} must be the same length"}\NormalTok{, }\AttributeTok{call. =} \ConstantTok{FALSE}\NormalTok{)}
\NormalTok{  \}}
  \FunctionTok{sum}\NormalTok{(w }\SpecialCharTok{*}\NormalTok{ x) }\SpecialCharTok{/} \FunctionTok{sum}\NormalTok{(w)}
\NormalTok{\}}
\end{Highlighting}
\end{Shaded}

\begin{Shaded}
\begin{Highlighting}[]
\NormalTok{wt\_mean }\OtherTok{\textless{}{-}} \ControlFlowTok{function}\NormalTok{(x, w, }\AttributeTok{na.rm =} \ConstantTok{FALSE}\NormalTok{) \{}
  \ControlFlowTok{if}\NormalTok{ (}\SpecialCharTok{!}\FunctionTok{is.logical}\NormalTok{(na.rm)) \{}
    \FunctionTok{stop}\NormalTok{(}\StringTok{"\textasciigrave{}na.rm\textasciigrave{} must be logical"}\NormalTok{)}
\NormalTok{  \}}
  \ControlFlowTok{if}\NormalTok{ (}\FunctionTok{length}\NormalTok{(na.rm) }\SpecialCharTok{!=} \DecValTok{1}\NormalTok{) \{}
    \FunctionTok{stop}\NormalTok{(}\StringTok{"\textasciigrave{}na.rm\textasciigrave{} must be length 1"}\NormalTok{)}
\NormalTok{  \}}
  \ControlFlowTok{if}\NormalTok{ (}\FunctionTok{length}\NormalTok{(x) }\SpecialCharTok{!=} \FunctionTok{length}\NormalTok{(w)) \{}
    \FunctionTok{stop}\NormalTok{(}\StringTok{"\textasciigrave{}x\textasciigrave{} and \textasciigrave{}w\textasciigrave{} must be the same length"}\NormalTok{, }\AttributeTok{call. =} \ConstantTok{FALSE}\NormalTok{)}
\NormalTok{  \}}
  
  \ControlFlowTok{if}\NormalTok{ (na.rm) \{}
\NormalTok{    miss }\OtherTok{\textless{}{-}} \FunctionTok{is.na}\NormalTok{(x) }\SpecialCharTok{|} \FunctionTok{is.na}\NormalTok{(w)}
\NormalTok{    x }\OtherTok{\textless{}{-}}\NormalTok{ x[}\SpecialCharTok{!}\NormalTok{miss]}
\NormalTok{    w }\OtherTok{\textless{}{-}}\NormalTok{ w[}\SpecialCharTok{!}\NormalTok{miss]}
\NormalTok{  \}}
  \FunctionTok{sum}\NormalTok{(w }\SpecialCharTok{*}\NormalTok{ x) }\SpecialCharTok{/} \FunctionTok{sum}\NormalTok{(w)}
\NormalTok{\}}
\end{Highlighting}
\end{Shaded}

This is a lot of extra work for little additional gain. A useful
compromise is the built-in \texttt{stopifnot()}: it checks that each
argument is \texttt{TRUE}, and produces a generic error message if not.

\begin{Shaded}
\begin{Highlighting}[]
\NormalTok{wt\_mean }\OtherTok{\textless{}{-}} \ControlFlowTok{function}\NormalTok{(x, w, }\AttributeTok{na.rm =} \ConstantTok{FALSE}\NormalTok{) \{}
  \FunctionTok{stopifnot}\NormalTok{(}\FunctionTok{is.logical}\NormalTok{(na.rm), }\FunctionTok{length}\NormalTok{(na.rm) }\SpecialCharTok{==} \DecValTok{1}\NormalTok{)}
  \FunctionTok{stopifnot}\NormalTok{(}\FunctionTok{length}\NormalTok{(x) }\SpecialCharTok{==} \FunctionTok{length}\NormalTok{(w))}
  
  \ControlFlowTok{if}\NormalTok{ (na.rm) \{}
\NormalTok{    miss }\OtherTok{\textless{}{-}} \FunctionTok{is.na}\NormalTok{(x) }\SpecialCharTok{|} \FunctionTok{is.na}\NormalTok{(w)}
\NormalTok{    x }\OtherTok{\textless{}{-}}\NormalTok{ x[}\SpecialCharTok{!}\NormalTok{miss]}
\NormalTok{    w }\OtherTok{\textless{}{-}}\NormalTok{ w[}\SpecialCharTok{!}\NormalTok{miss]}
\NormalTok{  \}}
  \FunctionTok{sum}\NormalTok{(w }\SpecialCharTok{*}\NormalTok{ x) }\SpecialCharTok{/} \FunctionTok{sum}\NormalTok{(w)}
\NormalTok{\}}
\end{Highlighting}
\end{Shaded}

\hypertarget{dot-dot-dot}{%
\subsubsection{19.5.3 Dot-dot-dot(\ldots)}\label{dot-dot-dot}}

Many functions in R take an arbitrary number of inputs:

\begin{Shaded}
\begin{Highlighting}[]
\FunctionTok{sum}\NormalTok{(}\DecValTok{1}\NormalTok{, }\DecValTok{2}\NormalTok{, }\DecValTok{3}\NormalTok{, }\DecValTok{4}\NormalTok{, }\DecValTok{5}\NormalTok{, }\DecValTok{6}\NormalTok{, }\DecValTok{7}\NormalTok{, }\DecValTok{8}\NormalTok{, }\DecValTok{9}\NormalTok{, }\DecValTok{10}\NormalTok{)}
\end{Highlighting}
\end{Shaded}

\begin{verbatim}
## [1] 55
\end{verbatim}

\begin{Shaded}
\begin{Highlighting}[]
\NormalTok{stringr}\SpecialCharTok{::}\FunctionTok{str\_c}\NormalTok{(}\StringTok{"a"}\NormalTok{, }\StringTok{"b"}\NormalTok{, }\StringTok{"c"}\NormalTok{, }\StringTok{"d"}\NormalTok{, }\StringTok{"e"}\NormalTok{, }\StringTok{"f"}\NormalTok{)}
\end{Highlighting}
\end{Shaded}

\begin{verbatim}
## [1] "abcdef"
\end{verbatim}

Define a function to concatenate strings with commas

\begin{Shaded}
\begin{Highlighting}[]
\NormalTok{commas }\OtherTok{\textless{}{-}} \ControlFlowTok{function}\NormalTok{(...) stringr}\SpecialCharTok{::}\FunctionTok{str\_c}\NormalTok{(..., }\AttributeTok{collapse =} \StringTok{", "}\NormalTok{)}
\FunctionTok{commas}\NormalTok{(letters[}\DecValTok{1}\SpecialCharTok{:}\DecValTok{10}\NormalTok{])}
\end{Highlighting}
\end{Shaded}

\begin{verbatim}
## [1] "a, b, c, d, e, f, g, h, i, j"
\end{verbatim}

\begin{Shaded}
\begin{Highlighting}[]
\NormalTok{rule }\OtherTok{\textless{}{-}} \ControlFlowTok{function}\NormalTok{(..., }\AttributeTok{pad =} \StringTok{"{-}"}\NormalTok{) \{}
\NormalTok{  title }\OtherTok{\textless{}{-}} \FunctionTok{paste0}\NormalTok{(...)}
\NormalTok{  width }\OtherTok{\textless{}{-}} \FunctionTok{getOption}\NormalTok{(}\StringTok{"width"}\NormalTok{) }\SpecialCharTok{{-}} \FunctionTok{nchar}\NormalTok{(title) }\SpecialCharTok{{-}} \DecValTok{5}
  \FunctionTok{cat}\NormalTok{(title, }\StringTok{" "}\NormalTok{, stringr}\SpecialCharTok{::}\FunctionTok{str\_dup}\NormalTok{(pad, width), }\StringTok{"}\SpecialCharTok{\textbackslash{}n}\StringTok{"}\NormalTok{, }\AttributeTok{sep =} \StringTok{""}\NormalTok{)}
\NormalTok{\}}
\FunctionTok{rule}\NormalTok{(}\StringTok{"Important output"}\NormalTok{)}
\end{Highlighting}
\end{Shaded}

\begin{verbatim}
## Important output -----------------------------------------------------------
\end{verbatim}

\begin{Shaded}
\begin{Highlighting}[]
\NormalTok{x }\OtherTok{\textless{}{-}} \FunctionTok{c}\NormalTok{(}\DecValTok{1}\NormalTok{,}\DecValTok{2}\NormalTok{)}
\FunctionTok{sum}\NormalTok{(x,}\AttributeTok{na.rm=}\NormalTok{T)}
\end{Highlighting}
\end{Shaded}

\begin{verbatim}
## [1] 3
\end{verbatim}

Define a function `complicated\_function' with conditions to return 0 if
`x' or `y' is empty

\begin{Shaded}
\begin{Highlighting}[]
\NormalTok{complicated\_function }\OtherTok{\textless{}{-}} \ControlFlowTok{function}\NormalTok{(x,y,z)\{}
  \ControlFlowTok{if}\NormalTok{ (}\FunctionTok{lenth}\NormalTok{(x)}\SpecialCharTok{==}\DecValTok{0} \SpecialCharTok{||} \FunctionTok{length}\NormalTok{(y)}\SpecialCharTok{==}\DecValTok{0}\NormalTok{)\{}
    \FunctionTok{return}\NormalTok{(}\DecValTok{0}\NormalTok{)}
\NormalTok{  \}}
\NormalTok{\}}
\end{Highlighting}
\end{Shaded}

Improve readability of if-else blocks by using early return for simple
cases

\begin{Shaded}
\begin{Highlighting}[]
\NormalTok{f }\OtherTok{\textless{}{-}} \ControlFlowTok{function}\NormalTok{() \{}
  \ControlFlowTok{if}\NormalTok{ (x) \{}
    \CommentTok{\# Do }
    \CommentTok{\# something}
    \CommentTok{\# that}
    \CommentTok{\# takes}
    \CommentTok{\# many}
    \CommentTok{\# lines}
    \CommentTok{\# to}
    \CommentTok{\# express}
\NormalTok{  \} }\ControlFlowTok{else}\NormalTok{ \{}
    \CommentTok{\# return something short}
\NormalTok{  \}}
\NormalTok{\}}
\end{Highlighting}
\end{Shaded}

But if the first block is very long, by the time you get to the else,
you've forgotten the condition. One way to rewrite it is to use an early
return for the simple case:

\begin{Shaded}
\begin{Highlighting}[]
\NormalTok{f }\OtherTok{\textless{}{-}} \ControlFlowTok{function}\NormalTok{() \{}
  \ControlFlowTok{if}\NormalTok{ (}\SpecialCharTok{!}\NormalTok{x) \{}
    \FunctionTok{return}\NormalTok{(something\_short)}
\NormalTok{  \}}

  \CommentTok{\# Do }
  \CommentTok{\# something}
  \CommentTok{\# that}
  \CommentTok{\# takes}
  \CommentTok{\# many}
  \CommentTok{\# lines}
  \CommentTok{\# to}
  \CommentTok{\# express}
\NormalTok{\}}
\end{Highlighting}
\end{Shaded}

This tends to make the code easier to understand, because you don't need
quite so much context to understand it.

\hypertarget{writing-pipeable-functions}{%
\subsubsection{19.6.2 Writing pipeable
functions}\label{writing-pipeable-functions}}

Define a function to show the count of missing values in a data frame

\begin{Shaded}
\begin{Highlighting}[]
\NormalTok{show\_missing }\OtherTok{\textless{}{-}} \ControlFlowTok{function}\NormalTok{(df)\{}
\NormalTok{  n }\OtherTok{\textless{}{-}} \FunctionTok{sum}\NormalTok{(}\FunctionTok{is.na}\NormalTok{(df))}
  \FunctionTok{cat}\NormalTok{(}\StringTok{"Missing values:"}\NormalTok{,n,}\StringTok{"}\SpecialCharTok{\textbackslash{}n}\StringTok{"}\NormalTok{,}\AttributeTok{sep=}\StringTok{""}\NormalTok{)}
  
  \FunctionTok{invisible}\NormalTok{(df)}
\NormalTok{\}}
\end{Highlighting}
\end{Shaded}

If we call it interatively, the \texttt{invisible()} means that the
input \texttt{df} does not get printed out:

\begin{Shaded}
\begin{Highlighting}[]
\FunctionTok{show\_missing}\NormalTok{(mtcars)}
\end{Highlighting}
\end{Shaded}

\begin{verbatim}
## Missing values:0
\end{verbatim}

But it's still there, it's not just printed by default:

\begin{Shaded}
\begin{Highlighting}[]
\NormalTok{x }\OtherTok{\textless{}{-}} \FunctionTok{show\_missing}\NormalTok{(mtcars)}
\end{Highlighting}
\end{Shaded}

\begin{verbatim}
## Missing values:0
\end{verbatim}

\begin{Shaded}
\begin{Highlighting}[]
\FunctionTok{class}\NormalTok{(x)}
\end{Highlighting}
\end{Shaded}

\begin{verbatim}
## [1] "data.frame"
\end{verbatim}

\begin{Shaded}
\begin{Highlighting}[]
\FunctionTok{dim}\NormalTok{(x)}
\end{Highlighting}
\end{Shaded}

\begin{verbatim}
## [1] 32 11
\end{verbatim}

And we can still use it in a pipe:

\begin{Shaded}
\begin{Highlighting}[]
\FunctionTok{library}\NormalTok{(magrittr)}
\FunctionTok{library}\NormalTok{(tidyverse)}

\NormalTok{mtcars }\SpecialCharTok{\%\textgreater{}\%} 
  \FunctionTok{show\_missing}\NormalTok{() }\SpecialCharTok{\%\textgreater{}\%} 
  \FunctionTok{mutate}\NormalTok{(}\AttributeTok{mpg=}\FunctionTok{ifelse}\NormalTok{(mpg}\SpecialCharTok{\textless{}}\DecValTok{20}\NormalTok{,}\ConstantTok{NA}\NormalTok{,mpg)) }\SpecialCharTok{\%\textgreater{}\%} 
  \FunctionTok{show\_missing}\NormalTok{()}
\end{Highlighting}
\end{Shaded}

\begin{verbatim}
## Missing values:0
## Missing values:18
\end{verbatim}

\hypertarget{environment}{%
\subsection{19.7 Environment}\label{environment}}

Define a function `f' that takes an argument `x' and returns the sum of
`x' and `y'

\begin{Shaded}
\begin{Highlighting}[]
\NormalTok{f }\OtherTok{\textless{}{-}} \ControlFlowTok{function}\NormalTok{(x)\{}
\NormalTok{  x}\SpecialCharTok{+}\NormalTok{y}
\NormalTok{\}}
\end{Highlighting}
\end{Shaded}

Demonstrate how changing the value of `y' affects the result of calling
function `f'

\begin{Shaded}
\begin{Highlighting}[]
\NormalTok{y }\OtherTok{\textless{}{-}} \DecValTok{100}
\FunctionTok{f}\NormalTok{(}\DecValTok{10}\NormalTok{)}
\end{Highlighting}
\end{Shaded}

\begin{verbatim}
## [1] 110
\end{verbatim}

\begin{Shaded}
\begin{Highlighting}[]
\NormalTok{y }\OtherTok{\textless{}{-}} \DecValTok{1000}
\FunctionTok{f}\NormalTok{(}\DecValTok{10}\NormalTok{)}
\end{Highlighting}
\end{Shaded}

\begin{verbatim}
## [1] 1010
\end{verbatim}

Overload the `+' operator to behave differently based on a random
condition

\begin{Shaded}
\begin{Highlighting}[]
\StringTok{\textasciigrave{}}\AttributeTok{+}\StringTok{\textasciigrave{}} \OtherTok{\textless{}{-}} \ControlFlowTok{function}\NormalTok{(x, y) \{}
  \ControlFlowTok{if}\NormalTok{ (}\FunctionTok{runif}\NormalTok{(}\DecValTok{1}\NormalTok{) }\SpecialCharTok{\textless{}} \FloatTok{0.1}\NormalTok{) \{}
    \FunctionTok{sum}\NormalTok{(x, y)}
\NormalTok{  \} }\ControlFlowTok{else}\NormalTok{ \{}
    \FunctionTok{sum}\NormalTok{(x, y) }\SpecialCharTok{*} \FloatTok{1.1}
\NormalTok{  \}}
\NormalTok{\}}
\FunctionTok{table}\NormalTok{(}\FunctionTok{replicate}\NormalTok{(}\DecValTok{1000}\NormalTok{, }\DecValTok{1} \SpecialCharTok{+} \DecValTok{2}\NormalTok{))}
\end{Highlighting}
\end{Shaded}

\begin{verbatim}
## 
##   3 3.3 
##  90 910
\end{verbatim}

\begin{Shaded}
\begin{Highlighting}[]
\CommentTok{\#\textgreater{} }
\CommentTok{\#\textgreater{}   3 3.3 }
\CommentTok{\#\textgreater{} 100 900}
\FunctionTok{rm}\NormalTok{(}\StringTok{\textasciigrave{}}\AttributeTok{+}\StringTok{\textasciigrave{}}\NormalTok{)}
\end{Highlighting}
\end{Shaded}

\hypertarget{chapter-20-vectors}{%
\section{Chapter 20: Vectors}\label{chapter-20-vectors}}

\hypertarget{prerequisites}{%
\subsubsection{20.1.1 PRerequisites}\label{prerequisites}}

\begin{Shaded}
\begin{Highlighting}[]
\FunctionTok{library}\NormalTok{(tidyverse)}
\end{Highlighting}
\end{Shaded}

\hypertarget{vector-basics}{%
\subsection{20.2 Vector basics}\label{vector-basics}}

Determine the data type of different vectors

\begin{Shaded}
\begin{Highlighting}[]
\FunctionTok{typeof}\NormalTok{(letters)}
\end{Highlighting}
\end{Shaded}

\begin{verbatim}
## [1] "character"
\end{verbatim}

\begin{Shaded}
\begin{Highlighting}[]
\FunctionTok{typeof}\NormalTok{(}\DecValTok{1}\SpecialCharTok{:}\DecValTok{10}\NormalTok{)}
\end{Highlighting}
\end{Shaded}

\begin{verbatim}
## [1] "integer"
\end{verbatim}

Determine the length of a list and display its contents

\begin{Shaded}
\begin{Highlighting}[]
\NormalTok{x }\OtherTok{\textless{}{-}} \FunctionTok{list}\NormalTok{(}\StringTok{"a"}\NormalTok{,}\StringTok{"b"}\NormalTok{,}\DecValTok{1}\SpecialCharTok{:}\DecValTok{10}\NormalTok{)}
\FunctionTok{length}\NormalTok{(x)}
\end{Highlighting}
\end{Shaded}

\begin{verbatim}
## [1] 3
\end{verbatim}

\begin{Shaded}
\begin{Highlighting}[]
\NormalTok{x}
\end{Highlighting}
\end{Shaded}

\begin{verbatim}
## [[1]]
## [1] "a"
## 
## [[2]]
## [1] "b"
## 
## [[3]]
##  [1]  1  2  3  4  5  6  7  8  9 10
\end{verbatim}

Demonstrate modulo operation and creation of logical vectors

\begin{Shaded}
\begin{Highlighting}[]
\DecValTok{1}\SpecialCharTok{:}\DecValTok{10} \SpecialCharTok{\%\%} \DecValTok{3} \SpecialCharTok{==}\DecValTok{0}
\end{Highlighting}
\end{Shaded}

\begin{verbatim}
##  [1] FALSE FALSE  TRUE FALSE FALSE  TRUE FALSE FALSE  TRUE FALSE
\end{verbatim}

\begin{Shaded}
\begin{Highlighting}[]
\FunctionTok{c}\NormalTok{(T,T,F,}\ConstantTok{NA}\NormalTok{)}
\end{Highlighting}
\end{Shaded}

\begin{verbatim}
## [1]  TRUE  TRUE FALSE    NA
\end{verbatim}

\hypertarget{numeric}{%
\subsubsection{20.3.2 Numeric}\label{numeric}}

Integer and double vectors are known collectively as numeric vectors. In
R, numbers are doubles by default. To make an integer, place an L after
the number:

\begin{Shaded}
\begin{Highlighting}[]
\FunctionTok{typeof}\NormalTok{(}\DecValTok{1}\NormalTok{)}
\end{Highlighting}
\end{Shaded}

\begin{verbatim}
## [1] "double"
\end{verbatim}

\begin{Shaded}
\begin{Highlighting}[]
\FunctionTok{typeof}\NormalTok{(1L)}
\end{Highlighting}
\end{Shaded}

\begin{verbatim}
## [1] "integer"
\end{verbatim}

\begin{Shaded}
\begin{Highlighting}[]
\FloatTok{1.5}
\end{Highlighting}
\end{Shaded}

\begin{verbatim}
## [1] 1.5
\end{verbatim}

Demonstrate the behavior of floating point arithmetic

\begin{Shaded}
\begin{Highlighting}[]
\NormalTok{x }\OtherTok{\textless{}{-}} \FunctionTok{sqrt}\NormalTok{(}\DecValTok{2}\NormalTok{)}\SpecialCharTok{\^{}}\DecValTok{2}
\NormalTok{x}
\end{Highlighting}
\end{Shaded}

\begin{verbatim}
## [1] 2
\end{verbatim}

\begin{Shaded}
\begin{Highlighting}[]
\NormalTok{x}\DecValTok{{-}2}
\end{Highlighting}
\end{Shaded}

\begin{verbatim}
## [1] 4.440892e-16
\end{verbatim}

Demonstrate the behavior of division by zero

\begin{Shaded}
\begin{Highlighting}[]
\FunctionTok{c}\NormalTok{(}\SpecialCharTok{{-}}\DecValTok{1}\NormalTok{,}\DecValTok{0}\NormalTok{,}\DecValTok{1}\NormalTok{)}\SpecialCharTok{\%/\%} \DecValTok{0}
\end{Highlighting}
\end{Shaded}

\begin{verbatim}
## [1] -Inf  NaN  Inf
\end{verbatim}

\begin{Shaded}
\begin{Highlighting}[]
\CommentTok{\# [1] {-}Inf  NaN  Inf}
\end{Highlighting}
\end{Shaded}

\hypertarget{character}{%
\subsubsection{20.3.3 Character}\label{character}}

Determine the memory size of a string and a replicated string vector

\begin{Shaded}
\begin{Highlighting}[]
\NormalTok{x }\OtherTok{\textless{}{-}} \StringTok{"This is a reasonably long string."}
\NormalTok{pryr}\SpecialCharTok{::}\FunctionTok{object\_size}\NormalTok{(x)}
\end{Highlighting}
\end{Shaded}

\begin{verbatim}
## 152 B
\end{verbatim}

\begin{Shaded}
\begin{Highlighting}[]
\NormalTok{y }\OtherTok{\textless{}{-}} \FunctionTok{rep}\NormalTok{(x,}\DecValTok{1000}\NormalTok{)}
\NormalTok{pryr}\SpecialCharTok{::}\FunctionTok{object\_size}\NormalTok{(y)}
\end{Highlighting}
\end{Shaded}

\begin{verbatim}
## 8.14 kB
\end{verbatim}

\hypertarget{missing-values-1}{%
\subsubsection{20.3.4 Missing values}\label{missing-values-1}}

Note that each type of atomic vector has its own missing value:

\begin{Shaded}
\begin{Highlighting}[]
\ConstantTok{NA}            \CommentTok{\# logical}
\end{Highlighting}
\end{Shaded}

\begin{verbatim}
## [1] NA
\end{verbatim}

\begin{Shaded}
\begin{Highlighting}[]
\ConstantTok{NA\_integer\_}   \CommentTok{\# integer}
\end{Highlighting}
\end{Shaded}

\begin{verbatim}
## [1] NA
\end{verbatim}

\begin{Shaded}
\begin{Highlighting}[]
\ConstantTok{NA\_real\_}      \CommentTok{\# double}
\end{Highlighting}
\end{Shaded}

\begin{verbatim}
## [1] NA
\end{verbatim}

\begin{Shaded}
\begin{Highlighting}[]
\ConstantTok{NA\_character\_} \CommentTok{\# character}
\end{Highlighting}
\end{Shaded}

\begin{verbatim}
## [1] NA
\end{verbatim}

Calculate the number and proportion of elements in a vector greater than
10

\begin{Shaded}
\begin{Highlighting}[]
\NormalTok{x }\OtherTok{\textless{}{-}} \FunctionTok{sample}\NormalTok{(}\DecValTok{20}\NormalTok{,}\DecValTok{100}\NormalTok{,}\AttributeTok{replace=}\NormalTok{T)}
\NormalTok{y }\OtherTok{\textless{}{-}}\NormalTok{ x }\SpecialCharTok{\textgreater{}} \DecValTok{10}
\FunctionTok{sum}\NormalTok{(y) }\CommentTok{\# how many are greater than 10?}
\end{Highlighting}
\end{Shaded}

\begin{verbatim}
## [1] 51
\end{verbatim}

\begin{Shaded}
\begin{Highlighting}[]
\FunctionTok{mean}\NormalTok{(y) }\CommentTok{\# what proportion are greater than 10?}
\end{Highlighting}
\end{Shaded}

\begin{verbatim}
## [1] 0.51
\end{verbatim}

\begin{Shaded}
\begin{Highlighting}[]
\ControlFlowTok{if}\NormalTok{ (}\FunctionTok{length}\NormalTok{(x))\{}
\NormalTok{\}}
\end{Highlighting}
\end{Shaded}

\begin{verbatim}
## NULL
\end{verbatim}

Determine the data type of different vectors

\begin{Shaded}
\begin{Highlighting}[]
\FunctionTok{typeof}\NormalTok{(}\FunctionTok{c}\NormalTok{(}\ConstantTok{TRUE}\NormalTok{,1L))}
\end{Highlighting}
\end{Shaded}

\begin{verbatim}
## [1] "integer"
\end{verbatim}

\begin{Shaded}
\begin{Highlighting}[]
\FunctionTok{typeof}\NormalTok{(}\FunctionTok{c}\NormalTok{(1L,}\FloatTok{1.5}\NormalTok{))}
\end{Highlighting}
\end{Shaded}

\begin{verbatim}
## [1] "double"
\end{verbatim}

\begin{Shaded}
\begin{Highlighting}[]
\FunctionTok{typeof}\NormalTok{(}\FunctionTok{c}\NormalTok{(}\FloatTok{1.5}\NormalTok{,}\StringTok{"a"}\NormalTok{))}
\end{Highlighting}
\end{Shaded}

\begin{verbatim}
## [1] "character"
\end{verbatim}

Generate random numeric or logical vectors

\begin{Shaded}
\begin{Highlighting}[]
\FunctionTok{sample}\NormalTok{(}\DecValTok{10}\NormalTok{)}\SpecialCharTok{+}\DecValTok{100}
\end{Highlighting}
\end{Shaded}

\begin{verbatim}
##  [1] 110 107 109 105 106 102 108 101 103 104
\end{verbatim}

\begin{Shaded}
\begin{Highlighting}[]
\FunctionTok{runif}\NormalTok{(}\DecValTok{10}\NormalTok{)}\SpecialCharTok{\textgreater{}}\FloatTok{0.5}
\end{Highlighting}
\end{Shaded}

\begin{verbatim}
##  [1] FALSE  TRUE  TRUE  TRUE  TRUE FALSE  TRUE FALSE FALSE  TRUE
\end{verbatim}

Demonstrate vector arithmetic with vectors of different lengths

\begin{Shaded}
\begin{Highlighting}[]
\DecValTok{1}\SpecialCharTok{:}\DecValTok{10} \SpecialCharTok{+}\DecValTok{1}\SpecialCharTok{:}\DecValTok{2}
\end{Highlighting}
\end{Shaded}

\begin{verbatim}
##  [1]  2  4  4  6  6  8  8 10 10 12
\end{verbatim}

\begin{Shaded}
\begin{Highlighting}[]
\DecValTok{1}\SpecialCharTok{:}\DecValTok{10}\SpecialCharTok{+}\DecValTok{1}\SpecialCharTok{:}\DecValTok{3}
\end{Highlighting}
\end{Shaded}

\begin{verbatim}
## Warning in 1:10 + 1:3: longer object length is not a multiple of shorter object
## length
\end{verbatim}

\begin{verbatim}
##  [1]  2  4  6  5  7  9  8 10 12 11
\end{verbatim}

Create a tibble with two columns, `x' and `y', with different lengths

\begin{Shaded}
\begin{Highlighting}[]
\FunctionTok{library}\NormalTok{(tidyverse)}



\FunctionTok{tibble}\NormalTok{(}
  \AttributeTok{x=}\DecValTok{1}\SpecialCharTok{:}\DecValTok{4}\NormalTok{,}
  \AttributeTok{y=}\FunctionTok{rep}\NormalTok{(}\DecValTok{1}\SpecialCharTok{:}\DecValTok{2}\NormalTok{,}\AttributeTok{each=}\DecValTok{2}\NormalTok{)}
\NormalTok{)}
\end{Highlighting}
\end{Shaded}

\begin{verbatim}
## # A tibble: 4 x 2
##       x     y
##   <int> <int>
## 1     1     1
## 2     2     1
## 3     3     2
## 4     4     2
\end{verbatim}

\hypertarget{naming-vectors}{%
\paragraph{20.4.4 Naming vectors}\label{naming-vectors}}

All types of vectors can be named. You can name them during creatin with
\texttt{c()}:

\begin{Shaded}
\begin{Highlighting}[]
\FunctionTok{c}\NormalTok{(}\AttributeTok{x=}\DecValTok{1}\NormalTok{,}\AttributeTok{y=}\DecValTok{2}\NormalTok{,}\AttributeTok{z=}\DecValTok{4}\NormalTok{)}
\end{Highlighting}
\end{Shaded}

\begin{verbatim}
## x y z 
## 1 2 4
\end{verbatim}

Or after the fact with \texttt{purr::set\_names()}

\begin{Shaded}
\begin{Highlighting}[]
\FunctionTok{set\_names}\NormalTok{(}\DecValTok{1}\SpecialCharTok{:}\DecValTok{3}\NormalTok{,}\FunctionTok{c}\NormalTok{(}\StringTok{"a"}\NormalTok{,}\StringTok{"b"}\NormalTok{,}\StringTok{"c"}\NormalTok{))}
\end{Highlighting}
\end{Shaded}

\begin{verbatim}
## a b c 
## 1 2 3
\end{verbatim}

Named vectors are most useful for subsetting, described next.

\hypertarget{subsetting-1}{%
\subsubsection{20.4.5 Subsetting}\label{subsetting-1}}

Demonstrate subsetting vectors with integer indices

\begin{Shaded}
\begin{Highlighting}[]
\NormalTok{x }\OtherTok{\textless{}{-}} \FunctionTok{c}\NormalTok{(}\StringTok{"one"}\NormalTok{,}\StringTok{"two"}\NormalTok{,}\StringTok{"three"}\NormalTok{,}\StringTok{"four"}\NormalTok{,}\StringTok{"five"}\NormalTok{)}
\NormalTok{x[}\FunctionTok{c}\NormalTok{(}\DecValTok{3}\NormalTok{,}\DecValTok{2}\NormalTok{,}\DecValTok{5}\NormalTok{)]}
\end{Highlighting}
\end{Shaded}

\begin{verbatim}
## [1] "three" "two"   "five"
\end{verbatim}

By repeating a position, you can actually make a longer output than
input:

\begin{Shaded}
\begin{Highlighting}[]
\NormalTok{x[}\FunctionTok{c}\NormalTok{(}\DecValTok{1}\NormalTok{,}\DecValTok{1}\NormalTok{,}\DecValTok{5}\NormalTok{,}\DecValTok{5}\NormalTok{,}\DecValTok{5}\NormalTok{,}\DecValTok{2}\NormalTok{)]}
\end{Highlighting}
\end{Shaded}

\begin{verbatim}
## [1] "one"  "one"  "five" "five" "five" "two"
\end{verbatim}

Negative values drop the elements at the specified positions:

\begin{Shaded}
\begin{Highlighting}[]
\NormalTok{x[}\FunctionTok{c}\NormalTok{(}\SpecialCharTok{{-}}\DecValTok{1}\NormalTok{,}\SpecialCharTok{{-}}\DecValTok{3}\NormalTok{,}\SpecialCharTok{{-}}\DecValTok{5}\NormalTok{)]}
\end{Highlighting}
\end{Shaded}

\begin{verbatim}
## [1] "two"  "four"
\end{verbatim}

The error message mentions subsetting with zero, which returns no
values:

\begin{Shaded}
\begin{Highlighting}[]
\NormalTok{x[}\DecValTok{0}\NormalTok{]}
\end{Highlighting}
\end{Shaded}

\begin{verbatim}
## character(0)
\end{verbatim}

\begin{Shaded}
\begin{Highlighting}[]
\FunctionTok{library}\NormalTok{(tidyverse)}
\NormalTok{x }\OtherTok{\textless{}{-}} \FunctionTok{c}\NormalTok{(}\DecValTok{10}\NormalTok{,}\DecValTok{3}\NormalTok{,}\ConstantTok{NA}\NormalTok{,}\DecValTok{5}\NormalTok{,}\DecValTok{8}\NormalTok{,}\DecValTok{1}\NormalTok{)}

\CommentTok{\# tibble test}
\NormalTok{x }\OtherTok{\textless{}{-}} \FunctionTok{as.tibble}\NormalTok{(x,}\AttributeTok{ncol=}\DecValTok{1}\NormalTok{)}
\end{Highlighting}
\end{Shaded}

\begin{verbatim}
## Warning: `as.tibble()` was deprecated in tibble 2.0.0.
## i Please use `as_tibble()` instead.
## i The signature and semantics have changed, see `?as_tibble`.
## This warning is displayed once every 8 hours.
## Call `lifecycle::last_lifecycle_warnings()` to see where this warning was
## generated.
\end{verbatim}

\begin{Shaded}
\begin{Highlighting}[]
\FunctionTok{names}\NormalTok{(x)}\OtherTok{=}\StringTok{"v1"}
\FunctionTok{is.na}\NormalTok{(x)}
\end{Highlighting}
\end{Shaded}

\begin{verbatim}
##         v1
## [1,] FALSE
## [2,] FALSE
## [3,]  TRUE
## [4,] FALSE
## [5,] FALSE
## [6,] FALSE
\end{verbatim}

\begin{Shaded}
\begin{Highlighting}[]
\NormalTok{x }\SpecialCharTok{\%\textgreater{}\%} \FunctionTok{filter}\NormalTok{(v1 }\SpecialCharTok{==} \ConstantTok{NA}\NormalTok{)}
\end{Highlighting}
\end{Shaded}

\begin{verbatim}
## # A tibble: 0 x 1
## # i 1 variable: v1 <dbl>
\end{verbatim}

\begin{Shaded}
\begin{Highlighting}[]
\CommentTok{\# all non{-}missing values of x}
\NormalTok{x }\OtherTok{\textless{}{-}} \FunctionTok{c}\NormalTok{(}\DecValTok{10}\NormalTok{,}\DecValTok{3}\NormalTok{,}\ConstantTok{NA}\NormalTok{,}\DecValTok{5}\NormalTok{,}\DecValTok{8}\NormalTok{,}\DecValTok{1}\NormalTok{)}
\NormalTok{x[}\SpecialCharTok{!}\FunctionTok{is.na}\NormalTok{(x)]}
\end{Highlighting}
\end{Shaded}

\begin{verbatim}
## [1] 10  3  5  8  1
\end{verbatim}

\begin{Shaded}
\begin{Highlighting}[]
\CommentTok{\# all even (or missing) values of x}
\NormalTok{x[x }\SpecialCharTok{\%\%} \DecValTok{2}\SpecialCharTok{==}\DecValTok{0}\NormalTok{]}
\end{Highlighting}
\end{Shaded}

\begin{verbatim}
## [1] 10 NA  8
\end{verbatim}

\begin{enumerate}
\def\labelenumi{\arabic{enumi}.}
\setcounter{enumi}{2}
\tightlist
\item
  If you have a named vector, you can subset it with a character vector:
\end{enumerate}

\begin{Shaded}
\begin{Highlighting}[]
\NormalTok{x }\OtherTok{\textless{}{-}} \FunctionTok{c}\NormalTok{(}\AttributeTok{abc=}\DecValTok{1}\NormalTok{, }\AttributeTok{def=}\DecValTok{2}\NormalTok{,}\AttributeTok{xyz=}\DecValTok{5}\NormalTok{)}
\NormalTok{x[}\FunctionTok{c}\NormalTok{(}\StringTok{"xyz"}\NormalTok{,}\StringTok{"def"}\NormalTok{)]}
\end{Highlighting}
\end{Shaded}

\begin{verbatim}
## xyz def 
##   5   2
\end{verbatim}

\hypertarget{recursive-vectors-lists}{%
\subsection{20.5 Recursive vectors
(lists)}\label{recursive-vectors-lists}}

Create a list with numeric elements

\begin{Shaded}
\begin{Highlighting}[]
\NormalTok{x }\OtherTok{\textless{}{-}} \FunctionTok{list}\NormalTok{(}\DecValTok{1}\NormalTok{,}\DecValTok{2}\NormalTok{,}\DecValTok{3}\NormalTok{)}
\NormalTok{x}
\end{Highlighting}
\end{Shaded}

\begin{verbatim}
## [[1]]
## [1] 1
## 
## [[2]]
## [1] 2
## 
## [[3]]
## [1] 3
\end{verbatim}

Display the structure of lists with and without names

\begin{Shaded}
\begin{Highlighting}[]
\FunctionTok{str}\NormalTok{(x)}
\end{Highlighting}
\end{Shaded}

\begin{verbatim}
## List of 3
##  $ : num 1
##  $ : num 2
##  $ : num 3
\end{verbatim}

\begin{Shaded}
\begin{Highlighting}[]
\NormalTok{x\_named }\OtherTok{\textless{}{-}} \FunctionTok{list}\NormalTok{(}\AttributeTok{a=}\DecValTok{1}\NormalTok{,}\AttributeTok{b=}\DecValTok{2}\NormalTok{,}\AttributeTok{c=}\DecValTok{3}\NormalTok{)}
\FunctionTok{str}\NormalTok{(x\_named)}
\end{Highlighting}
\end{Shaded}

\begin{verbatim}
## List of 3
##  $ a: num 1
##  $ b: num 2
##  $ c: num 3
\end{verbatim}

Unlike atomic vectors, \texttt{list()} can contain a mix of objects:

\begin{Shaded}
\begin{Highlighting}[]
\NormalTok{y }\OtherTok{\textless{}{-}} \FunctionTok{list}\NormalTok{(}\StringTok{"a"}\NormalTok{,1L,}\FloatTok{1.5}\NormalTok{,T)}
\FunctionTok{str}\NormalTok{(y)}
\end{Highlighting}
\end{Shaded}

\begin{verbatim}
## List of 4
##  $ : chr "a"
##  $ : int 1
##  $ : num 1.5
##  $ : logi TRUE
\end{verbatim}

List can even contain other lists!

\begin{Shaded}
\begin{Highlighting}[]
\NormalTok{z }\OtherTok{\textless{}{-}} \FunctionTok{list}\NormalTok{(}\FunctionTok{list}\NormalTok{(}\DecValTok{1}\NormalTok{,}\DecValTok{2}\NormalTok{),}\FunctionTok{list}\NormalTok{(}\DecValTok{3}\NormalTok{,}\DecValTok{4}\NormalTok{))}
\FunctionTok{str}\NormalTok{(z)}
\end{Highlighting}
\end{Shaded}

\begin{verbatim}
## List of 2
##  $ :List of 2
##   ..$ : num 1
##   ..$ : num 2
##  $ :List of 2
##   ..$ : num 3
##   ..$ : num 4
\end{verbatim}

\hypertarget{visualizing-lists}{%
\subsubsection{20.5.1 Visualizing lists}\label{visualizing-lists}}

\begin{Shaded}
\begin{Highlighting}[]
\NormalTok{x1 }\OtherTok{\textless{}{-}} \FunctionTok{list}\NormalTok{(}\FunctionTok{c}\NormalTok{(}\DecValTok{1}\NormalTok{,}\DecValTok{2}\NormalTok{),}\FunctionTok{c}\NormalTok{(}\DecValTok{3}\NormalTok{,}\DecValTok{4}\NormalTok{))}
\NormalTok{x2 }\OtherTok{\textless{}{-}} \FunctionTok{list}\NormalTok{(}\FunctionTok{list}\NormalTok{(}\DecValTok{1}\NormalTok{,}\DecValTok{2}\NormalTok{),}\FunctionTok{list}\NormalTok{(}\DecValTok{3}\NormalTok{,}\DecValTok{4}\NormalTok{))}
\NormalTok{x3 }\OtherTok{\textless{}{-}} \FunctionTok{list}\NormalTok{(}\DecValTok{1}\NormalTok{,}\FunctionTok{list}\NormalTok{(}\DecValTok{2}\NormalTok{,}\FunctionTok{list}\NormalTok{(}\DecValTok{3}\NormalTok{)))}
\NormalTok{x1}
\end{Highlighting}
\end{Shaded}

\begin{verbatim}
## [[1]]
## [1] 1 2
## 
## [[2]]
## [1] 3 4
\end{verbatim}

\begin{Shaded}
\begin{Highlighting}[]
\NormalTok{x2}
\end{Highlighting}
\end{Shaded}

\begin{verbatim}
## [[1]]
## [[1]][[1]]
## [1] 1
## 
## [[1]][[2]]
## [1] 2
## 
## 
## [[2]]
## [[2]][[1]]
## [1] 3
## 
## [[2]][[2]]
## [1] 4
\end{verbatim}

\begin{Shaded}
\begin{Highlighting}[]
\NormalTok{x3}
\end{Highlighting}
\end{Shaded}

\begin{verbatim}
## [[1]]
## [1] 1
## 
## [[2]]
## [[2]][[1]]
## [1] 2
## 
## [[2]][[2]]
## [[2]][[2]][[1]]
## [1] 3
\end{verbatim}

\hypertarget{subsetting-2}{%
\subsubsection{20.5.2 Subsetting}\label{subsetting-2}}

Create a list `a' with named elements and demonstrate subsetting

\begin{Shaded}
\begin{Highlighting}[]
\NormalTok{a }\OtherTok{\textless{}{-}} \FunctionTok{list}\NormalTok{(}\AttributeTok{a =} \DecValTok{1}\SpecialCharTok{:}\DecValTok{3}\NormalTok{, }\AttributeTok{b =} \StringTok{"a string"}\NormalTok{, }\AttributeTok{c =}\NormalTok{ pi, }\AttributeTok{d =} \FunctionTok{list}\NormalTok{(}\SpecialCharTok{{-}}\DecValTok{1}\NormalTok{, }\SpecialCharTok{{-}}\DecValTok{5}\NormalTok{))}
\end{Highlighting}
\end{Shaded}

\begin{Shaded}
\begin{Highlighting}[]
\FunctionTok{str}\NormalTok{(a)}
\end{Highlighting}
\end{Shaded}

\begin{verbatim}
## List of 4
##  $ a: int [1:3] 1 2 3
##  $ b: chr "a string"
##  $ c: num 3.14
##  $ d:List of 2
##   ..$ : num -1
##   ..$ : num -5
\end{verbatim}

\begin{Shaded}
\begin{Highlighting}[]
\FunctionTok{str}\NormalTok{(a[}\DecValTok{1}\SpecialCharTok{:}\DecValTok{2}\NormalTok{])}
\end{Highlighting}
\end{Shaded}

\begin{verbatim}
## List of 2
##  $ a: int [1:3] 1 2 3
##  $ b: chr "a string"
\end{verbatim}

\begin{Shaded}
\begin{Highlighting}[]
\FunctionTok{str}\NormalTok{(a[}\DecValTok{4}\NormalTok{])}
\end{Highlighting}
\end{Shaded}

\begin{verbatim}
## List of 1
##  $ d:List of 2
##   ..$ : num -1
##   ..$ : num -5
\end{verbatim}

Demonstrate subsetting lists using double square brackets

\begin{Shaded}
\begin{Highlighting}[]
\FunctionTok{str}\NormalTok{(a[[}\DecValTok{1}\NormalTok{]])}
\end{Highlighting}
\end{Shaded}

\begin{verbatim}
##  int [1:3] 1 2 3
\end{verbatim}

\begin{Shaded}
\begin{Highlighting}[]
\FunctionTok{str}\NormalTok{(a[[}\DecValTok{4}\NormalTok{]])}
\end{Highlighting}
\end{Shaded}

\begin{verbatim}
## List of 2
##  $ : num -1
##  $ : num -5
\end{verbatim}

Access list elements by name using \$ or {[}{[} {]}{]}

\begin{Shaded}
\begin{Highlighting}[]
\NormalTok{a}\SpecialCharTok{$}\NormalTok{a}
\end{Highlighting}
\end{Shaded}

\begin{verbatim}
## [1] 1 2 3
\end{verbatim}

\begin{Shaded}
\begin{Highlighting}[]
\NormalTok{a[[}\StringTok{"a"}\NormalTok{]]}
\end{Highlighting}
\end{Shaded}

\begin{verbatim}
## [1] 1 2 3
\end{verbatim}

\hypertarget{attributes}{%
\subsection{20.6 Attributes}\label{attributes}}

Demonstrate setting and retrieving attributes of vectors

\begin{Shaded}
\begin{Highlighting}[]
\NormalTok{x }\OtherTok{\textless{}{-}} \DecValTok{1}\SpecialCharTok{:}\DecValTok{10}
\FunctionTok{attr}\NormalTok{(x,}\StringTok{"greeting"}\NormalTok{)}
\end{Highlighting}
\end{Shaded}

\begin{verbatim}
## NULL
\end{verbatim}

\begin{Shaded}
\begin{Highlighting}[]
\FunctionTok{attr}\NormalTok{(x,}\StringTok{"greeting"}\NormalTok{) }\OtherTok{\textless{}{-}} \StringTok{"Hi!"}
\FunctionTok{attr}\NormalTok{(x,}\StringTok{"farewell"}\NormalTok{) }\OtherTok{\textless{}{-}} \StringTok{"Bye!"}
\FunctionTok{attributes}\NormalTok{(x)}
\end{Highlighting}
\end{Shaded}

\begin{verbatim}
## $greeting
## [1] "Hi!"
## 
## $farewell
## [1] "Bye!"
\end{verbatim}

Demonstrate methods for class `Date'

\begin{Shaded}
\begin{Highlighting}[]
\NormalTok{as.Date}
\end{Highlighting}
\end{Shaded}

\begin{verbatim}
## function (x, ...) 
## UseMethod("as.Date")
## <bytecode: 0x00000263204f7048>
## <environment: namespace:base>
\end{verbatim}

\begin{Shaded}
\begin{Highlighting}[]
\FunctionTok{methods}\NormalTok{(}\StringTok{"as.Date"}\NormalTok{)}
\end{Highlighting}
\end{Shaded}

\begin{verbatim}
## [1] as.Date.character   as.Date.default     as.Date.factor     
## [4] as.Date.numeric     as.Date.POSIXct     as.Date.POSIXlt    
## [7] as.Date.vctrs_sclr* as.Date.vctrs_vctr*
## see '?methods' for accessing help and source code
\end{verbatim}

Retrieve specific methods for `as.Date'

\begin{Shaded}
\begin{Highlighting}[]
\FunctionTok{getS3method}\NormalTok{(}\StringTok{"as.Date"}\NormalTok{,}\StringTok{"default"}\NormalTok{)}
\end{Highlighting}
\end{Shaded}

\begin{verbatim}
## function (x, ...) 
## {
##     if (inherits(x, "Date")) 
##         x
##     else if (is.null(x)) 
##         .Date(numeric())
##     else if (is.logical(x) && all(is.na(x))) 
##         .Date(as.numeric(x))
##     else stop(gettextf("do not know how to convert '%s' to class %s", 
##         deparse1(substitute(x)), dQuote("Date")), domain = NA)
## }
## <bytecode: 0x0000026323d95488>
## <environment: namespace:base>
\end{verbatim}

\begin{Shaded}
\begin{Highlighting}[]
\FunctionTok{getS3method}\NormalTok{(}\StringTok{"as.Date"}\NormalTok{,}\StringTok{"numeric"}\NormalTok{)}
\end{Highlighting}
\end{Shaded}

\begin{verbatim}
## function (x, origin, ...) 
## if (missing(origin)) .Date(x) else as.Date(origin, ...) + x
## <bytecode: 0x0000026340985148>
## <environment: namespace:base>
\end{verbatim}

\hypertarget{factors}{%
\subsubsection{20.7.1 Factors}\label{factors}}

Demonstrate creating a factor and inspecting its attributes

\begin{Shaded}
\begin{Highlighting}[]
\NormalTok{x }\OtherTok{\textless{}{-}} \FunctionTok{factor}\NormalTok{(}\FunctionTok{c}\NormalTok{(}\StringTok{"ab"}\NormalTok{,}\StringTok{"cd"}\NormalTok{,}\StringTok{"ab"}\NormalTok{),}\AttributeTok{levels=}\FunctionTok{c}\NormalTok{(}\StringTok{"ab"}\NormalTok{,}\StringTok{"cd"}\NormalTok{,}\StringTok{"ed"}\NormalTok{))}
\FunctionTok{typeof}\NormalTok{(x)}
\end{Highlighting}
\end{Shaded}

\begin{verbatim}
## [1] "integer"
\end{verbatim}

\begin{Shaded}
\begin{Highlighting}[]
\FunctionTok{attributes}\NormalTok{(x)}
\end{Highlighting}
\end{Shaded}

\begin{verbatim}
## $levels
## [1] "ab" "cd" "ed"
## 
## $class
## [1] "factor"
\end{verbatim}

\hypertarget{dates-and-date-times}{%
\subsubsection{20.7.2 Dates and date-times}\label{dates-and-date-times}}

Dates in R are numeric vectors that represent the number of days since 1
January 1970.

\begin{Shaded}
\begin{Highlighting}[]
\NormalTok{x }\OtherTok{\textless{}{-}} \FunctionTok{as.Date}\NormalTok{(}\StringTok{"1971{-}01{-}01"}\NormalTok{)}
\FunctionTok{unclass}\NormalTok{(x)}
\end{Highlighting}
\end{Shaded}

\begin{verbatim}
## [1] 365
\end{verbatim}

\begin{Shaded}
\begin{Highlighting}[]
\FunctionTok{typeof}\NormalTok{(x)}
\end{Highlighting}
\end{Shaded}

\begin{verbatim}
## [1] "double"
\end{verbatim}

\begin{Shaded}
\begin{Highlighting}[]
\FunctionTok{attributes}\NormalTok{(x)}
\end{Highlighting}
\end{Shaded}

\begin{verbatim}
## $class
## [1] "Date"
\end{verbatim}

Demonstrate creating and inspecting a date-time object

\begin{Shaded}
\begin{Highlighting}[]
\NormalTok{x }\OtherTok{\textless{}{-}}\NormalTok{ lubridate}\SpecialCharTok{::}\FunctionTok{ymd\_hm}\NormalTok{(}\StringTok{"1970{-}01{-}01 01:00"}\NormalTok{)}
\FunctionTok{unclass}\NormalTok{(x)}
\end{Highlighting}
\end{Shaded}

\begin{verbatim}
## [1] 3600
## attr(,"tzone")
## [1] "UTC"
\end{verbatim}

\begin{Shaded}
\begin{Highlighting}[]
\FunctionTok{typeof}\NormalTok{(x)}
\end{Highlighting}
\end{Shaded}

\begin{verbatim}
## [1] "double"
\end{verbatim}

\begin{Shaded}
\begin{Highlighting}[]
\FunctionTok{attributes}\NormalTok{(x)}
\end{Highlighting}
\end{Shaded}

\begin{verbatim}
## $class
## [1] "POSIXct" "POSIXt" 
## 
## $tzone
## [1] "UTC"
\end{verbatim}

Demonstrate setting and retrieving time zone for date-time object

\begin{Shaded}
\begin{Highlighting}[]
\FunctionTok{attr}\NormalTok{(x,}\StringTok{"tzone"}\NormalTok{) }\OtherTok{\textless{}{-}} \StringTok{"US/Pacific"}
\NormalTok{x}
\end{Highlighting}
\end{Shaded}

\begin{verbatim}
## [1] "1969-12-31 17:00:00 PST"
\end{verbatim}

\begin{Shaded}
\begin{Highlighting}[]
\FunctionTok{attr}\NormalTok{(x,}\StringTok{"tzone"}\NormalTok{) }\OtherTok{\textless{}{-}} \StringTok{"US/Eastern"}
\NormalTok{x}
\end{Highlighting}
\end{Shaded}

\begin{verbatim}
## [1] "1969-12-31 20:00:00 EST"
\end{verbatim}

There is another type of date-times called POSIXIt. There are built on
top of named lists:

\begin{Shaded}
\begin{Highlighting}[]
\NormalTok{y }\OtherTok{\textless{}{-}} \FunctionTok{as.POSIXlt}\NormalTok{(x)}
\FunctionTok{typeof}\NormalTok{(y)}
\end{Highlighting}
\end{Shaded}

\begin{verbatim}
## [1] "list"
\end{verbatim}

\begin{Shaded}
\begin{Highlighting}[]
\CommentTok{\#\textgreater{} [1] "list"}
\FunctionTok{attributes}\NormalTok{(y)}
\end{Highlighting}
\end{Shaded}

\begin{verbatim}
## $names
##  [1] "sec"    "min"    "hour"   "mday"   "mon"    "year"   "wday"   "yday"  
##  [9] "isdst"  "zone"   "gmtoff"
## 
## $class
## [1] "POSIXlt" "POSIXt" 
## 
## $tzone
## [1] "US/Eastern" "EST"        "EDT"       
## 
## $balanced
## [1] TRUE
\end{verbatim}

\hypertarget{tibbles-1}{%
\subsubsection{20.7.3 Tibbles}\label{tibbles-1}}

Tibbles are augmented lists: they have class ``tbl\_df'' + ``tbl'' +
``data.frame'', and \texttt{names} (column) and \texttt{row.names}
attributes:

\begin{Shaded}
\begin{Highlighting}[]
\NormalTok{tb }\OtherTok{\textless{}{-}}\NormalTok{ tibble}\SpecialCharTok{::}\FunctionTok{tibble}\NormalTok{(}\AttributeTok{x =} \DecValTok{1}\SpecialCharTok{:}\DecValTok{5}\NormalTok{, }\AttributeTok{y =} \DecValTok{5}\SpecialCharTok{:}\DecValTok{1}\NormalTok{)}
\FunctionTok{typeof}\NormalTok{(tb)}
\end{Highlighting}
\end{Shaded}

\begin{verbatim}
## [1] "list"
\end{verbatim}

\begin{Shaded}
\begin{Highlighting}[]
\FunctionTok{attributes}\NormalTok{(tb)}
\end{Highlighting}
\end{Shaded}

\begin{verbatim}
## $class
## [1] "tbl_df"     "tbl"        "data.frame"
## 
## $row.names
## [1] 1 2 3 4 5
## 
## $names
## [1] "x" "y"
\end{verbatim}

\begin{Shaded}
\begin{Highlighting}[]
\NormalTok{df }\OtherTok{\textless{}{-}} \FunctionTok{data.frame}\NormalTok{(}\AttributeTok{x =} \DecValTok{1}\SpecialCharTok{:}\DecValTok{5}\NormalTok{, }\AttributeTok{y =} \DecValTok{5}\SpecialCharTok{:}\DecValTok{1}\NormalTok{)}
\FunctionTok{typeof}\NormalTok{(df)}
\end{Highlighting}
\end{Shaded}

\begin{verbatim}
## [1] "list"
\end{verbatim}

\begin{Shaded}
\begin{Highlighting}[]
\FunctionTok{attributes}\NormalTok{(df)}
\end{Highlighting}
\end{Shaded}

\begin{verbatim}
## $names
## [1] "x" "y"
## 
## $class
## [1] "data.frame"
## 
## $row.names
## [1] 1 2 3 4 5
\end{verbatim}

\hypertarget{chapter-21-iteration}{%
\section{Chapter 21: Iteration}\label{chapter-21-iteration}}

\hypertarget{prerequisites-1}{%
\subsubsection{21.1.1 Prerequisites}\label{prerequisites-1}}

\begin{Shaded}
\begin{Highlighting}[]
\FunctionTok{library}\NormalTok{(tidyverse)}
\end{Highlighting}
\end{Shaded}

\hypertarget{for-loops}{%
\subsection{21.2 For loops}\label{for-loops}}

Imagine we have this simple tibble:

\begin{Shaded}
\begin{Highlighting}[]
\NormalTok{df }\OtherTok{\textless{}{-}} \FunctionTok{tibble}\NormalTok{(}
  \AttributeTok{a=}\FunctionTok{rnorm}\NormalTok{(}\DecValTok{10}\NormalTok{),}
  \AttributeTok{b=}\FunctionTok{rnorm}\NormalTok{(}\DecValTok{10}\NormalTok{),}
  \AttributeTok{c=}\FunctionTok{rnorm}\NormalTok{(}\DecValTok{10}\NormalTok{),}
  \AttributeTok{d=}\FunctionTok{rnorm}\NormalTok{(}\DecValTok{10}\NormalTok{)}
\NormalTok{)}
\end{Highlighting}
\end{Shaded}

Calculate the median for each column in a tibble

\begin{Shaded}
\begin{Highlighting}[]
\FunctionTok{median}\NormalTok{(df}\SpecialCharTok{$}\NormalTok{a)}
\end{Highlighting}
\end{Shaded}

\begin{verbatim}
## [1] 0.2189312
\end{verbatim}

\begin{Shaded}
\begin{Highlighting}[]
\FunctionTok{median}\NormalTok{(df}\SpecialCharTok{$}\NormalTok{b)}
\end{Highlighting}
\end{Shaded}

\begin{verbatim}
## [1] -0.01030153
\end{verbatim}

\begin{Shaded}
\begin{Highlighting}[]
\FunctionTok{median}\NormalTok{(df}\SpecialCharTok{$}\NormalTok{c)}
\end{Highlighting}
\end{Shaded}

\begin{verbatim}
## [1] -0.3109431
\end{verbatim}

\begin{Shaded}
\begin{Highlighting}[]
\FunctionTok{median}\NormalTok{(df}\SpecialCharTok{$}\NormalTok{d)}
\end{Highlighting}
\end{Shaded}

\begin{verbatim}
## [1] 0.2376858
\end{verbatim}

Calculate the median for each column in the data frame `df' using a for
loop

\begin{Shaded}
\begin{Highlighting}[]
\NormalTok{df}
\end{Highlighting}
\end{Shaded}

\begin{verbatim}
## # A tibble: 10 x 4
##          a       b      c       d
##      <dbl>   <dbl>  <dbl>   <dbl>
##  1 -0.686  -0.145  -0.967  1.66  
##  2  0.282   0.0111 -1.68  -0.827 
##  3 -1.36    0.335   0.236 -0.289 
##  4  0.156   0.380  -0.614  1.08  
##  5  1.06   -0.937  -0.957  0.502 
##  6  1.12   -0.157  -0.448 -1.66  
##  7  1.13   -0.566   0.473  2.23  
##  8  1.77   -0.0317 -0.174 -0.0263
##  9 -0.384   1.07    0.377 -1.29  
## 10 -0.0527  1.62    0.738  0.945
\end{verbatim}

\begin{Shaded}
\begin{Highlighting}[]
\NormalTok{output }\OtherTok{\textless{}{-}} \FunctionTok{vector}\NormalTok{(}\StringTok{"double"}\NormalTok{,}\FunctionTok{ncol}\NormalTok{(df))}
\ControlFlowTok{for}\NormalTok{ (i }\ControlFlowTok{in} \FunctionTok{seq\_along}\NormalTok{(df))\{}
\NormalTok{  output[[i]] }\OtherTok{\textless{}{-}} \FunctionTok{median}\NormalTok{(df[[i]])}
\NormalTok{\}}
\NormalTok{output }\OtherTok{\textless{}{-}} \FunctionTok{tibble}\NormalTok{(output)}
\end{Highlighting}
\end{Shaded}

Demonstrate the behavior of seq\_along and length functions with an
empty vector `y'

\begin{Shaded}
\begin{Highlighting}[]
\NormalTok{y }\OtherTok{\textless{}{-}} \FunctionTok{vector}\NormalTok{(}\StringTok{"double"}\NormalTok{, }\DecValTok{0}\NormalTok{)}
\FunctionTok{seq\_along}\NormalTok{(y)}
\end{Highlighting}
\end{Shaded}

\begin{verbatim}
## integer(0)
\end{verbatim}

\begin{Shaded}
\begin{Highlighting}[]
\CommentTok{\#\textgreater{} integer(0)}
\DecValTok{1}\SpecialCharTok{:}\FunctionTok{length}\NormalTok{(y)}
\end{Highlighting}
\end{Shaded}

\begin{verbatim}
## [1] 1 0
\end{verbatim}

\begin{Shaded}
\begin{Highlighting}[]
\CommentTok{\#\textgreater{} [1] 1 0}
\end{Highlighting}
\end{Shaded}

\hypertarget{v-modifying-an-existing-object}{%
\subsubsection{21.3.1v Modifying an existing
object}\label{v-modifying-an-existing-object}}

Sometimes, you want to use a for loop to modify an existing object. For
example, remember our challenges from functions. We wanted to rescale
every column in a data frame:

\begin{Shaded}
\begin{Highlighting}[]
\FunctionTok{library}\NormalTok{(tidyverse)}

\NormalTok{df }\OtherTok{\textless{}{-}} \FunctionTok{tibble}\NormalTok{(}
  \AttributeTok{a=}\FunctionTok{rnorm}\NormalTok{(}\DecValTok{10}\NormalTok{),}
  \AttributeTok{b=}\FunctionTok{rnorm}\NormalTok{(}\DecValTok{10}\NormalTok{),}
  \AttributeTok{c=}\FunctionTok{rnorm}\NormalTok{(}\DecValTok{10}\NormalTok{),}
  \AttributeTok{d=}\FunctionTok{rnorm}\NormalTok{(}\DecValTok{10}\NormalTok{)}
\NormalTok{)}

\NormalTok{rescale01 }\OtherTok{\textless{}{-}} \ControlFlowTok{function}\NormalTok{(x)\{}
\NormalTok{  rng }\OtherTok{\textless{}{-}} \FunctionTok{range}\NormalTok{(x,}\AttributeTok{na.rm=}\NormalTok{T)}
\NormalTok{  (x}\SpecialCharTok{{-}}\NormalTok{rng[}\DecValTok{1}\NormalTok{])}\SpecialCharTok{/}\NormalTok{(rng[}\DecValTok{2}\NormalTok{]}\SpecialCharTok{{-}}\NormalTok{rng[}\DecValTok{1}\NormalTok{])}
\NormalTok{\}}

\NormalTok{df}\SpecialCharTok{$}\NormalTok{a }\OtherTok{\textless{}{-}} \FunctionTok{rescale01}\NormalTok{(df}\SpecialCharTok{$}\NormalTok{a)}
\NormalTok{df}\SpecialCharTok{$}\NormalTok{b }\OtherTok{\textless{}{-}} \FunctionTok{rescale01}\NormalTok{(df}\SpecialCharTok{$}\NormalTok{b)}
\NormalTok{df}\SpecialCharTok{$}\NormalTok{c }\OtherTok{\textless{}{-}} \FunctionTok{rescale01}\NormalTok{(df}\SpecialCharTok{$}\NormalTok{c)}
\NormalTok{df}\SpecialCharTok{$}\NormalTok{d }\OtherTok{\textless{}{-}} \FunctionTok{rescale01}\NormalTok{(df}\SpecialCharTok{$}\NormalTok{d)}

\NormalTok{df}
\end{Highlighting}
\end{Shaded}

\begin{verbatim}
## # A tibble: 10 x 4
##        a       b     c     d
##    <dbl>   <dbl> <dbl> <dbl>
##  1 0.222 1       0.845 0.930
##  2 1     0.794   0     0.938
##  3 0     0.802   0.999 0.952
##  4 0.344 0.650   0.682 0.346
##  5 0.247 0.373   0.919 0    
##  6 0.283 0.441   0.383 0.575
##  7 0.726 0.530   0.558 1    
##  8 0.580 0       0.144 0.805
##  9 0.361 0.320   1     0.201
## 10 0.126 0.00143 0.233 0.890
\end{verbatim}

\begin{Shaded}
\begin{Highlighting}[]
\ControlFlowTok{for}\NormalTok{ ( i }\ControlFlowTok{in} \FunctionTok{seq\_along}\NormalTok{(df))\{}
\NormalTok{  df[[i]] }\OtherTok{\textless{}{-}} \FunctionTok{rescale01}\NormalTok{(df[[i]])}
\NormalTok{\}}
\end{Highlighting}
\end{Shaded}

\hypertarget{looping-patterns}{%
\subsubsection{21.3.2 Looping patterns}\label{looping-patterns}}

\begin{Shaded}
\begin{Highlighting}[]
\NormalTok{x}
\end{Highlighting}
\end{Shaded}

\begin{verbatim}
## [1] "1969-12-31 20:00:00 EST"
\end{verbatim}

\begin{Shaded}
\begin{Highlighting}[]
\NormalTok{results }\OtherTok{\textless{}{-}} \FunctionTok{vector}\NormalTok{(}\StringTok{"list"}\NormalTok{,}\FunctionTok{length}\NormalTok{(x))}
\FunctionTok{names}\NormalTok{(results) }\OtherTok{\textless{}{-}} \FunctionTok{names}\NormalTok{(x)}
\end{Highlighting}
\end{Shaded}

Demonstrate looping patterns using a for loop to iterate over a list `x'
and store results in a list `results'

\begin{Shaded}
\begin{Highlighting}[]
\ControlFlowTok{for}\NormalTok{(i }\ControlFlowTok{in} \FunctionTok{seq\_along}\NormalTok{(x))\{}
\NormalTok{  name }\OtherTok{\textless{}{-}} \FunctionTok{names}\NormalTok{(x)[[i]]}
\NormalTok{  value }\OtherTok{\textless{}{-}}\NormalTok{ x[[i]]}
\NormalTok{\}}
\end{Highlighting}
\end{Shaded}

\hypertarget{unknown-output-length}{%
\subsubsection{21.3.3 Unknown output
length}\label{unknown-output-length}}

Create a vector `output' with unknown length and store results from a
for loop in it

\begin{Shaded}
\begin{Highlighting}[]
\NormalTok{means }\OtherTok{\textless{}{-}} \FunctionTok{c}\NormalTok{(}\DecValTok{0}\NormalTok{,}\DecValTok{1}\NormalTok{,}\DecValTok{2}\NormalTok{)}

\NormalTok{output }\OtherTok{\textless{}{-}} \FunctionTok{double}\NormalTok{()}
\ControlFlowTok{for}\NormalTok{ (i }\ControlFlowTok{in} \FunctionTok{seq\_along}\NormalTok{(means))\{}
\NormalTok{  n }\OtherTok{\textless{}{-}} \FunctionTok{sample}\NormalTok{(}\DecValTok{100}\NormalTok{,}\DecValTok{1}\NormalTok{)}
\NormalTok{  output }\OtherTok{\textless{}{-}} \FunctionTok{c}\NormalTok{(output,}\FunctionTok{rnorm}\NormalTok{(n,means[[i]]))}
\NormalTok{\}}
\FunctionTok{str}\NormalTok{(output)}
\end{Highlighting}
\end{Shaded}

\begin{verbatim}
##  num [1:128] 0.6158 -0.7805 0.1823 -0.2956 0.0559 ...
\end{verbatim}

\begin{Shaded}
\begin{Highlighting}[]
\NormalTok{output}
\end{Highlighting}
\end{Shaded}

\begin{verbatim}
##   [1]  0.61582674 -0.78047640  0.18228177 -0.29562487  0.05594639 -0.48140807
##   [7] -0.09051582  0.67999728 -1.67943733  1.81396694  0.13586641 -0.73846085
##  [13] -0.61509220 -1.33925687  0.26033659  0.70101585 -1.25424430  1.51570661
##  [19]  0.51895076 -0.34172393 -1.50201189 -0.14116152  0.62951704 -1.12379907
##  [25] -0.01805414  0.26097205  0.73434172 -1.51373852 -0.73372791  1.18037834
##  [31]  0.88929074  0.41874228 -1.65097327 -0.04030223 -0.27045575  0.91525329
##  [37]  0.79578351 -0.99538089 -0.79716664  0.17925505  1.78466357  2.55908675
##  [43]  1.76918766  1.71467776  0.33900961  2.68559765  2.96093865  2.90562907
##  [49]  3.55631911  0.37287058  2.30954520  1.81790636  2.67288953  1.52678392
##  [55]  1.05678780  2.16116583  1.31317637  2.68956504  2.31283857  2.91352838
##  [61]  2.22754887  3.83479471  1.97041409 -0.02015087  3.25686896  2.59372717
##  [67]  2.15779973  1.16100936  1.18081275  1.52723230  0.84426514  3.21334569
##  [73]  2.80462917  2.97049116  2.20622231  2.79844480  1.68555259  4.94365714
##  [79]  1.73746569  2.74940752  2.41660279  1.28411777  2.55209464  1.75773063
##  [85]  2.94131717  2.03039867  1.05504582  1.94878120  2.62598434  1.81928481
##  [91]  1.40844511  3.66986784  1.05186589  1.19823582  3.21195120  2.37373575
##  [97]  1.74204634  2.99961369  2.09534975  1.29671173  0.07008914  0.67379662
## [103]  1.72363310  2.58144972  2.16986318  3.09965159  2.43099757  2.16481770
## [109]  1.73930984  2.21865057  1.93365365  0.68555963  1.34448193  4.73259373
## [115]  2.01458244  2.10469181  2.13343698  1.84188728  0.13505446  2.26946565
## [121]  1.92215927  1.53103667  2.87640200  2.23772969  2.04567187  2.05161209
## [127]  2.86501059  3.37049297
\end{verbatim}

Create a list `out' with unknown length and store results from a for
loop in it

\begin{Shaded}
\begin{Highlighting}[]
\NormalTok{out }\OtherTok{\textless{}{-}} \FunctionTok{vector}\NormalTok{(}\StringTok{"list"}\NormalTok{,}\FunctionTok{length}\NormalTok{(means))}
\ControlFlowTok{for}\NormalTok{ (i }\ControlFlowTok{in} \FunctionTok{seq\_along}\NormalTok{(means))\{}
\NormalTok{  n }\OtherTok{\textless{}{-}} \FunctionTok{sample}\NormalTok{(}\DecValTok{100}\NormalTok{,}\DecValTok{1}\NormalTok{)}
\NormalTok{  out[[i]] }\OtherTok{\textless{}{-}} \FunctionTok{rnorm}\NormalTok{(n,means[[i]])}
\NormalTok{\}}
\FunctionTok{str}\NormalTok{(out)}
\end{Highlighting}
\end{Shaded}

\begin{verbatim}
## List of 3
##  $ : num [1:13] -0.0185 -0.8276 -0.8112 0.0327 -0.8446 ...
##  $ : num [1:27] 0.964 0.522 1.145 0.082 0.185 ...
##  $ : num [1:32] 0.0756 2.4434 1.2202 3.4739 1.2994 ...
\end{verbatim}

\begin{Shaded}
\begin{Highlighting}[]
\FunctionTok{str}\NormalTok{(}\FunctionTok{unlist}\NormalTok{(out))}
\end{Highlighting}
\end{Shaded}

\begin{verbatim}
##  num [1:72] -0.0185 -0.8276 -0.8112 0.0327 -0.8446 ...
\end{verbatim}

\hypertarget{unknown-sequence-length}{%
\subsubsection{21.3.4 Unknown sequence
length}\label{unknown-sequence-length}}

A while loop is also more general than a for loop, because you can
rewrite any for loop as a while loop, but you can't rewrite every while
loop as for loop:

\begin{Shaded}
\begin{Highlighting}[]
\ControlFlowTok{for}\NormalTok{ (i }\ControlFlowTok{in} \FunctionTok{seq\_along}\NormalTok{(x)) \{}
  \CommentTok{\# body}
\NormalTok{\}}

\CommentTok{\# Equivalent to}
\NormalTok{i }\OtherTok{\textless{}{-}} \DecValTok{1}
\ControlFlowTok{while}\NormalTok{ (i }\SpecialCharTok{\textless{}=} \FunctionTok{length}\NormalTok{(x)) \{}
  \CommentTok{\# body}
\NormalTok{  i }\OtherTok{\textless{}{-}}\NormalTok{ i }\SpecialCharTok{+} \DecValTok{1} 
\NormalTok{\}}
\end{Highlighting}
\end{Shaded}

Herhow we could use a while loop to find how many tries it takes to get
three heads in a row:

\begin{Shaded}
\begin{Highlighting}[]
\NormalTok{flip }\OtherTok{\textless{}{-}} \ControlFlowTok{function}\NormalTok{() }\FunctionTok{sample}\NormalTok{(}\FunctionTok{c}\NormalTok{(}\StringTok{"T"}\NormalTok{, }\StringTok{"H"}\NormalTok{), }\DecValTok{1}\NormalTok{)}

\NormalTok{flips }\OtherTok{\textless{}{-}} \DecValTok{0}
\NormalTok{nheads }\OtherTok{\textless{}{-}} \DecValTok{0}

\ControlFlowTok{while}\NormalTok{ (nheads }\SpecialCharTok{\textless{}} \DecValTok{3}\NormalTok{) \{}
  \ControlFlowTok{if}\NormalTok{ (}\FunctionTok{flip}\NormalTok{() }\SpecialCharTok{==} \StringTok{"H"}\NormalTok{) \{}
\NormalTok{    nheads }\OtherTok{\textless{}{-}}\NormalTok{ nheads }\SpecialCharTok{+} \DecValTok{1}
\NormalTok{  \} }\ControlFlowTok{else}\NormalTok{ \{}
\NormalTok{    nheads }\OtherTok{\textless{}{-}} \DecValTok{0}
\NormalTok{  \}}
\NormalTok{  flips }\OtherTok{\textless{}{-}}\NormalTok{ flips }\SpecialCharTok{+} \DecValTok{1}
\NormalTok{\}}
\NormalTok{flips}
\end{Highlighting}
\end{Shaded}

\begin{verbatim}
## [1] 61
\end{verbatim}

\hypertarget{for-loops-vs.-functionals}{%
\subsection{21.4 For loops
vs.~functionals}\label{for-loops-vs.-functionals}}

Compare for loop and functional approaches for calculating column means
in a data frame

\begin{Shaded}
\begin{Highlighting}[]
\NormalTok{df }\OtherTok{\textless{}{-}} \FunctionTok{tibble}\NormalTok{(}
  \AttributeTok{a=}\FunctionTok{rnorm}\NormalTok{(}\DecValTok{10}\NormalTok{),}
  \AttributeTok{b=}\FunctionTok{rnorm}\NormalTok{(}\DecValTok{10}\NormalTok{),}
  \AttributeTok{c=}\FunctionTok{rnorm}\NormalTok{(}\DecValTok{10}\NormalTok{),}
  \AttributeTok{d=}\FunctionTok{rnorm}\NormalTok{(}\DecValTok{10}\NormalTok{)}
\NormalTok{)}
\end{Highlighting}
\end{Shaded}

Using for loop

\begin{Shaded}
\begin{Highlighting}[]
\NormalTok{output }\OtherTok{\textless{}{-}} \FunctionTok{vector}\NormalTok{(}\StringTok{"double"}\NormalTok{,}\FunctionTok{length}\NormalTok{(df))}
\ControlFlowTok{for}\NormalTok{ (i }\ControlFlowTok{in} \FunctionTok{seq\_along}\NormalTok{(df))\{}
\NormalTok{  output[[i]] }\OtherTok{\textless{}{-}} \FunctionTok{mean}\NormalTok{(df[[i]])}
\NormalTok{\}}
\NormalTok{output}
\end{Highlighting}
\end{Shaded}

\begin{verbatim}
## [1] -0.02836029 -0.04503321  0.08110423  0.04142965
\end{verbatim}

Using functional approach with a custom function `col\_mean'

\begin{Shaded}
\begin{Highlighting}[]
\NormalTok{col\_mean }\OtherTok{\textless{}{-}} \ControlFlowTok{function}\NormalTok{(df)\{}
\NormalTok{  output }\OtherTok{\textless{}{-}} \FunctionTok{vector}\NormalTok{(}\StringTok{"double"}\NormalTok{,}\FunctionTok{length}\NormalTok{(df))}
  \ControlFlowTok{for}\NormalTok{ (i }\ControlFlowTok{in} \FunctionTok{seq\_along}\NormalTok{(df))\{}
\NormalTok{    output[i] }\OtherTok{\textless{}{-}} \FunctionTok{mean}\NormalTok{(df[[i]])}
\NormalTok{  \}}
\NormalTok{  output}
\NormalTok{\}}
\end{Highlighting}
\end{Shaded}

Define a function `col\_median' to calculate the median for each column
in the data frame `df'

\begin{Shaded}
\begin{Highlighting}[]
\NormalTok{col\_median }\OtherTok{\textless{}{-}} \ControlFlowTok{function}\NormalTok{(df)\{}
\NormalTok{  output }\OtherTok{\textless{}{-}} \FunctionTok{vector}\NormalTok{(}\StringTok{"double"}\NormalTok{,}\FunctionTok{hh}\NormalTok{(df))}
  \ControlFlowTok{for}\NormalTok{ (i }\ControlFlowTok{in} \FunctionTok{seq\_along}\NormalTok{(df))\{}
\NormalTok{    output[i] }\OtherTok{\textless{}{-}} \FunctionTok{median}\NormalTok{(df[[i]])}
\NormalTok{  \}}
\NormalTok{  output}
\NormalTok{\}}

\NormalTok{col\_sd }\OtherTok{\textless{}{-}} \ControlFlowTok{function}\NormalTok{(df)\{}
\NormalTok{  output }\OtherTok{\textless{}{-}} \FunctionTok{vector}\NormalTok{(}\StringTok{"double"}\NormalTok{,}\FunctionTok{length}\NormalTok{(df))}
  \ControlFlowTok{for}\NormalTok{ (i }\ControlFlowTok{in} \FunctionTok{seq\_along}\NormalTok{(df))\{}
\NormalTok{    output[i] }\OtherTok{\textless{}{-}} \FunctionTok{sd}\NormalTok{(df[[i]])}
\NormalTok{  \}}
\NormalTok{  output}
\NormalTok{\}}

\NormalTok{df}
\end{Highlighting}
\end{Shaded}

\begin{verbatim}
## # A tibble: 10 x 4
##         a      b       c      d
##     <dbl>  <dbl>   <dbl>  <dbl>
##  1 -1.94  -0.908 -1.24    0.961
##  2 -0.557 -0.116 -0.161  -1.61 
##  3 -0.391  0.225 -0.354   0.224
##  4  1.51   0.551  1.20    0.372
##  5 -0.686 -0.323 -0.0188  0.333
##  6  0.181  0.870  1.03    0.117
##  7  1.33   0.547  0.164  -0.123
##  8 -0.417  0.616 -1.14   -0.469
##  9  0.299 -0.894 -1.08   -0.324
## 10  0.386 -1.02   2.42    0.939
\end{verbatim}

Define functions f1, f2, and f3 for calculating different powers of
absolute deviation from the mean

\begin{Shaded}
\begin{Highlighting}[]
\NormalTok{f1 }\OtherTok{\textless{}{-}} \ControlFlowTok{function}\NormalTok{(x) }\FunctionTok{abs}\NormalTok{(x}\SpecialCharTok{{-}}\FunctionTok{mean}\NormalTok{(x))}\SpecialCharTok{\^{}}\DecValTok{1}
\NormalTok{f2 }\OtherTok{\textless{}{-}} \ControlFlowTok{function}\NormalTok{(x) }\FunctionTok{abs}\NormalTok{(x}\SpecialCharTok{{-}}\FunctionTok{mean}\NormalTok{(x))}\SpecialCharTok{\^{}}\DecValTok{2}
\NormalTok{f3 }\OtherTok{\textless{}{-}} \ControlFlowTok{function}\NormalTok{(x) }\FunctionTok{abs}\NormalTok{(x}\SpecialCharTok{{-}}\FunctionTok{mean}\NormalTok{(x))}\SpecialCharTok{\^{}}\DecValTok{3}
\end{Highlighting}
\end{Shaded}

Define a function `f' to calculate the absolute deviation from the mean
raised to a given power `i'

\begin{Shaded}
\begin{Highlighting}[]
\NormalTok{f }\OtherTok{\textless{}{-}} \ControlFlowTok{function}\NormalTok{(x,i) }\FunctionTok{abs}\NormalTok{(x}\SpecialCharTok{{-}}\FunctionTok{mean}\NormalTok{(x))}\SpecialCharTok{\^{}}\NormalTok{i}
\end{Highlighting}
\end{Shaded}

Define a function `col\_summary' to apply a summary function `fun' to
each column of the data frame `df'

\begin{Shaded}
\begin{Highlighting}[]
\NormalTok{col\_summary }\OtherTok{\textless{}{-}} \ControlFlowTok{function}\NormalTok{(df, fun) \{}
\NormalTok{  out }\OtherTok{\textless{}{-}} \FunctionTok{vector}\NormalTok{(}\StringTok{"double"}\NormalTok{, }\FunctionTok{length}\NormalTok{(df))}
  \ControlFlowTok{for}\NormalTok{ (i }\ControlFlowTok{in} \FunctionTok{seq\_along}\NormalTok{(df)) \{}
\NormalTok{    out[i] }\OtherTok{\textless{}{-}} \FunctionTok{fun}\NormalTok{(df[[i]])}
\NormalTok{  \}}
\NormalTok{  out}
\NormalTok{\}}
\FunctionTok{col\_summary}\NormalTok{(df, median)}
\end{Highlighting}
\end{Shaded}

\begin{verbatim}
## [1] -0.10523309  0.05453164 -0.09009023  0.17023910
\end{verbatim}

\begin{Shaded}
\begin{Highlighting}[]
\FunctionTok{col\_summary}\NormalTok{(df, mean)}
\end{Highlighting}
\end{Shaded}

\begin{verbatim}
## [1] -0.02836029 -0.04503321  0.08110423  0.04142965
\end{verbatim}

Demonstrate the use of `map\_dbl' from the `purrr' package to apply a
function to each column of the data frame `df'

\begin{Shaded}
\begin{Highlighting}[]
\FunctionTok{library}\NormalTok{(purrr)}
\FunctionTok{head}\NormalTok{(df)}
\end{Highlighting}
\end{Shaded}

\begin{verbatim}
## # A tibble: 6 x 4
##        a      b       c      d
##    <dbl>  <dbl>   <dbl>  <dbl>
## 1 -1.94  -0.908 -1.24    0.961
## 2 -0.557 -0.116 -0.161  -1.61 
## 3 -0.391  0.225 -0.354   0.224
## 4  1.51   0.551  1.20    0.372
## 5 -0.686 -0.323 -0.0188  0.333
## 6  0.181  0.870  1.03    0.117
\end{verbatim}

\begin{Shaded}
\begin{Highlighting}[]
\CommentTok{\# Reference {-} for loop()}
\NormalTok{output }\OtherTok{\textless{}{-}} \FunctionTok{vector}\NormalTok{(}\StringTok{"double"}\NormalTok{,}\FunctionTok{length}\NormalTok{(df))}
\ControlFlowTok{for}\NormalTok{ (i }\ControlFlowTok{in} \FunctionTok{seq\_along}\NormalTok{(df))\{}
\NormalTok{  output[[i]] }\OtherTok{\textless{}{-}} \FunctionTok{mean}\NormalTok{(df[[i]])}
\NormalTok{\}}
\NormalTok{output}
\end{Highlighting}
\end{Shaded}

\begin{verbatim}
## [1] -0.02836029 -0.04503321  0.08110423  0.04142965
\end{verbatim}

\begin{Shaded}
\begin{Highlighting}[]
\FunctionTok{map\_dbl}\NormalTok{(df,mean)}
\end{Highlighting}
\end{Shaded}

\begin{verbatim}
##           a           b           c           d 
## -0.02836029 -0.04503321  0.08110423  0.04142965
\end{verbatim}

\begin{Shaded}
\begin{Highlighting}[]
\FunctionTok{map\_dbl}\NormalTok{(df,median)}
\end{Highlighting}
\end{Shaded}

\begin{verbatim}
##           a           b           c           d 
## -0.10523309  0.05453164 -0.09009023  0.17023910
\end{verbatim}

\begin{Shaded}
\begin{Highlighting}[]
\FunctionTok{map\_dbl}\NormalTok{(df,sd)}
\end{Highlighting}
\end{Shaded}

\begin{verbatim}
##         a         b         c         d 
## 1.0106399 0.7110738 1.1786225 0.7487341
\end{verbatim}

\begin{Shaded}
\begin{Highlighting}[]
\NormalTok{df }\SpecialCharTok{\%\textgreater{}\%} \FunctionTok{map\_dbl}\NormalTok{(mean)}
\end{Highlighting}
\end{Shaded}

\begin{verbatim}
##           a           b           c           d 
## -0.02836029 -0.04503321  0.08110423  0.04142965
\end{verbatim}

\begin{Shaded}
\begin{Highlighting}[]
\NormalTok{df }\SpecialCharTok{\%\textgreater{}\%} \FunctionTok{map\_dbl}\NormalTok{(median)}
\end{Highlighting}
\end{Shaded}

\begin{verbatim}
##           a           b           c           d 
## -0.10523309  0.05453164 -0.09009023  0.17023910
\end{verbatim}

\begin{Shaded}
\begin{Highlighting}[]
\NormalTok{df }\SpecialCharTok{\%\textgreater{}\%} \FunctionTok{map\_dbl}\NormalTok{(sd)}
\end{Highlighting}
\end{Shaded}

\begin{verbatim}
##         a         b         c         d 
## 1.0106399 0.7110738 1.1786225 0.7487341
\end{verbatim}

Demonstrate the use of `map\_dbl' from the `purrr' package with
additional arguments

\begin{Shaded}
\begin{Highlighting}[]
\FunctionTok{map\_dbl}\NormalTok{(df,mean,}\AttributeTok{trim=}\FloatTok{0.5}\NormalTok{)}
\end{Highlighting}
\end{Shaded}

\begin{verbatim}
##           a           b           c           d 
## -0.10523309  0.05453164 -0.09009023  0.17023910
\end{verbatim}

Demonstrate the use of `map\_int' from the `purrr' package to apply a
function that returns integers to each element of a list

\begin{Shaded}
\begin{Highlighting}[]
\NormalTok{z }\OtherTok{\textless{}{-}} \FunctionTok{list}\NormalTok{(}\AttributeTok{x=}\DecValTok{1}\SpecialCharTok{:}\DecValTok{3}\NormalTok{,}\AttributeTok{y=}\DecValTok{4}\SpecialCharTok{:}\DecValTok{5}\NormalTok{)}
\NormalTok{z}
\end{Highlighting}
\end{Shaded}

\begin{verbatim}
## $x
## [1] 1 2 3
## 
## $y
## [1] 4 5
\end{verbatim}

\begin{Shaded}
\begin{Highlighting}[]
\FunctionTok{map\_int}\NormalTok{(z,length)}
\end{Highlighting}
\end{Shaded}

\begin{verbatim}
## x y 
## 3 2
\end{verbatim}

\hypertarget{shortcuts}{%
\subsubsection{21.5.1 Shortcuts}\label{shortcuts}}

Demonstrate the use of `safely' from the `purrr' package to create a
safe version of a function

\begin{Shaded}
\begin{Highlighting}[]
\NormalTok{safe\_log }\OtherTok{\textless{}{-}} \FunctionTok{safely}\NormalTok{(log)}
\FunctionTok{str}\NormalTok{(}\FunctionTok{safe\_log}\NormalTok{(}\DecValTok{10}\NormalTok{))}
\end{Highlighting}
\end{Shaded}

\begin{verbatim}
## List of 2
##  $ result: num 2.3
##  $ error : NULL
\end{verbatim}

\begin{Shaded}
\begin{Highlighting}[]
\FunctionTok{str}\NormalTok{(}\FunctionTok{safe\_log}\NormalTok{(}\StringTok{"a"}\NormalTok{))}
\end{Highlighting}
\end{Shaded}

\begin{verbatim}
## List of 2
##  $ result: NULL
##  $ error :List of 2
##   ..$ message: chr "non-numeric argument to mathematical function"
##   ..$ call   : language .Primitive("log")(x, base)
##   ..- attr(*, "class")= chr [1:3] "simpleError" "error" "condition"
\end{verbatim}

Demonstrate the use of `map' from the `purrr' package with `safely' to
apply a safe version of a function to each element of a list

\begin{Shaded}
\begin{Highlighting}[]
\NormalTok{x }\OtherTok{\textless{}{-}} \FunctionTok{list}\NormalTok{(}\DecValTok{1}\NormalTok{,}\DecValTok{10}\NormalTok{,}\StringTok{"a"}\NormalTok{)}
\NormalTok{y }\OtherTok{\textless{}{-}}\NormalTok{ x }\SpecialCharTok{\%\textgreater{}\%} \FunctionTok{map}\NormalTok{(}\FunctionTok{safely}\NormalTok{(log))}
\FunctionTok{str}\NormalTok{(y)}
\end{Highlighting}
\end{Shaded}

\begin{verbatim}
## List of 3
##  $ :List of 2
##   ..$ result: num 0
##   ..$ error : NULL
##  $ :List of 2
##   ..$ result: num 2.3
##   ..$ error : NULL
##  $ :List of 2
##   ..$ result: NULL
##   ..$ error :List of 2
##   .. ..$ message: chr "non-numeric argument to mathematical function"
##   .. ..$ call   : language .Primitive("log")(x, base)
##   .. ..- attr(*, "class")= chr [1:3] "simpleError" "error" "condition"
\end{verbatim}

Demonstrate the use of `transpose' from the `purrr' package to transpose
a list of lists

\begin{Shaded}
\begin{Highlighting}[]
\NormalTok{y }\OtherTok{\textless{}{-}}\NormalTok{ x }\SpecialCharTok{\%\textgreater{}\%} \FunctionTok{transpose}\NormalTok{()}
\FunctionTok{str}\NormalTok{(y)}
\end{Highlighting}
\end{Shaded}

\begin{verbatim}
## List of 1
##  $ :List of 3
##   ..$ : num 1
##   ..$ : num 10
##   ..$ : chr "a"
\end{verbatim}

Demonstrate the use of error handling with `map\_lgl' and `is\_null'
from the `purrr' package

\begin{Shaded}
\begin{Highlighting}[]
\NormalTok{is\_ok }\OtherTok{\textless{}{-}}\NormalTok{ y}\SpecialCharTok{$}\NormalTok{error }\SpecialCharTok{\%\textgreater{}\%} \FunctionTok{map\_lgl}\NormalTok{(is\_null)}
\NormalTok{x[}\SpecialCharTok{!}\NormalTok{is\_ok]}
\end{Highlighting}
\end{Shaded}

\begin{verbatim}
## list()
\end{verbatim}

\begin{Shaded}
\begin{Highlighting}[]
\CommentTok{\# y$result[is\_ok] \%\textgreater{}\% flatten\_dbl()}
\end{Highlighting}
\end{Shaded}

Purrr provides two other useful adverbs:

\begin{Shaded}
\begin{Highlighting}[]
\NormalTok{x }\OtherTok{\textless{}{-}} \FunctionTok{list}\NormalTok{(}\DecValTok{1}\NormalTok{,}\DecValTok{10}\NormalTok{,}\StringTok{"a"}\NormalTok{)}
\NormalTok{x }\SpecialCharTok{\%\textgreater{}\%} \FunctionTok{map\_dbl}\NormalTok{(}\FunctionTok{possibly}\NormalTok{(log,}\ConstantTok{NA\_real\_}\NormalTok{))}
\end{Highlighting}
\end{Shaded}

\begin{verbatim}
## [1] 0.000000 2.302585       NA
\end{verbatim}

Demonstrate the use of `quietly' from the `purrr' package to suppress
errors and return results with warnings

\begin{Shaded}
\begin{Highlighting}[]
\NormalTok{x }\OtherTok{\textless{}{-}} \FunctionTok{list}\NormalTok{(}\DecValTok{1}\NormalTok{,}\SpecialCharTok{{-}}\DecValTok{1}\NormalTok{)}
\NormalTok{x }\SpecialCharTok{\%\textgreater{}\%} \FunctionTok{map}\NormalTok{(}\FunctionTok{quietly}\NormalTok{(log)) }\SpecialCharTok{\%\textgreater{}\%} \FunctionTok{str}\NormalTok{()}
\end{Highlighting}
\end{Shaded}

\begin{verbatim}
## List of 2
##  $ :List of 4
##   ..$ result  : num 0
##   ..$ output  : chr ""
##   ..$ warnings: chr(0) 
##   ..$ messages: chr(0) 
##  $ :List of 4
##   ..$ result  : num NaN
##   ..$ output  : chr ""
##   ..$ warnings: chr "NaNs produced"
##   ..$ messages: chr(0)
\end{verbatim}

\hypertarget{mapping-over-multiple-arguments}{%
\subsection{21.7 Mapping over multiple
arguments}\label{mapping-over-multiple-arguments}}

Generate random numbers from normal distributions with different means
using `map' from the `purrr' package

\begin{Shaded}
\begin{Highlighting}[]
\NormalTok{mu }\OtherTok{\textless{}{-}} \FunctionTok{list}\NormalTok{(}\DecValTok{5}\NormalTok{,}\DecValTok{10}\NormalTok{,}\SpecialCharTok{{-}}\DecValTok{3}\NormalTok{)}
\NormalTok{mu }\SpecialCharTok{\%\textgreater{}\%} 
  \FunctionTok{map}\NormalTok{(rnorm,}\AttributeTok{n=}\DecValTok{5}\NormalTok{) }\SpecialCharTok{\%\textgreater{}\%} 
  \FunctionTok{str}\NormalTok{()}
\end{Highlighting}
\end{Shaded}

\begin{verbatim}
## List of 3
##  $ : num [1:5] 3.54 5.85 4.5 5.36 4.73
##  $ : num [1:5] 11.23 8.14 10.77 10.15 10.92
##  $ : num [1:5] -2.97 -2.71 -1.68 -3.33 -2.99
\end{verbatim}

Generate random numbers from normal distributions with different means
and standard deviations using `map2' from the `purrr' package

\begin{Shaded}
\begin{Highlighting}[]
\NormalTok{sigma }\OtherTok{\textless{}{-}} \FunctionTok{list}\NormalTok{(}\DecValTok{1}\NormalTok{,}\DecValTok{5}\NormalTok{,}\DecValTok{10}\NormalTok{)}
\FunctionTok{seq\_along}\NormalTok{(mu) }\SpecialCharTok{\%\textgreater{}\%} 
  \FunctionTok{map}\NormalTok{(}\SpecialCharTok{\textasciitilde{}}\FunctionTok{rnorm}\NormalTok{(}\DecValTok{5}\NormalTok{,mu[[.]],sigma[[.]])) }\SpecialCharTok{\%\textgreater{}\%} 
  \FunctionTok{str}\NormalTok{()}
\end{Highlighting}
\end{Shaded}

\begin{verbatim}
## List of 3
##  $ : num [1:5] 5.28 4.52 5.04 5.29 4.52
##  $ : num [1:5] 9.39 13.24 12.99 12.34 10.81
##  $ : num [1:5] -5.2 14.81 -4.32 -3.04 7.05
\end{verbatim}

Define a custom `map2' function to apply a binary function to
corresponding elements of two lists

\begin{Shaded}
\begin{Highlighting}[]
\FunctionTok{map2}\NormalTok{(mu,sigma,rnorm,}\AttributeTok{n=}\DecValTok{5}\NormalTok{) }\SpecialCharTok{\%\textgreater{}\%} \FunctionTok{str}\NormalTok{()}
\end{Highlighting}
\end{Shaded}

\begin{verbatim}
## List of 3
##  $ : num [1:5] 4.29 5.07 4.91 4.45 5.75
##  $ : num [1:5] 1.59 7.94 14.17 11.87 8.96
##  $ : num [1:5] -10.602 -0.224 11.694 2.94 18.341
\end{verbatim}

\begin{Shaded}
\begin{Highlighting}[]
\NormalTok{map2 }\OtherTok{\textless{}{-}} \ControlFlowTok{function}\NormalTok{(x,y,f,...)\{}
\NormalTok{  out }\OtherTok{\textless{}{-}} \FunctionTok{vector}\NormalTok{(}\StringTok{"list"}\NormalTok{,}\FunctionTok{length}\NormalTok{(x))}
  \ControlFlowTok{for}\NormalTok{ (i }\ControlFlowTok{in} \FunctionTok{seq\_along}\NormalTok{(x))\{}
\NormalTok{    out[[i]] }\OtherTok{\textless{}{-}} \FunctionTok{f}\NormalTok{(x[[i]],y[[i]],...)}
\NormalTok{  \}}
\NormalTok{  out}
\NormalTok{\}}
\end{Highlighting}
\end{Shaded}

Apply a function to corresponding elements of multiple lists using
`pmap' from the `purrr' package

\begin{Shaded}
\begin{Highlighting}[]
\FunctionTok{library}\NormalTok{(magrittr)}
\FunctionTok{library}\NormalTok{(purrr)}

\NormalTok{n }\OtherTok{\textless{}{-}} \FunctionTok{list}\NormalTok{(}\DecValTok{1}\NormalTok{,}\DecValTok{3}\NormalTok{,}\DecValTok{5}\NormalTok{)}
\NormalTok{args1 }\OtherTok{\textless{}{-}} \FunctionTok{list}\NormalTok{(n,mu,sigma)}
\NormalTok{args1 }\SpecialCharTok{\%\textgreater{}\%} 
  \FunctionTok{pmap}\NormalTok{(rnorm) }\SpecialCharTok{\%\textgreater{}\%} 
  \FunctionTok{str}\NormalTok{()}
\end{Highlighting}
\end{Shaded}

\begin{verbatim}
## List of 3
##  $ : num 5.09
##  $ : num [1:3] 5.84 18.96 9.48
##  $ : num [1:5] -0.18 12.54 -3.36 -5.34 4.03
\end{verbatim}

Apply a function to corresponding elements of multiple lists with named
parameters using `pmap' from the `purrr' package

\begin{Shaded}
\begin{Highlighting}[]
\NormalTok{args2 }\OtherTok{\textless{}{-}} \FunctionTok{list}\NormalTok{(}\AttributeTok{mean=}\NormalTok{mu, }\AttributeTok{sd=}\NormalTok{sigma,}\AttributeTok{n=}\NormalTok{n)}
\NormalTok{args2 }\SpecialCharTok{\%\textgreater{}\%} 
  \FunctionTok{pmap}\NormalTok{(rnorm) }\SpecialCharTok{\%\textgreater{}\%} 
  \FunctionTok{str}\NormalTok{()}
\end{Highlighting}
\end{Shaded}

\begin{verbatim}
## List of 3
##  $ : num 4.45
##  $ : num [1:3] 10.2 15.9 10.8
##  $ : num [1:5] 5.09 -25.56 3.17 -11.85 -2.14
\end{verbatim}

Apply a function to corresponding rows of a data frame using `pmap' from
the `purrr' package with a tibble

\begin{Shaded}
\begin{Highlighting}[]
\FunctionTok{library}\NormalTok{(tidyverse)}
\NormalTok{parms }\OtherTok{\textless{}{-}} \FunctionTok{tribble}\NormalTok{(}
  \SpecialCharTok{\textasciitilde{}}\NormalTok{mean,}\SpecialCharTok{\textasciitilde{}}\NormalTok{sd,}\SpecialCharTok{\textasciitilde{}}\NormalTok{n,}
  \DecValTok{5}\NormalTok{,}\DecValTok{1}\NormalTok{,}\DecValTok{1}\NormalTok{,}
  \DecValTok{10}\NormalTok{,}\DecValTok{5}\NormalTok{,}\DecValTok{3}\NormalTok{,}
  \SpecialCharTok{{-}}\DecValTok{3}\NormalTok{,}\DecValTok{10}\NormalTok{,}\DecValTok{5}
\NormalTok{)}

\NormalTok{parms }\SpecialCharTok{\%\textgreater{}\%} 
  \FunctionTok{pmap}\NormalTok{(rnorm)}
\end{Highlighting}
\end{Shaded}

\begin{verbatim}
## [[1]]
## [1] 6.114783
## 
## [[2]]
## [1] 12.98033 11.20400 17.56886
## 
## [[3]]
## [1] -4.110093 -5.752027 -2.416376 -7.106618 -6.984135
\end{verbatim}

\hypertarget{involing-different-functions}{%
\subsubsection{21.7.1 Involing different
functions}\label{involing-different-functions}}

Invoke different functions with different parameters using `invoke\_map'
from the `purrr' package

\begin{Shaded}
\begin{Highlighting}[]
\NormalTok{f }\OtherTok{\textless{}{-}} \FunctionTok{c}\NormalTok{(}\StringTok{"runif"}\NormalTok{,}\StringTok{"rnorm"}\NormalTok{,}\StringTok{"rpois"}\NormalTok{)}
\NormalTok{param }\OtherTok{\textless{}{-}} \FunctionTok{list}\NormalTok{(}
  \FunctionTok{list}\NormalTok{(}\AttributeTok{min=}\SpecialCharTok{{-}}\DecValTok{1}\NormalTok{,}\AttributeTok{max=}\DecValTok{1}\NormalTok{),}
  \FunctionTok{list}\NormalTok{(}\AttributeTok{sd=}\DecValTok{5}\NormalTok{),}
  \FunctionTok{list}\NormalTok{(}\AttributeTok{lambda=}\DecValTok{10}\NormalTok{)}
\NormalTok{)}

\NormalTok{f}
\end{Highlighting}
\end{Shaded}

\begin{verbatim}
## [1] "runif" "rnorm" "rpois"
\end{verbatim}

\begin{Shaded}
\begin{Highlighting}[]
\NormalTok{param}
\end{Highlighting}
\end{Shaded}

\begin{verbatim}
## [[1]]
## [[1]]$min
## [1] -1
## 
## [[1]]$max
## [1] 1
## 
## 
## [[2]]
## [[2]]$sd
## [1] 5
## 
## 
## [[3]]
## [[3]]$lambda
## [1] 10
\end{verbatim}

To handle this case, you can use \texttt{invoke\_map()}:

\begin{Shaded}
\begin{Highlighting}[]
\FunctionTok{invoke\_map}\NormalTok{(f,param,}\AttributeTok{n=}\DecValTok{5}\NormalTok{) }\SpecialCharTok{\%\textgreater{}\%} 
  \FunctionTok{str}\NormalTok{()}
\end{Highlighting}
\end{Shaded}

\begin{verbatim}
## Warning: `invoke_map()` was deprecated in purrr 1.0.0.
## i Please use map() + exec() instead.
## This warning is displayed once every 8 hours.
## Call `lifecycle::last_lifecycle_warnings()` to see where this warning was
## generated.
\end{verbatim}

\begin{verbatim}
## List of 3
##  $ : num [1:5] 0.7019 -0.0297 0.0484 -0.2979 0.8275
##  $ : num [1:5] -2.45 7.19 -3.24 3.81 8.34
##  $ : int [1:5] 14 9 14 22 14
\end{verbatim}

Invoke different functions with different parameters using `pmap' from
the `purrr' package and a tibble

\begin{Shaded}
\begin{Highlighting}[]
\NormalTok{sim }\OtherTok{\textless{}{-}} \FunctionTok{tribble}\NormalTok{(}
  \SpecialCharTok{\textasciitilde{}}\NormalTok{f,      }\SpecialCharTok{\textasciitilde{}}\NormalTok{params,}
  \StringTok{"runif"}\NormalTok{, }\FunctionTok{list}\NormalTok{(}\AttributeTok{min =} \SpecialCharTok{{-}}\DecValTok{1}\NormalTok{, }\AttributeTok{max =} \DecValTok{1}\NormalTok{),}
  \StringTok{"rnorm"}\NormalTok{, }\FunctionTok{list}\NormalTok{(}\AttributeTok{sd =} \DecValTok{5}\NormalTok{),}
  \StringTok{"rpois"}\NormalTok{, }\FunctionTok{list}\NormalTok{(}\AttributeTok{lambda =} \DecValTok{10}\NormalTok{)}
\NormalTok{)}
\NormalTok{sim }\SpecialCharTok{\%\textgreater{}\%} 
  \FunctionTok{mutate}\NormalTok{(}\AttributeTok{sim =} \FunctionTok{invoke\_map}\NormalTok{(f, params, }\AttributeTok{n =} \DecValTok{10}\NormalTok{))}
\end{Highlighting}
\end{Shaded}

\begin{verbatim}
## # A tibble: 3 x 3
##   f     params           sim       
##   <chr> <list>           <list>    
## 1 runif <named list [2]> <dbl [10]>
## 2 rnorm <named list [1]> <dbl [10]>
## 3 rpois <named list [1]> <int [10]>
\end{verbatim}

\hypertarget{walk}{%
\subsection{21.8 Walk}\label{walk}}

Perform side effects without returning a value for each element of a
list using `walk' from the `purrr' package

\begin{Shaded}
\begin{Highlighting}[]
\NormalTok{x }\OtherTok{\textless{}{-}} \FunctionTok{list}\NormalTok{(}\DecValTok{1}\NormalTok{,}\StringTok{"a"}\NormalTok{,}\DecValTok{3}\NormalTok{)}
\NormalTok{x }\SpecialCharTok{\%\textgreater{}\%} 
  \FunctionTok{walk}\NormalTok{(print)}
\end{Highlighting}
\end{Shaded}

\begin{verbatim}
## [1] 1
## [1] "a"
## [1] 3
\end{verbatim}

Perform side effects on each element of a list using `walk' from the
`purrr' package, then save the results

\begin{Shaded}
\begin{Highlighting}[]
\FunctionTok{library}\NormalTok{(ggplot2)}
\NormalTok{plots }\OtherTok{\textless{}{-}}\NormalTok{ mtcars }\SpecialCharTok{\%\textgreater{}\%} 
  \FunctionTok{split}\NormalTok{(.}\SpecialCharTok{$}\NormalTok{cyl) }\SpecialCharTok{\%\textgreater{}\%} 
  \FunctionTok{map}\NormalTok{(}\SpecialCharTok{\textasciitilde{}}\FunctionTok{ggplot}\NormalTok{(., }\FunctionTok{aes}\NormalTok{(mpg, wt)) }\SpecialCharTok{+} \FunctionTok{geom\_point}\NormalTok{())}
\NormalTok{paths }\OtherTok{\textless{}{-}}\NormalTok{ stringr}\SpecialCharTok{::}\FunctionTok{str\_c}\NormalTok{(}\FunctionTok{names}\NormalTok{(plots), }\StringTok{".pdf"}\NormalTok{)}

\FunctionTok{pwalk}\NormalTok{(}\FunctionTok{list}\NormalTok{(paths, plots), ggsave, }\AttributeTok{path =} \FunctionTok{tempdir}\NormalTok{())}
\end{Highlighting}
\end{Shaded}

\begin{verbatim}
## Saving 6.5 x 4.5 in image
## Saving 6.5 x 4.5 in image
## Saving 6.5 x 4.5 in image
\end{verbatim}

Retain or remove elements of a list based on a predicate function using
`keep' and `discard' from the `purrr' package

\begin{Shaded}
\begin{Highlighting}[]
\NormalTok{iris }\SpecialCharTok{\%\textgreater{}\%} 
  \FunctionTok{keep}\NormalTok{(is.factor) }\SpecialCharTok{\%\textgreater{}\%} 
  \FunctionTok{str}\NormalTok{()}
\end{Highlighting}
\end{Shaded}

\begin{verbatim}
## 'data.frame':    150 obs. of  1 variable:
##  $ Species: Factor w/ 3 levels "setosa","versicolor",..: 1 1 1 1 1 1 1 1 1 1 ...
\end{verbatim}

\begin{Shaded}
\begin{Highlighting}[]
\NormalTok{iris }\SpecialCharTok{\%\textgreater{}\%} 
  \FunctionTok{discard}\NormalTok{(is.factor) }\SpecialCharTok{\%\textgreater{}\%}
  \FunctionTok{str}\NormalTok{()}
\end{Highlighting}
\end{Shaded}

\begin{verbatim}
## 'data.frame':    150 obs. of  4 variables:
##  $ Sepal.Length: num  5.1 4.9 4.7 4.6 5 5.4 4.6 5 4.4 4.9 ...
##  $ Sepal.Width : num  3.5 3 3.2 3.1 3.6 3.9 3.4 3.4 2.9 3.1 ...
##  $ Petal.Length: num  1.4 1.4 1.3 1.5 1.4 1.7 1.4 1.5 1.4 1.5 ...
##  $ Petal.Width : num  0.2 0.2 0.2 0.2 0.2 0.4 0.3 0.2 0.2 0.1 ...
\end{verbatim}

\begin{Shaded}
\begin{Highlighting}[]
\FunctionTok{library}\NormalTok{(tidyverse)}
\FunctionTok{library}\NormalTok{(magrittr)}
\end{Highlighting}
\end{Shaded}

\hypertarget{reduce-and-accumulate}{%
\subsubsection{21.9.2 Reduce and
accumulate}\label{reduce-and-accumulate}}

Iteratively combine elements of a list using a binary function with
`reduce' from the `purrr' package

\begin{Shaded}
\begin{Highlighting}[]
\NormalTok{dfs }\OtherTok{\textless{}{-}} \FunctionTok{list}\NormalTok{(}
  \AttributeTok{age=}\FunctionTok{tibble}\NormalTok{(}\AttributeTok{name=}\StringTok{"John"}\NormalTok{,}\AttributeTok{age=}\DecValTok{30}\NormalTok{),}
  \AttributeTok{sex=}\FunctionTok{tibble}\NormalTok{(}\AttributeTok{name=}\FunctionTok{c}\NormalTok{(}\StringTok{"John"}\NormalTok{,}\StringTok{"Mary"}\NormalTok{),}\AttributeTok{sex=}\FunctionTok{c}\NormalTok{(}\StringTok{"M"}\NormalTok{,}\StringTok{"F"}\NormalTok{)),}
  \AttributeTok{trt=}\FunctionTok{tibble}\NormalTok{(}\AttributeTok{name=}\StringTok{"Mary"}\NormalTok{,}\AttributeTok{treatment=}\StringTok{"A"}\NormalTok{)}
\NormalTok{)}

\NormalTok{dfs }\SpecialCharTok{\%\textgreater{}\%} \FunctionTok{reduce}\NormalTok{(full\_join)}
\end{Highlighting}
\end{Shaded}

\begin{verbatim}
## Joining with `by = join_by(name)`
## Joining with `by = join_by(name)`
\end{verbatim}

\begin{verbatim}
## # A tibble: 2 x 4
##   name    age sex   treatment
##   <chr> <dbl> <chr> <chr>    
## 1 John     30 M     <NA>     
## 2 Mary     NA F     A
\end{verbatim}

Find the intersection of multiple vectors using `reduce' from the
`purrr' package

\begin{Shaded}
\begin{Highlighting}[]
\NormalTok{vs }\OtherTok{\textless{}{-}} \FunctionTok{list}\NormalTok{(}
  \FunctionTok{c}\NormalTok{(}\DecValTok{1}\NormalTok{,}\DecValTok{3}\NormalTok{,}\DecValTok{5}\NormalTok{,}\DecValTok{6}\NormalTok{,}\DecValTok{10}\NormalTok{),}
  \FunctionTok{c}\NormalTok{(}\DecValTok{1}\NormalTok{,}\DecValTok{2}\NormalTok{,}\DecValTok{3}\NormalTok{,}\DecValTok{7}\NormalTok{,}\DecValTok{8}\NormalTok{,}\DecValTok{10}\NormalTok{),}
  \FunctionTok{c}\NormalTok{(}\DecValTok{1}\NormalTok{,}\DecValTok{2}\NormalTok{,}\DecValTok{3}\NormalTok{,}\DecValTok{4}\NormalTok{,}\DecValTok{8}\NormalTok{,}\DecValTok{9}\NormalTok{,}\DecValTok{10}\NormalTok{)}
\NormalTok{)}
\NormalTok{vs }\SpecialCharTok{\%\textgreater{}\%} \FunctionTok{reduce}\NormalTok{(intersect)}
\end{Highlighting}
\end{Shaded}

\begin{verbatim}
## [1]  1  3 10
\end{verbatim}

Iteratively apply a function to elements of a list using `accumulate'
from the `purrr' package

\begin{Shaded}
\begin{Highlighting}[]
\NormalTok{x }\OtherTok{\textless{}{-}} \FunctionTok{sample}\NormalTok{(}\DecValTok{10}\NormalTok{)}
\NormalTok{x}
\end{Highlighting}
\end{Shaded}

\begin{verbatim}
##  [1]  2  8  3  4 10  1  7  5  6  9
\end{verbatim}

\begin{Shaded}
\begin{Highlighting}[]
\NormalTok{x }\SpecialCharTok{\%\textgreater{}\%} \FunctionTok{accumulate}\NormalTok{(}\StringTok{\textasciigrave{}}\AttributeTok{+}\StringTok{\textasciigrave{}}\NormalTok{)}
\end{Highlighting}
\end{Shaded}

\begin{verbatim}
##  [1]  2 10 13 17 27 28 35 40 46 55
\end{verbatim}

\end{document}
