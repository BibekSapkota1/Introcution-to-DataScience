% Options for packages loaded elsewhere
\PassOptionsToPackage{unicode}{hyperref}
\PassOptionsToPackage{hyphens}{url}
%
\documentclass[
]{article}
\usepackage{amsmath,amssymb}
\usepackage{iftex}
\ifPDFTeX
  \usepackage[T1]{fontenc}
  \usepackage[utf8]{inputenc}
  \usepackage{textcomp} % provide euro and other symbols
\else % if luatex or xetex
  \usepackage{unicode-math} % this also loads fontspec
  \defaultfontfeatures{Scale=MatchLowercase}
  \defaultfontfeatures[\rmfamily]{Ligatures=TeX,Scale=1}
\fi
\usepackage{lmodern}
\ifPDFTeX\else
  % xetex/luatex font selection
\fi
% Use upquote if available, for straight quotes in verbatim environments
\IfFileExists{upquote.sty}{\usepackage{upquote}}{}
\IfFileExists{microtype.sty}{% use microtype if available
  \usepackage[]{microtype}
  \UseMicrotypeSet[protrusion]{basicmath} % disable protrusion for tt fonts
}{}
\makeatletter
\@ifundefined{KOMAClassName}{% if non-KOMA class
  \IfFileExists{parskip.sty}{%
    \usepackage{parskip}
  }{% else
    \setlength{\parindent}{0pt}
    \setlength{\parskip}{6pt plus 2pt minus 1pt}}
}{% if KOMA class
  \KOMAoptions{parskip=half}}
\makeatother
\usepackage{xcolor}
\usepackage[margin=1in]{geometry}
\usepackage{color}
\usepackage{fancyvrb}
\newcommand{\VerbBar}{|}
\newcommand{\VERB}{\Verb[commandchars=\\\{\}]}
\DefineVerbatimEnvironment{Highlighting}{Verbatim}{commandchars=\\\{\}}
% Add ',fontsize=\small' for more characters per line
\usepackage{framed}
\definecolor{shadecolor}{RGB}{248,248,248}
\newenvironment{Shaded}{\begin{snugshade}}{\end{snugshade}}
\newcommand{\AlertTok}[1]{\textcolor[rgb]{0.94,0.16,0.16}{#1}}
\newcommand{\AnnotationTok}[1]{\textcolor[rgb]{0.56,0.35,0.01}{\textbf{\textit{#1}}}}
\newcommand{\AttributeTok}[1]{\textcolor[rgb]{0.13,0.29,0.53}{#1}}
\newcommand{\BaseNTok}[1]{\textcolor[rgb]{0.00,0.00,0.81}{#1}}
\newcommand{\BuiltInTok}[1]{#1}
\newcommand{\CharTok}[1]{\textcolor[rgb]{0.31,0.60,0.02}{#1}}
\newcommand{\CommentTok}[1]{\textcolor[rgb]{0.56,0.35,0.01}{\textit{#1}}}
\newcommand{\CommentVarTok}[1]{\textcolor[rgb]{0.56,0.35,0.01}{\textbf{\textit{#1}}}}
\newcommand{\ConstantTok}[1]{\textcolor[rgb]{0.56,0.35,0.01}{#1}}
\newcommand{\ControlFlowTok}[1]{\textcolor[rgb]{0.13,0.29,0.53}{\textbf{#1}}}
\newcommand{\DataTypeTok}[1]{\textcolor[rgb]{0.13,0.29,0.53}{#1}}
\newcommand{\DecValTok}[1]{\textcolor[rgb]{0.00,0.00,0.81}{#1}}
\newcommand{\DocumentationTok}[1]{\textcolor[rgb]{0.56,0.35,0.01}{\textbf{\textit{#1}}}}
\newcommand{\ErrorTok}[1]{\textcolor[rgb]{0.64,0.00,0.00}{\textbf{#1}}}
\newcommand{\ExtensionTok}[1]{#1}
\newcommand{\FloatTok}[1]{\textcolor[rgb]{0.00,0.00,0.81}{#1}}
\newcommand{\FunctionTok}[1]{\textcolor[rgb]{0.13,0.29,0.53}{\textbf{#1}}}
\newcommand{\ImportTok}[1]{#1}
\newcommand{\InformationTok}[1]{\textcolor[rgb]{0.56,0.35,0.01}{\textbf{\textit{#1}}}}
\newcommand{\KeywordTok}[1]{\textcolor[rgb]{0.13,0.29,0.53}{\textbf{#1}}}
\newcommand{\NormalTok}[1]{#1}
\newcommand{\OperatorTok}[1]{\textcolor[rgb]{0.81,0.36,0.00}{\textbf{#1}}}
\newcommand{\OtherTok}[1]{\textcolor[rgb]{0.56,0.35,0.01}{#1}}
\newcommand{\PreprocessorTok}[1]{\textcolor[rgb]{0.56,0.35,0.01}{\textit{#1}}}
\newcommand{\RegionMarkerTok}[1]{#1}
\newcommand{\SpecialCharTok}[1]{\textcolor[rgb]{0.81,0.36,0.00}{\textbf{#1}}}
\newcommand{\SpecialStringTok}[1]{\textcolor[rgb]{0.31,0.60,0.02}{#1}}
\newcommand{\StringTok}[1]{\textcolor[rgb]{0.31,0.60,0.02}{#1}}
\newcommand{\VariableTok}[1]{\textcolor[rgb]{0.00,0.00,0.00}{#1}}
\newcommand{\VerbatimStringTok}[1]{\textcolor[rgb]{0.31,0.60,0.02}{#1}}
\newcommand{\WarningTok}[1]{\textcolor[rgb]{0.56,0.35,0.01}{\textbf{\textit{#1}}}}
\usepackage{graphicx}
\makeatletter
\def\maxwidth{\ifdim\Gin@nat@width>\linewidth\linewidth\else\Gin@nat@width\fi}
\def\maxheight{\ifdim\Gin@nat@height>\textheight\textheight\else\Gin@nat@height\fi}
\makeatother
% Scale images if necessary, so that they will not overflow the page
% margins by default, and it is still possible to overwrite the defaults
% using explicit options in \includegraphics[width, height, ...]{}
\setkeys{Gin}{width=\maxwidth,height=\maxheight,keepaspectratio}
% Set default figure placement to htbp
\makeatletter
\def\fps@figure{htbp}
\makeatother
\setlength{\emergencystretch}{3em} % prevent overfull lines
\providecommand{\tightlist}{%
  \setlength{\itemsep}{0pt}\setlength{\parskip}{0pt}}
\setcounter{secnumdepth}{-\maxdimen} % remove section numbering
\ifLuaTeX
  \usepackage{selnolig}  % disable illegal ligatures
\fi
\IfFileExists{bookmark.sty}{\usepackage{bookmark}}{\usepackage{hyperref}}
\IfFileExists{xurl.sty}{\usepackage{xurl}}{} % add URL line breaks if available
\urlstyle{same}
\hypersetup{
  pdftitle={Outlier Detection (Lab-7)},
  pdfauthor={Bibek Sapkota},
  hidelinks,
  pdfcreator={LaTeX via pandoc}}

\title{Outlier Detection (Lab-7)}
\author{Bibek Sapkota}
\date{}

\begin{document}
\maketitle

\hypertarget{outlier-detection-part-2}{%
\section{Outlier Detection (Part-2)}\label{outlier-detection-part-2}}

\hypertarget{boxplot}{%
\subsection{BoxPlot}\label{boxplot}}

\begin{Shaded}
\begin{Highlighting}[]
\FunctionTok{library}\NormalTok{(}\StringTok{"tidyverse"}\NormalTok{)}
\end{Highlighting}
\end{Shaded}

\begin{verbatim}
## -- Attaching core tidyverse packages ------------------------ tidyverse 2.0.0 --
## v dplyr     1.1.4     v readr     2.1.5
## v forcats   1.0.0     v stringr   1.5.1
## v ggplot2   3.5.0     v tibble    3.2.1
## v lubridate 1.9.3     v tidyr     1.3.1
## v purrr     1.0.2     
## -- Conflicts ------------------------------------------ tidyverse_conflicts() --
## x dplyr::filter() masks stats::filter()
## x dplyr::lag()    masks stats::lag()
## i Use the conflicted package (<http://conflicted.r-lib.org/>) to force all conflicts to become errors
\end{verbatim}

task 1:Ploting the data.

\begin{Shaded}
\begin{Highlighting}[]
\FunctionTok{hist}\NormalTok{(starwars}\SpecialCharTok{$}\NormalTok{height)}
\end{Highlighting}
\end{Shaded}

\includegraphics{OutliersDetection_files/figure-latex/unnamed-chunk-2-1.pdf}

task 2:Creating a new dataset that only includes males and feamles and
creating a boxplot using R

\begin{Shaded}
\begin{Highlighting}[]
\NormalTok{starwars\_mf}\OtherTok{=}\NormalTok{ starwars }\SpecialCharTok{\%\textgreater{}\%} \FunctionTok{filter}\NormalTok{(sex }\SpecialCharTok{\%in\%} \FunctionTok{c}\NormalTok{(}\StringTok{"male"}\NormalTok{, }\StringTok{"female"}\NormalTok{))}

\FunctionTok{boxplot}\NormalTok{(starwars\_mf}\SpecialCharTok{$}\NormalTok{height)}
\end{Highlighting}
\end{Shaded}

\includegraphics{OutliersDetection_files/figure-latex/unnamed-chunk-3-1.pdf}

task 3:creating seprate boxplot for both

\begin{Shaded}
\begin{Highlighting}[]
\FunctionTok{boxplot}\NormalTok{(height}\SpecialCharTok{\textasciitilde{}}\NormalTok{sex, }\AttributeTok{data =}\NormalTok{ starwars\_mf)}
\end{Highlighting}
\end{Shaded}

\includegraphics{OutliersDetection_files/figure-latex/unnamed-chunk-4-1.pdf}

\hypertarget{q4-how-many-outliers-in-males-and-females}{%
\paragraph{Q4: How many outliers in males and
females?}\label{q4-how-many-outliers-in-males-and-females}}

Ans- There is 1 outliers in females and 12 in males.

task 4:Filtering out outlier values

\begin{Shaded}
\begin{Highlighting}[]
\NormalTok{outliers }\OtherTok{\textless{}{-}} \FunctionTok{boxplot}\NormalTok{(starwars\_mf}\SpecialCharTok{$}\NormalTok{height, }\AttributeTok{plot=}\ConstantTok{FALSE}\NormalTok{)}\SpecialCharTok{$}\NormalTok{out}
\end{Highlighting}
\end{Shaded}

\hypertarget{q5-what-does-plotfalse-do}{%
\paragraph{Q5: What does plot=FALSE
do?}\label{q5-what-does-plotfalse-do}}

Ans- Boxplot function in R, the plot=FALSE parameter is used to suppress
the creation of a plot.

\hypertarget{removing-outliers-from-the-dataset}{%
\subsubsection{Removing outliers from the
dataset}\label{removing-outliers-from-the-dataset}}

task 1:First you need find in which rows the outliers are

\begin{Shaded}
\begin{Highlighting}[]
\NormalTok{starwars\_mf[}\FunctionTok{which}\NormalTok{(starwars\_mf}\SpecialCharTok{$}\NormalTok{height }\SpecialCharTok{\%in\%}\NormalTok{ outliers),] }
\end{Highlighting}
\end{Shaded}

\begin{verbatim}
## # A tibble: 10 x 14
##    name     height  mass hair_color skin_color eye_color birth_year sex   gender
##    <chr>     <int> <dbl> <chr>      <chr>      <chr>          <dbl> <chr> <chr> 
##  1 Chewbac~    228   112 brown      unknown    blue             200 male  mascu~
##  2 Yoda         66    17 white      green      brown            896 male  mascu~
##  3 Wicket ~     88    20 brown      brown      brown              8 male  mascu~
##  4 Sebulba     112    40 none       grey, red  orange            NA male  mascu~
##  5 Ratts T~     79    15 none       grey, blue unknown           NA male  mascu~
##  6 Dud Bolt     94    45 none       blue, grey yellow            NA male  mascu~
##  7 Gasgano     122    NA none       white, bl~ black             NA male  mascu~
##  8 Yarael ~    264    NA none       white      yellow            NA male  mascu~
##  9 Lama Su     229    88 none       grey       black             NA male  mascu~
## 10 Tarfful     234   136 brown      brown      blue              NA male  mascu~
## # i 5 more variables: homeworld <chr>, species <chr>, films <list>,
## #   vehicles <list>, starships <list>
\end{verbatim}

task 2:you can remove the rows containing the outliers, one possible
option is:

\begin{Shaded}
\begin{Highlighting}[]
\NormalTok{starwars\_mf\_new }\OtherTok{\textless{}{-}}\NormalTok{ starwars\_mf[}\SpecialCharTok{{-}}\FunctionTok{which}\NormalTok{(starwars\_mf}\SpecialCharTok{$}\NormalTok{ height }\SpecialCharTok{\%in\%}\NormalTok{ outliers),]}
\end{Highlighting}
\end{Shaded}

task 3:check outliers with boxplot

\begin{Shaded}
\begin{Highlighting}[]
\FunctionTok{boxplot}\NormalTok{(starwars\_mf\_new}\SpecialCharTok{$}\NormalTok{height) }
\end{Highlighting}
\end{Shaded}

\includegraphics{OutliersDetection_files/figure-latex/unnamed-chunk-8-1.pdf}

\hypertarget{ux3c3-rule}{%
\subsection{3σ Rule}\label{ux3c3-rule}}

task 1: calculating standard deviation and mean

\begin{Shaded}
\begin{Highlighting}[]
\NormalTok{sd\_value }\OtherTok{\textless{}{-}} \FunctionTok{sd}\NormalTok{(starwars\_mf}\SpecialCharTok{$}\NormalTok{height, }\AttributeTok{na.rm =} \ConstantTok{TRUE}\NormalTok{)}
\NormalTok{sd\_value}
\end{Highlighting}
\end{Shaded}

\begin{verbatim}
## [1] 33.06843
\end{verbatim}

\begin{Shaded}
\begin{Highlighting}[]
\NormalTok{mean\_value }\OtherTok{\textless{}{-}} \FunctionTok{mean}\NormalTok{(starwars\_mf}\SpecialCharTok{$}\NormalTok{height, }\AttributeTok{na.rm =} \ConstantTok{TRUE}\NormalTok{)}
\NormalTok{mean\_value}
\end{Highlighting}
\end{Shaded}

\begin{verbatim}
## [1] 177.6338
\end{verbatim}

\hypertarget{q6-in-above-equations-what-should-be-given-as-the-value-of-na.rm-and-why}{%
\paragraph{Q6: In above equations, what should be given as the value of
na.rm and
why?}\label{q6-in-above-equations-what-should-be-given-as-the-value-of-na.rm-and-why}}

Ans- In the given equations, the value of na.rm should be TRUE. This is
because na.rm stands for ``NA remove'', and setting it to TRUE will
instruct the functions to remove any NA (missing) values before
performing the calculation. If NA values are not removed, the functions
sd and mean will return NA as the result because they cannot compute the
standard deviation or mean with missing values present.

task 2:Then calculate upper and lower bounds

\begin{Shaded}
\begin{Highlighting}[]
\NormalTok{upper\_bound }\OtherTok{\textless{}{-}}\NormalTok{ mean\_value }\SpecialCharTok{+} \DecValTok{3}\SpecialCharTok{*}\NormalTok{sd\_value }
\NormalTok{upper\_bound}
\end{Highlighting}
\end{Shaded}

\begin{verbatim}
## [1] 276.8391
\end{verbatim}

\begin{Shaded}
\begin{Highlighting}[]
\NormalTok{lower\_bound }\OtherTok{\textless{}{-}}\NormalTok{ mean\_value }\SpecialCharTok{{-}} \DecValTok{3}\SpecialCharTok{*}\NormalTok{sd\_value }
\NormalTok{lower\_bound}
\end{Highlighting}
\end{Shaded}

\begin{verbatim}
## [1] 78.42851
\end{verbatim}

task 3: Extract outliers

\begin{Shaded}
\begin{Highlighting}[]
\NormalTok{outliers\_sigma }\OtherTok{\textless{}{-}}\NormalTok{starwars\_mf }\SpecialCharTok{\%\textgreater{}\%} \FunctionTok{filter}\NormalTok{((height }\SpecialCharTok{\textgreater{}}\NormalTok{ upper\_bound)}\SpecialCharTok{|}\NormalTok{ (height }\SpecialCharTok{\textless{}}\NormalTok{ lower\_bound)) }
\NormalTok{outliers\_sigma}
\end{Highlighting}
\end{Shaded}

\begin{verbatim}
## # A tibble: 1 x 14
##   name  height  mass hair_color skin_color eye_color birth_year sex   gender   
##   <chr>  <int> <dbl> <chr>      <chr>      <chr>          <dbl> <chr> <chr>    
## 1 Yoda      66    17 white      green      brown            896 male  masculine
## # i 5 more variables: homeworld <chr>, species <chr>, films <list>,
## #   vehicles <list>, starships <list>
\end{verbatim}

\hypertarget{hampel-identifier}{%
\subsection{Hampel Identifier}\label{hampel-identifier}}

task 1: Calculate median and MAD

\begin{Shaded}
\begin{Highlighting}[]
\NormalTok{median\_value }\OtherTok{\textless{}{-}} \FunctionTok{median}\NormalTok{(starwars\_mf}\SpecialCharTok{$}\NormalTok{height, }\AttributeTok{na.rm =} \ConstantTok{TRUE}\NormalTok{)}
\NormalTok{median\_value}
\end{Highlighting}
\end{Shaded}

\begin{verbatim}
## [1] 183
\end{verbatim}

\begin{Shaded}
\begin{Highlighting}[]
\NormalTok{MAD\_value }\OtherTok{\textless{}{-}} \FunctionTok{mad}\NormalTok{ (starwars\_mf}\SpecialCharTok{$}\NormalTok{height, }\AttributeTok{na.rm =} \ConstantTok{TRUE}\NormalTok{) }
\NormalTok{MAD\_value}
\end{Highlighting}
\end{Shaded}

\begin{verbatim}
## [1] 19.2738
\end{verbatim}

\end{document}
