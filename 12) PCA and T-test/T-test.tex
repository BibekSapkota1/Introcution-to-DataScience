% Options for packages loaded elsewhere
\PassOptionsToPackage{unicode}{hyperref}
\PassOptionsToPackage{hyphens}{url}
%
\documentclass[
]{article}
\usepackage{amsmath,amssymb}
\usepackage{iftex}
\ifPDFTeX
  \usepackage[T1]{fontenc}
  \usepackage[utf8]{inputenc}
  \usepackage{textcomp} % provide euro and other symbols
\else % if luatex or xetex
  \usepackage{unicode-math} % this also loads fontspec
  \defaultfontfeatures{Scale=MatchLowercase}
  \defaultfontfeatures[\rmfamily]{Ligatures=TeX,Scale=1}
\fi
\usepackage{lmodern}
\ifPDFTeX\else
  % xetex/luatex font selection
\fi
% Use upquote if available, for straight quotes in verbatim environments
\IfFileExists{upquote.sty}{\usepackage{upquote}}{}
\IfFileExists{microtype.sty}{% use microtype if available
  \usepackage[]{microtype}
  \UseMicrotypeSet[protrusion]{basicmath} % disable protrusion for tt fonts
}{}
\makeatletter
\@ifundefined{KOMAClassName}{% if non-KOMA class
  \IfFileExists{parskip.sty}{%
    \usepackage{parskip}
  }{% else
    \setlength{\parindent}{0pt}
    \setlength{\parskip}{6pt plus 2pt minus 1pt}}
}{% if KOMA class
  \KOMAoptions{parskip=half}}
\makeatother
\usepackage{xcolor}
\usepackage[margin=1in]{geometry}
\usepackage{color}
\usepackage{fancyvrb}
\newcommand{\VerbBar}{|}
\newcommand{\VERB}{\Verb[commandchars=\\\{\}]}
\DefineVerbatimEnvironment{Highlighting}{Verbatim}{commandchars=\\\{\}}
% Add ',fontsize=\small' for more characters per line
\usepackage{framed}
\definecolor{shadecolor}{RGB}{248,248,248}
\newenvironment{Shaded}{\begin{snugshade}}{\end{snugshade}}
\newcommand{\AlertTok}[1]{\textcolor[rgb]{0.94,0.16,0.16}{#1}}
\newcommand{\AnnotationTok}[1]{\textcolor[rgb]{0.56,0.35,0.01}{\textbf{\textit{#1}}}}
\newcommand{\AttributeTok}[1]{\textcolor[rgb]{0.13,0.29,0.53}{#1}}
\newcommand{\BaseNTok}[1]{\textcolor[rgb]{0.00,0.00,0.81}{#1}}
\newcommand{\BuiltInTok}[1]{#1}
\newcommand{\CharTok}[1]{\textcolor[rgb]{0.31,0.60,0.02}{#1}}
\newcommand{\CommentTok}[1]{\textcolor[rgb]{0.56,0.35,0.01}{\textit{#1}}}
\newcommand{\CommentVarTok}[1]{\textcolor[rgb]{0.56,0.35,0.01}{\textbf{\textit{#1}}}}
\newcommand{\ConstantTok}[1]{\textcolor[rgb]{0.56,0.35,0.01}{#1}}
\newcommand{\ControlFlowTok}[1]{\textcolor[rgb]{0.13,0.29,0.53}{\textbf{#1}}}
\newcommand{\DataTypeTok}[1]{\textcolor[rgb]{0.13,0.29,0.53}{#1}}
\newcommand{\DecValTok}[1]{\textcolor[rgb]{0.00,0.00,0.81}{#1}}
\newcommand{\DocumentationTok}[1]{\textcolor[rgb]{0.56,0.35,0.01}{\textbf{\textit{#1}}}}
\newcommand{\ErrorTok}[1]{\textcolor[rgb]{0.64,0.00,0.00}{\textbf{#1}}}
\newcommand{\ExtensionTok}[1]{#1}
\newcommand{\FloatTok}[1]{\textcolor[rgb]{0.00,0.00,0.81}{#1}}
\newcommand{\FunctionTok}[1]{\textcolor[rgb]{0.13,0.29,0.53}{\textbf{#1}}}
\newcommand{\ImportTok}[1]{#1}
\newcommand{\InformationTok}[1]{\textcolor[rgb]{0.56,0.35,0.01}{\textbf{\textit{#1}}}}
\newcommand{\KeywordTok}[1]{\textcolor[rgb]{0.13,0.29,0.53}{\textbf{#1}}}
\newcommand{\NormalTok}[1]{#1}
\newcommand{\OperatorTok}[1]{\textcolor[rgb]{0.81,0.36,0.00}{\textbf{#1}}}
\newcommand{\OtherTok}[1]{\textcolor[rgb]{0.56,0.35,0.01}{#1}}
\newcommand{\PreprocessorTok}[1]{\textcolor[rgb]{0.56,0.35,0.01}{\textit{#1}}}
\newcommand{\RegionMarkerTok}[1]{#1}
\newcommand{\SpecialCharTok}[1]{\textcolor[rgb]{0.81,0.36,0.00}{\textbf{#1}}}
\newcommand{\SpecialStringTok}[1]{\textcolor[rgb]{0.31,0.60,0.02}{#1}}
\newcommand{\StringTok}[1]{\textcolor[rgb]{0.31,0.60,0.02}{#1}}
\newcommand{\VariableTok}[1]{\textcolor[rgb]{0.00,0.00,0.00}{#1}}
\newcommand{\VerbatimStringTok}[1]{\textcolor[rgb]{0.31,0.60,0.02}{#1}}
\newcommand{\WarningTok}[1]{\textcolor[rgb]{0.56,0.35,0.01}{\textbf{\textit{#1}}}}
\usepackage{graphicx}
\makeatletter
\def\maxwidth{\ifdim\Gin@nat@width>\linewidth\linewidth\else\Gin@nat@width\fi}
\def\maxheight{\ifdim\Gin@nat@height>\textheight\textheight\else\Gin@nat@height\fi}
\makeatother
% Scale images if necessary, so that they will not overflow the page
% margins by default, and it is still possible to overwrite the defaults
% using explicit options in \includegraphics[width, height, ...]{}
\setkeys{Gin}{width=\maxwidth,height=\maxheight,keepaspectratio}
% Set default figure placement to htbp
\makeatletter
\def\fps@figure{htbp}
\makeatother
\setlength{\emergencystretch}{3em} % prevent overfull lines
\providecommand{\tightlist}{%
  \setlength{\itemsep}{0pt}\setlength{\parskip}{0pt}}
\setcounter{secnumdepth}{-\maxdimen} % remove section numbering
\ifLuaTeX
  \usepackage{selnolig}  % disable illegal ligatures
\fi
\IfFileExists{bookmark.sty}{\usepackage{bookmark}}{\usepackage{hyperref}}
\IfFileExists{xurl.sty}{\usepackage{xurl}}{} % add URL line breaks if available
\urlstyle{same}
\hypersetup{
  pdftitle={T-test},
  pdfauthor={Bibek Sapkota},
  hidelinks,
  pdfcreator={LaTeX via pandoc}}

\title{T-test}
\author{Bibek Sapkota}
\date{}

\begin{document}
\maketitle

task 1: Load built-in sleep data set

\begin{Shaded}
\begin{Highlighting}[]
\NormalTok{sleep}
\end{Highlighting}
\end{Shaded}

\begin{verbatim}
##    extra group ID
## 1    0.7     1  1
## 2   -1.6     1  2
## 3   -0.2     1  3
## 4   -1.2     1  4
## 5   -0.1     1  5
## 6    3.4     1  6
## 7    3.7     1  7
## 8    0.8     1  8
## 9    0.0     1  9
## 10   2.0     1 10
## 11   1.9     2  1
## 12   0.8     2  2
## 13   1.1     2  3
## 14   0.1     2  4
## 15  -0.1     2  5
## 16   4.4     2  6
## 17   5.5     2  7
## 18   1.6     2  8
## 19   4.6     2  9
## 20   3.4     2 10
\end{verbatim}

task 2:Making a wide version of the sleep data; below we'll see how to
work with data in both long and wide formats

\begin{Shaded}
\begin{Highlighting}[]
\NormalTok{sleep\_wide }\OtherTok{\textless{}{-}} \FunctionTok{data.frame}\NormalTok{(}
    \AttributeTok{ID=}\DecValTok{1}\SpecialCharTok{:}\DecValTok{10}\NormalTok{,}
    \AttributeTok{group1=}\NormalTok{sleep}\SpecialCharTok{$}\NormalTok{extra[}\DecValTok{1}\SpecialCharTok{:}\DecValTok{10}\NormalTok{],}
    \AttributeTok{group2=}\NormalTok{sleep}\SpecialCharTok{$}\NormalTok{extra[}\DecValTok{11}\SpecialCharTok{:}\DecValTok{20}\NormalTok{]}
\NormalTok{)}
\NormalTok{sleep\_wide}
\end{Highlighting}
\end{Shaded}

\begin{verbatim}
##    ID group1 group2
## 1   1    0.7    1.9
## 2   2   -1.6    0.8
## 3   3   -0.2    1.1
## 4   4   -1.2    0.1
## 5   5   -0.1   -0.1
## 6   6    3.4    4.4
## 7   7    3.7    5.5
## 8   8    0.8    1.6
## 9   9    0.0    4.6
## 10 10    2.0    3.4
\end{verbatim}

\hypertarget{compare-two-samples-using-t-test}{%
\subsection{Compare two samples using
T-test}\label{compare-two-samples-using-t-test}}

task 1:Welch t-test

\begin{Shaded}
\begin{Highlighting}[]
\FunctionTok{t.test}\NormalTok{(extra }\SpecialCharTok{\textasciitilde{}}\NormalTok{ group, sleep)}
\end{Highlighting}
\end{Shaded}

\begin{verbatim}
## 
##  Welch Two Sample t-test
## 
## data:  extra by group
## t = -1.8608, df = 17.776, p-value = 0.07939
## alternative hypothesis: true difference in means between group 1 and group 2 is not equal to 0
## 95 percent confidence interval:
##  -3.3654832  0.2054832
## sample estimates:
## mean in group 1 mean in group 2 
##            0.75            2.33
\end{verbatim}

task 2: Student t-test

\begin{Shaded}
\begin{Highlighting}[]
\FunctionTok{t.test}\NormalTok{(extra }\SpecialCharTok{\textasciitilde{}}\NormalTok{ group, sleep, }\AttributeTok{var.equal=}\ConstantTok{TRUE}\NormalTok{)}
\end{Highlighting}
\end{Shaded}

\begin{verbatim}
## 
##  Two Sample t-test
## 
## data:  extra by group
## t = -1.8608, df = 18, p-value = 0.07919
## alternative hypothesis: true difference in means between group 1 and group 2 is not equal to 0
## 95 percent confidence interval:
##  -3.363874  0.203874
## sample estimates:
## mean in group 1 mean in group 2 
##            0.75            2.33
\end{verbatim}

\hypertarget{q1what-is-the-p-value-of-the-t-test}{%
\subsubsection{Q1:What is the p-value of the
t-test?}\label{q1what-is-the-p-value-of-the-t-test}}

0.s- The p value of t-test is 0.07919

\hypertarget{q2if-significance-level-equals-0.05-do-you-accept-alternative-hypothesis-or-reject-null-hypothesis}{%
\subsubsection{Q2:If significance level (∝) equals 0.05, do you accept
alternative hypothesis (or reject null
hypothesis)}\label{q2if-significance-level-equals-0.05-do-you-accept-alternative-hypothesis-or-reject-null-hypothesis}}

Ans- If the p-value is less than or equal to 0.05, you reject the null
hypothesis and accept the alternative hypothesis, indicating statistical
significance. If the p-value is greater than 0.05, you do not reject the
null hypothesis, meaning there isn't enough evidence to support the
alternative hypothesis.

\end{document}
